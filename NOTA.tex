\chapter{Nota à segunda edição}

Quase uma década decorreu desde a primeira edição deste volume de traduções,
acompanhadas de pequenos comentários, dos fragmentos da poesia mélica
arcaica de Safo, precedidos por detalhada introdução abarcando a poeta, seu
universo cultural, o gênero poético que praticou, as demais poetas
mulheres e a transmissão do \textit{corpus} de sua obra até nós.

A antologia teve por critério oferecer ao leitor todo fragmento
minimamente legível, destacando no título a única canção completa de Safo, 
salvo por um problema num de seus versos. Isso significa a exclusão de fragmentos em que mal
vemos uma palavra inteira, de outros em que há apenas
palavras esparsas que, poucas e soltas, mal permitem a reconstituição de
qualquer leitura, e ainda de outros que contêm uma única palavra --
estes incluídos somente quando podem integrar um conjunto que viabilize
alguma compreensão. Logo, o termo ``antologia'' é usado não para indicar 
a escolha de textos mais famosos ou algo desse
tipo, mas a reunião dos que podemos ler e entender
em algum nível, nos quais há algo de palpável.

O critério se mantém, por certo, mas um número razoável de fragmentos, revisitados,
foram incluídos desta vez, e distribuídos nas seções temáticas já
existentes ou abertas para acomodá"-los. Nesse processo, fragmentos já 
traduzidos na primeira edição foram nesta reorganizados, e 
em muitos casos os comentários foram expandidos.

Todo o material passou por revisão, e mesmo as traduções, parte delas,
trazem modificações nesta edição. Mas nisso nada há a estranhar. Para alguns, a
tradução é tarefa que, uma vez feita, não mais se altera. Para mim,
é tarefa contínua, porque articulada à compreensão, à análise,
e mesmo à sensibilidade ao texto original que, quero crer, sofre os
efeitos positivos do amadurecimento da tradutora e pesquisadora. Claro, restrinjo as
alterações ao mínimo necessário ou irresistível, na oportunidade que
me é dada.

A revisão da introdução que preparei quando da 1ª edição não
se limitou à redação -- que, seguramente, sempre pode melhorar --, mas
procurou rever certos problemas e elaborações que hoje me pareceram
insatisfatórias. De novo, só interferi naquilo que de fato merecia ser
reformulado, sobretudo o que é relativo à nova e mais consistente
apreciação da coralidade na mélica de Safo e da natureza do grupo por
ela liderado de \textit{parthénoi} -- termo que nomeia tecnicamente o
estágio transicional das moças não casadas, virgens, mas na puberdade
e, portanto, prontas para a boda, \textit{gámos}. Tal apreciação é fruto
do impacto nos estudos da poeta causado pelas pesquisas sobre a canção
coral e a \textit{performance}, e, em especial, da descoberta em 2004
da ``Canção sobre a velhice'', fragmento não numerado -- como é o caso também do último --, 
porque anterior à edição de autoridade adotada para os textos gregos, 
de Eva"-Maria Voigt, \textit{Sappho et Alcaeus}, de 1971.

No mundo acadêmico, a produção de conhecimento não para, e não há que
ansiar por verdades inamovíveis, absolutas -- menos ainda no universo
das Humanidades, que lidam com a cultura e seus objetos,
e no campo das Letras Clássicas que, vez por outra, é obrigado a rever
teorias, posturas e o mais, seja porque algo novo vem à luz, como o 
referido fragmento sáfico, seja porque para algo se elabora uma compreensão mais
sólida.

Esta 2ª edição deu"-me a chance de mostrar esse movimento natural e
esperado das pesquisas em torno de Safo e da mélica, que estimulou
estudos recentes sobre a poeta, incluindo os meus próprios, listados no
último acréscimo que fiz a este volume -- um adendo à bibliografia
da primeira edição.

Que o leitor possa, com a mediação deste trabalho de tradução organizada
e abordagem contextualizada, admirar uma das maiores vozes da poesia grega, que
atravessa os séculos, e que já os antigos poetas celebravam com versos
como estes com os quais encerro estas breves linhas.
Na geografia dos montes diletos às Musas, na imagem do deus do enlace dos noivos 
e a perda da virgindade da noiva,\footnote{Himeneu.} na figura de Afrodite, deusa 
que rege \textit{éros} e que dele é vítima pelo arrebatamento diante 
da beleza do jovem Adônis, filho do cíprio Ciniras, tais versos celebram 
linhas de força da mélica de Safo -- o desejo, o casamento, Afrodite -- como 
mostrará esta coletânea de seus fragmentos.

\pagebreak

\chapter*{}
\thispagestyle{empty}

\vspace*{\fill}
\begin{verse}
\small{Ó Safo, aos jovens que amam o mais doce travesseiro das paixões,\\
a ti, junto às Musas, a Piéria adorna, ou o\\
Hélicon coberto de hera --- a ti que sopras tal qual\\
elas a ti, Musa na Ereso eólia.\\
Ou Hímen Himeneu, portando sua tocha brilhante,\\
contigo fica sobre o tálamo nupcial;\\
ou junto a Afrodite enlutada, lamentando o jovem rebento de\\
Ciniras, contemplas o bosque sacro dos venturosos.\\
Em toda parte, ó soberana, te saúdo como aos deuses, pois tuas\\
canções ainda hoje consideramos filhas dos imortais.}\footnote{O verso acima alude ao epigrama 407 do livro \textsc{vii} da \textit{Antologia palatina}.}
\end{verse}

\begin{flushright}
\small{\textit{Dioscúrides, século \textsc{iii} a.C.}}
\end{flushright}
