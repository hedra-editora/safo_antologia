\documentclass[semcabeco,showtrims,trimframe,12pt,conselho,spreadimages]{memoir}

\usepackage[largepost]{hedraoptions} %% << %%%%%%%%%%%%%%%%
\usepackage[baruch]{hedrastyles}
\usepackage[xetex,chicagofootnotes]{tipografia}
\usepackage[standart,compontinhos]{toc}
\usepackage{hedraextra}
\usepackage{penalidades}
\usepackage{graficos}
\usepackage{hedralogo}
\usepackage{hifensextras}
\usepackage{fichatecnica}
\usepackage[standart]{aparatos}
\usepackage{tabelas}
\usepackage{versos}
\usepackage{gitrevisioninfo}


%Pequeno resumo abaixo dos itens do sumário

\makeatletter
\newcommand*\l@chapterinfo{\@nodottedtocline{0}{0.0em}{1.5em}}
\newcommand*\l@sectioninfo{\@nodottedtocline{1}{1.5em}{2.3em}}
\newcommand*\l@subsectioninfo{\@nodottedtocline{2}{3.8em}{3.2em}}
\newcommand*\l@subsubsectioninfo{\@nodottedtocline{3}{7.0em}{4.1em}}
\newcommand*\l@paragraphinfo{\@nodottedtocline{4}{10em}{5em}}
\newcommand*\l@subparagraphinfo{\@nodottedtocline{5}{12em}{6em}}

\def\@nodottedtocline#1#2#3#4#5{%
  \ifnum #1>\c@tocdepth \else
    \vskip \z@ \@plus.2\p@
    {\leftskip #2\relax \rightskip \@tocrmarg \parfillskip -\rightskip
     \parindent #2\relax\@afterindenttrue
     \interlinepenalty\@M
     \leavevmode
     \@tempdima #3\relax
     \advance\leftskip \@tempdima \null\nobreak\hskip -\leftskip
     {#4}\nobreak
     \leaders\hbox{$\m@th
        \mkern \@dotsep mu\hbox{\,}\mkern \@dotsep
        mu$}\hfill
     \nobreak
     \hb@xt@\@pnumwidth{\hfil\normalfont \normalcolor }%
     \par}%
  \fi}

\makeatother

\def\chapterinfo#1{%
    \addcontentsline{toc}{chapterinfo}{%
    \noexpand\small\itshape\numberline{}#1 \smallskip}%
}
\def\sectioninfo#1{%
    \addcontentsline{toc}{sectioninfo}{%
    \noexpand\numberline{}#1}%
}
\def\subsectioninfo#1{%
    \addcontentsline{toc}{subsectioninfo}{%
    \noexpand\numberline{}#1}%
}


\newcommand{\forceindent}{\leavevmode{\parindent=1,4em\indent}}

\linespread{1.15}

\usepackage{endnotes}
\renewcommand{\notesname}{Notas}

\setlength{\marginparwidth}{2em}
\newcommand{\num}[1]{\marginpar{\scriptsize#1}}

\usepackage{makeidx,hedraindex}  % cria índice
\makeindex	

%\counterwithin*{endnote}{part}
%\counterwithin*{endnote}{chapter}

\let\latexchapter\chapter
\makeatletter
\renewcommand\enoteheading{%
  \setcounter{secnumdepth}{-2}
  \latexchapter*{\notesname\markboth{NOTAS}{}}
  \mbox{}\par\vskip-\baselineskip
  \let\@afterindentfalse\@afterindenttrue
}
\makeatother
%\usepackage{fancyhdr}
%\pagestyle{fancy}
%\setlength{\headheight}{9mm}
%\fancyhf{}
%\fancyhead[R]{\thepage}
%\renewcommand{\headrulewidth}{0pt}

%\lhead[\fancyplain{}]{}
%\chead[\fancyplain{}]{}
%\rhead[\fancyplain{}]{\cnvt{\thepage} -- \thepage}

%\newcommand*{\cnvt}[1]{\the\numexpr#1-1\relax}

%\fancypagestyle{chapter}{
%\pagestyle{fancy}
%\setlength{\headheight}{5mm}
%\fancyhf{}
%\fancyhead[R]{\thepage}
%\renewcommand{\headrulewidth}{0pt}}


\usepackage{footmisc}

\renewcommand*\footnoterule{}
%\fancyhf[RO]{\cnvt{\thepage} -- \thepage}
%\fancyfoot{}
%\renewcommand{\headrulewidth}{0pt}
%\renewcommand{\footrulewidth}{0pt}}

\usepackage{fontspec}

%\usepackage{Formular}
\newcommand{\slsc}[1]{\fontspec[SmallCapsFeatures={FakeSlant=0.6}]{Formular-LightItalic}\textsc{#1}\fontspec[]{Formular-LightItalic}}

\newcommand{\bold}[1]{\fontspec[SmallCapsFeatures={FakeSlant=0.6}]{Formular-BoldItalic}\textsc{#1}\fontspec[]{Formular-BoldItalic}}

\newcommand{\italic}[1]{\fontspec[SmallCapsFeatures={FakeSlant=0.6}]{Formular-Italic}\textsc{#1}\fontspec[]{Formular-Italic}}


%\usepackage{Formular}
\newfontfamily\Formular{Formular-Regular}[
BoldFont = Formular-Bold.otf,
ItalicFont = Formular-Light.otf]

%--------------------------------------------ALTERAR DISTÃNCIA ENTRE TÍTULO DO SUMÁRIO E CAPÍTULOS
%\addtocontents{toc}{\vskip-15pt}
%--------------------------------------------
\usepackage{afterpage}

\newcommand\blankpage{%
    \null
    \thispagestyle{empty}%
    \addtocounter{page}{0}%
    \newpage}

    

%\usepackage{imakeidx} 
%\makeindex[program=xindy, options=-C utf8 -L portuguese]
%\newcommand\gobbleone[1]{}
%\newcommand*{\seeonly}[2]{\ (\emph{\seename} #1)}
%\newcommand*{\also}[2]{\emph{cf.} #1}
%\newcommand{\Also}[2]{\emph{See also} #1}
%\renewcommand\indexname{Índice onomástico}
%\makeindex[intoc]

\setcounter{tocdepth}{1}
\setcounter{secnumdepth}{-2} 
%\linespread{1.08}
\usepackage{commands}

\usepackage{setspace}

\makeatletter
\newenvironment{Parskip}{%
   \par
   \parskip=0.3\baselineskip \advance\parskip by 0pt plus 2pt
   \parindent=\z@
   \def\@listI{\leftmargin\leftmargini
      \topsep\z@ \parsep\parskip \itemsep\z@}
   \let\@listi\@listI
   \@listi
   \def\@listii{\leftmargin\leftmarginii
      \labelwidth\leftmarginii\advance\labelwidth-\labelsep
      \topsep\z@ \parsep\parskip \itemsep\z@}
   \def\@listiii{\leftmargin\leftmarginiii
       \labelwidth\leftmarginiii\advance\labelwidth-\labelsep
       \topsep\z@ \parsep\parskip \itemsep\z@}
   \partopsep=\z@
}{\par}
\makeatother

\makeatletter
\newenvironment{myParskip}{%
   \par
   \parskip=0.2\baselineskip \advance\parskip by 0pt plus 2pt
   \parindent=\z@
   \def\@listI{\leftmargin\leftmargini
      \topsep\z@ \parsep\parskip \itemsep\z@}
   \let\@listi\@listI
   \@listi
   \def\@listii{\leftmargin\leftmarginii
      \labelwidth\leftmarginii\advance\labelwidth-\labelsep
      \topsep\z@ \parsep\parskip \itemsep\z@}
   \def\@listiii{\leftmargin\leftmarginiii
       \labelwidth\leftmarginiii\advance\labelwidth-\labelsep
       \topsep\z@ \parsep\parskip \itemsep\z@}
   \partopsep=\z@
}{\par}
\makeatother

\newcommand{\mystar}{{\fontfamily{lmr}\selectfont$\star$}}

%\makeatletter
%\renewcommand{\@chapapp}{}% Not necessary...
%\newenvironment{chapquote}[2][2em]
%  {\setlength{\@tempdima}{#1}%
%   \def\chapquote@author{#2}%
%   \parshape 1 \@tempdima \dimexpr\textwidth-2\@tempdima\relax%
%   \itshape}
%  {\par\scriptsize\hfill-- \chapquote@author\hspace*{\@tempdima}\par\bigskip}
%\makeatother

%\newcommand\Chapter[2]{\chapter
%  [#1\hfil\hbox{}\protect\linebreak{\itshape#1}]%
%  {#1\\[2ex]\Large\itshape#2}%
%}

\begin{document}

%!TEX root=./LIVRO.tex
\chapter{Nota à segunda edição}

Quase uma década decorreu da publicação deste volume de traduções,
acompanhadas de pequenos comentários, dos fragmentos da poesia mélica
arcaica de Safo, precedidos por longa introdução abarcando a poeta, seu
universo cultural, o gênero poético que praticou, as demais poetas
mulheres e a transmissão do \emph{corpus} de sua obra até nós.

A antologia teve por critério oferecer ao leitor todo fragmento
minimamente legível. Isso significa a exclusão de fragmentos em que mal
vemos uma palavra inteira preservada, de outros em que há apenas
palavras esparsas que, poucas e soltas, mal permitem a reconstituição de
qualquer leitura, e ainda de outros que contêm uma única palavra ---
estes incluídos somente quando podem integrar um conjunto que viabilize
alguma compreensão. Logo, a palavra ``antologia'' é usada em termos
relativos: não indica uma escolha de textos mais famosos ou algo desse
tipo, mas uma reunião dos textos que podemos ler e entender pelo menos
de modo a vislumbrar algo de palpável.

O critério se mantém, por certo, mas alguns fragmentos, reconsiderados,
foram incluídos desta vez, e distribuídos nas seções temáticas já
existentes ou abertas para acomodá"-los e a alguns outros que foram
reorganizados.

Todo o material passou por revisão, e mesmo as traduções, em alguns
casos, trazem pequenas modificações nesta edição. Para alguns, a
tradução é tarefa que, uma vez feita, não mais se altera. Para mim, a
tradução é tarefa contínua, porque articulada à compreensão, à análise,
e mesmo à sensibilidade ao texto original que, quero crer, sofre os
efeitos positivos do amadurecimento da tradutora. Claro, restrinjo as
alterações ao mínimo necessário --- ou irresistível na oportunidade que
me é dada. Mas elas se fazem presentes, aqui e ali, no conjunto dos
fragmentos apresentados.

A revisão da introdução detalhada que preparei quando da 1ª edição não
se limitou à redação --- que, seguramente, sempre pode melhorar ---, mas
procurou rever certos problemas e elaborações que hoje me pareceram
insatisfatórias. De novo, só interferi naquilo que de fato merecia ser
reformulado, sobretudo o que é relativo à nova e mais consistente
apreciação da coralidade na mélica de Safo e na natureza do grupo por
ela liderado de \emph{parthénoi} --- o termo que nomeia tecnicamente o
estágio transicional das moças não casadas, virgens, mas já na puberdade
e, portanto, prontas para a boda (\emph{gámos}). Tal apreciação é fruto
do impacto nos estudos da poeta causado pelas pesquisas sobre a canção
coral e a \emph{performance}, e, sobretudo, da descoberta em 2004 de um
novo fragmento, a ``Canção sobre a velhice'' --- a única não numerada,
porque, recente, não está incluída nas edições de autoridade ---, o
último traduzido neste volume.

No mundo acadêmico, a produção de conhecimento não para, e não há que
ansiar por verdades inamovíveis, absolutas --- menos ainda no universo
das Humanidades, que lidam com a cultura e seus objetos, como a poesia,
e no campo das Letras Clássicas que, vez por outra, é obrigado a rever
teorias, posturas e o mais, seja porque algo novo vem à luz, como o
fragmento sáfico, seja porque para algo se elabora uma compreensão mais
sólida.

Esta 2ª edição deu"-me a chance de mostrar esse movimento natural e
esperado das pesquisas em torno de Safo e da mélica, que estimulou
estudos recentes sobre a poeta, incluindo alguns meus, listados no
último acréscimo que fiz a este volume --- um adendo à bibliografia
inicialmente lançada.

Que o leitor possa, com a mediação deste trabalho de tradução organizada
e abordagem contextualizada, admirar uma das maiores vozes da poesia que
atravessa os séculos, e que já os antigos poetas celebravam com versos
como estes que deixo, encerrando estas breves linhas.

\begin{verse}
Ó Safo, o mais doce travesseiro das paixões \qb{}aos jovens que amam,\\
a ti, junto às Musas, a Piéria adorna, ou o\\
Hélicon coberto de hera --- a ti que sopras tal \qb{}qual\\
elas, a ti, Musa na Ereso eólia.\\
Ou Hímen Himeneu, portando sua tocha \qb{}brilhante,\\
contigo fica sobre o tálamo nupcial;\\
ou junto a Afrodite enlutada, lamentando o \qb{}jovem rebento de\\
Ciniras, contemplas o bosque sacro dos \qb{}venturosos.\\
Em toda parte, ó soberana, te saúdo como aos \qb{}deuses, pois tuas canções\\
ainda hoje consideramos filhas dos imortais.\\[5pt]
(Dioscúrides, século \versal{III} a.C., epigrama 407 do livro \versal{VII} da
\emph{Antologia palatina})
\end{verse}

\chapter[Introdução, \emph{por Giuliana Ragusa} \medskip]{Introdução}
\hedramarkboth{introdução}{giuliana ragusa}

\begin{flushright}
\textsc{giuliana ragusa}
\end{flushright}

\section*{Safo revisitada: viagem pela\break poesia grega
antiga\footnoteInSection{ Amparam estas páginas os seguintes estudos:
Ragusa (2005, pp.~23--53, 55--78; 2010, pp.~23--53, 55--97).}}

\epigraph{Estas mulheres de divinas línguas o Hélicon nutriu --- e o  %\\
rochedo macedônio de Piéria --- com hinos: %\\
Praxila, Mero, Anite eloquente, feminino Homero, %\\
Safo, adorno das lésbias de belos cachos, %\\
Erina, Telesila mui gloriosa e tu, Corina, %\\
o impetuoso escudo de Atena cantando, %\\
Nóssis de feminina língua, e Mirtes, doce de ouvir ---  %\\
todas fazedoras de eternos escritos. %\\
Nove Musas do grande Urano, e nove mesmas %\\
Gaia pariu, para a imperecível alegria dos mortais.}
{\textit{Antologia palatina} (\textit{\textsc{ap}} livro \textsc{ix}, epigrama 26, de Antípatro de Tessalônica,
séculos \textsc{i} a.C.--\textsc{i} d.C.)\footnotemark}
\footnotetext{ Tradução de Ragusa (2005, p. 57) aqui com pequenas alterações. Todas as
traduções neste texto, salvo quando indicado, são minhas.}


\noindent{}Ao contrário do que faz supor o mito --- e quando se trata da poeta a que se
dedica este livro, mito e realidade se confundem sem cessar, mal se
distinguindo entre si ---, Safo não é o único nome feminino da poesia da Grécia
	antiga, mas de sua primeira fase histórica, a arcaica\footnote{ O adjetivo é
	usado no sentido de “antiga, remota”; a era arcaica é, por assim dizer, a mais
	antiga da Grécia antiga, e divide"-se em duas etapas: a arcaica, até \textit{c.}
	550 a.C., e a tardo"-arcaica, \textit{c.} 550--450 a.C.; ver Shapiro (2007, pp.
	1--3) a respeito.} (\textit{c.} 800--480 a.C.). Nascida em 630~a.C., de família
aristocrática, na costeira Êresos, oeste da ilha de Lesbos, ela viveu na
proeminente Mitilene, costa oriental, contemporaneamente ao poeta e guerreiro
Alceu. Ambos são os primeiros poetas lésbios dos quais sobreviveram, para cada
um, corpos de obra substanciosos; suas práticas, porém, se beneficiaram,
ressalta Angus M. Bowie, de uma forte e bem reputada tradição
poética lésbio"-eólica, em que se inserem nomes como os dos célebres citaredos
Terpandro (séculos \textsc{viii}--\textsc{vii} a.C.) e Árion (séculos
\textsc{vii}--\textsc{vi} a.C.), que levaram a
outras geografias do mundo grego, e a dois polos culturais da era arcaica ---
	Esparta e Corinto ---, suas práticas métrico"-musicais.\footnote{ Bowie (1984,
	pp.~7--10).} Mais não podemos dizer,
pois do primeiro há só dois fragmentos de autoria duvidosa, e do segundo, nada
resta. De todo modo, a relevância dessas figuras e o peso que conferiram a uma
tradição lésbio"-eólica de canção bem conhecida e firmada se
fazem sentir na imagem que os antigos projetaram de Terpandro, tido como
inovador da música grega num século \textsc{vii} a.C. de ricas experimentações, e
inventor da lira de sete cordas, algo que a arqueologia prova insustentável,
uma vez que o instrumento era já conhecido no mundo minoico"-micênico, que
antecede o que chamamos “Grécia histórica”. E Árion é dado como o
poeta do ditirambo, canção de forte aspecto narrativo.

Safo e Alceu são, ainda, dois dos nomes notáveis de um gênero poético, a
\textit{mélica} ou a lírica propriamente dita --- a canção destinada à
\textit{performance} em solo ou coral,
com o acompanhamento da lira (e de outros instrumentos e da dança, na
modalidade em coro). Se falo em \textit{performance} é porque, recorde"-se
desde já, sobretudo no período arcaico e depois no clássico (\textit{c.}
480--323 a.C.), pelo menos até \textit{c}. 400 a.C., a poesia grega é eminentemente de tradição oral e inserida no
que John Herington chama de “cultura da canção”,\footnote{ Herington (1985, p. 3).} na
qual, recitada ou cantada na \textit{performance}, disseminava
“ideias morais, políticas e sociais”. A oralidade, portanto, marca
a composição e a circulação dessa poesia em \textit{performances} e
\textit{reperformances} profissionais e/ou amadoras a determinada
audiência, de certo modo, em dada ocasião, colocada assim em ligação estreita
com a vida cotidiana da comunidade em que se fazia e pela qual passava, ligação
esta que lhe confere um caráter em essência pragmático. A mélica grega,
como bem ressalta Bruno Gentili, “não foi intimista, no senso
moderno”,\footnote{ Gentili (1990b, p. 9).} uma vez que só existia integrada na
vida da comunidade em meio à qual
circulava oralmente. Não por acaso, a voz poética, apresentada numa
situação de diálogo entre o \textit{performer} e sua audiência, está sempre em
diálogo; em vez do “eu” a falar consigo mesmo ou a ninguém, o “eu”/“nós” sempre
se dirige ao outro, ao “tu”/“vós” com que estabelece a interlocução. Se por
vezes esta não nos é de todo discernível, isso se deve aos problemas
materiais de preservação dos textos. Ora, o diálogo, dimensão viva da
comunicação verbal humana, é um elemento crucial da oralidade, incorporado com
grande força aos gêneros poéticos da Grécia antiga desde a épica homérica e
seus poemas monumentais, a \textit{Ilíada} e a \textit{Odisseia}, em que há uma
divisão quase equivalente entre narrativa de ação e diálogo. 

A oralidade se evidencia na composição da mélica, que se vale regularmente de
estruturas e procedimentos estilísticos de caráter mnemônico, que,
de maneira mais flagrante na era arcaica, refletem a tradição poética oral,
mesmo que já possamos pensar, naquele momento, no uso da escrita --- o alfabeto
grego, adaptação do fenício, se disseminava desde fins do século \textsc{ix} a.C. ---
pelos poetas nos processos e técnicas de construção de seus
versos.\footnote{ Ver a respeito Gentili (1990a, pp.~14--23) e Svenbro (1993, pp.
27--30).} Pensando o caso de Safo, Jesper Svenbro acredita
que ela teve seus textos escritos à sua época e com sua interferência direta,
ela mesma os escrevendo --- algo que pode ser excessivo e que não podemos
comprovar; de todo modo, afirma ele:

\begin{quote}
Um grego que vivesse por volta de 600 a.C., se refletisse sobre o problema de
registrar o poema sob a forma escrita, provavelmente consideraria a questão em
termos de uma \textit{transcrição} de algo que já tinha uma existência
socialmente reconhecida e que tenha sido tecnicamente controlado num estado
oral ou memorizado. Considerar a transcrição como uma operação que tornava o
poema duradouro e famoso não seria necessário; a tradição oral era bastante
capaz de fazer isso, sem o auxílio da escrita.\footnote{ Svenbro (1993, pp.~145--59).}
\end{quote}

Em outras palavras, ainda que aceitemos a possibilidade de que Safo e outros
poetas arcaicos, principalmente, tenham feito uso da escrita, o estudo atento
aos elementos estruturais e estilísticos de suas obras dá a perceber que a oralidade as gera e
sustenta.

\section*{O problemático nome «lírica»}

Decorre do modo de \textit{performance}, justamente, o nome tardio,
\textit{lírica}, que prevalecerá na referência moderna ao variado corpo de
textos --- de vários autores, metros, dialetos, tradições culturais, temas,
\textit{performances}, tons --- que nem são poesia hexamétrica --- como a
épica ---, nem dramática --- tragédia e comédia. Esse uso moderno do nome é decerto
prático, mas acaba por vestir, com um mesmo manto,
gêneros poéticos autônomos e distintos, entre os quais destacam"-se, na era
arcaica, a elegia, o jambo e a mélica. \textit{Mélica}, essa palavra não
dicionarizada em nosso vernáculo, é o termo que os antigos identificavam à
\textit{lírica}, rigorosamente, o gênero da canção para a lira. 

Fica, então, exposta a primeira armadilha de “lírica”, que por isso grafo
entre aspas: suas acepções antiga e moderna não são correspondentes uma
à outra. A segunda reside no fato de que uma mesma designação é empregada para
um gênero ou um conjunto de gêneros de poesia antiga --- fruto de uma cultura em
que tudo se realizava no trânsito entre boca e ouvido --- e para um gênero de
poesia moderna --- fruto de uma cultura da escrita. Isso cria uma falsa impressão
de familiaridade no que tange à poesia antiga, que, se não revertida, acarreta
para sua leitura um olhar modernizante potencialmente equivocado, sobretudo se
guiado pelas expectativas de uma noção comum --- não menos errônea para certa
poesia moderna --- sobre a “lírica” e o “lírico”. Tal noção agrega as ideias da
brevidade, da subjetividade --- amparada na abundante e intensa constância da voz
poética em 1ª pessoa do singular ---, da explosão dos sentimentos. Pautar"-se por
ela, todavia, é esquecer o filtro da dimensão estética, que faz com que
experiências e sentimentos, ainda que pessoais, conhecidos, vividos pelo poeta,
passem pelo processo da elaboração artística. Nele, a linguagem é trabalhada
estilisticamente, formalmente, transformando a experiência ou a emoção --- pouco
importa, a rigor, se vividas ou não pelo poeta --- em experiência
\textit{representada} no presente da composição. Há, portanto, uma filtragem
que impõe um distanciamento que precisa ser observado, mesmo que a voz poética
assuma o nome do próprio poeta, pois, ainda assim, estamos diante de sua
\textit{persona}, artisticamente elaborada em linguagem diversa --- não importa
quão próxima ela se pretenda --- da cotidiana.

Se é preciso atentar para tudo isso já na poesia moderna, mais ainda o é na antiga, em que, dada sua composição genérica, devem
estar articuladas as escolhas que faz o poeta da matéria, do metro, da ocasião
e do modo de \textit{performance}, bem como da linguagem, do caráter, do tom, e
assim por diante. Noutras palavras, a composição dessa poesia --- que, à diferença
da moderna, tem notável caráter pragmático --- centra"-se no gênero, cujas regras,
na época arcaica, são tradicionalmente preservadas e praticadas, sem que
estejam escritas, o que nos obriga a pensar os gêneros de poesia arcaica e
clássica menos como identidades de leis severamente fixas e mais como
“tendências firmadas o suficiente para permitir que afinidades e influências
sejam discerníveis”, observa Chris Carey, gerando
“expectativas na audiência”, mas sem prejuízo da flexibilidade que dá margem “à
frustração e à redefinição de tais expectativas” pelo poeta e pela sua
audiência.\footnote{ Carey (2009, p. 22).}

Claro está, a esta altura, que há uma defasagem considerável, para não dizer
enorme, entre nossas prática e cultura literárias e as antigas, às quais não
podem ser associados, sem grande prejuízo para a compreensão e leitura dos
textos --- na letra fria, artificial e estática de uma poesia feita para a
\textit{performance} em viva voz --- e de seus universos, conceitos relativos à
ideia moderna de “literatura”, tais como originalidade e criatividade. A poesia
elegíaca, jâmbica e mélica da Grécia arcaica, sobretudo, e clássica, resume
Anne P.~Burnett: 

\begin{quote}
[\ldots{}] é mais engenhosa e menos apaixonada,
mais convencional e menos individual do que desejariam os que advogam essa noção [da explosão da individualidade].\footnote{ Burnett (1983, p. 2).} 
\end{quote}

Essa poesia antiga, oral e de
ocasião, “fincada no sistema social de uma \textit{pólis} grega arcaica”,
recorda Wolfgang Rösler, é essencialmente discurso.\footnote{ Ver
Rösler (1985, p. 139), Johnson (1982, p. 72) e Clay \mbox{(1998, p. 11)}.}

Em conclusão a essas palavras, cabe esclarecer que não se trata, aqui, de negar
qualquer medida de identificação entre o poeta e seu “eu” poético, mas de
afirmar que, ao lidarmos com os antigos, o somatório do modo de composição, da
falta de conhecimento sobre a biografia dos poetas e seus contextos
histórico"-sociais, e da precariedade mais ou menos intensa dos textos
sobreviventes, torna essa identificação ainda mais complexa do que é.
E o terceiro ingrediente dessa soma não é de importância menor, como
bem mostra este aviso aos que buscam a obra de Safo que, reunida com rigor,

\begin{quote}
contém apenas um poema completo, aproximadamente dez fragmentos substanciais,
uma centena de citações breves de autores antigos e cerca de 50 peças de textos
em papiro, que emergiram das areias do deserto egípcio.
\end{quote}

Daí ser “mais exato
falar em fragmentos de Safo”,\footnote{ Lardinois (1995, p. 29).} e não em
poemas. Essa síntese citada de André Lardinois é acurada, e de modo algum restrita ao
caso da poeta de Lesbos.
Antes, o cenário da preservação das obras de outros poetas da “lírica” antiga ---
a poesia jâmbica, a elegíaca e a mélica --- é similar ou pior do que
o da obra de Safo. A propósito disso, cito a fala marcante de Walter R. Johnson:

\begin{quote}
Nós todos, quando lemos a lírica grega, ficamos desapontados em certo sentido:
não porque a poesia não impressione --- antes, é supremamente bela ---, mas
porque existe para nós apenas em cacos e farrapos. [\ldots{}] E quando a comparamos
aos outros remanescentes da literatura e da cultura gregas, essas ruínas são de
machucar o coração. Nenhuma experiência de leitura, talvez, é mais deprimente e
mais frustrante do que a de abrir um volume dos fragmentos de Safo e
reconhecer, ainda uma vez --- pois sempre se espera que desta vez seja diferente
---, que essa poesia está perdida para nós.

Esse é um fato que escolhemos não encarar --- não de frente e constantemente.
Logo, divisamos uma ficção --- [\ldots{}] que chamamos poesia lírica grega ---,
mesmo que saibamos que ela está em meros fragmentos. Na verdade,
\textit{porque} sabemos que está em meros fragmentos, agimos, falamos
e escrevemos como se o impensável não tivesse acontecido, como se bispos pios,
monges descuidados e ratos famintos não tivessem consignado Safo e seus colegas
líricos ao esquecimento irremediável. [\ldots{}] Naturalmente, qualquer helenista a
quem você perguntar admitirá o fato da fragmentação. Pode até ter prazer em
descrever o estado verdadeiro dos textos e a incerteza fascinante de
restaurações e conjecturas. Mas se você persistir em seu escrutínio e conversar
sobre a poesia, sobre os poemas não existentes, à medida que a conversa
esquentar, o esqueleto ganhará carne e cor, e a ruína se esvaecerá. Isso não é
prevaricação ou enganação, isso é a natureza humana: nós queremos aqueles
poemas e, nos momentos em que nos desarmamos, nós os imaginamos de volta à
existência. [\ldots{}]

É o leitor, então, que deve se lembrar, quando eu me esquecer, que a lírica
grega [\ldots{}] nos é essencialmente inacessível.\footnote{ Johnson (1982, pp.~25--6).}
\end{quote}

Nesta introdução e na tradução que se lhe seguirá ---
para não esquecer essa realidade e para que o leitor dela tenha consciência,
acrescento à síntese de Lardinois esta outra: aos que procuram a poesia de
Safo, restarão seus fragmentos vivos, desafiadores, pulsantes há séculos, a
despeito de sua fragilidade material; aos que procuram Safo, a mulher, ou Safo,
a \textit{lésbica}, restará pouco mais do que ficções e anedotas. Penso que,
entre a substância não obstante precária dos fragmentos e a névoa tão sedutora
quanto impalpável da biografia, mais profícuo será privilegiar a primeira
opção. Mas falemos um pouco da névoa sáfica.


\section*{Em busca de Safo: poeta de Lesbos}

Um papiro encontrado na antiga cidade egípcia de Oxirrinco, próxima a\EP[]
Alexandria, isto nos conta da vida de Safo (\textit{Papiro de Oxirrinco} 1800,
fr.~1, século \textsc{ii} ou início do \textsc{iii} d.C.): seu pai seria Escamandro ou
Escamandrônimo; seus irmãos, Cáraxo, o mais velho, Erígio e Lárico, o mais
novo; sua filha, Cleis, que levaria o nome da mãe de Safo ``foi acusada
por alguns de ser irregular e amante de mulheres”; era feia, mirrada e de
compleição escura. Já o léxico bizantino \textit{Suda}, compilado no século \textsc{x},
no verbete à poeta repete o segundo nome para o pai de Safo, acrescentando à
lista outros sete; reitera o nome Cleis como sendo de sua mãe e de sua filha, e
os de seus três irmãos; diz ainda que Safo teria sido casada; que teria mantido
amizades impuras com jovens meninas, como Átis, e, por isso, adquirido má
reputação; e teria tido pupilas, como Gongila. Alguns desses nomes se registram
na obra de Safo, ou nas fontes que as preservaram. Mas, como se vê, os
testemunhos antigos se retomam uns aos outros, discordando aqui e ali, tornando
mais intricada a rede de inconsistências, e embasando"-se, claramente, na
leitura dos fragmentos de Safo, em circularidade viciosa. O pequeno painel
biográfico composto pelo papiro e pelo léxico é, pois, indigno de confiança,
para dizer o mínimo, mesmo que associado a outros testemunhos.\footnote{ Na
edição bilíngue de Campbell (1994; 1ª ed.: 1982), são arrolados 61 testemunhos
sobre a poeta e sua obra. O volume é, \mbox{comparado} aos testemunhos de outros
poetas, considerável; de seu contemporâneo Alceu, há 27 testemunhos na mesma edição. 
Quando me referir a testemunhos sobre Safo, uso
sempre a compilação de Campbell.}

O mistério, então, persiste e as perguntas que estimula --- e que com frequência
ganham dimensão desproporcional à própria inviabilidade de solução
corroborada em sólidas evidências --- dificilmente podem ser respondidas: quem
foi Safo e que figura portava? Como se fez a poeta? Como construiu e fez
circular suas canções? Como viveu na Lesbos arcaica? Sustenta"-se a imagem da
Safo \textit{lésbica}? Vejamos um pouco do muito que se diz, e do
pouco que podemos dizer.

Embora seja a mais velha das poetas mulheres listadas no epigrama que abre esta
introdução, atuante numa Grécia da oralidade, em quem a escrita ainda estava
por se consolidar como veículo principal de produção artística, Safo é aquela
cujo \textit{corpus} é o mais extenso (cerca de 200 fragmentos, um dos quais é,
na verdade, uma canção completa, o “Hino a Afrodite”) e cujos testemunhos são
os mais numerosos, referindo"-a desde sua época até a romana. Será a poeta
privilegiada pela sorte, ou sua reconhecida superioridade garantiu sua
preservação mais generosa do que a de outros poetas arcaicos e das poetas
mulheres? Impossível simplificar assim a explicação
para a sobrevivência de Safo ou de qualquer outro poeta grego antigo: nem a
mera sorte, nem a mera fama podem ser responsabilizadas exclusivamente por
nossa boa ou má fortuna quanto ao estado e volume do \textit{corpus}
remanescente de cada um deles.

Se não é Safo a única poeta grega antiga, é a única do período
arcaico. Esse dado, todavia, não deve ser superestimado, pois desconhecemos as
razões que guiaram a sobrevivência da obra dos poetas antigos.
Em princípio, nada impede que Safo tenha sido, de fato, a primeira
poeta grega; mais equilibrada se revela, no entanto, a ideia de que tenha
herdado, como os demais poetas gregos, uma tradição\footnote{ Ver West (1973, pp.~179--92), que mostra ser a
tradição métrica lésbio"-eólica mais antiga do que outras, como a dórica e a
ático"-jônica. Para Skinner (1996, p. 183), haveria, dado o caso de Safo, “uma
longa linha de predecessoras mulheres” poetas, mas nada há que permita sustentar em evidências essa ideia, nem as ideias de uma tradição separada de mulheres na poesia grega e de audiências igualmente separadas e femininas. Ver a propósito Bowman (2004, pp. 1--27).} que, em seu fazer --- haja vista
sua grande reputação firmemente atestada desde sua época ---, alcançou um grau
elevado de maturidade que, no caso das poetas mulheres, pode ter alargado os
caminhos depois dela percorridos.\footnote{ Ver Cantarella (1991, pp.~73--6), De
Martino (1991, pp.~17--75), Bowman (2004, pp. 1--27) e o estudo de Klinck (2008).} A reputação de Safo, atestam os
testemunhos, não encontrou em nenhuma poeta mulher um nome que a
superasse; Estrabão (\textsc{i} a.C.--\textsc{i} d.C.) ressalta isso: 

\begin{quote}
em todo o tempo recordado não sei de mulher
alguma aparecida que seja páreo ou que se aproxime um pouco dela, de Safo,
quanto à graça de sua poesia.\footnote{ \textit{Geografia} (13. 2. 3).}
\end{quote}

Considerado o desenho geográfico composto pelos nomes de poetas mulheres
lembrados no epigrama, Albin Lesky atenta para um detalhe
notável: há poetas mulheres de Lesbos, da Beócia e do Peloponeso, mas não da
Ática. “Assim se manifesta uma posição diferente, mais livre, da mulher”
nessas outras regiões, do que aquela “que conhecemos no mundo de Atenas”.\footnote{ Lesky (1995, p. 210).} A
conclusão do helenista, em princípio possível, leva a outra: nas sociedades
que geraram poetas mulheres, deve ter havido acesso a uma forma de educação
feminina e abertura para a participação maior das mulheres na vida da
comunidade, incluindo as de origem aristocrática, como Safo; daí a provável
aceitação do fazer poético exercido por uma mulher.

A circulação da obra de Safo, e de outras poetas mulheres mais tarde, parece
apontar nessa direção, bem como o silêncio que ressoa da Ática, de Atenas, onde
prevalecia, ao menos na era clássica, o confinamento das mulheres ao \textit{oîkos} (“casa”), espaço
feminino por excelência na Grécia; salvo em ocasiões específicas, como um rito
religioso"-cultual, as mulheres atenienses não deveriam ser vistas, nem suas
vozes ouvidas.\footnote{ Para mais sobre a mulher ateniense, ver os estudos de
Mossé (1991, pp.~49--61) e Murray (1993, p. 41). Reitero que é eloquente --- se
não for obra do acaso --- a ausência de poetas provindas da Ática em nosso
\textit{corpus}, considerada a exclusão social feminina em Atenas. E ressalta
Mossé (pp.~152--3) que só uma esfera de atuação civil das mulheres lhes era
permitida em Atenas: a religiosa, da qual participavam ativamente, em funerais,
casamentos, rituais e festivais públicos e privados.} Na casa, talvez elas
tivessem acesso à educação e mesmo à escrita, mas, à diferença do que teria
ocorrido noutras partes do diversificado mundo grego, não haveria em Atenas
condições, segundo o que se sabe, para que suas manifestações artísticas
pudessem circular.

Nada disso pode ser afirmado com segurança, mas é plausível pensar que o acesso
à educação aristocrática e a maior liberdade de ação e inserção social
contribuíram e muito para que poetas mulheres tivessem condições de existir,
poetas como Safo,\footnote{ Diz Bennett (1994, p. 346):
“Para tornar"-se uma poeta, Safo teve de ser treinada, em expressão
e composição, e nós naturalmente suporíamos que tal treino era aquele de outras
meninas aristocráticas de Mitilene”.} cujas composições, como as dos demais
poetas arcaicos e clássicos, tinham que ser, necessariamente, apresentadas em
determinados modo e ocasião de \textit{performance} --- e dadas a celebridade e a variedade da mélica sáfica, não pode ter sido esta uma só, e nem limitada
a grupos secretos de mulheres. 

Lamentavelmente, não é fácil avaliar a condição feminina na Grécia
antiga, menos ainda na era arcaica, e menos ainda na lésbia Mitilene, cuja
especificidade, em qualquer de suas dimensões, nos escapa quase que
de todo. As evidências escritas ou iconográficas são em toda parte muito
escassas, mas não podemos deixar de mencionar que nos vasos atenienses que
retratam figuras femininas associadas à escrita e/ou à leitura, estas são Safo
--- porque é a poeta de grande fama --- e as Musas --- porque são deusas.
Essas imagens, em última análise, nada provam quanto a Safo --- e outras mulheres
aristocratas --- no que concerne às habilidades da escrita e da leitura --- as
quais são separadas na Antiguidade, o domínio de uma não implicando
automaticamente o da outra, lembra Susan G. Cole, limitando"-se a
leitura, sempre em voz alta, a poucos textos.\footnote{ Cole (1992, p. 220).} Na representação iconográfica,
prossegue ela, os livros --- na verdade, rolos de papiros --- amparam a
recitação, e não a “leitura solitária” e silenciosa; e nas imagens de mulheres,
é típico o desenho de uma que lê para uma moça ou para um grupo feminino.
Quando a leitora é Safo, diz Cole, a imagem quer antes celebrar a poeta, do que
retratar “uma cena familiar ou tipicamente doméstica”, até porque, ressalta,
mesmo em vasos onde figuram cenas domésticas, além dos que trazem a poeta ou as
deusas da poesia, jamais vemos mulheres escrevendo.\footnote{ Cole (1992, p. 224).}

Da busca de Safo, a mulher, e da formação da poeta, voltamos de mãos
praticamente vazias --- praticamente, porque podem carregar ideias plausíveis,
mas estas serão inverificáveis. Da busca da Safo \textit{lésbica}, à qual
aludem os testemunhos tardios do papiro e do \textit{Suda}, o que traremos nas
mãos no retorno será um resultado semelhante, ou até mais esvaziado.

A condição de mulher poeta cujas canções têm na temática erótico"-amorosa e no
universo feminino suas linhas de força --- canções estas lidas em chave
biografizante já pelos antigos que nelas buscavam a substância da figura
histórica para eles, como para nós, esvaecida --- faz emergir a questão da
sexualidade feminina e de seu exercício a partir da figura de Safo,
amiúde reelaborada em tramas antigas e modernas tecidas no correr dos
séculos, desprovidas quase de historicidade atestada, mas
inseridas na rede que se pretende explicativa da poeta e de seus textos. Judith
P.~Hallett\footnote{ Hallett (1996, pp.~125--7). Ao tratar das
imagens da poeta, ela fornece as indicações das fontes antigas.} comenta as
imagens que se formaram, sobretudo, do século \textsc{iv} a.C. em diante; e assim resume
Glenn W.~Most as díspares e desencontradas --- não raro,
extremadas --- imagens de Safo: 

\begin{quote}
As várias fontes que fluíram juntas para criá"-la creditaram"-na com um marido,
uma filha, muitos irmãos, numerosas amigas e companheiras (com as quais, ao
menos segundo alguns relatos, ela teve relações sexuais), numerosos amantes, um
homem que rejeitou as investidas de Safo, e um salto suicida de um penhasco. Em
princípio, decerto, não há razão para que uma vida social tão variada e rica
não tenha sido possível --- embora se pudesse cogitar como, entre um e outro
compromisso, Safo teria encontrado tempo para compor sua poesia [\ldots{}] Mas tanta
complexidade apresenta um desafio a qualquer um que tente imaginar um retrato
coerente da vida de Safo, pois requer que elementos potencialmente divergentes
sejam trazidos a uma relação plausível uns com os outros. Mais
fundamentalmente, a recepção de Safo pode ser interpretada como uma série de
tentativas de chegar a termo com a complexidade dessa gama de informações.\footnote{ Most (1996, p. 14).}
\end{quote}

Ora, a leitura dos modernos dessa intricada trama biográfica não foi, no mais
das vezes, menos infeliz que a dos antigos. Most comenta, por exemplo,
a dos românticos:

\begin{quote}
\textit{Condensando} numa única pessoa as muitas contradições com as quais a
tradição tinha suprido Safo, inventaram uma figura intensamente paradoxal
[\ldots{}] A Safo romântica é a primeira que é, essencialmente, uma poeta --- mas uma
poeta romântica, insatisfeita com a realidade banal e lutando para alcançar a
perfeição espiritual incompatível com a vida e somente alcançável às custas da
morte.\footnote{ Most (1996, p. 20).}
\end{quote}

Em verdade, uma revisão das leituras modernas de Safo mostra que cada época,
cada contexto histórico"-social e cultural criou para si a imagem desejada da
poeta; tal liberdade explica"-se pela carência de conhecimento consistente sobre
Safo e sua vida na arcaica Mitilene --- liberdade, diga"-se ainda, criativa, que
em sedutoras e intrigantes projeções pseudobiográficas, acabaram por
ganhar mais atenção do que merecem, do que sua arte
muitas vezes usada para alavancar --- eis novamente a circularidade viciosa que
já outrora mencionei --- tais projeções. 

A imagem da Safo \textit{lésbica} construiu"-se nesse movimento, e ganhou fôlego
na esteira de vogas na crítica literária --- \textit{gay studies}, \textit{women
studies} --- que valorizam aspectos do entorno dos textos que passam a ser
estudados de modo secundário, e não no primeiro plano, para sustentar a análise
de tais aspectos. Nessa linha, certa leitura de Safo e de certa fatia de
testemunhos sobre a poeta amparam a afirmação de que trata"-se da primeira poeta
\textit{lésbica} do Ocidente, não porque seja filha da ilha de Lesbos, mas
porque, segundo um olhar modernizante e romântico --- que
desconsidera ou minimiza o fato de que a defasagem entre nós e os clássicos
reside também no modo como percebemos a sexualidade, e que o uso da poesia como
história é, no mínimo, imprudente ---, Safo dividia seu leito com mulheres e por
elas era tomada de paixão. Digo certa leitura, porque há outros modos, mais
seguros e coerentes com o que de fato sabemos do mundo antigo grego, de
compreender a prevalência de meninas --- e não mulheres ---  na poesia erótica de Safo; e digo
certa fatia de testemunhos antigos, porque naqueles que enfocam sua
sexualidade, Safo é retratada majoritariamente como promíscua e engajada em relações hetero, e não homoeróticas. Vejamos. 

A partir de Aristófanes (séculos \textsc{v}--\textsc{iv} a.C.) e da comédia clássica, pelo menos, a
referência às mulheres de Lesbos e o uso de verbos como \textit{lesbiázein} e
\textit{lesbízein} (“agir como uma mulher de Lesbos”) conotavam luxúria e
lascívia; em particular, diz Hallett, a prática da
felação, que as lésbias teriam inventado\footnote{ Hallett (1996, p. 129).} --- algo sem qualquer respaldo
histórico. A esse respeito, Gentili observa: 

\begin{quote}
Já na segunda metade do século \textsc{v} a.C. --- e seu uso é certamente bem mais antigo
--- as palavras \textit{Lésbia} ou \textit{moça de Lesbos} tinham a típica
conotação de \textit{fellatrix}, e não de \textit{Lésbica} no sentido moderno
do termo. \textit{Lesbís} e \textit{lesbiázein} eram essencialmente peças de
terminologia sociológica com um significado específico e inequivocamente
erótico.\footnote{ Gentili (1990a, p. 95).}
\end{quote}

Não há como precisar a razão dessa ligação das mulheres de Lesbos a práticas
sexuais específicas, mas talvez isso se deva à fama da beleza incomparável e da
sensualidade das mulheres da ilha, já atestada na \textit{Ilíada} (\textsc{ix}, 128--30),
e/ou à intensidade erótica da poesia de Safo. Seja como for, a habilidade
sexual --- com o sexo oposto --- atribuída às lésbias e a imagem da poeta parecem
emanar, confundidas, no uso do adjetivo “Lésbia” mesmo tarde, em Catulo
(século \textsc{i} a.C.) --- repare"-se, sempre em contextos heteroeróticos, frisa Hallett.\footnote{ Hallet 
(1996, (pp.~129--30). Ver ainda os estudos de Brasete (2003, pp. 17--26; 2009, pp. 289--303) e Ragusa (2019b, pp. 211--39).} 

A despeito dessas ressalvas, \textit{Safo de Lesbos} é designação que
recorrentemente projeta em nosso imaginário certa Safo, a poeta
\textit{lésbica}. E é essa projeção que vem sendo aqui
problematizada, uma vez que não só carece de respaldo em evidências antigas,
confiáveis e consistentes, mas que também resulta de uma percepção da
sexualidade que em muito se distingue da dos antigos, seja em largos termos ---
os concernentes à sexualidade feminina e à masculina ---, seja em termos
restritos, os relativos ao caso de Safo. 

Repare"-se, a propósito, que o adjetivo \textit{lésbica} não existia na
Antiguidade, tampouco o \textit{lesbianismo} em sua concepção moderna --- o amor
entre mulheres, que exclui o amor heterossexual, dado que a moderna
categorização do homossexualismo se faz a partir da oposição a este, lembra Mossé.\footnote{ Mossé (1991, p. 156).} Não há,
ademais, qualquer base com lastro mínimo na afirmação de que teria sido esta
uma opção sexual da poeta que, enquanto sujeito histórico, é pouco mais, se
tanto, do que uma neblina para nossos olhos, como já para os dos antigos. O
termo \textit{lésbica}, enfim, frisa Sue Blundell, “é uma
invenção moderna” que passa a nomear “uma mulher homossexual no final do século
\textsc{xix}, como resultado da publicidade criada por uma controvérsia acadêmica em
torno da sexualidade da própria Safo”\footnote{ Blundell (1995, p. 83).} --- polêmica esta que permanece tão acesa
nos debates quanto estéril nos seus resultados.
Lardinois precisa as datas em que, em
língua inglesa, o adjetivo se verifica, a partir de 1890, e o substantivo
\textit{lesbianismo} (homossexualismo feminino), primeiro em 1870, escrito com
inicial maiúscula, em clara alusão à ilha de Lesbos; e ele indaga:

\begin{quote}
Será justificada a relação entre a ilha de Lesbos e o homossexualismo das
mulheres? Existiriam razões para crer que Safo de Lesbos fosse uma ‘lésbica’?

É essa a Grande Questão Sáfica [\ldots{}] [que] já era debatida na Antiguidade, mas
os estudiosos ainda não foram capazes de chegar a um consenso. Provavelmente,
jamais o consigam, não apenas porque as evidências sejam muito escassas, mas
porque existe algo intrinsecamente errôneo na forma de colocar a questão.\footnote{ Lardinois (1995, pp.~27--8).}
\end{quote}

Concluindo seu estudo, afirma Lardinois:

\begin{quote}
Podemos concluir que, no caso de Safo, estamos, no máximo, diante de
relacionamentos breves entre uma mulher adulta e uma jovem prestes a se casar.
Chamar de ‘lésbicas’ essas relações é um anacronismo. É impossível avaliar se a
palavra se aplica à própria Safo ou à sua vida íntima. Na verdade, essa é uma
questão sem sentido. Mesmo se, pelos padrões modernos, Safo devesse ser
considerada lésbica, sua experiência deve ter sido muito diferente, vivendo,
como viveu, em uma era diferente com diferentes noções e tipos de sexualidade.\footnote{ Lardinois (1995, p. 50).}
\end{quote}

O grau elevado de complexidade da chamada “Questão Sáfica” é um alerta
fundamental contra afirmações simplistas como as que anunciam em Safo a
primeira poeta (engajadamente) \textit{lésbica} da literatura ocidental,
servindo"-se de modo igualmente simplista e redutor de sua poesia para provar a alegada
verdade do rótulo. E se não basta tudo o que aqui se disse, cito as palavras
enfáticas de Holt Parker: 

\begin{quote}
O texto de Safo está em fragmentos [\ldots{}] A linguagem é difícil, a sociedade,
obscura. Voltamo"-nos a manuais e comentários em busca de auxílio. Isso
significa, porém, que chegamos a Safo já cegos pelas assunções largamente não
examinadas de gerações prévias de estudiosos; e no caso de Safo, o acúmulo de
assunções é profundamente milenar e inclui comédias gregas, romances italianos
e pornografia francesa. O caso é pior com Safo do que com qualquer outro autor,
incluindo Homero. Pois aqui não lidamos apenas com a literatura arcaica, mas
com a sexualidade; os comentários são pesadamente carregados de emoção e de
nossos preconceitos. Mais importante, estamos lidando com homossexualidade (ou
melhor, o que construímos como homossexualidade) e sexualidade feminina.\footnote{ Parker (1996, p. 149).}
\end{quote}

As considerações de Parker e Lardinois são de válidas, embora raras, lucidez e
prudência, tanto mais se considerarmos que, segundo Blundell, em
textos antigos, a única “clara alusão a um comportamento homossexual de uma
mulher de Lesbos ocorre num diálogo entre duas prostitutas escrito pelo
satirista Luciano, no século \textsc{ii} d.C.”, o \textit{Diálogo das cortesãs}
(5).\footnote{ Blundell (1995, p. 82).} A
alusão é clara, mas sua historicidade está comprometida pela natureza
literária, mais do que isso, satírica, da fonte em que se encerra --- gêneros como a
comédia e a sátira, nunca é demais frisar, valem"-se com frequência da liberdade
para tratar do sexo para fazer rir e, no caso do segundo, que visa ao riso dos
cúmplices do satirista e à destruição do alvo de seu texto, da linguagem do
vitupério. 

Deve estar evidente, agora, por que evito termos como
\textit{lésbica}, por que deve ser relativizada a imagem da nomenclatura “Safo
de Lesbos”, e por que prefiro falar em homoerotismo, em vez de homossexualismo
que, entre as mulheres da Grécia antiga é muito difícil de ser estudado, dada a
escassez de fontes que induz, muitas vezes, às analogias decerto problemáticas
com o abundante material do atestado homossexualismo masculino --- que,
diga"-se logo, também não funciona entre os antigos como funciona no imaginário
que com olhos modernizantes o contempla. Isso tudo está bastante
bem explorado, documentado e analisado no estudo fundamental de Kenneth J.
Dover, originalmente publicado em 1978.\footnote{ Dover (1994).}

Quanto à referida preferência, reitero: embora haja em Safo forte erotismo
dirigido a outras mulheres --- mas não exclusivamente a elas ---, suas canções não
são registros biográficos ou documentos de sua sexualidade, mas discursos
poéticos em condição fragmentária, que compromete o alcance de nossas leituras,
filiados a determinado gênero poético, o mélico, e inseridos em certa tradição
poética de linguagem erótica. Estamos fadados a fracassar e a nos perder em
especulações, se tentarmos explicar Safo e sua poesia a partir das
características da vida cotidiana das mulheres em Lesbos, ou pela sua
biografia, mesmo porque teciam"-na os antigos extraindo"-a das canções de Safo, e
sem qualquer constrangimento preenchiam as lacunas biográficas com narrativas
que criavam segundo a verossimilhança, que é categoria discursiva. Mas entramos em terreno mais seguro
quando atentamos para o fato de que a poesia grega antiga é de composição
genérica, predominantemente: metro, matéria, linguagem, tema, tudo se articula
em termos das práticas dos gêneros e das expectativas que criam nas audiências.

Tendo isso em mente, e passeando por versos de poetas que ou precederam Safo ou
a sucederam, melhor entenderemos a similaridade tremenda entre as imagens, a
linguagem e o tom de seus fragmentos eróticos e de versos marcados pelo
erotismo em Hesíodo (ativo em \textit{c.} 700 a.C.), Arquíloco (\textit{c.} 680--640
a.C.), Íbico (ativo em \textit{c.} 550 a.C.) e Eurípides (\textit{c.} 482--406 a.C.). Eis
uma pequena amostragem de como, no registro erótico de gêneros, poetas e
tradições distintos, é símil a linguagem para falar da paixão erótica, do
desejo nas traduções possíveis para o grego \textit{éros} que, longe de nomear
o amor romântico, designa a força controlada por Afrodite --- mesmo na forma do
deus Eros, sempre a ela subordinado --- que, segundo seus desígnios, toma de
assalto sua vítima e se apodera de seu corpo e de sua mente, como “uma invasão,
uma doença, uma insanidade, um animal selvagem, um desastre natural”, que vem a
“provocar o colapso, consumir, queimar, devorar, exaurir, entontecer, picar,
aguilhoar, [\ldots{}]”, resume Anne Carson:\footnote{ Carson (1998, p. 148).}

\begin{quote} \parindent=0em
%\pagebreak
{\centering
\textsc{Hesíodo, \textit{Teogonia} (poesia didática)\footnote{ Tradução:
Torrano (2003).}}
\par}

\smallskip

	[\ldots{}] Eros: o mais belo entre os Deuses imortais,\\
	solta-membros, dos Deuses todos e dos homens todos\\
	ele doma no peito o espírito e a prudente vontade.\\
	\mbox{}\hfill (vv. 120--2)

\smallskip

	\hspace*{2em}[\ldots{}] Eurínome de amável beleza virgem de Oceano\\
	\hspace*{4.5em}terceira esposa gerou-lhe Graças de belas faces:\\
	\hspace*{4.5em}Esplendente, Agradábil e Festa amorosa\\
	\hspace*{4.5em}de seus olhos brilhantes esparge-se o amor \textit{[éros]}\\
	\hspace*{4.5em}solta-membros, belo brilha sob os cílios o olhar.\\
	\mbox{}\hfill (vv. 907--11)

\bigskip

{\centering
\textsc{Arquíloco, Fragmentos 193 e 196 W (poesia jâmbica)}
\par}

\smallskip

	\hspace*{4.5em}mísero jazo em desejo,\\
	\hspace*{2em}sem ar, de atrozes dores por querer dos deuses\\
	\hspace*{4.5em}transido até os ossos.\footnote{ Tradução: Souza (1984, p. 92). Edição: West (1998).}

\smallskip

	Mas o desejo solta-membros, ó companheiro, subjuga-me
	\ldots{}\footnote{ Tradução: Corrêa (2010, p.~504). Edição: West (1998)}

\bigskip

{\centering
\textsc{Íbico, Fragmento 287 Dav. (poesia mélica)\footnote{ Tradução: Ragusa (2010, p.
650; ver estudo às pp.~480--507). Edição: Davies (1991).}}
\par}

\smallskip

Eros, de novo, de sob escuras\\
pálpebras, com olhos me fitando derretidamente,\\ %[F020?][F020?][F020?][F020?][F020?] 
com encantos de toda sorte, às inextricáveis\\
redes de Cípris \textup{[Afrodite]} me atira.\\ %[F020?][F020?][F020?][F020?] 
Sim, tremo quando ele ataca, \num{5}\\
tal qual atrelado cavalo vencedor, perto da velhice,\\
contrariado vai para a corrida com carros velozes.

\bigskip

{\centering
\textsc{Eurípides, \textit{Hipólito} (tragédia)}\footnote{ Tradução: Oliveira (2010), em
edição bilíngue que segue o texto grego de Barrett (1992).}
\par}

\smallskip

Eros, Eros que nos olhos \num{525}\\
destilas desejo, incutindo doce\\ 
prazer n’alma dos que atacas,\\
que jamais me apareças com dano\\
nem venhas desmedido.\\
Pois nem dardo de fogo e nem dos astros é forte \num{530}\\
como o de Afrodite, que atira das mãos\\
Eros, filho de Zeus.\\
Em vão, em vão às margens do Alfeu \num{535}\\
e na morada pítica de Febo\\
bovinos sacrifícios prodigaliza a terra heládica,\\
se não veneramos Eros,\\
o soberano dos homens,\\
claviculário da deleitosa alcova de Afrodite \num{540}\\
exterminador que atira os mortais\\
em todos os desastres quando vem.
\end{quote}

Nos fragmentos traduzidos de Safo, o leitor decerto perceberá os ecos intensos
dessas linguagem e imagens sobre a paixão erótica, da qual lançam mão
reiteradamente os poetas gregos, como mostra a seleção citada. Esse movimento é
característico da prática genérica de composição dessa poesia de tradição oral, que lida sobretudo com motivos consolidados e embasados em
percepções mantidas pela repetição, pelo seu retomar por vezes incrementado em
escolhas estilísticas que parecem singulares a um ou outro poeta --- parecem, é
prudente dizer, em vista de nosso magro \textit{corpus} de textos, a partir do
qual afirmar a inovação é um risco.

Confrontados com os versos que reproduzi de Hesíodo, de Arquíloco, de Eurípides,
poetas homens de gêneros poéticos distintos, os fragmentos de Safo que envolvem
a temática da ação da paixão sobre sua vítima e a concepção sobre \textit{éros}
não se distinguem deles a ponto de tornar sustentável o argumento questionável
de uma “literatura feminina”; ao contrário, ecoam versos
dos dois primeiros poetas, e ressoam nos do terceiro, pois trabalham com
suas imagens e, se lhes acrescentam outras, aparentemente singulares à poeta
lésbia, fazem"-no calcados num tratamento da temática referida que não se pode
dizer masculino, nem feminino, mas tradicional. Daí porque é tão complicado
inferir, a partir da leitura das canções em que o “eu” --- nem sempre
identificável para além de uma voz em 1ª pessoa do singular --- acontece de ser
feminino e de se relacionar a figuras femininas eroticamente, que Safo seja
\textit{lésbica}, que represente a “literatura homossexual” --- outro rótulo
demasiado modernizante e anacrônico, em se tratando dos poetas antigos
(homens ou mulheres).

O recorte de um universo que é sobretudo feminino em sua poesia, inclusive
a de caráter intensamente erótico, poderia ser pensado de outro modo. No mundo
de Safo, como na Grécia antiga em geral, estavam cindidos o universo masculino
e o feminino, e Jan N. Bremmer observa que mesmo as relações
heterossexuais no casamento eram distanciadas. Conclui o helenista:
“Dificilmente foi por acaso que o homossexualismo moderno”, que em geral impõe
a rejeição ao heterossexualismo, “se desenvolveu na mesma época em que a
relação heterossexual no casamento adquiria um caráter muito mais íntimo”.\footnote{ Bremmer (1995, p. 24).} Na
Grécia, prossegue Bremmer, evidências literárias e iconográficas --- vasos
atenienses da segunda metade do século \textsc{vi} a.C., “que circulavam nos banquetes
aristocráticos”\footnote{ Bremmer (1995, pp.~20--21).} --- “mostram que as relações homossexuais normalmente
ocorriam só entre adultos e rapazes”,\footnote{ Bremmer (1995, p. 12).} sendo “um aspecto estabelecido do
caminho de um rapaz rumo à idade adulta”.\footnote{ Bremmer (1995, p. 12). Ao menos, na visão
externa dessas relações; da interna, pouco se pode dizer, lembra Bremmer (p.
20).} E frisa Bremmer que tais relações não implicavam a rejeição “do contato
heterossexual”, necessário à procriação e à preservação das linhagens e das
comunidades, mas eram vivenciadas no mundo dos homens, nitidamente distinto do
mundo das mulheres.\footnote{ Bremmer (1995, p. 26) ainda observa que era,
porém, a “relação pederasta que transformava o rapaz em um verdadeiro homem” ---
pois se voltava ao benefício do ensinamento intelectual e social do jovem, e
não apenas ao benefício do prazer sexual ao adulto --- e abria as portas para o
universo da “elite social”.}

Ao falar de \textit{éros} numa linguagem tradicional e a partir de uma concepção
reafirmada na poesia de temática erótica, Safo segue as práticas poéticas de
seu tempo e de lugar histórico, viabilizando assim a circulação de sua poesia
que por todos poderia ser entendida e fruída. É decerto nesse sentido que
devemos tomar uma conhecida frase --- na verdade, um verso hexamétrico --- do poeta
latino Horácio (século \textsc{i} a.C.), numa \textit{Epístola} (1. 19. 28):
``Regula a Musa pelo pé de Arquíloco a máscula Safo”,\footnote{ Agradeço
ao colega latinista Prof.~Dr.~Marcos Martinho dos Santos (\versal{USP}) a discussão sobre esse verso e
sua tradução.} ou seja, a poeta lésbia regulou seus ritmos pelos de Arquíloco,
numa compreensão possível de um verso controverso. A rigor, porém,
estudos métricos mostram que a tradição lésbio"-eólica é mais antiga que a
jônica, de que se vale Arquíloco, e que a dórica.\footnote{ Ver o estudo de West
(1973, pp.~179--92).} Mas talvez possamos, apoiados nos versos dos dois poetas
gregos arcaicos, pensar a ideia do ritmo em termos mais largos: a linguagem e
as imagens que Safo emprega no tratamento da paixão são as mesmas que
encontramos em Arquíloco e em outros poetas homens, antes ou depois dela. E
quanto à conhecida expressão ``máscula Safo”, a primeira das duas
explicações oferecidas por Porfírio (século \textsc{iii} d.C.), comentador de Horácio,
de que a poeta foi excelente na poesia em que homens mais amiúde se
destacam, é, decerto, aquela que pode ser entendida coerentemente e com
respaldo vasto e sólido das práticas poéticas antigas testemunhadas na
composição dos textos sobreviventes. A segunda, de que Safo foi difamada como
dissoluta tríbade --- termo grego de que se vale o comentador, que significa
mulher homossexual ou \textit{lésbica} --- é de pouca valia, inclusive para a
compreensão do verso epistolar horaciano, conduzindo"-nos para longe de sua
poesia e do fazer poético de que resultam suas canções, e para perto das
ficções de Safo e de polêmicas e ansiedades de nosso tempo. 


\section*{A mélica de Safo}

Melhor é, portanto, ir para perto da obra da poeta lésbia, como espero que esta
antologia evidencie aos que pensam conhecer Safo, mas talvez tenham se perdido
nas tramas lendárias em que foi enredada, antes mesmo de chegar às ruínas da
matéria viva que resta da poeta, de um sopro quente e fragrante que os séculos
não abafaram.

No começo desta introdução, observei o fato de que o gênero poético praticado
por Safo é o que os antigos chamavam simplesmente “canção”
(\textit{mélos},\footnote{ Termo mais usado nos períodos arcaico e clássico da
Grécia. O sentido primeiro de \textit{mélos} é “membro” do corpo, daí “membro
musical, frase” e “canção” (palavra, melodia e ritmo), na compreensão
explicitada na \textit{República} (398c), de Platão (séculos \textsc{v}--\textsc{iv} a.C.).
Budelmann (2009, p. 2) lembra que \textit{mélos} “é usado por vários líricos
arcaicos com referência a suas composições”.} \textit{âisma, oidé}) --- daí a
designação \textit{mélica} --- e que se destinava à \textit{performance} ao som
da lira --- daí \textit{lírica} em acepção específica, termo mais tardio, em
circulação da era helenística (323--31 a.C.) em diante. A propósito
desses nomes, Rudolf Pfeiffer afirma: “Em tempos
modernos, toda a poesia não épica e não dramática é usualmente chamada lírica.
Mas os antigos teóricos e editores faziam a distinção entre, de um lado, poemas
elegíacos e jâmbicos, e, de outro, mélicos”.\footnote{ Pfeiffer (1998, pp.~182--3).} Mélica ou lírica, prossegue,
designava o “verso que era cantado para a música e, muito frequentemente, a
dança, e era composto de elementos de ritmos e tamanhos variados”:
\textit{mélos} era, na literatura grega arcaica, o poema lírico; o poeta,
\textit{melopoiós} --- o “fazedor de canções”, literalmente --- ou mélico; o
gênero, mélica ou poesia mélica. Ainda segundo Pfeiffer, tais nomes
“permaneceram usuais em mais tardias pesquisas sobre a teoria poética e a
classificação da poesia”, mas “líricos” era o termo que designava os autores
“em referências a edições de textos e em listas de ‘fazedores’\,”; e do século \textsc{i}
a.C. em diante, lírica sobrepõe"-se a mélica para designar a canção
cantada ao som da lira, o instrumento mais importante de seu acompanhamento, e
os latinos acabam por adotar o primeiro, a despeito do uso ocasional do
segundo.

Como se vê, o modo de \textit{performance} nomeia o gênero;
mas o que mais o caracteriza? Na \textit{performance} ainda, a canção podia ser
entoada em solo e com acompanhamento da lira, ou em coro, com o acréscimo de
outros instrumentos e da dança, a configurarem um espetáculo. Na métrica, a
composição da mélica dá"-se em estrofes mais breves e menos
complexas na modalidade monódica do que na coral. No conteúdo, a canção
monódica apresenta grande variedade de temas --- principalmente os vinculados à
vida cotidiana na \textit{pólis}, a eventos de um passado recente, à
experiência humana, sempre em relação direta com a voz poética ---, formas e
linguagem, enquanto à coral e seus subgêneros são comuns a celebração, a
narrativa mítica e o canto de autorreferência à \textit{performance} em
execução pelo coro. Não são mais estas palavras do que uma síntese para dar uma ideia ao leitor de um cenário mais complexo do que aparenta ser à primeira vista. 

Considerando que as composições da mélica grega arcaica “estavam destinadas,
desde o início, à execução pública ou privada, e constituíam por definição uma
poesia \textit{de} e \textit{para} a voz”, Gustavo Guerrero
afirma decorrer daí que a mélica

\begin{quote}
apareça dominada por um maior traço --- o caráter circunstancial do discurso --- de
literatura oral, que reflete a relação direta do texto com um local e um
momento precisos, um espaço e um tempo ritualizados [\ldots{}]; daí os índices
textuais de um discurso situacional, que se expressam através do uso de certas
figuras pronominais e de marcas do presente, signos que traduzem a interação
geral entre o sujeito da enunciação e seus destinatários.\footnote{ Guerrero (1998, pp.~20--1).}
\end{quote}

Inserida e movida “culturalmente no seio de um sistema de comunicação oral”, a
mélica se concretizava, “portanto, numa prática artística performática”,
conclui Guerrero.\footnote{ Guerrero (1998, p. 21).} Carecemos, porém, de informações que permitam uma
reconstrução clara e precisa de sua \textit{performance}, entre as quais, as
relativas à música, que logo se perdeu. No que se refere à mélica monódica,
lembra Giovan B. D’Alessio que se destinava à apresentação “em
contextos mais próximos aos da comunicação espontânea, face a face”; logo, nela
“espera"-se grau maior de imediatismo”.\footnote{ D’Alessio (2004, p. 270).} 

Recompondo minimamente o cenário, haveria para a canção solo várias audiências e
ocasiões de \textit{performance} possíveis, com destaque para o simpósio,
central também para gêneros como a elegia e o jambo; como anota Massimo Vetta,
no mundo grego de até meados do século \textsc{v} a.C., em que não
estava previsto “um público de leitores”, o simpósio 

\begin{quote}
é o lugar de conservação
e evolução da cultura ‘literária’ relativa a todos os temas que resultam
alternativos ao interesse ecumênico do \textit{epos} e à ambientação
exclusivamente pública do canto religioso oficial e da lírica agonística
[temas estes muito trabalhados na elegia, no jambo e na mélica monódica].\footnote{ Vetta (1995, p. \textsc{xiii}).}
\end{quote}

Entenda"-se por simpósio o que Pauline Schimitt"-Pantel define, em
sentido restrito e etimológico, como o “momento, após a refeição, em que todos
passam a beber”, e, em sentido mais amplo e corrente, “uma prática, de beber
junto, e uma instituição” que “é a expressão do modo de vida aristocrático” dos
homens na \textit{pólis}.\footnote{ Schimitt"-Pantel (1990, p. 15).} O simpósio
é, portanto, coletivo do ponto de vista do
evento, mas restrito do ponto de vista da classe e do gênero;\footnote{ Só
indivíduos do sexo masculino podiam dele tomar parte, reclinando"-se nos sofás ---
prática oriental --- ou servindo aos comensais comida e bebida. A presença
feminina limitava"-se às tocadoras de \textit{aulós}, instrumento de sopro, às
dançarinas --- normalmente cortesãs ---, e às servidoras de bebida. Não por acaso a
sala da casa do aristocrata em que se dava o simpósio era chamada
\textit{andrṓn} (“sala dos homens”).} e era pautado por um
“código rígido e próprio de honra”, ressalta Oswyn Murray, visando
garantir a moderação, caminho para a harmonia essencial à atmosfera
simposiástica.\footnote{ Murray (1990, p. 7).}

No andamento do simpósio, o beber era privilegiado e, por isso, tornou"-se
altamente ritualizado. A \textit{(re)performance} amadora ou profissional da
poesia, pelo poeta ou não, tinha lugar em meio a essa fase, e o simpósio acaba
por exercer um papel fundamental na sua preservação e difusão. Relaxados,
sorvendo o vinho, os gregos ouviam e cantavam e/ou recitavam elegias, trechos
dos poemas homéricos e, claro, exemplares da mélica, competindo uns com os
outros, e demonstrando a habilidade e desenvoltura esperadas de sua formação
aristocrática. Entre os textos escolhidos pelos simposiastas, poderia haver
versos em que o “eu” poético fosse de uma mulher, admitindo"-se, no contexto, a
representação de um papel feminino.\footnote{ Ver Bowie (1986, pp.~16--7).}
Nesse sentido, mais uma vez, não haveria obstáculo às canções de Safo em que a
1ª pessoa do singular é feminina. E não por acaso se diz ser o simpósio uma
ocasião bastante adequada ao caráter mais informal e privado da mélica
monódica; mas Most ressalta:

\begin{quote}
\mbox[\ldots{}] a aparente privacidade da canção monódica não é do individual espontâneo,
introspectivo, mas, antes, do pequeno grupo fora do qual o sujeito grego
arcaico mal pode ser concebido. [\ldots{}] Por sua própria natureza, portanto,
centra"-se nas relações pessoais entre um poeta individual e outro membro de seu
grupo de amigos, ou entre ele e o grupo como um todo, ou ainda entre ele e
indivíduos de fora desse grupo. [\ldots{}] Logo, em geral, a poesia monódica tem
dois modos principais: erótico para com os de dentro do mesmo grupo, de
invectiva, contra os de fora.\footnote{ Most (1982, p. 90).}
\end{quote}

O simpósio é mais voltado ao “mundo do privado”, frisa Schimitt"-Pantel, menos
formal e oficial do que o festival próprio à \textit{performance}
da canção coral; nem por isso, todavia, deixa de ser público e
ritualizado.\footnote{ Schimitt"-Pantel (1990, p. 25).} Na
“tentativa de generalização”, Most diz sobre a \textit{performance} e função da
mélica monódica: essa poesia 

\begin{quote}
era apresentada em ocasiões informais, para
pequenos grupos ligados por laços de amizade e interesse comum, e cumpria a
função social de unir esses grupos em todos coesos e separá"-los ou colocá"-los
em oposição a outros grupos numa mesma cidade.
\end{quote}

A mélica nas modalidades solo e coral está representada no \textit{corpus} sobrevivente
de Safo, que conta ainda com ao menos um fragmento de narrativa de
caráter epicizante (não sabemos se coral ou solo na \textit{performance}).

Feita para a \textit{performance} no festival cívico"-religioso, patrocinado
pelos governos e aristocracias das cidades gregas, no qual se
desenvolviam muitas atividades --- como o \textit{agṓn} (“competição”) poético
para cada gênero poético, para a música e para o corpo (os jogos) \mbox{---,} a canção coral tinha no poeta o compositor das palavras e
da música, e o diretor do coro de cidadãos, que, liderado por um de seus
membros, conduzia a apresentação, cantando e dançando ao som dos instrumentos.
Nessa atmosfera festiva e de celebração pública e coletiva, o canto tinha por
“tônica dominante”, frisa Herington, “o \textit{prazer}, humano e
divino”, pois eram homens e deuses homenageados a um só tempo.\footnote{ Herington (1985, p. 6).} Mas, para além
do festival, também podiam servir de ocasião à
\textit{performance} da mélica coral os grandes funerais e as grandes bodas,
menos públicos e mais privados. São notáveis nessas três ocasiões a ideia da
solenidade, o caráter religioso e a comemoração em chave de elogio; e esses
três elementos permeiam a construção das canções corais, refletindo"-se
em sua linguagem altamente elaborada em registro elevado, em seu forte
componente mítico, no canto autodramático do coro, e nas máximas de tom moral
e ético que pontuam seus versos, a (re)validar e reiterar valores e
ensinamentos compartilhados pela comunidade, pela audiência, pelos
\textit{performers}, pelo poeta cuja voz ganha dimensão mais
pública do que privada, na medida em que sua poesia, diz Most,
“tem papel vital na autoconsciência pública da cidade”.\footnote{ Most (1982, p. 94).}

As composições de Safo, volto a afirmar, são canções, mais
precisamente, fragmentos de algo mais próximo do que chamamos “canção” do que
daquilo que chamamos “poema”, retirada do primeiro termo a ideia moderna da
prevalência da música sobre o texto, ou de uma relação de paridade entre ambos,
diz a advertência de Guerrero,\footnote{ Guerrero (1998, p. 18).} uma vez que, na era arcaica,
completa Eric A. Havelock, 

\begin{quote}
a melodia permaneceu serva
das palavras, e seus ritmos foram moldados para obedecer à pronúncia
quantitativa da fala.\footnote{ Havelock (1996, p. 132).}
\end{quote}

 A poesia grega antiga, cabe recordar, não se
escandia por acento, mas por duração breve ou longa de pronúncia da sílaba, o
que já lhe confere uma sonoridade bem ritmada. Em Safo, até onde permite
afirmar seu \textit{corpus} preservado,
há canções solo e corais, majoritariamente de estrofes breves, de variados temas,
linguagens, tons e modos, \textit{performance} em modalidade solo e coral.
Permanecem, contudo, problemáticas a
audiência e a ocasião específicas da apresentação original das canções
de Safo, presumivelmente pela própria poeta, na Mitilene arcaica, com participação de seu grupo de \textit{parthénoi} --- meninas virgens, condição transicional na rota ao casamento que as insere no mundo do sexo e da sociedade ---, em festivais cívico"-cultuais públicos e em cerimônias de casamento, ocasiões valorizadas no estudo de Franco Ferrari.\footnote{Ferrari (2010) se baseia nas novas reflexões a que levaram os avanços nos estudos de \textit{performance} e a descoberta do novo fragmento da poeta (``Canção sobre a velhice''), em 2004, que nos fez reavaliar a coralidade em Safo e a natureza coral de sua associação de \textit{parthénoi}, e traz um olhar fresco ao aproximá"-la de Álcman, o poeta dos partênios --- canções para coros de virgens --- que atuava em sua época. Ver ainda Ragusa (2019a, pp. 85--111; 2019b, pp. 211--39).}

Se olharmos para a canção poética, seus temas mais recorrentes e a inserção da
1ª pessoa do singular, e para a canção popular --- algumas do universo grego nos
foram transmitidas ---, perceberemos as origens pré"-literárias da mélica. Afinal,
o canto é uma forma de expressão verbal própria do homem, e a canção popular é
“atributo quase universal de sociedades tradicionais”, frisa Bowie.\footnote{
Bowie (1984, p. 3).} E, pensando na mélica grega, Lesky comenta suas formas
pré"-literárias, como os cantos de culto aos deuses, de lamento ou de celebração
nos “momentos culminantes da vida e da morte”, o “canto que acompanha o
trabalho” nos teares, na colheita das uvas.\footnote{ Lesky (1995, pp.~133--4).} Esses cantos ligam"-se ao ritmo da
vida cotidiana e às festividades das comunidades envolvidas na sua
\textit{performance}. Note"-se, enfim, que boa parte da música entoada pelas
personagens épicas da \textit{Ilíada} e da \textit{Odisseia} ilustra as formas
lembradas por Lesky, e, por conseguinte, relaciona"-se principalmente à mélica
arcaica e a seus subgêneros.

Em síntese, a mélica é um gênero de evidentes raízes
encravadas em tradições populares, visíveis em subgêneros
como o epitalâmio ou canção de casamento, que muito pode nos contar, tomada
junto a outras fontes escritas e iconográficas, sobre a cerimônia
da boda. Segundo Cecil M. Bowra, a despeito das variações
impostas por tradições locais, os epitalâmios apontam características gerais:
o banquete inicial na casa do pai da noiva; os sacrifícios aos
deuses do casamento na festa; a noiva, escondida por um véu, sentada junto às
outras virgens, aguardando a apresentação ao noivo, feita por parentes e amigos
dela; a procissão que acompanhava os noivos a seguirem em carruagem rumo à nova
morada; a condução dos noivos ao aposento nupcial, para consumar a boda,
condução esta feita pelos amigos de ambos, com danças, cantos, brincadeiras e
tochas, visando amenizar a tensão dos noivos --- em geral, estranhos um ao outro
--- e propiciar sua união. Aliás, a temática dos epitalâmios varia entre a
solene, quando centrada na procissão da carruagem dos noivos, jocosa, quando na
condução deles ao leito, ou elogiosa, quando na aparência física dos noivos de
modo a estimular a atração entre eles, passo importante à sua necessária união
sexual.\footnote{ Bowra (1961, pp.~214--8).}

Como se percebe, a execução dos epitalâmios tende a ser sobretudo coral, dado
que reforça seus elos com a tradição popular; e no caso dos fragmentos sáficos
dessas canções, anota Lesky, 

\begin{quote}
\mbox[\ldots{}] vemos como a poesia popular
tradicional é captada em toda a sua frescura e espontaneidade, por uma grande
poetisa que, no âmbito de sua arte, a modela em composições que alcançam uma
forma perfeita, sem perderem o encanto daquilo que surgiu do povo.\footnote{ Lesky (1995, p. 168).}
\end{quote}

Considerados os epitalâmios sáficos, bem como o já referido fragmento do que
poderíamos chamar mélica narrativa (“As bodas de Heitor e Andrômaca”), neles
encontramos uma poesia menos “pessoal”, por assim dizer, logo, bem menos
propícia à tão frequente leitura romântico"-biografista do \textit{corpus} de
Safo. O oposto disso ocorre com a maior parte da mélica da poeta. Nela, a temática
erótico"-amorosa --- uma de suas linhas de força ---, a autonomeação da
\textit{persona} como “Safo”, a força da 1ª pessoa do singular, o caráter
aparentemente intimista, as numerosas figuras
femininas contempladas em chave erótica ---, tudo isso estimula o estudo das
canções de Safo em viés flagrantemente romântico"-biografista. Mas tais elementos, incluindo o homoerotismo, mais coerentemente devem ser pensados, como se tem argumentado após a descoberta da ``Canção sobre a velhice'', no âmbito da coralidade e das associações corais femininas, atestadas com evidências consistentes pelo mundo arcaico afora.\footnote{Ver estudos indicados à nota 64 e ainda Lardinois (1994, pp. 57--84; 1996, pp. 150--72). Todos tratam da relação das meninas e de sua líder, Safo, a partir da ideia da coralidade e das associações corais em que se dava a \textit{paideía} feminina, atestadas no mundo arcaico. Tal formação, sentido do termo grego, era conduzida por uma mulher adulta, e tinha no casamento uma de suas orientações, bem como na formação ético"-moral e nas atividades femininas, no cantar e no dançar, nos ritos e em outros âmbitos.} A propósito
disso, afirma Odysseus Tsagarakis, em lúcida advertência à
problemática leitura da 1ª pessoa do singular da canção
monódica:

\begin{quote}
É bastante compreensível, dada a atitude do poeta lírico grego diante da vida,
que alguma experiência humana real subjaza à situação \textit{poética} (ela não
precisa ser específica). A relação da experiência poética com uma experiência
real é de importância secundária no que se refere à apreciação de um poema como
peça de arte. Mas quando se trata de julgar o caráter de um poeta e de
reconstruir sua biografia a partir do poema, a questão se torna séria. Mesmo se
assumirmos, em prol da argumentação, que Safo relembra uma situação pessoal,
não podemos provar que o que está descrito e o modo como está descrito refletem
a experiência real. Uma experiência real [\ldots{}] foi transformada em peça de
poesia que --- este agora é o problema --- tem um propósito.\footnote{ Tsagarakis (1977, p. 71).}
\end{quote}

Adiante, ele completa: “O caráter de Safo e de sua vida privada não pode ser
julgado a partir de sua poesia”.\footnote{ Tsagarakis (1977, pp.~81--2).} Se a identificação é problemática,
a separação radical também o é; na verdade, há que reconhecer certa medida de
proximidade e outra de distância entre o “eu” do poeta e o de seus versos. Qual
medida? Eis o enigma próprio da natureza poética dos textos; arriscar
responder à esfinge, em se tratando dos poetas antigos, cujas figuras nossos
olhos bem mal alcançam, é deixar devorar"-se a frágil matéria que deles nos foi
transmitida pela especulação e pela simplificação.

Vale reiterar que a relação na poesia entre a 1ª pessoa do
singular de um poema e o sujeito empírico do poeta é, desde o século \textsc{xix}, um
dos grandes nós no estudo da elegia, do jambo e da mélica arcaicos,\footnote{
Ver Slings (1990, pp.~1--30). Corrêa (2009, pp.~31--93) discute, com indicação de
bibliografia pertinente, os trabalhos de Fränkel (1975; 1ª edição
1951) e Snell (2001; 1ª edição 1951), da “escola Fränkel"-Snell”, de
caráter romântico"-hegeliano; também Ragusa (2005, pp.~23--44), com o foco em
Safo.} mas que o biografismo é já corrente entre os antigos, que alimentavam a
imagem dos poetas tão admirados quanto desconhecidos a partir de seus poemas,
agindo esses estudiosos como “poetas da ficção biográfica”, sintetiza Diskin
Clay.\footnote{ Clay (1998, p. 10). Não por acaso “a separação entre poeta e
\textit{persona} chegou tarde e com grande dificuldade”, prossegue ele (p. 16),
tendo sido o latino Catulo (século \textsc{i} a.C.) o primeiro poeta a clamar por tal
separação, num protesto poético em chave de virulenta vituperação de tradição
jâmbica, o que, portanto, compromete a argumentação. Cito o Poema 16 (tradução
Oliva, 1996): ``Meu pau no cu, na boca, eu vou meter"-vos,/ Aurélio bicha
e Fúrio chupador,/ Que por meus versos breves, delicados,/ Me julgastes não
ter nenhum pudor./ A um poeta pio convém ser casto/ Ele mesmo, aos seus
versos não há lei./ Estes só têm sabor e graça quando/ São delicados, sem
nenhum pudor,/ E quando incitam o que excite não/ Digo aos meninos, mas esses
peludos/ Que jogo de cintura já não têm/ E vós, que muitos beijos (aos
milhares!)/ Já lestes, me julgais não ser viril?/ Meu pau no cu, na boca, eu
vou meter"-vos''. As questões que se impõem, insolúveis, são: quem fala
aqui? Catulo ou seu “eu” poético? Quão identificados estão um e outro? E quanto
pesa na linguagem o gênero com o qual dialoga o poema, o jambo, de vituperação,
de ataque aos inimigos, de sarcasmo e de sátira?} Como bem resume Simon R.~Slings, 
todavia, as modernas teorias sobre a poesia “lírica”
pressupõem a leitura; mas a “lírica” grega arcaica e clássica destinava"-se à
\textit{performance} perante uma audiência, em dada situação e de certo modo
que se relacionavam ao gênero --- por conseguinte, à linguagem, à matéria e ao
metro --- do poema apresentado de viva voz. Assim, o poeta grego arcaico,
“precisamente porque é o causador de uma experiência estética, é, em certa
medida, despersonalizado”:

\begin{quote}
Se olharmos para o problema desse ângulo, tornar"-se"-á claro, de imediato, que
a oposição Eu ficcional x Eu biográfico é, na verdade, uma simplificação
irresponsável. O Eu é o Eu do \textit{performer}, que se move através de um
\textit{continuum} no qual o Eu biográfico e o Eu ficcional são os dois
extremos: na maior parte do tempo, ele não é nenhum deles.\footnote{ Slings (1990, pp.~11--12).}
\end{quote}

\section*{outras poetas}

No epigrama da \textit{Antologia palatina} --- compilação de quinze livros de
epigramas datados dos séculos \textsc{vii} a.C. ao \textsc{v} d.C. ---, que abre esta introdução,
nove poetas mulheres são nomeadas --- nove são as Musas, e nove é o número comum
às listas de cânones de poetas, como a do epigrama, que proliferam como reflexo
dos trabalhos na Biblioteca de Alexandria, da qual dentro em pouco falarei.
Somadas, elas compõem um modesto conjunto de obra e de testemunhos biográficos
não de todo confiáveis, dado o caráter altamente ficcionalizante da
biografia antiga, que, sendo antes um gênero de discurso do que um relato
factual, está comprometido com o verossímil.

\subsection*{1. Grécia clássica:\break Mirtes, Praxila, Telesila (e
Corina?)\footnoteInSection{ Para todas essas poetas, suas obras e testemunhos sobre
elas, ver a edição bilíngue de Campbell (1992).}}

\smallskip

Mirtes (Beócia, fins do século \textsc{vi} a.C.), da qual nada resta, teria sido mestra
de Píndaro (séculos \textsc{vi}--\textsc{v} a.C.) e de Corina --- neste caso, datada do século \textsc{v}
a.C., não do \textsc{iii} a.C., como parece também ser possível. Todos os três são
mélicos predominantemente corais, e estão relacionados no Fr.~664(a)
P\footnote{ Abreviação para a edição Page (1962).}  de Corina, cuja compreensão
é bastante nebulosa, como a própria imagem de Mirtes:


\begin{quote}
\mbox[\ldots{}] eu censuro também a de clara-voz,\\
Mirtes, porque, sendo mulher,\\
entrou em rivalidade com Píndaro [\ldots{}]
\end{quote}

Praxila, ativa em meados do século \textsc{v} a.C., assim surge num passo do gramático
Ateneu (séculos \textsc{ii}--\textsc{iii} d.C.), no \textit{Banquete dos eruditos} (\textsc{xv}. 694a):
``Praxila de Sícion era também admirada pela composição de cantos
convivais” para acompanhar o beber do vinho, nos simpósios. De sua obra,
porém, temos apenas o fragmento de um “Hino a Adônis” (747 P), outro
de um “Ditirambo: Aquiles” (748 P), três de canções convivais. No fragmento
hínico, o mítico jovem amante de Afrodite, Adônis, arrola as mais belas coisas
que deixou no mundo dos vivos ao morrer, no auge de sua juventude e virilidade:

\begin{quote}
\mbox[\ldots{}] a mais bela coisa que eu deixo é a luz do sol,\\
a segunda, as estrelas brilhantes e da lua sua face,\\
e também pepinos maduros e maçãs e peras [\ldots{}]
\end{quote}

O retórico Zenóbio (século \textsc{ii} d.C., \textit{Provérbios} 4. 21), fonte única
desses versos, cita"-os para explicar o antigo provérbio ``mais ingênuo
que o Adônis de Praxila”, pois “é fraco da cabeça quem igualmente
lista com o sol e com a lua os pepinos e o resto”.

Do fragmento de ditirambo --- gênero mélico de difícil classificação, com forte
componente narrativo ---, há um verso citado por Heféstion (século \textsc{ii} d.C.,
\textit{Sobre os metros}, \textsc{ii}. 3): ``[\ldots{}] mas teu coração no peito nunca
eles persuadiram [\ldots{}]”. Nele, Praxila lembra a intolerância de Aquiles e sua
inabalável rejeição à ajuda aos gregos na luta contra os troianos, em razão da
grave ofensa à honra que lhe havia feito o chefe da expedição, o Atrida
Agamêmnon, ao arrebatar"-lhe Briseida, seu \textit{géras} (“prêmio”) --- parte do
botim de guerra que é a medida da honra de quem o recebe ---, conforme o canto \textsc{i}
da \textit{Ilíada}.

Telesila (Argos, meados do século \textsc{v} a.C.), é associada a um episódio marcial
supostamente biográfico: segundo Plutarco (séculos \textsc{i}--\textsc{ii} d.C., \textit{As
virtudes das mulheres}, 4. 245c"-f) e Pausânias (século \textsc{ii} d.C.,
\textit{Descrição da Grécia}, 2. 20. 8--10), ela teria liderado um conflito
militar contra Cleomenes, de Esparta, o que jogaria sua datação para \textit{c.}
494 a.C. Pausânias descreve, no santuário de Afrodite em Argos, diante da
estátua sentada da deusa, a imagem de Telesila, ``a compositora de
canções”, numa estela:\footnote{ Placa funerária comum nos túmulos gregos, com
inscrições e desenhos sobre os mortos.} ``seus livros caídos aos seus
pés, ela olha para um elmo, segurando"-o com a mão e prestes a pô"-lo sobre sua
cabeça”. Para o viajante, a poesia deu à poeta ainda ``maior honra” do
que outras coisas. Interessa notar, sobre a imagem de guerreira e poeta de
Telesila, que a temática marcial deve ter sido relevante em sua produção, a
confiarmos em Máximo de Tiro (século \textsc{ii} d.C., \textit{Oração} 37. 5), segundo
quem ``os espartanos eram exaltados pelos versos de Tirteu” --- 
poeta elegíaco (meados do século \textsc{vii} a.C.) ---, ``os argivos, pelas
canções de Telesila, e os lésbios, pelas odes de Alceu”. 

A referida imagem de Telesila, dada na fonte tardia, pode ter se nutrido dos
versos da poeta, tanto quanto o episódio biográfico, cuja historicidade não
podemos comprovar. Infelizmente, não se percebe a temática guerreira na obra
remanescente da poeta: cinco fragmentos, o maior, com dois versos (Fr.~717 P),
ao que se percebe, sobre o mito da paixão do deus"-rio Alfeu pela irmã de Apolo,
a virgem caçadora Ártemis.

Por fim, se aceitarmos uma de suas possíveis datações, temos Corina (Beócia) na
era clássica --- mas ela pode ter vivido na helenística, disse
anteriormente.\footnote{ Se a primeira opção proceder, há que supor que “seu
trabalho se perdeu logo depois de ser escrito, e só em fins do século \textsc{iii} a.C.
foi redescoberto”, comenta Campbell (1998, p. 409), pois a análise da
ortografia dos papiros (séculos \textsc{i} e \textsc{ii} d.C.) de Corina “prova que o nosso texto
foi escrito em \textit{c.} 200 a.C.” (p. 408). A questão cronológica permanece
em disputa, mas fato é que o nome de Corina apenas se registra, para nós, em
fontes de 50 a.C. em diante; Lesky (1995, p. 209) ressalta que os “grandes
gramáticos alexandrinos não se ocuparam” dela, que somente na Antiguidade
tardia parece ter desfrutado de certa notoriedade.} De sua obra, temos
fragmentos em que se nota a presença bem marcada de mitos beócios. O mais longo
(654 P) deles, cuja fonte é o bastante precário \textit{Papiro de Berlim}
(século \textsc{ii} d.C.), traz a competição poética entre duas montanhas beócias,
Hélicon (abrigo das Musas) e Citero (no limite com a vizinha região da Ática),
e algo sobre as filhas de Asopo, deus"-rio beócio. Outro fragmento (655 P)
canta, no trecho legível:

\begin{quote}
\mbox[\ldots{}] a musa Terpsícore me (chama?) [\ldots{}]\\
belos contos cantar\\
às de Tanagra, de brancas vestes;\\
e grandemente a cidade se alegrou com minha\\
clara sedutora voz.
\end{quote}

Há, ainda, fragmentos de alguns versos, uma única linha ou palavra, os quais
formam a maioria dos pequenos e corrompidos textos do \textit{corpus} de
Corina, como o 666 P, conservado na citação de Apolônio Díscolo (século \textsc{ii}
d.C., \textit{Pronomes}, 95bc):

\begin{quote}
\mbox[\ldots{}] por tua causa Hermes contra Ares\\
boxeia [\ldots{}]
\end{quote}

Destaco, por fim, o Fr.~664(b), também citado por Apolônio; nele, a voz poética
declara o tema de sua mélica:

\begin{quote}
\mbox[\ldots{}] eu mesma canto dos heróis as excelências\\
e das heroínas [\ldots{}]
\end{quote}

\smallskip

\subsection*{2. Grécia helenística:\break Mero, Erina, Anite, Nóssis}

\smallskip

Esse grupo nos remete à primeira parte da era helenística, cujo centro não é
mais Atenas, mas Alexandria, no Egito ptolomaico, onde o grego foi tornado
língua oficial da administração, do comércio e da educação, e onde proliferaram
escribas, sobretudo na cidade, sede da Biblioteca que era, na
verdade, uma sala entre outras do \textit{Mouseîon} (“a casa das Musas”,
“Museu”) erguido pelo faraó Ptolomeu \textsc{i}, o Sóter, que reinou entre 305--285 a.C.,
e que fora general de Alexandre, o pupilo de Aristóteles. Seu objetivo era um
só: edição e cópia das grandes obras dos antigos, em organização de forte
inspiração aristotélica; e Ptolomeu \textsc{ii}, o Filadelfo, no poder entre 285--246
a.C., ampliou a Biblioteca, de modo a permitir a intensificação desses
trabalhos depois interrompidos por uma catástrofe em 47 a.C., mas retomados
eventualmente e ativos até meados do século \textsc{v} d.C. 

Nesse mundo, “a poesia do passado”, declara Gentili, “passou a
ser lida como literatura pura e simples”, embora fosse ainda recitada --- reflexo
ainda vivo da cultura oral em que se produziu.\footnote{ Gentili (1990a, p. 37).} E em termos da produção poética,
destaca"-se entre os gêneros praticados no período o epigrama, “poema curto em
dísticos elegíacos” --- metro próprio da poesia elegíaca --- e de conteúdos
variados, lembra Jane M. Snyder, entre os quais, “lamentos,
dedicatórias, casos amorosos, animais de estimação e assim por diante”.\footnote{ Snyder (1989, p. 66).}
Originalmente, o epigrama “limitava"-se a servir de epitáfio”, anota a
helenista; e “a palavra grega \textit{epigramma} significa ‘inscrição’\,”. Os
epigramas das poetas de que passo a me ocupar encontram"-se na \textit{Antologia
palatina}.

Mero, de Bizâncio, é a mais desconhecida; dela só há dois epigramas na
\textit{\textsc{ap}} (livro \textsc{vi}, 119 e 189): uma dedicatória às uvas viníferas,
outra às ninfas das águas. Mas em Ateneu (\textsc{xi}. 491b) encontram"-se ainda dez
versos hexamétricos --- metro da épica grega e da poesia didático"-sapiencial,
sobretudo ---, que tratam da constelação das Plêiades. Esse trânsito por entre os
gêneros não é novidade nem na produção poética dos gregos, nem no período, mas
se atesta desde a era arcaica.

Outra poeta, Erina --- não sabemos ao certo sua origem ---, também praticou o
epigrama e a poesia épica. Magro, porém, é seu \textit{corpus}: três epigramas
provavelmente de sua lavra, um fragmento com dois versos hexamétricos, pedaços
d’“O fuso do tear”, longo poema (cerca de 300 versos) em hexâmetros e dialetos
lésbio"-eólico e dórico, preservado num papiro em que se leem, quando muito,
algumas palavras. Um epigrama anônimo (\textit{\textsc{ap}} livro \textsc{ix}, 190)
sobre Erina a dá por lésbia, conta sua morte ainda virgem aos 19 anos, e
afirma que os versos d’“O fuso” são ``iguais” aos de Homero. Por fim,
declara que ``tanto quanto Safo supera Erina em canções, Erina supera
Safo em hexâmetros” (vv. 7--8), metro da epopeia, que teria sido contemplado na
obra sáfica.

Sobre os epigramas, atribuídos a Erina, Snyder observa
serem todos concernentes às mulheres: o 352 do livro \textsc{vi}
retrata uma mulher de nome Agatárquis; o 710 e o 712 do \textsc{vii}, a morte da jovem
Báucis, personagem que reencontramos n’“O fuso do tear”, poema cujo sentido de
todo nos escapa.\footnote{ Snyder (1989, p. 90).}

Já a Anite estão dados cerca de vinte e quatro
epigramas, informa Snyder.\footnote{ Snyder (1989, p. 67).} Segundo o \textit{Onomástico} (5. 48) de
Pólux (século \textsc{ii} d.C.), ela seria de Tegeia, na Arcádia, dados o dialeto dos
versos, as referências a elementos naturais e a imagem do deus Pã, própria da
mitologia local. Cito dois epigramas da poeta, na tradução de José Paulo Paes:

\begin{quote}\parindent=0em 
Para o seu gafanhoto, rouxinol dos campos, e a sua\\
cigarra das árvores, fez Miro um duplo túmulo\\
e o regou com lágrimas de menina: pois o cruel Hades\\
levou-lhe embora os dois bichinhos de estimação.\\
\mbox{}\hfill (\textsc{\textit{ap} vii}, 190)

\smallskip

Vivo, este homem era Manes, um escravo; morto,\\
vale agora o mesmo que o grande Dario.\\
\mbox{}\hfill (\textsc{\textit{ap} vii}, 538)\footnote{ Paes (1995, pp.~34--5).}
\end{quote}

Vê"-se acima a diversidade de conteúdos, própria do gênero; e são os temas
pastorais especialmente relevantes nessa poeta da Arcádia --- região configurada
como “o ideal da paisagem bucólica do pastor”, resume Snyder. Eis o epigrama
313 (\textsc{\textit{ap}}, \textsc{ix}):

\begin{quote}
Senta-te, de todo sob as belas folhas vicejantes do loureiro\\
e tira doce porção d’água de beber da graciosa nascente,\\
para que descansem teus membros cansados da labuta\\
do verão, tocados pelo sopro de Zéfiro.
\end{quote}

Merecem nota, ainda, os curiosos epitáfios para animais --- dos quais o epigrama
190 do livro \textsc{vii}, já citado, é exemplo ---, o que pode ser visto como uma
brincadeira de Anite com a expectativa de quem ouve um epigrama e espera
“grande solenidade”, observa Snyder.\footnote{ Snyder (1989, p. 70).} 

Nossa última poeta, Nóssis, nasceu em Lócris, “colônia grega no sul da Itália”,
ressalta Snyder,\footnote{ Snyder (1989, p. 77).} fundada no século \textsc{vii} a.C. Seu \textit{corpus} se
compõe de doze epigramas, quase todos centrados no
universo feminino e nas deusas Hera e Afrodite. E, como Safo, ela em três deles
se autonomeia, observa a helenista, “criando a vívida \textit{persona} de uma
mulher que celebra as delícias de Eros e que se proclama, ela própria,
seguidora da tradição poética” da poeta de Lesbos. Cito dois epigramas,
novamente em tradução de Paes:

\begin{quote}\parindent=0em
Nada mais doce que o amor; tudo quanto haja de ditoso\\
lhe fica atrás e eu cuspo da boca até o mel.\\
Eis o que diz Nóssis; aquela a quem a Cípria\footnote{ Outro nome de
Afrodite, a deusa do amor erótico, da beleza, da sedução.} não beijou,\\
essa não sabe sequer que flores são as rosas.\\
\mbox{}\hfill (\textsc{\textit{ap} v}, 170)

\smallskip

Se fores, estrangeiro, à Mitilene de formosas danças,\\
a qual fez Safo, a flor das Graças, consumir-se,\\ 
diz que a terra locriana produziu, dileta das Musas,\\
alguém que lhe é igual, de nome Nóssis. Vai!\\
\mbox{}\hfill (\textsc{\textit{ap} vii}, 718)\footnote{ Paes (1995, pp.~36--7).}
\end{quote}

Ambos sugerem que o tema amoroso deve ter preenchido alguns de seus textos. No
primeiro epigrama, os dois versos finais parecem referir"-se exatamente à
paixão, pois alinhavam Afrodite, o beijo e as rosas, flores prediletas da
deusa. No segundo, ao igualar"-se a Safo, cujo tema principal, até onde o
\textit{corpus} de sua poesia mélica e a sua reputação na Antiguidade permitem
afirmar, gira em torno de \textit{éros} (paixão, amor, desejo erótico), Nóssis
declara, indiretamente, que sua poesia se afina à mesma temática: Mitilene
gerou uma poeta de \textit{éros}; Lócris, outra que lhe é ``igual”.

Muitos nomes, escassa substância:\footnote{ Alguns nomes mais nebulosos ainda são
os de Megalóstrata, mencionada por um poeta mélico ativo na Esparta de fins do
século \textsc{vii} a.C., Álcman, no Fr.~59(b) (edição Davies, 1991;
tradução Ragusa, 2010, p. 653): ``[\ldots{}] isto mostrou, das doces
Musas/ o dom, uma das virgens venturosa ---/ ela, a loira Megalóstrata [\ldots{}]'' Há
também o de Cleobulina, filha de Cleóbulo de Lindos, o colecionador de enigmas,
e Carixena, a quem o \textit{Léxico} de Fócio (patriarca de Constantinopla,
século \textsc{ix}) se refere para explicar a expressão ``do tempo de Carixena”:
``Carixena foi uma antiquada tocadora de flauta e compositora de música,
mas alguns a dizem também poeta lírica”. Um provérbio no \textit{Léxico} de
Hesíquio (século \textsc{v} d.C.) buscava já explicar essa expressão depois lembrada em
Fócio: ``Carixena foi famosa por sua estupidez, porque não sabia que era
antiquada. Alguns dizem que ela fazia canções eróticas. Há um provérbio também,
‘o tipo de coisa que é do tempo de Carixena’\,”. Nada resta dessas duas poetas
provavelmente do século \textsc{v} a.C.} é o que se pode espremer do acúmulo dos
séculos que encobrem as obras e as poetas do epigrama de abertura desta
introdução. Distinta e bem mais feliz fortuna teve a obra de Safo --- não sua
figura que, decerto pelo fascínio exercido por sua poesia, ao menos em parte,
tem sido preenchida com múltiplas ficções desde a Antiguidade, as quais a
tornam ainda mais impalpável. 

\section*{A transmissão da mélica de Safo}

\epigraph{Ó Píndaro, boca sacra das Musas, e loquaz Sirena --- %\\
Baquílides! ---, e graças eólias de Safo, %\\
e escrita de Anacreonte, e quem da fonte homérica %\\
extraiu sua própria obra --- Estesícoro! ---, %\\
e doce página de Simônides, e quem de Peitó e dos %\\
meninos colheu a doce flor --- Íbico! ---, %\\
e espada de Alceu, que o sangue de tiranos muitas vezes %\\
derramou, protegendo as leis da pátria, %\\
e rouxinóis de suaves cantos de Álcman --- sede graciosos, vós %\\
que fincastes o início e o fim de toda a lírica. %\\
\mbox{}\hfill (\textsc{\textit{ap}} \textsc{ix}, 184)}

\smallskip

\epigraph{Gritou alto de Tebas Píndaro; soprou deleites %\\
com voz doce"-mel a musa de Simônides; %\\
brilha Estesícoro e também Íbico; era doce Álcman; %\\
deleitáveis sons dos lábios entoou Baquílides; %\\
e Peitó falou junto a Anacreonte; e coisas variegadas canta %\\
Alceu, cisne lésbio na Eólida; %\\
e dentre os homens Safo não é a nona, mas entre as amáveis %\\
Musas a décima Musa registrada. %\\
\mbox{}\hfill (\textsc{\textit{ap}} \textsc{ix}, 571)}%


Não podemos precisar as razões que favoreceram ou não a
preservação das obras dos poetas gregos. Mas somou"-se aos fatores favoráveis,
em princípio, junto à reputação do poeta e outros elementos, sua edição na
Biblioteca de Alexandria. Aliás, ao tratar do termo \textit{lírica}, disse"-o
tardio, porque seu uso nos remete justamente a esse trabalho de cópia e estudo
que, no caso dos poetas mélicos, teve em Aristófanes de Bizâncio
(\textit{c.} 258--180 a.C.) seu principal executor. Segundo Pfeiffer, desse
erudito pode ser a autoria do cânone dos “nove líricos”,
dado nos dois epigramas declamatórios anônimos que citei
acima.\footnote{ Pfeiffer (1998, p. 205). Traduções: Ragusa (2010, pp.~27--8). Peitó (v. 5 de cada
epigrama) é a deusa Persuasão.}

A edição dos mélicos listados, além de tardia, seguiu critérios variados e
arbitrários, para nós nem sempre discerníveis: a compilação de Safo pautou"-se
pelo critério métrico, e foi dividida no eloquente número de nove
livros\footnote{ O nono livro seria de epitalâmios, segundo uma hipótese cuja
aceitação não é consensual; ver Lesky (1995, pp.~168--9). Sobre os critérios
adotados para a edição de Safo, ver ainda Nicosia (1976, pp.~31--2).}  --- rolos de papiros; já seu contemporâneo Alceu pode ter
tido dez, mas o número é incerto e os critérios de
organização, ignorados; as obras de outros, como as de Píndaro e Baquílides,
poetas dos séculos \textsc{vi}--\textsc{v} a.C., organizaram"-se pelo gênero mélico, o
que é muito problemático, pois a definição é usualmente discutível,
em virtude da ausência de sistematização deles antes dos trabalhos na
Biblioteca.\footnote{ Ver os estudos que discutem as definições de gêneros entre os
antigos: Harvey (1955, pp.~157--75), Rossi (1971, pp.~69--94), Calame (1974, pp.
113--28).}

Antes da Biblioteca, a circulação da poesia grega antiga, incluindo a jâmbica,
elegíaca e mélica, foi viabilizada, como vimos, por \textit{performances} e
\textit{reperformances} --- nos mesmos moldes ou não, profissionais ou amadoras ---, pela
simples repetição propiciada pela memória,\footnote{ Ver Herington (1985, pp.
45--8) a respeito.} por inscrições comemorativas em monumentos, por possíveis
edições\footnote{ Harvey (1955, p. 159) afirma que “não há razão para pensar
que as edições alexandrinas foram as primeiras a existir”; e na Atenas clássica
circulavam edições disponíveis dos grandes poetas.} --- termo que deve ser
entendido como cópias de um registro original, em quantidade e difusão muito
restritas e, certamente, custosa.\footnote{ Ver Havelock (1996, p. 26). Tais
cópias eram feitas sobretudo em papiro, material do qual o Egito, sua fonte,
detinha o monopólio, e que, a partir do século \textsc{vi} a.C., adentra o mundo heleno.
O “livro” é, na verdade, um \textit{bíblos} ou \textit{biblíon}, isto é, rolo
de papiro, sendo tardio o formato do \textit{codex}, do século \textsc{ii} d.C. Sobre
os copistas, muitos devem ter sido escravos, e não necessariamente saberiam ler
o que copiavam.} Tudo isso contribui para a sobrevivência dos textos até os
alexandrinos e para o trabalho destes em suas próprias edições e classificações
numa época em que mudaram demais “as condições fundamentais de produção
poética, assim como a relação entre o poeta e sua audiência”, anota Clay.\footnote{ Clay (1998, p. 28).} 

Tendo a obra de Safo se inserido nesse cenário geral --- que não pode ser mais
específico, por falta de conhecimento nosso --- de circulação e
preservação, como nos foi transmitida? Como chegou até nós? Por dois caminhos
trilhados por toda a literatura produzida na Grécia antiga: por fontes de
transmissão direta --- papiros, manuscritos, inscrições em monumentos, e assim
por diante --- e por fontes de transmissão indireta --- citações. Vejamos. 

Desde a década final do século \textsc{xix} a meados do século \textsc{xx}, sobretudo,
intensos trabalhos de escavações conduzidos no Egito trouxeram à luz uma
incrível massa de papiros literários e não literários, provindos,
majoritariamente, da cidade de Oxirrinco que, para Salvatore Nicosia, “tinha
estreito contato com Alexandria”; diz ele ainda que, “em geral, os
textos lá descobertos reportam à atividade filológica e crítica dos grandes
gramáticos alexandrinos”.\footnote{ Nicosia (1976, p. 32).}

Com os acréscimos, que hoje ocorrem em ritmo bem mais lento --- em 2005, foi
publicada uma nova elegia de Arquíloco, o fragmento do Télefo, contido no
\textit{Papiro de Oxirrinco} 4708 (século \textsc{ii} d.C.),\footnote{ Ver tradução de
Corrêa (2009, pp.~337--9).} e em 2004, uma nova canção de Safo, o fragmento da
``Canção sobre a velhice'', preservado no \textit{Papiro de Colônia} 21351 (século \textsc{ii} d.C.) ---,
muitas obras passaram a ser de fato conhecidas, outras ganharam mais
substância, como a da poeta lésbia. Por outro lado, o volume recuperado demanda
o reconhecimento das pesadas perdas sofridas, com as quais devemos conviver. A
condição do \textit{corpus} da poesia jâmbica, elegíaca e mélica, em termos
quantitativos, melhorou --- o estado material dos papiros, porém, trouxe textos
em geral precários; Frederic G. Kenyon frisa:
“É no período lírico, talvez, que as nossas perdas foram maiores; e aqui os
papiros não fizeram muito por nós”.\footnote{ Kenyon (1919, p. 9).} Animado com os acontecimentos então
recentes, a despeito das frustrações, Kenyon afirmava, ao final de seu
artigo: 

\begin{quote}
Verdadeiramente, para todos aqueles que amam a literatura e reconhecem
na literatura grega a mais alta expressão do pensamento humano, os desertos do
Egito floresceram como uma rosa.\footnote{ Kenyon (1919, p. 13).}
\end{quote}


Quase cinquenta anos depois, William H.~Willis oferecia ao
leitor um censo dos papiros literários encontrados no Egito, com cerca de 3000
exemplares publicados.\footnote{ Willis (1968, pp.~205--41).} Eis sua avaliação: 

\begin{quote}
Devemos, é claro, ter em mente as severas limitações de nossa evidência. Quase
todos os nossos papiros vêm de uma única província do mundo greco"-romano; e o
Egito, de muitas maneiras --- na geografia, na tradição e no isolamento político
--- foi uma província atípica. Tampouco podem os nossos textos preservados
derivar uniformemente de todo o Egito. Uma vez que a sobrevivência dos papiros
depende da completa proteção da umidade, as chuvas de Alexandria e a nascente
do Delta, as inundações anuais do Nilo, a irrigação, e o crescimento gradual do
lençol freático ao longo dos séculos --- para não mencionar os inimigos naturais
--- devem, necessariamente, ter"-nos roubado a vasta maioria dos textos antigos.\footnote{ Willis (1968, pp.~205--6).}
\end{quote}

Além dos fatores relativos ao clima, há que se considerar a sorte, as limitações
relacionadas às próprias escavações e o interesse das equipes quanto ao que
gostariam de ver renascer das areias egípcias. Tudo somado, temos uma medida
dos estragos sofridos pelos papiros: a maior parte desapareceu, e os que
sobreviveram estão corrompidos, mutilados, demasiado escurecidos. Mesmo assim,
tê"-los descoberto foi grande fortuna; e grande foi a sorte dos jâmbicos,
elegíacos e mélicos, que contaram com os esforços de Edgar Lobel, helenista
inglês que trabalhou intensamente com os papiros de Oxirrinco, recorda Willis,
em cujo censo os de Safo concentram"-se nos períodos
romano (31 a.C.--476 d.C.) e bizantino (476--1453).\footnote{ Willis (1968, pp.~211--3).}

Quanto às fontes de transmissão indireta, paráfrases e citações em escritos
antigos variados, Nicosia observa que devem ter dependido
sobretudo da memória falível e seletiva de quem cita, da versão do texto por
ele conhecida e/ou disponível em cópia escrita, e das suas necessidades para o
uso dos textos citados, as quais influíram no tamanho destes, em geral
reduzido. Tais textos sofreram ainda, lembra ele, alterações decorrentes da
aticização dos dialetos nos quais os poemas foram compostos --- no caso de Safo,
o lésbio"-eólico, e não o ático que, em parte pelo impulso de uma Atenas
culturalmente muito poderosa na era clássica, prevaleceu sobre os demais
dialetos gregos. Não obstante os problemas, a maioria dos poetas
arcaicos, notadamente, têm nesse tipo de transmissão uma grande aliada.\footnote{ Nicosia (1976, pp.~23--5).}

Diante desse quadro, os textos que contam com mais de uma fonte comumente
apresentam variações, diferenças, que precisam ser resolvidas por escolhas do
editor no trabalho com as obras, ressalta Nicosia.\footnote{ Nicosia (1976, pp.~28).} Insere"-se na lista
de dificuldades do trabalho com a poesia jâmbica, elegíaca, mélica
ainda isto: o problema do estabelecimento dos textos fragmentários, salvo raras
exceções. 

Vamos, pois, às canções de Safo.


\begin{bibliohedra}
\tit{battistini}, Y. (introd., trad., notas). \textit{Poétesses grecques:
Sapphô, Corinne, Anytè \ldots{}}. Paris: Imprimerie Nationale Éditions, 1998.

\tit{bennett}, C. “Concerning ‘Sappho schoolmistress’\,”. \textit{TAPhA} 124,
1994, pp.~345--7.

\tit{blundell}, S. \textit{Women in ancient Greece.} London: British Museum
Press, 1995.

\tit{bowie}, A. M. \textit{The poetic dialect of Sappho and Alcaeus.} Salem:
Ayer, 1984. 

\tit{bowie}, E. L. “Early Greek elegy, symposium and public festival”.
\textit{JHS} 106, 1986, pp.~13--35.

\tit{bowra}, C. M. \textit{Greek lyric poetry.} 2ª
ed. Oxford: Clarendon Press, 1961.

\tit{bremmer}, J. N. “Pederastia grega e homossexualismo moderno”. In:
\line(1,0){25}. (org.). \textit{De Safo a Sade: momentos na história da
sexualidade.} Campinas: Papirus, 1995, pp.~11--26.

\tit{budelmann}, F. “Introducing Greek lyric”. In:
\line(1,0){25}. (ed.). \textit{The Cambridge
Companion to Greek lyric.} Cambridge: Cambridge University Press, 2009, pp.
1--18.

\tit{burnett}, A. P. \textit{Three archaic poets: Archilochus, Alcaeus,
Sappho.} Cambridge: Harvard University Press, 1983. 

\tit{calame}, C. “Refléxions sur les genres littéraires en Grèce archaïque”.
\textit{QUCC} 17, 1974, pp.~113--28.

\tit{campbell}, D. A. (ed. e trad.). \textit{Greek lyric \textsc{i}.} Cambridge:
Harvard University Press, 1994. [1ª ed.: 1982].

\titidem. (ed. e trad.). \textit{Greek lyric \textsc{iv}.} Cambridge: Harvard
University Press, 1992.

\titidem. (coment.). \textit{Greek lyric poetry.} London: Bristol
Classical Press, 1998. [1ª ed.: 1967]. 

\tit{cantarella}, E. \textit{Pandora’s daughters. The rôle and status of
women in Greek and Roman antiquity.} Trad. M. B. Fant. New York: Penguin, 1991.

\tit{carey}, C. “Genre, occasion and performance”. In:
\textsc{budelmann}, F. (ed.). \textit{The Cambridge Companion to Greek
lyric.} Cambridge: Cambridge University Press, 2009, pp.~21--38.

\tit{carson}, A. \textit{Eros, the bittersweet: an essay.} Chicago:
Dalkey Archive Press, 1998.

\tit{clay}, D. “The theory of the literary \textit{persona} in Antiquity”.
\textit{MD} 40, 1998, pp.~9--40.

\tit{cole}, S. G. “Could Greek women read and write?”. In: \textsc{foley}, H.
P. (ed.). \textit{Reflections of women in antiquity.} Philadelphia: Gordon and
Breach, 1992, pp.~219--45.

\textsc{corrêa}, P. da C. \textit{Armas e varões: a guerra na
lírica de Arquíloco.} 2ª ed. revista e ampliada. São Paulo: Ed. da Unesp, 2009.

\titidem. \textit{Um bestiário arcaico: fábulas e imagens
de animais na poesia de Arquíloco}. Campinas: Editora da Unicamp, 2010. (Apoio:
Fapesp)

\tit{d’alessio}, G. B. “Past, future and present past: temporal
\textit{deixis} in Greek archaic lyric”. \textit{Arethusa} 37, 2004, pp.
267--94.

\tit{davies}, M. (ed.). \textit{Poetarum melicorum Graecorum fragmenta \textsc{i}.}
Oxford: Clarendon Press, 1991.

\tit{de martino}, F. “Appunti sulla scrittura al femminile nel mondo antico''.
In: \line(1,0){25}. (ed.). \textit{Rose} \textit{de Pieria.} Bari: Levante Editori,
1991, pp.~17--75.

\tit{dover}, K. J.  \textit{A homossexualidade na Grécia antiga.}
Trad. L. S. Krausz. São Paulo: Nova Alexandria, 1994. [1ª ed. orig.: 1978].

\tit{easterling}, P. E.; \textsc{knox}, B.W. (ed.). \textit{The Cambridge History of
classical literature --- \textsc{i}: Greek literature.} Cambridge: Cambridge University
Press, 1990.

\tit{ferrari}, F. \textit{Sappho’s gift: the poet and her community.} Trad.
B. Acosta"-Hughes e L. Prauscello. Ann Arbor: Michigan University Press, 2010.

\tit{foley}, H. P. (ed.). \textit{Reflections of women in antiquity.}
Philadelphia: Gordon and Breach, 1992.

\tit{fränkel}, H. \textit{Early Greek poetry and philosophy.} Trad. M. Hadas
e J. Willis. Oxford: Basil Blackwell, 1975. [1ª ed. orig.: 1951]. 

\tit{gentili}, B. \textit{Poetry and its public in ancient Greece.} Trad. A.
T. Cole. Baltimore: The Johns Hopkins University Press, 1990a.
[1ª ed. orig.: 1985]

\titidem. “Lo ‘io’ nella poesia lirica greca”. \textit{AION (filol)} 12,
1990b, pp.~9--24.

\tit{guerrero}, G. \textit{Teorías de la lírica.} México: Fondo de
Cultura Económica, 1998.

\tit{johnson}, W. R. \textit{The idea of lyric. Lyric modes in ancient and
modern poetry.} Berkeley: University of California Press, 1982.

\tit{hallett}, J. P. “Sappho and her social context”. In: \textsc{greene}, E.
(ed.). \textit{Reading Sappho: contemporary approaches.} Berkeley: University
of California Press, 1996, pp.~125--42.

\tit{harvey}, A. E. “The classification of Greek lyric poetry”. \textit{CQ}
5, 1955, pp.~157--75.

\tit{havelock}, E. \textit{A revolução da escrita na Grécia e suas
consequências culturais.} Trad. O. J. Serra. São Paulo, Rio de Janeiro: Editora
da Unesp, Paz e Terra, 1996.

\tit{henderson}, W. J. ``Received responses: ancient testimony on Greek lyric imagery.'' 
\textit{AClass} 41, 1998, pp.~5--27. 

\tit{herington}, J. \textit{Poetry into drama.} Berkeley: University of
California Press, 1985.

\tit{kenyon}, F. G. “Greek papyri and classical literature”.
\textit{JHS} 39, 1919, pp.~1--15.

\tit{lardinois}, A. “Safo lésbica e Safo de Lesbos”. In: \textsc{bremmer}, J.
(org.). \textit{De Safo a Sade: momentos na história da sexualidade.} Campinas:
Papirus, 1995, pp.~27--50.

\tit{lesky}, A. \textit{História da literatura grega.} Trad. M. Losa. Lisboa:
Fundação Calouste Gulbenkian, 1995. [1ª ed. orig.: 1957].

\tit{lourenço}, F. (trad.). \textit{Poesia grega de Álcman a Teócrito.}
Lisboa: Livros Cotovia, 2006.

\tit{mossé}, C. \textit{La femme dans la Grèce antique.} Paris: Éditions
Complexe, 1991.

\tit{most}, G. W. “Greek lyric poets”. In: \textsc{luce}, T. J. (ed.). \textit{Ancient writers -- \textsc{i}: Greece and Rome}. New York: Charles Scribner's Sons, 1982, pp. 75--98. 

\titidem. ``Reflecting Sappho.'' In: \textsc{greene}, E.~(ed.) 
\textit{Re"-reading Sappho: reception and transmission}. Berkeley: University of California Press,
1996, pp.~11--35.

\tit{murray}, O. “Sympotic history”. In: \line(1,0){25}. (ed.).
\textit{Sympotica. A symposium on the symposion.} Oxford: Clarendon Press,
1990, pp.~3--13.

\titidem. \textit{Early Greece.} 2ª ed.
Cambridge: Harvard University Press, 1993.

\tit{nicosia}, S. \textit{Tradizione testuale diretta e indiretta dei poeti
di Lesbo.} Roma: Ateneo, 1976. 

\tit{oliva neto}, J. A. (trad., introd. e notas). \textit{Catulo. O Livro de
Catulo.} São Paulo: Edusp, 1996.

\tit{oliveira}, F. R. (introd., trad. e notas). \textit{Hipólito. Eurípides.}
São Paulo: Odysseus, 2010.

\tit{paes}, J. P. (trad., notas, posfácio). \textit{Poemas da Antologia grega
ou palatina, séculos \textsc{vii} a.C. a \textsc{v} d.C.} São Paulo: Companhia das Letras, 1995.

\tit{page}, D. L. (ed.). \textit{Poetae melici Graeci.} Oxford: Clarendon
Press, 1962.

\titidem. \textit{Sappho and Alcaeus.} Oxford: Clarendon
Press, 2001. [1ª ed.: 1955].

\tit{parker}, H. “Sappho schoolmistress”. In: \textsc{greene}, E. (ed.).
\textit{Re"-reading Sappho: reception and transmission.} Berkeley: University of
California Press, 1996, pp.~146--83.

\tit{pfeiffer}, R. \textit{A history of classical scholarship --- \textsc{i}.} Oxford:
Clarendon Press, 1998. [1ª ed.: 1968]

\tit{ragusa}, G. \textit{Fragmentos de uma deusa: a representação de Afrodite
na lírica de Safo}. Campinas: Editora da Unicamp, 2005. (Apoio: Fapesp).

\titidem. \textit{Lira, mito e erotismo: Afrodite na poesia mélica grega
arcaica}. Campinas: Editora da Unicamp, 2010. (Apoio: Fapesp).

\tit{robb}, K. \textit{Literacy and paideia in ancient Greece.} New York,
Oxford: Oxford University Press, 1994.

\tit{rösler}, W. “Persona reale o persona poetica?”. \textit{QUCC} 19, 1985,
pp.~131--44.

\tit{rossi}, L. E. “I generi letterari e le loro leggi scritte e non scritte
nelle letterature classiche”. \textit{BICS} 18, 1971, pp.~69--94.

\tit{schimitt"-pantel}, P. “Sacrificial meal and symposion: two models of
civic institutions in the archaic city?” In: \textsc{murray}, O. (ed.).
\textit{Sympotica. A symposium on the symposion.} Oxford: Clarendon Press,
1990, pp.~14--33.

\tit{shapiro}, H. A. “Introduction”. In: \line(1,0){25}.
(ed.). \textit{The Cambridge Companion to archaic Greece.} Cambridge: Cambridge
University Press, 2007, pp.~1--9.

\tit{skinner}, M. B. “Woman and language in archaic Greece, or, Why is Sappho
a woman?”. In: \textsc{greene}, E. (ed.). \textit{Reading Sappho: contemporary
approaches.} Berkeley: University of California Press, 1996, pp.~175--92.

\tit{slings}, S. R. “The \textit{I} in personal archaic lyric: an
introduction”. In: \line(1,0){25}. (ed.). \textit{The poet’s I in archaic Greek
lyric.} Amsterdam: VU University Press, 1990, pp.~1--30.

\tit{snell}, B (ed.). \textit{A cultura grega e as origens do pensamento
europeu.} Trad. P. de Carvalho. São Paulo: Perspectiva, 2001. [1ª ed. orig.:
1955].

\tit{snyder}, J. M. \textit{The woman and the lyre: women writers in
classical Greece and Rome}. Carbondale: Southern Illinois University Press,
1989.

\tit{stehle}, E. “Romantic sensuality, poetic sense”. In: \textsc{greene}, E.
(ed.). \textit{Reading Sappho: contemporary approaches.} Berkeley: University
of California Press, 1996, pp.~143--9.

\tit{svenbro}, J. \textit{Phrasiklea. An anthropology of reading in ancient
Greece.} Trad. J. Lloyd. Ithaca: Cornell University Press, 1993.

\tit{torrano}, J. (estudo e trad.). \textit{Hesíodo. Teogonia --- A origem dos
deuses.} 5ª ed. São Paulo: Iluminuras, 2003.

\tit{tsagarakis}, O. \textit{Self"-expression in early Greek lyric, elegiac
and iambic poetry.} Wiesbaden: Franz Steiner, 1977.

\tit{vetta}, M. “Poesia simposiale nella Grecia arcaica e classica”. In:
\line(1,0){25}. (ed.). \textit{Poesia e simposio nella Grecia arcaica.}
Bari: Laterza, 1995, pp.~xi--lx.

\tit{west}, M. L. “Corinna”. \textit{CQ} 20, 1970, pp.~277--87.

\titidem.“Greek poetry 2000--700 \textsc{b.c.”.} \textit{CQ} 23, 1973,
pp.~179--92.

\titidem. (ed.). \textit{Iambi et elegi Graeci.} Oxford: Oxford University
Press, 1998. vols. 1--2. [1ª ed.: 1971].

\tit{willis}, W. H. “A census of the literary papyri from Egypt”.
\textit{GRBS} 9, 1968, pp.~205--41.



\section*{adendo bibliográfico à 2ª edição}


\tit{BOWMAN}, L. “The ‘women’s tradition’ in Greek poetry”. \textit{Phoenix} 58, 2004, pp. 1--27.

\tit{BRASETE}, M. F. “O amor na poesia de Safo”. In: \textsc{ferreira}, A. M. (ed.). \textit{Percursos de Eros -- representação do erotismo}. Aveiro: Universidade de Aveiro, 2003, pp. 17--26.

\titidem. “Homoerotismo feminino na lírica grega arcaica: a poesia de Safo”. In: \textsc{fialho}, M. do Céu et alii (eds.). \textit{A sexualidade no mundo antigo}. Lisboa, Coimbra: Centro de História da Universidade de Lisboa/Centro de Estudos Clássicos e Humanísticos U. Coimbra, 2009, pp. 289--303. 

\tit{BRUSSE}, J. S. “Epigram”. In: \textsc{clauss}, J. J.; \textsc{cuypers}, M. (eds.). \textit{A companion to Hellenistic literature}. Malden: Wiley"-Blackwell, 2010, pp. 117--35. 

\tit{BUZZI}, S. et alii (eds.). \textit{Nuove acquisizioni di Saffo e della lirica greca}. Alessandria: Edizioni dell'Orso, 2008.

\tit{CAZZATO}, V.; \textsc{lardinois}, A. (eds.) \textit{The look of lyric: Greek song and the visual. Studies in archaic and classical Greek song, vol. 1}. Leiden: Brill, 2016.

\tit{GREENE}, E.; \textsc{skinner}, M. B. (eds.). \textit{The new Sappho on old age. Textual and philosophical issues}. Washington, \textsc{d.c.}: Center for Hellenic Studies, 2009.

\tit{KLINCK}, A. \textit{Woman’s song in ancient Greece}. Montreal: McGill-Queen’s University Press, 2008. 

\tit{LARDINOIS}, A. “Subject and circumstance in Sappho’s poetry”. \textit{TAPhA} 124, 1994, pp. 57--84.

\titidem. “Who sang Sappho’s songs?”. In: \textsc{greene}, E. (ed.). \textit{Reading Sappho}. Berkeley: University of California Press, 1996, pp. 150--72.

\titidem. ``Lesbian Sappho reviseted''. In: \textsc{dijkstra}, J. et alii (eds.). \textit{Myths, Martyrs, and Modernity. Studies in the History of Religions in Honour of Jan N. Bremmer}. Leiden: Brill, 2010, pp. 13--30.

\tit{RAGUSA}, G. (org., trad.). \textit{Lira grega: antologia de poesia arcaica}. São Paulo: Hedra, 2013.

\titidem. “Memória, a terra prometida dos poetas: o tema na mélica grega arcaica”. \textit{Forma Breve} 15, 2018, pp. 143--52. 

\titidem. “A coralidade e o mundo das \textit{parthénoi} na poesia mélica de Safo”. \textit{Revista Aletria} 29.4, 2019a, pp. 85--111. 

\titidem. “Safo de Lesbos: de liras e neblinas”. In: \textsc{rede}, M. (org.). \textit{Vidas Antigas. Ensaios Biográficos da Antiguidade}. São Paulo: Editora Intermeios, 2019b, pp. 211--39.

\tit{REEDER}, E. D. (ed.). \textit{Pandora’s box. Women in classical Greece}. Baltimore, Princeton: The Walters Art Gallery/University Press, 1995.

\end{bibliohedra}


\part{\textsc{hino a afrodite e outros poemas}}
\chapter*{}
\thispagestyle{empty}

\vspace*{\fill}
\begin{verse}
\small{Ó Safo, aos jovens que amam o mais doce travesseiro das paixões,\\
a ti, junto às Musas, a Piéria adorna, ou o\\
Hélicon coberto de hera -- a ti que sopras tal qual\\
elas a ti, Musa na Ereso eólia.\\
Ou Hímen Himeneu, portando sua tocha brilhante,\\
contigo fica sobre o tálamo nupcial;\\
ou junto a Afrodite enlutada, lamentando o jovem rebento de\\
Ciniras, contemplas o bosque sacro dos venturosos.\\
Em toda parte, ó soberana, te saúdo como aos deuses, pois tuas\\
canções ainda hoje consideramos filhas dos imortais.}
\end{verse}

\begin{flushright}
\small{\textsc{dioscúrides, séc.\,iii a.c.}}\\
\small{\textit{Antologia palatina}, \textsc{vii}, ep.\,407}
\end{flushright}

\setbeforesecskip{1.2\onelineskip}%
  \setaftersecskip{.35\onelineskip}%
  \setsecheadstyle{\large\baselineskip=.7\baselineskip\bfseries\scshape\MakeTextLowercase}%

\oneside
\pagestyle{plain}

\chapter{Afrodite}

\paragraph{\textsc{nota introdutória}} 
Nenhuma outra divindade grega aparece nas canções de Safo com a mesma
frequência, nem do mesmo modo: Afrodite é a mais presente\footnote{ Afrodite é
personagem dos fragmentos 1, 2, 5, 15, 22, 33, 73\textsc{a}, 86, 96, 102, 112, 133, 134
e 140. Neles, o tema da presença da deusa \textit{corpus} foi estudado em Ragusa (2005), em que se baseiam as traduções dos e as notas aos fragmentos legíveis desse \textit{corpus}. Tais traduções, como outras publicadas previamente a esta nova edição da antologia, podem estar aqui ligeiramente alteradas.}. O fato
decerto reflete três das linhas de força da mélica sáfica: a
paixão erótica, a beleza e o universo feminino. Ora, Afrodite, em Safo e nos
demais poetas gregos, para não falar da iconografia e dos cultos, é
multifacetada -- como são em geral os deuses gregos --, mas é, fundamentalmente,
deidade da beleza física, da feminilidade, da sensualidade, da sedução, da
paixão erótica, do desejo -- características que constituem seus poderes
principais e sua esfera central de atuação, a do erotismo. Há, portanto,
estreita afinidade entre o fazer poético de Safo e a imagem de Afrodite, que se
traduz em notável e inigualável cumplicidade entre a deusa dileta e a voz
poética dos versos.\footnote{ A diferença é evidente se comparamos a
representação da deusa em Safo à encontrada nos demais poetas arcaicos que, como ela, praticaram a poesia mélica; para estes e o estudo de Afrodite em seus fragmentos, ver Ragusa (2010).}

\pagebreak
\section{«\,Hino a Afrodite\,» (Fr.\,1)} 

\begin{gkverse}
Ποικιλόθρον’ ἀθανάτ’ Ἀφρόδιτα,\\
παῖ Δίος δολόπλοκε, λίσσομαί σε,\\
μή μ’ ἄσαισι μηδ’ ὀνίαισι δάμνα,\\
	πότνια, θῦμον,

ἀλλὰ τυίδ’ ἔλθ’, αἴ ποτα κἀτέρωτα\\
τὰς ἔμας αὔδας ἀίοισα πήλοι\\
ἔκλυες, πάτρος δὲ δόμον λίποισα\\
	χρύσιον ἦλθες

ἄρμ’ ὐπασδεύξαισα· κάλοι δέ σ’ ἆγον\\
ὤκεες στροῦθοι περὶ γᾶς μελαίνας\\
πύκνα δίννεντες πτέρ’ ἀπ’ ὠράνω αἴθε-\\
ρος διὰ μέσσω·

αἶψα δ’ ἐξίκοντο· σὺ δ’, ὦ μάκαιρα,\\
μειδιαίσαισ’ ἀθανάτωι προσώπωι\\
ἤρε’ ὄττι δηὖτε πέπονθα κὤττι\\
δηὖτε κάλημμι

κὤττι μοι μάλιστα θέλω γένεσθαι\\
μαινόλαι θύμωι· τίνα δηὖτε πείθω\\
\ldots{} σ̣άγην ἐς σὰν φιλότατα; τίς σ’, ὦ\\
Ψάπφ’, ἀδικήει;

καὶ γὰρ αἰ φεύγει, ταχέως διώξει,\\
αἰ δὲ δῶρα μὴ δέκετ’, ἀλλὰ δώσει,\\
αἰ δὲ μὴ φίλει, ταχέως φιλήσει\\
κωὐκ ἐθέλοισα.

ἔλθε μοι καὶ νῦν, χαλέπαν δὲ λῦσον\\
ἐκ μερίμναν, ὄσσα δέ μοι τέλεσσαι\\
θῦμος ἰμέρρει, τέλεσον, σὺ δ’ αὔτα\\
σύμμαχος ἔσσο. 
\end{gkverse}

\pagebreak
\section*{}

\begin{verse}
De flóreo manto furta-cor, ó imortal Afrodite,\\
filha de Zeus, tecelã de ardis, suplico-te:\\
não me domes com angústias e náuseas,\\*
veneranda, o coração,

mas para cá vem, se já outrora --\\
a minha voz ouvindo de longe -- me\\
atendeste, e de teu pai deixando a casa\\*
áurea a carruagem

atrelando vieste. E belos te conduziram\\
velozes pardais em torno da terra negra --\\
rápidas asas turbilhonando, céu abaixo e\\*
pelo meio do éter.

De pronto chegaram. E tu, ó venturosa,\\
sorrindo em tua imortal face,\\
indagaste por que de novo sofro e por que\\*
de novo te invoco,

e o que mais quero que me aconteça em meu\\
desvairado coração. ``Quem de novo devo persuadir\\
\ldots{} ao teu amor? Quem, ó\\*
Safo, te maltrata?

Pois se ela foge, logo perseguirá;\\
e se presentes não aceita, em troca os dará;\\
e se não ama, logo amará,\\*
mesmo que não queira''.

Vem até mim também agora, e liberta-me dos\\
duros pesares, e tudo o que cumprir meu\\
coração deseja, cumpre; e, tu mesma,\\*
sê minha aliada de lutas.
\end{verse}

\pagebreak
\textls[-24]{\paragraph{Comentário} Esse \textit{hino clético} -- prece que invoca a deidade para instá"-la a vir à \EP[1]
presença de quem suplica -- estrutura"-se em três etapas
fundamentais, mostra a tradição: identificação do deus,\footnote{Versos 1--2.} essencial
num sistema politeísta; recordação de relação previamente firmada com a
deidade, de modo a nela suscitar o sentido de obrigação para com quem apela;\footnote{Versos 5--24.} explicitação do(s) pedido(s). Essa cuidadosa elaboração
formal explica"-se por constituir a própria prece em presente à divindade que, com
tal agrado, pode se tornar propícia. No hino, a suplicante, que se autonomeia
``Safo'',\footnote{Trata"-se de procedimento muito usado em variados gêneros poéticos, por meio dos quais a \textit{persona} dramatizada do próprio poeta torna"-se parte dos versos que podem imortalizá"-lo. A ideia da poesia como instrumento de imortalização de heróis está no cerne da épica, mas logo vemos a ideia associada de que a poesia confere memória e fama não apenas àquilo que canta, mas àquele que a canta. Safo vale"-se de autonomeação em outros fragmentos, bem como Álcman, poeta mélico ativo em c.\,620 a.C., Teógnis, poeta elegíaco de fins de 600 a.C., e mesmo antes, Hesíodo, ativo em fins de 700 a.C., na sua poesia didática"-comosmogônica e didático"-sapiencial, para mencionar apenas esses mais antigos poetas. A prática se perpetua, claro, no correr dos séculos, em várias tradições poéticas, nas quais vai sendo ressignificada. Discuti recentemente o tema da memória e imortalização do poeta pela poesia na Grécia arcaica (Ragusa, 2018, pp. 143--152).} invoca a deusa
a vir à sua presença, para junto a ela lutar pela sedução da amada que ora a
rejeita, objetivo em torno do qual giram todos os pedidos.\footnote{Versos 1--5, 25--8.}
Atente"-se para a fala de Afrodite,\footnote{Versos 21--4.} dita no passado, mas
revalidada no presente e a cada nova rodada da vinda da intermitente paixão
erótica; tal fala guarda um valor universal de caráter punitivo"-consolatório: o
consolo do amador rejeitado pelo amado está na reversão de papéis que a
experiência erótica em tempo produz. Também vale a pena reparar na visão de
\textit{éros} como patologia de corpo e mente,\footnote{Versos 3--4.} que, em princípio,
torna sua vítima impotente; e no modo como a sedução é pensada como uma batalha
e uma caçada, arenas às quais são comuns o ataque violento, a perseguição e a
dominação do outro -- o inimigo, a presa, o objeto de desejo do sedutor. Tudo
isso está muito presente na linguagem erótica da poesia grega antiga, mas se
maximiza no Fr.\,1 de Safo, em que a suplicante chama Afrodite a ser sua
``aliada de lutas'', \textit{sýmmakhos},\footnote{Verso 28.} em estreita parceria. Eis o último pedido da prece, que, ao contrário dos demais que se retomam, só surge nas penúltimas palavras. Tudo no hino à deusa é, pois, arquitetado para culminar neste pedido que, se concedido, como será (já o foi no passado!), garantirá o sucesso da empreitada de sedução à qual quer se lançar a suplicante"-amada. Pedido que é proferido estrategicamente, depois que ``gradualmente'' construiu ``uma relação segura e protetora com a poderosa e caprichosa deusa''.\footnote{Thomas, 1999, p.\,9} Finalmente, nos versos 1--2, veja"-se
que os epítetos estabelecem Afrodite não só como poderosa, mas também bela
sedutora ardilosa. A sedução e a deidade caminham, pois, no âmbito do fugidio,
do oblíquo, da dissimulação, da trama. A arte do engano é imprescindível na
esfera da sedução regida por Afrodite; e nessa arte, ninguém superará a deusa,
preciosa aliada.

Não posso deixar de ressaltar que o ``Hino a Afrodite'' é não só
a mais famosa canção de Safo, mas a única integralmente preservada em citação, feita no tratado \textit{Sobre o arranjo das palavras},\footnote{Número 23.} de Dionísio de Halicarnasso (retórico, século \textsc{i} a.C.). Mais: é a primeira canção do livro \textsc{i} de Safo,
compilado na célebre Biblioteca de Alexandria, provavelmente na virada dos
séculos \textsc{iii}--\textsc{ii} a.C.
Todo plasmado como é em Afrodite, a deusa \textit{sýmmakhos} de ``Safo'' que lhe pede e antecipa sua presença, que pode se pensar o epíteto e o hino como um todo em dimensão metapoética, à semelhança do que se passa com o Fr.\,2, adiante. Nessa dimensão, ``Safo'' dramatiza"-se como ``uma mulher líder de um grupo feminino e uma poeta totalmente dedicada à esfera da beleza, do amor, e de Afrodite'',\footnote{Bierl 2018, p. 929.} e o faz nas derradeiras palavras, ``drasticamente revertendo seu sentimento inicial de devastação e depressão'', para se afirmar qual ``autoconfiante \textit{choregos}, ‘líder do coro’, e cantora poética''.\footnote{Bierl, p. 930.} ``Reencenando essa divina aliada de luta na própria \textit{performance}, a canção faz Safo se fundir a Afrodite. Assim, em qualquer \textit{performance}, Safo se torna Afrodite, como a cantora perfeita, plena de encantamento poético e erótico''.\footnote{Bierl, p. 948.}}

\pagebreak
\section{«\,Ode do óstraco\,» (Fr.\,2)}

\begin{gkverse}
\dagger{}δευρυμμ̣εκρητε̣σιπ̣[.]ρ[̣ \quad].\dagger{} ναῦον\\
ἄγνον ὄππ̣[αι\quad] χάριεν μὲν ἄλσος\\
μαλί[αν], βῶμοι δ’ ἔ<ν>ι θυμιάμε-\\
νοι [λι]β̣ανώτω<ι>·

ἐν δ’ ὔδωρ ψῦχρον κελάδει δι’ ὔσδων\\
μαλίνων, βρόδοισι δὲ παῖς ὀ χῶρος\\
ἐσκίαστ’, αἰθυσσομένων δὲ φύλλων\\
κῶμα \dagger{}καταιριον·

ἐν δὲ λείμων ἰπ̣π̣όβοτος τέθαλε\\
\dagger{}τω̣τ...(.)ριννοις\dagger{} ἄνθεσιν, αἰ <δ’> ἄηται\\
μέλλιχα πν[έο]ισιν [\\
\quad[\qquad\qquad]

ἔνθα δὴ σὺ \dagger{}συ.αν\dagger{} ἔλοισα Κύπρι\\
χρυσίαισιν ἐν κυλίκεσσιν ἄβρως\\
<ὀ>μ<με>μείχμενον θαλίαισι νέκταρ\\
οἰνοχόεισα
\end{gkverse}

\begin{verse}
\ldots{} Para cá, até mim, de Creta \ldots{} templo\\
sacro onde \ldots{} e agradável bosque\\
de macieiras, e altares nele são esfumeados\\*
com incenso.

E nele água fria murmura por entre ramos\\
de macieiras, e pelas rosas todo o lugar\\
está sombreado, e das trêmulas folhas\\*
torpor divino desce.

E nele o prado pasto de cavalos viceja\\
\ldots{} com flores, e os ventos\\*
docemente sopram \ldots{}

Aqui tu [\ldots{}] pegando, ó Cípris,\\ \EP[1]
nos áureos cálices, delicadamente,\\
néctar, misturado às festividades,\\
vinho-vertendo \ldots{}
\end{verse}


\pagebreak
{\paragraph{Comentário} Eis outro \textit{hino clético}, em que se destaca o detalhamento do local ao
qual Afrodite é convidada a vir, saindo de Creta. O espaço, porém, não é
definido cartograficamente, mas se desenha como cenário primaveril idealizado
em chave sacroerótica, inerente à visão grega da natureza, suspenso em
temporalidade própria, impregnado de Afrodite, de cujas imagens poéticas e
mítico"-religiosas se desprendem seus elementos constitutivos. Desse cenário
emana uma atmosfera carregada de sensualidade e do divino, algo ampliado pela
antecipação da epifania da deusa\footnote{Verso 13.} invocada como ``Cípris'' --
nome mais frequente na literatura grega antiga, além de ``Afrodite'' --, que evoca
seus elos com um de seus locais de culto mais importantes, a ilha de Chipre,
onde são particularmente fortes suas ligações com o mundo vegetal. Atente"-se
para o caráter ativo da epifania, que reforça a ideia da fusão num fragmento de intensa linguagem sinestésica. Há o desejo de proximidade
entre a voz poética e a deusa. Proximidade que assume um caráter metapoético na diluição dos limites do mundo divino e do sagrado -- a mistura do néctar, nutrição divina, e do vinho, nutrição mortal -- na festividade compartilhada da mélica sáfica que celebra o universo de Afrodite. A fonte principal do texto é um \textit{óstraco} (século \textsc{iii} a.C.) ou caco de cerâmica, material abundante na Grécia antiga e muito utilizado para a
escrita.}


\pagebreak
%\section{Dórica e Cáraxo (Fr.\,5 e 15)}
\section{Prece a Afrodite e às Nereidas (Fr.\,5)}

\begin{gkverse}
Κύπρι καὶ] Νηρήϊδες ἀβλάβη[ν μοι\\
		τὸν κασί]γνητον δ[ό]τε τυίδ’ ἴκεσθα[ι\\
κὤσσα ϝ]ο̣ι ̣θύμω<ι> κε θέλη γένεσθαι\\
πάντα τε]λέσθην,

ὄσσα δὲ πρ]όσθ’ ἄμβροτε πάντα λῦσα[\\
καὶ φίλοισ]ι ϝοῖσι χάραν γένεσθαι\\
\ldots{} ἔ]χθροισι, γένοιτο δ’ ἄμμι\\
\ldots{} μ]ηδ’ εἴς·

τὰν κασιγ]νήταν δὲ θέλοι πόησθαι\\
 ]τίμας, [ὀν]ίαν δὲ λύγραν\\
 ]οτοισι π[ά]ροιθ’ ἀχεύων\\
 ].να\\
 ].εισαΐω[ν] τὸ κέγχρω\\
  ]λεπαγ̣[..( .́ )]αι πολίταν\\
  ]λλωσ̣[...]νηκε δ’ αὖτ’ οὐ\\
 ]κρω[]\\
 ]οναικ[\quad]εο[\quad]. ι\\
 ]..[.]ν· σὺ [δ] ὲ̣ Κύπ̣[ρι]..[..(.)]να\\
       ]θεμ[έν]α κάκαν [\\
	       ]ι.	
\end{gkverse}

\begin{verse}
Ó Cípris e Nereidas, ileso, a mim,\\
o meu irmão concedei aqui chegar,\\
e o que no coração ele queira que seja --\\*
tudo cumpri;

e que seus passados erros todos ele repare\\
e que aos amigos uma alegria ele seja,\\*
\ldots{} aos inimigos, e que não nos seja \ldots{}

[\ldots{}] e a irmã -- que ele a queira fazer
\end{verse}

\pagebreak
{\paragraph{Comentário} No fragmento preservado no \textit{Papiro de Oxirrinco} 7 (século \textsc{iii} d.C.),
temos uma canção"-prece a Afrodite -- chamada pelo nome que nos remete à sua
geografia mítico"-poética e religiosa insular, ``Cípris'' -- e às Nereidas
-- netas de Oceano e filhas do velho do mar, Nereu. Quem apela às deidades o faz
em benefício da 3ª pessoa do singular, a quem se refere como ``meu
irmão''.\footnote{Verso 2.} Uma vez que o \textit{eu} da poesia de Safo é habitualmente
identificado à própria poeta, é leitura corrente que o \textit{ele} é Cáraxo, com base
em Heródoto (século \textsc{v} a.C., \textit{Histórias}, \textsc{ii}, 134--135), segundo quem uma
cortesã de nome Rodopis, feita escrava, foi levada para o Egito do faraó Amásis
(\textit{c.} 570 a.C.) por um homem que, depois, a libertou em troca de vultoso
pagamento efetuado por ``Cáraxo de Mitilene, filho de Escamandrônimo e
irmão de Safo, a poeta''; e Heródoto diz: Cáraxo, em seguida, ``retornou
a Mitilene, e numa canção Safo muito o atacou, de maneira severa''. Há nesse
relato, porém, sérios problemas de cronologia; ademais, nada prova que Heródoto se refira ao
nosso Fr.\,5; e tampouco se justifica de fato a identificação automática de
Safo ao \textit{eu} da canção, o qual pode ser outra personagem.
Na prece, identificadas as deidades, seguem"-se os
pedidos. Primeiro, de proteção ao navegante que retorna -- função própria da
atuação de Afrodite e das Nereidas, indicam seus cultos e imagens
mítico"-poéticas e iconográficas. Mas nos pedidos seguintes, nada há que os
ligue de maneira especial às deidades invocadas; eles falam no
cumprimento de todos os desejos do ``irmão'' -- algo que recorda os
versos 17--18 e 26--27 do Fr.\,1 --, na reparação de seus erros passados, na
atitude para com amigos e inimigos -- ``alegria'' àqueles,
males, a estes, de acordo com a ética grega tradicional. Falam ainda em algo
relativo à sua ``irmã'', que não sabemos quem é.\footnote{Mantenho para o Fr.\,5 (e outros) a edição Voigt (1971). Preferi não adotar reedições com base em novos papiros envoltos em complexas questões, das quais dá ideia a nota de retratação de Anton Bierl e André Lardinois, editores de \textit{The newest Sappho} (Brill, 2016). Recomenda"-me a prudência, mesmo se excessiva, aguardar a estabilização do cenário.}}


\pagebreak
\section[Prece a Afrodite, uma punição para Dórica (Fr.\,15)]{Prece a Afrodite, uma punição\\para Dórica (Fr.\,15)}

\begin{gkverse}
]α̣ μάκαι̣[ρ

\textnormal{[\textit{versos 2--8: ilegíveis e lacunares}]}

Κύ]πρι κα[ί σ]ε πι[κροτ..́]α̣ν ἐπεύρ[οι\\
μη]δὲ καυχάσ[α]ιτο τόδ’ ἐννέ\,[ποισα\\
Δ]ω̣ρίχα τὸ δεύ[τ]ερον ὠς πόθε[\\
]ερον ἦλθε.

\end{gkverse}

\begin{verse}
\ldots{} [tu] venturosa \ldots{}\\ 
\ldots{}\\
Ó Cípris, e a mais amarga te descubra\\
e não se vanglorie isto contando -- ela,\\
Dórica: como a segunda vez \ldots{}\\*
\ldots{} veio.
\end{verse}

\medskip

{\paragraph{Comentário} A única fonte do Fr.\,15 é o \textit{Papiro de Oxirrinco} 1231 (século \textsc{ii} d.C.).
Acredita"-se que se ligaria à mesma narrativa que enreda o Fr.\,5, mas essa
leitura carece de substância historicamente comprovada, e sofre de
questionável biografismo ficcionalizante. Aqui,
parece ser feita uma prece a Afrodite, ou ``Cípris'': o \textit{eu} pede"-lhe a
punição de Dórica.\footnote{Esse nome talvez esteja presente também no Fr.\,7, se certa a emenda. O fragmento é de resto ilegível.} Segundo compreensões usuais, Safo estaria pedindo a punição
de Dórica/\,Rodopis, em benefício de Cáraxo e de si mesma. Como no caso do Fr.\,5,
todavia, além do biografismo, há os dados contraditórios dos testemunhos
antigos, em geral, de caráter francamente anedótico. Note"-se, por fim, que se
revela de forma implícita no segundo pedido uma ação própria de Afrodite e dos
deuses gregos como um todo, na sua relação com os mortais: gabar"-se,
vangloriar"-se, é atitude em potencial perigosa para um mortal que,
deixando"-se levar pela arrogância e prepotência, pode bem esquecer
os limites de sua condição e, por isso, ser punido.}

\pagebreak


\section{Fragmento 22}

\begin{gkverse}

\textnormal{[\textit{verso 1: ilegível e lacunar}]}

]εργον, ..λ’α..[\\
  ]ν ῤέθος δοκιμ̣[\\
    ]ησθαι\\
]ν̣ ἀυάδην χ.[\\
δ]ὲ μή, χείμων[\\
  ].οισαναλγεα.[\\
  ]δε\\
.]. ε̣.[\ldots{}].[\ldots{}κ]έλομαι σ.[\\
\ldots{}].γυλα.[\ldots{}]α̣νθι λάβοισα.α[\\
πᾶ]κτιν, ἆς̣ σε δηὖτε πόθος τ̣.[\\
ἀμφιπόταται

τὰν κάλαν· ἀ γὰρ κατάγωγις αὔτ̣α[\\
ἐπτόαισ’ ἴδοισαν, ἔγω δὲ χαίρω,\\
καὶ γ̣ὰρ αὔτ̣α δήπο̣[τ’] ἐμεμφ[\\
Κ]υπρογέν[ηα

ὠ̣ς ἄραμα̣[ι\\
τοῦτο τῶ[\\
β]όλλομα̣[ι

\end{gkverse}


\begin{verse}
\ldots{} tarefa \ldots{}\\
\ldots{}\\ 
\ldots{}  rosto \ldots{}\\  
\ldots{}  desagradável \ldots{}\\
\ldots{}  e não, inverno \ldots{}\\
\ldots{}  dor \ldots{}\\
\ldots{}\\ 
\ldots{}  peço a ti\\
\ldots{}  [ela], após pegar \ldots{}\\
\ldots{} harpa, enquanto de novo o desejo \ldots{}\\
voa ao redor de ti --\\
a bela --; pois o vestido \ldots{}\\
vendo tremeste, e eu me alegro,\\
pois, certa vez, a própria \ldots{} \\
Ciprogênia \ldots{}\\
como rezo \ldots{}\\
isto \ldots{}\\*
quero \ldots{}\\
\end{verse}

\medskip

{\paragraph{Comentário} Nessa canção fragmentária, também preservada no \textit{Papiro de Oxirrinco}
1231, dignas de notas são a referência à ``harpa'',\footnote{[\textit{Pâ}]\,d\textit{ktin}, verso 11.} instrumento muito
antigo e de procedência oriental, e ao voo do desejo, na imagem poética
recorrente da excitação erótica, ao redor de um \textit{tu} -- voo este que se repete
no presente da canção, como antes, no passado, revela o advérbio ``de
novo'', \textit{dēûte},\footnote{Verso 11.} a marcar, como no Fr.\,1,\footnote{Versos 15, 16, 18.} a intermitência de \textit{éros}. A sensualidade da cena se intensifica com o vestido captado pelos olhos do \textit{eu}, motivo de sua alegria e gatilho de uma recordação em que se insere Afrodite, a
``Ciprogênia'' ou ``nascida em Chipre''. São ligações tradicionais na
poesia erótica de Safo e de outros poetas os binômios beleza--desejo e
desejo--olhar. Afinal, sendo \textit{éros} o desejo sexual, é pelos olhos que entra -- os que contemplam a beleza do corpo físico. Mas o elemento do vestido e o modo como é enfocado na canção é muito característico da mélica de Safo, que amiúde realça peças do vestuário feminino. Sobre isso, um conjunto de fragmentos adiante traduzidos dirá algo mais.}


\pagebreak

\section{Fragmento 33}

\begin{gkverse}
αἴθ’ ἔγω, χρυσοστέφαν’ Ἀφρόδιτα,\\
τόνδε τὸν πάλον <......> λαχοίην
\end{gkverse}

\begin{verse}
\ldots{}se ao menos eu, ó auricoroada Afrodite,\\
este lote \ldots{} obtivesse por parte \ldots{}
\end{verse}

\medskip

{\paragraph{Comentário} Estamos uma vez mais diante de uma canção"-prece, em prol da obtenção de algo,
com o auxílio da deidade invocada, em imagem dourada. O fragmento tem por fonte
o tratado \textit{Sobre a sintaxe} (3.247), do gramático Apolônio Díscolo (século
\textsc{ii} d.C.).}

\pagebreak


\section{Fragmento 65}

\begin{gkverse}

\textnormal{[\textit{versos 1--4 e 11: ilegíveis e lacunares}]}

]Ψάπφοι, σεφίλ[\\
Κύπρωι ̣β̣[α]σίλ[\\
]κ̣αίτοι μέγα δ.[\\
ὄ]σσοις φαέθων̣ [\\
πάνται κλέος [\\
καί σ’ ἐνν Ἀχέρ[οντ 
\end{gkverse}

\begin{verse}
ó Safo \ldots{}\\
[em] Chipre a rainha \ldots{}\\
e deveras grande \ldots{}\\
para tantos [ele] brilhando \ldots{}\\
em toda parte glória \ldots{}

e a ti, no Aqueronte \ldots{}\\
\end{verse}

\medskip

{\paragraph{Comentário} Preservado no \textit{Papiro de Oxirrinco} 1787  (século \textsc{iii} d.C.), cujos textos acham"-se muito mutilados, o fragmento traz uma voz que se dirige a Safo e menciona a ilha de Chipre -- referida, junto a duas outras localidades\footnote{Pafos, na ilha, e Pânormos, na Sicília.} na única linha em que consiste o Fr.\,35. Em seguida, talvez mencione sua mais importante deusa, Afrodite, no verso 2, como sua soberana. A ideia do \textit{kléos} -- a glória alcançada por grandes feitos, referida no Fr.\,44, em sua narrativa mítica -- se sobressai e pode estar vinculada à imortalização pela poesia pela fama, talvez o tema do último fragmento traduzido nesta antologia que congrega todas as canções de Safo, exceto pelos ilegíveis. Nesse sentido poderíamos entender a menção de um dos rios do Hades, o mundo dos mortos, no verso final, o Aqueronte, e igualmente a inserção da poeta como personagem dos versos por meio de seu próprio nome, na abertura que resta.}

\pagebreak

\section{Fragmento 73\,\textsc{a}} 

\begin{gkverse}
\textnormal{[\textit{versos 1--2: ilegíveis e lacunares}]}

]αν Ἀφροδι[τα\\
          ἀ]δύλογοι δ’ ἐρ[\\
        ]β̣αλλοι\\
      α]ις̣ ἔχοισα\\
     ].έ̣να θαασ[ς\\
     ]άλλει\\
     ]ας ἐέρσας [

\end{gkverse}

\begin{verse}
\ldots{} ó Afrodite\\
de doce fala \ldots{}\\
\ldots{} lançaria \ldots{}\\
\ldots{} ela tendo\\
\ldots{}\\
\ldots{} orvalho \ldots{}\\
\end{verse}

\medskip

{\paragraph{Comentário} Preservado na mesma fonte do Fr.\,65, o 73\,\textsc{a} mostra a persona a se dirigir a Afrodite no primeiro verso minimamente legível,\footnote{Verso 3.} mas nada podemos saber da cena desenhada. O adjetivo \textit{adýlogoi}, ``de doce fala'', no verso 4, talvez qualifique o termo \textit{éros} na forma plural, \textit{érōtes}, que, todavia, não se atesta em Safo.}

\pagebreak

\section{Fragmento 86}

\begin{gkverse}
\textnormal{[\textit{verso 1: ilegível e lacunar}]}

] α̣ἰ̣γιόχω λ̣α̣[\\
]. Κ̣υ̣θέρη’ εὐχομ[\\
  ]ο̣ν ἔχοισα θῦμο̣[ν\\
κλ]ῦθί μ̣’ ἄρας αἴ π[οτα κἀτέρωτα\\
    ]ας π̣ρ̣ολίποισ̣α κ[\\
    ]. πεδ’ ἔμαν ἰώ[\\
        ]ν χα̣λέπαι.[

\end{gkverse}

\begin{verse}
\ldots{} porta-égide \ldots{}\\
\ldots{} ó Citereia, eu rezo \ldots{}\\
\ldots{} [tu], tendo \ldots{} coração \ldots{}\\
\ldots{} ouve-me as preces, se já outrora\\
\ldots{} [tu] tendo deixado \ldots{}\\
\ldots{} para minha \ldots{}\\
\ldots{} pesares \ldots{}
\end{verse}

\medskip

{\paragraph{Comentário} Tendo por fonte a mesma do fragmento anterior, o 86 de novo revela a \textit{persona} a falar a Afrodite, nomeada Citereia -- como no Fr.\,140 e em boa parte da poesia grega antiga, pela sua ligação mítico"-cultual com a ilha de Citera --, e em linguagem muito similar à do Fr.\,1, da segunda estrofe.\footnote{Versos 5--8.} É certo que é uma prece, possivelmente um hino clético à deusa, tal qual os Frs. 1 e 2, que igualmente refere Zeus, a quem o epíteto ``porta"-égide'', \textit{aigiókhō},\footnote{Verso 2.} costuma ser atribuído.}

\pagebreak

\section{Fragmento 101}

\begin{gkverse}
χερρόμακτρα δὲ \dagger{}καγγόνων\dagger{}\\
πορφύραι \dagger{}καταυταμενἀ-\\
τατιμάσεις\dagger{} ἔπεμψ’ ἀπὺ Φωκάας\\
δῶρα τίμια \dagger{}καγγόνων\dagger{}\\
\end{gkverse}

\begin{verse}
\ldots{} e panos de mão \ldots{}\\
purpúreas \ldots{}\\
\ldots{} enviou da Fócea\\
dons preciosos \ldots{}
\end{verse}

\medskip

{\paragraph{Comentário} Citado no tratado \textit{Sobre o estilo} (140\,\textsc{ss}.), de Demétrio (século \textsc{iii} a.C. ou \textsc{i} d.C.), os versos estariam no Livro 5 da compilação da obra de Safo na Biblioteca de Alexandria, e seriam, segundo ele, ditos pela poeta a Afrodite. Vestes e dons, tão importantes em Safo e marcados na deusa no Fr.\,1 estão em cena, mas não podemos recuperar seu sentido. Demétrio explica que o primeiro dos elementos destina"-se a enfeitar a cabeça; os versos estão severamente corrompidos; traduzo os termos compreensíveis.}

\pagebreak

\section{Fragmento 102}

\begin{gkverse}
Γλύκηα μᾶτερ, οὔ τοι δύναμαι κρέκην τὸν ἴστον\\
πόθωι δάμεισα παῖδος βραδίναν δι’ Ἀφροδίταν
\end{gkverse}

\begin{verse}
Ó doce mãe, não posso mais tecer a trama -- \\*
domada pelo desejo de um menino, graças à esguia \mbox{Afrodite \ldots{}}
\end{verse}

\medskip

{\paragraph{Comentário} O \textit{eu} feminino desse fragmento, cuja fonte principal é o \textit{Inquérito
sobre os metros} (10.5), de Heféstion (século \textsc{ii} d.C.), dirige uma queixa à sua
``doce mãe'', \textit{glýkēa mâter}:\footnote{Verso 1.} domada pela paixão por um menino, por ação de Afrodite, acha"-se impotente, incapaz de prosseguir com o trabalho no tear. Na dupla de
versos, compõe"-se um cenário erótico muito apropriado a Afrodite, tramado no
entrelaçamento da ação de domar, do desejo, da intervenção da deusa
-- a ``tecelã de ardis'', \textit{dolóploke},\footnote{Verso 2.} no Fr.\,1 --, e do tecer de tramas, tarefa feminina, que, metaforicamente, alude à ação de enganar, própria da sedução.
Vale reparar ainda em como o fragmento ecoa a canção popular, da qual
temos vestígios, tanto na referência ao trabalho, quanto no tema -- a queixa da
menina doente de paixão. O \textit{eu} do fragmento é, pois, dramático, como o é
na poesia grega antiga. Cabe atentar para a figura da mãe: na
casa, espaço das mulheres, a ela cabia supervisionar os trabalhos e os deveres
domésticos da família e de seus servos. E para o contraste entre a violenta
ação de domar, executada por Afrodite, e sua imagem física delicada.}

\pagebreak

\section{Fragmento 112}

\begin{gkverse}
Ὄλβιε γάμβρε, σοι μὲν δὴ γάμος ὠς ἄραο\\
ἐκτετέλεστ’, ἔχηις δὲ πάρθενον, ἂν ἄραο.\\
σοὶ χάριεν μὲν εἶδος, ὄππατα <δ’ \ldots{}>\\
μέλλιχ’, ἔρος δ’ ἐπ’ ἰμέρτωι κέχυται προσώπωι\\
<...> τετίμακ’ ἔξοχά σ’ Ἀφροδίτα
\end{gkverse}

\begin{verse}
Ó feliz noivo, tua boda, como pediste,\\
se cumpriu, e tens a virgem que pediste.\\
Tua forma é graciosa, e \ldots{} olhos de\\
mel, e desejo se derrama na desejável face\\*
\ldots{} honra-te em especial Afrodite \ldots{}
\end{verse}

\medskip

\paragraph{Comentário} Esse fragmento é de uma canção de casamento ou epitalâmio, como outros no item a essa espécie mélica dedicado nesta antologia, a ser entoado no decorrer dos eventos da cerimônia.
Da importância do epitalâmio no \textit{corpus} de Safo e, mais, no universo em que a poeta e seu coro de \textit{parthénoi} se encontram, falei na introdução e na nota a esta segunda edição deste trabalho, lembrando, entre outras evidências que se somam às próprias canções de Safo, a existência de um livro só de mélica epitalâmica na compilação de sua obra  em Alexandria, na famosa Biblioteca. Recordo ainda testemunhos antigos como o de Himério, retórico do século \textsc{iv} d.C., que diz na \textit{Oração 9},\footnote{33--47.} ou Epitalâmio para Severo, seu aluno: ``[\ldots{}] os ritos de Afrodite foram deixados a Safo de Lesbos, para que os cantasse para a lira e os fizesse para o tálamo. Após os concursos, ela seguia para o tálamo, ornamentava o aposento nupcial, preparava o leito, recrutava as virgens, liderava"-as ao quarto da noiva, e também Afrodite sobre o carro das Cárites -- deusas Graças --, e o coro de Erotes, companheiro de folguedos''.

No Fr.\,112, canta"-se a felicidade do noivo qualificado por ela, \textit{ólbios};\footnote{Verso 1.} trata"-se de um exemplo de \textit{makarismós}, a tradicional ``enunciação do casal como feliz e abençoado'',\footnote{Swift, 2010, p. 246.} que vem da tradição popular, tal qual o elogio dos noivos, também feito nos versos.
Vale a pena recordar que a fama dos epitalâmios sáficos parece ter
sido ampla e duradoura; afinal, tardiamente, o sofista Corício de Gaza (século
\textsc{vi} d.C.) dizia, na obra \textit{Epitalâmios em Zacarias} (19): ``Portanto, eu
-- para que a ti de novo agrade -- com um canto sáfico adornarei a noiva\ldots{}''.
Ou seja, não apenas as palavras do texto de um epitalâmio adornam os noivos,
mas a própria canção, uma vez que sua função é louvar a aparência física e
sedutora deles, de modo a estimulá"-los ao enlace sexual, que concretiza a
boda. O epitalâmio integra, assim, outros recursos voltados a essa mesma finalidade,
como o banho ritual. A presença de Afrodite, deusa da beleza e do sexo,
justifica"-se plenamente, portanto. As fontes do fragmento são as já indicadas
obras de Heféstion, referido no Fr.\,102, e de Corício.
Seus versos mostram que os epitalâmios sáficos não são exclusivamente femininos, pois ``o casamento era uma etapa importante de transição na vida do homem, como na da mulher''.\footnote{Swift, 2010, p. 249.} Porém, a vida masculina se consolidava na esfera política e bélica, na feminina  ``a boda e a geração de prole era o foco final, o objetivo'',\footnote{Swift, p. 250.} razão pela qual a noiva tem proeminência sobre o noivo nas festividades, algo que se reflete nas canções.

\pagebreak

\section{Fragmento 133}

\begin{gkverse}
Ἔχει μὲν Ἀνδρομέδα κάλαν ἀμοίβαν

\hspace*{21mm}\adforn{47}

Ψάπφοι, τί τὰν πολύολβον Ἀφροδίταν\ldots;
\end{gkverse}

\begin{verse}
``Tem Andrômeda bela paga \ldots{}''

\hspace*{21mm}\adforn{47}

``Ó Safo, por que a multiafortunada Afrodite \ldots{}?''
\end{verse}

\medskip

{\paragraph{Comentário} O Fr.\,133 é um canto coral dialogado, similarmente ao 140, à frente. No verso 1, uma voz -- ou um coro -- faz uma constatação acerca de ``Andrômeda''; no seguinte,
lança uma pergunta a ``Safo'', \textit{persona }da poeta no fragmento de
uma canção de evidente caráter dramático. De acordo com leituras correntes,
``Andrômeda'' seria a poeta"-líder de um grupo de
meninas rival ao de Safo. Esse nome aparece também no Fr.\,130, em contexto
erótico que envolve outra personagem feminina, ``Átis'', referida em
outros fragmentos de Safo. Mas a fragilidade de nosso conhecimento sobre Lesbos
e sua sociedade precisa ser admitida: nada resta de Andrômeda, exceto seu nome
na mélica sáfica materialmente frágil. O fragmento é de difícil leitura:
fala"-se em punição ou benefício? E o que encobre a lacuna do verso 2, em que
Afrodite é dita -- como o são os demais deuses gregos -- venturosa, afortunada, 
feliz (\textit{polýolbon})?\footnote{Verso 2.} Sua fonte de transmissão é o tratado do metricista Heféstion (14.7), referido
no Fr.\,102.}

\pagebreak

\section{Fragmento 134}

\begin{gkverse}
Ζὰ <.> ἐλεξάμαν ὄναρ Κυπρογενηα
\end{gkverse}

\begin{verse}
Em sonho falei à Ciprogênia \ldots{}
\end{verse}

\medskip

{\paragraph{Comentário} O sonho, como percebido pelos antigos, surge como uma das formas de 
comunicação entre deuses e homens. O fragmento foi preservado em Heféstion (12.4), já
referido no Fr.\,102. Nele, projeta"-se a intimidade da \textit{persona} de Safo com Afrodite, pois não é a deusa que lhe fala, como tradicionalmente, mas a poeta que a ela se dirigiu no passado do sonho sonhado, que parece recontar a alguém.}


\pagebreak
\section{Afrodite e Adônis, paixão e morte (Fr.\,140)}

\begin{gkverse}
Κατθνάσκει, Κυθέρη’, ἄβρος Ἄδωνις· τί κε θεῖμεν;

καττύπτεσθε, κόραι, καὶ κατερείκεσθε χίθωνας.
\end{gkverse}

\begin{verse}
``Morre, Citereia, delicado Adônis. Que podemos fazer?''

``Golpeai, ó virgens, vossos seios, e lacerai vossas vestes \ldots{}''
\end{verse}

\medskip

{\paragraph{Comentário} Segundo o viajante grego do século \textsc{ii} d.C., Pausânias, Safo ``sobre
Adônis cantou'';\footnote{\textit{Descrição da Grécia}, \textsc{ix}, 29, 8.} prova isso o Fr.\,140, 
centrado no contexto mítico da paixão de Afrodite por ele, e da sua morte.
O belo jovem, associado estreitamente ao universo sacroerótico da
flora, das plantas, dos arômatas, dos jardins -- muito caro à deusa --,
tornado alvo da paixão dela, acaba morto violentamente no auge de sua
virilidade; na versão mais retomada, matou"-o um javali, fato que deve se ligar
à interdição do porco nos rituais a Afrodite, em muitos cultos gregos. Morto
Adônis, a deusa sofre. Esse mito se acha nas tradições poéticas, iconográficas
e religiosas, o que mostra a influência oriental em terras gregas, uma vez que
a figura de Adônis é fenícia, em termos de sua origem -- e Afrodite associa"-se
estreitamente ao Oriente. Ora, repare"-se que a deidade do Fr.\,140 é chamada
``Citereia'', em referência ao seu culto proeminente na ilha de Citera,
de forte influência fenícia.
Não é demais ressaltar que é também oriental o
mito da deusa do sexo arrebatada por um jovem deus precocemente morto. O
cenário mítico está plasmado no fragmento em registro fúnebre e dramático por um coro feminino, a morte de Adônis é afirmada, e a isso
se segue uma indagação; a resposta vem em seguida, dita pela própria Afrodite.

\pagebreak
Tal resposta, imperativa, demanda das \textit{kórai}, ``meninas'' ou ``virgens'', a gestualidade fúnebre tradicional do lamento feminino, chamada \textit{góos}), atestada na iconografia e na Ilíada \textsc{xviii},\footnote{30--1, 50--1.} no pranto das servas e das Nereidas pela morte de Pátroclo.
Do ponto de vista das práticas cultuais relacionadas à morte de Adônis, o festival das Adonias 
era celebrado no início do verão, por mulheres adultas. Essa festa se atesta 
desde o século \textsc{vi} a.C., em várias partes do
mundo grego, e ainda é feita no Egito do \textsc{iii} a.C., agora de modo público e
oficial. Durante as Adonias, as mulheres pranteavam Adônis, representando o
sofrimento de Afrodite, exposto na canção sáfica. A mais antiga
menção ao mito, ao culto e ao rito fúnebre dá"-se no Fr.\,140, cuja fonte é
Heféstion (10.4), a mesma do Fr.\,102. E como Herington\footnote{1985, p. 57.} afirma, a estrutura dramática do Fr.\,140,  como a do 114, ``antecipa'' cenas semelhantemente trabalhadas na tragédia ática: ``Somos colocados num evento mítico [\ldots{}] que não está sendo narrado, mas imediatamente apresentado''; logo,  temos que creditar a Safo, e a mais ninguém, o mais antigo fragmento de drama em verso, preservado em toda a tradição europeia [\ldots{}]''. Canto dialogado coral fúnebre, o fragmento anuncia a própria \textit{performance} pública -- possivelmente ``uma \textit{reperformance} da morte em vida de Adônis por um grupo de suas adoradoras''.\footnote{Herington.} Diga"-se, por fim: ``Um dos gêneros de discurso público associados à mulher tanto na Grécia arcaica, quanto na rural de hoje, é o lamento'',\footnote{Lardinois 1996, p. 172.} justamente o gênero do Fr.\,140. Em conformidade com a lógica
paidêutica do grupo coral liderada por Safo, a dramatização da dor lutuosa de Afrodite que ensina as moças a prantear a morte de Adônis permiti"-lhes vivenciar no mito a experiência fúnebre, parte da vida cotidiana. Talvez a presença das meninas, em vez das mulheres -- como nos cultos das Adonias -- se deva à dimensão da \textit{paideía} das jovens do coro que, desse modo, ``lamenta com ou como Afrodite a perda de Adônis''.\footnote{Stehle 1997, p. 224.}}

\pagebreak
\section{Fragmento 168}

\begin{gkverse}
ὦ τὸν Ἄδωνιν
\end{gkverse}

\begin{verse}
Ó por Adônis!
\end{verse}

\medskip

{\paragraph{Comentário} Preservado por Mário Plotino Sacerdo (século \textsc{iii} d.C.), em sua
\textit{Gramática} (\textsc{iii}. 3), e atribuído a Safo pela presença de Adônis em
sua mélica -- o anterior Fr.\,140 --, as palavras parecem lamentosas e
talvez também pranteassem a morte do belo amado de Afrodite, lá cantado. Na canção perdida, de que temos só a sequência acima, a deusa poderia estar presente junto a ele.}


\pagebreak
\section{Fragmento 117\,\textsc{b}}

\begin{gkverse}
 Ἔσπερ’ ὐμήναον

\hspace*{10mm}\adforn{47}

ὦ τὸν Ἀδώνιον
\end{gkverse}

\begin{verse}
\ldots{} Ó Vésper! Ó Himeneu!

\hspace*{10mm}\adforn{47}

\ldots{} Ó pelo Adônio!
\end{verse}

\medskip

{\paragraph{Comentário} Preservados pela mesma fonte do Fr.\,168, como exemplares de métrica de
himeneus, as canções de casamento -- adiante as veremos -- que na
Biblioteca de Alexandria passaram a ser chamadas pelo termo epitalâmio,\footnote{Literalmente, ``sobre o leito nupcial, o tálamo''.} as linhas não contíguas (\textbf{a}, \textbf{b}) viriam de uma composição talvez enlaçada ao mito de Adônis, de que falei no
comentário ao Fr.\,140, que canta sua morte e o luto de Afrodite, da qual foi o noivo em boda não consumada, segundo as tradições. Note"-se
a presença de Vésper, a Estrela da Tarde, que veremos na canção
epitalâmica Fr.\,104\textsc{a}. Não necessariamente é lamentosa a segunda linha
(\textbf{b}); em vista do contexto epitalâmico que traz o deus da boda, do
enlace sexual que a sela -- ver o comentário ao Fr.\,111, entre os
epitalâmios --, pode ser celebratória à beleza do noivo que talvez projete, maximizando"-a. No entendimento da edição Voigt,\footnote{De 1971.} as duas linhas de algum modo integrariam o Fr.\,117, epitalâmio de verso único de saudações a noivo e noiva, como se verá na sua tradução, no item dedicado às canções desse tipo.}




\chapter{Eros}

\paragraph{\textsc{nota introdutória}} 
É forte em Safo a temática erótica, mas não tem a mesma presença marcante que
Afrodite o deus Eros, que lhe será sempre subordinado, como filho ou servo, na
tradição mítico"-poética, iconográfica e cultual da Grécia antiga, como se vê no
Fr.\,159, abaixo. A relevância do universo feminino certamente favorece a
proeminência de Afrodite sobre Eros. Na imagem sáfica do deus, projeta"-se a
concepção grega sobre o desejo, a paixão erótica: prazerosa e doce, mas
sobretudo violenta, acre, dolorosa, doença de corpo e mente, força dominadora e
intermitente.

\section{Fragmento 47}

\begin{gkverse}
\vinphantom{φρένας, ὠς ἄνεμος κὰτ}Ἔρος δ’ ἐτίναξέ <μοι>\\
φρένας, ὠς ἄνεμος κὰτ ὄρος δρύσιν ἐμπέτων.
\end{gkverse}

\begin{verse}
\ldots{} Eros sacudiu meus\\*
sensos, qual vento montanha abaixo caindo sobre as \mbox{árvores \ldots{}}
\end{verse}

\medskip

\paragraph{Comentário} A fonte do fragmento é uma \textit{Oração} (18), do retórico Máximo de Tiro (século \textsc{ii} d.C.).
Note"-se como a imagem da tempestade ou ventania violenta é usada para falar de Eros a se abater sobre a \textit{persona}, comprometendo sua estabilidade mental, na medida em que lhe atingi os ``sensos'', \textit{phrénas}.\footnote{Verso 2.} Veremos sobretudo no Fr.\,31 a síntese do sofrimento erótico, mas já no Fr.\,1\footnote{Versos 3--4.} se faz presente.



\pagebreak
\section{Fragmento 54}

\begin{gkverse}
ἔλθοντ’ ἐξ ὀράνω πορφυρίαν περθέμενον χλάμυν
\end{gkverse}

\begin{verse}
\ldots{} vindo do céu, envolto em purpúreo manto \ldots{} 
\end{verse}

\medskip

{\paragraph{Comentário} Segundo a fonte, Pólux,\footnote{\textit{Onomástico} 10.124, século \textsc{ii} d.C.} Safo descreve Eros quando canta o verso acima.}

\section{Fragmento 130}

\begin{gkverse}
Ἔρος δηὖτέ μ’ ὀ λυσιμέλης δόνει,\\
			γλυκύπικρον ἀμάχανον ὄρπετον\\

\hspace*{22mm}\adforn{47}

			Ἄτθι, σοὶ δ’ ἔμεθεν μὲν ἀπήχθετο\\
φροντίσδην, ἐπὶ δ’ Ἀνδρομέδαν πότη<ι>

\end{gkverse}

\begin{verse}
\ldots{} Eros de novo -- deslassa-membros -- me agita,\\*
dulciamara inelutável criatura \ldots{}

\hspace*{22mm}\adforn{47}

\ldots{} ó Átis, mas a ti tornou-se odioso meu\\*
pensamento, e para Andrômeda alças voo \ldots{}
\end{verse}

\pagebreak
{\paragraph{Comentário} O primeiro par de versos sintetiza os elementos que marcam a imagem poética de
\textit{éros}, o deus ou a força. O segundo par traz duas figuras femininas
outras vezes mencionadas nas canções de Safo,\footnote{Frs.\,133, 49, 96.} entre
as quais se insere, em chave de conflito, a 1ª pessoa do singular, cuja
identidade nos escapa. O referido conflito pode ter caráter
erótico, como indica a linguagem dos dois primeiros versos. E se Andrômeda é mesmo poeta rival de Safo, esta pode estar a censurar Átis,\footnote{Além dos fragmentos que veremos aqui, o nome é a única palavra legível no Fr.\,8.} nome de uma das virgens do coro outras vezes presente nas canções, por ter
por ter saído de seu grupo para passar ao daquela líder. O fragmento está preservado no tratado do metricista Heféstion (7.7), já
referido no Fr.\,102.
Abaixo, na caracterização de Eros, merecem destaques os adjetivos  compostos \textit{lysimelḗs } -- que Hesíodo já lhe dá na \textit{Teogonia} (121) --, que marca a fragmentação do sujeito, e \textit{glýkypikron} -- primeiro doce, depois, amargo --, o adjetivo \textit{amákhanon}, que marca a impotência em que se acha o amador diante de deus; e \textit{órpeton}, que nomeia criaturas rastejantes, arrepiantes, como as serpentes que paralisam de temor o sujeito que as vê.}


\pagebreak
\section{Fragmento 159}

\begin{gkverse}
σύ τε κἆμος θεράπων Ἔρος
\end{gkverse}

\begin{verse}
``\ldots{} tu e meu servo, Eros \ldots{}''
\end{verse}

\medskip

{\paragraph{Comentário} Máximo de Tiro, fonte do fragmento 47 e deste, afirma que, numa canção da poeta,
Afrodite fala a Safo, estabelecendo a hierarquia corrente entre si e o deus que
menciona.}

\section{Fragmento 172}

\begin{gkverse}
ἀλγεσίδωρος
\end{gkverse}

\begin{verse}
\ldots{} doa-dores \ldots{} 
\end{verse}

\medskip

{\paragraph{Comentário} A mesma fonte dos Frs.~47 e 159 preservou como sendo atribuída por Safo a Eros -- o deus ou o desejo -- a seguinte qualificação.}


\section{Fragmento 188}

\begin{gkverse}
μυθόπλοκος
\end{gkverse}

\begin{verse}
\ldots{} tecelão de palavras \ldots{}
\end{verse}

\medskip

{\paragraph{Comentário} Citado por Máximo de Tiro, fonte do Fr.\,47, o 188 traz apenas uma palavra, \textit{mythoplókon}, que teria sido dita sobre Eros por Safo, diz o antigo retórico. O adjetivo pode se referir à eloquência da sedução, potencialmente enganadora, como indicaria o tecer que no mundo erótico tem tal conotação, como vimos no Fr.\,1 e no epíteto \textit{dolóploke} para Afrodite, a ``tecelã de ardis''.\footnote{Verso 2.}}


\chapter{Ártemis}

\section{Fragmento 44\,\textsc{a}}

\begin{gkverse}
\textnormal{[\textit{verso 1: ilegível e lacunar}]}

Φοίβωι χρυσοκό]μαι τὸν ἔτικτε Κόω .[\\
μίγεισ(α)         Κρ]ονίδαι μεγαλωνύμω̣<ι>\\
Ἄρτεμις δὲ θέων] μέγαν ὄρκον ἀπώμοσε\\
    κεφά]λαν· ἄϊ πάρθενος ἔσσομαι\\
             ].ων ὀρέων κορύφα̣ι̣σ’ ἔπι\\
    ]δ̣ε νεῦσον ἔμαν χάριν·\\
         ένευ]σ̣ε θέων μακάρων πάτηρ·\\
ἐλαφάβ]ολον ἀγροτέραν θέο̣ι\\
            ].σιν ἐπωνύμιον μέγα·\\
   ]ερος οὐδάμα πίλναται·\\
     ].[.]...μα̣φόβε[..]έ̣ρω·

\textnormal{[\textit{versos 1--4 e 10: ilegíveis e lacunares}]}

Μοισαν ἀγλα[\\
πόει καὶ Χαρίτων̣[\\
βραδίνοις ἐπεβ.[\\
ὄργας μὴ ’πιλάθε.[\\
θ̣ν̣άτοισιν· πεδ.́χ[\\
      ]δ̣αλίω[
\end{gkverse}
\pagebreak
\begin{verse}
\ldots{} a Febo auricomado, a quem [\ldots{}] gerou de \rlap{Coios \ldots{}}\\
\ldots{} unida ao Cronida de grande nome.\\
\ldots{} mas Ártemis jurou a grande jura dos deuses:\\
``\ldots{} para sempre virgem serei\\
\ldots{} sobre os cimos das montanhas\\
\ldots{} concede-me este favor''.\\
\ldots{} concedeu-o o pai dos deuses venturosos\\
\ldots{} flecha-cervos, a selvagem, os deuses\\
\ldots{} grande título\\*
[Eros] dela nunca se achega \ldots{}

\hspace*{16mm}\adforn{47}

\ldots{} Musas esplêndidas \ldots{}\\
faz\ldots{} das Cárites \ldots{}\\
esguios \ldots{}\\
a cólera não esquecer \ldots{}\\
mortais \ldots{}
\end{verse}

\medskip

{\paragraph{Comentário} Apenas neste fragmento, e no de resto ilegível Fr.\,84, cuja fonte é o \textit{Papiro Fouad} 239 (séculos \textsc{ii} ou
\textsc{iii} d.C.), surge Ártemis em seu desenho tradicional de eterna virgem caçadora,
irmã de Apolo, filha de Leto, que bosques percorre, cercada de animais. A
virgindade, desejada pela deusa, concedeu"-a Zeus à filha.}




\chapter{As Cárites ou «\,Graças\,»}

\paragraph{\textsc{nota introdutória}}
Quase sempre nomeadas coletivamente, as deusas integram o séquito de Afrodite.
Esse elo reflete afinidades sobretudo no universo erótico, uma vez que as
deusas, cujo nome é a forma do plural de \textit{kháris }-- conceito que vai de
``graça, charme, regozijo, prazer'', no sentido físico, a ``favor dos deuses'', no
âmbito da reciprocidade das relações --, favorecem a beleza e a sedução na
esfera do sexo, a festa e a alegria na da vida cotidiana, o vigor, o
crescimento e a renovação, nas esferas humana e vegetal. São as Cárites
filhas de Zeus e integrantes da ordem olímpica, indica o Fr.\,53, em que
recebem um epíteto dado por Safo também a Eos, a Aurora, na ``Canção sobre a velhice'';
tal epíteto, ao evocar a rosa, flor predileta de Afrodite, joga tintas eróticas
sobre o desenho das deusas. E são as Cárites, segundo a \textit{Teogonia},\footnote{Versos 64--67.} de Hesíodo, vizinhas das Musas e de Hímeros (Desejo) no Olimpo -- imagem refletida na presença daquelas deusas e do erotismo no Fr.\,128. Deste -- cuja fonte é Heféstion (9.2), como no caso do Fr.\,102 -- e do
Fr.\,53 -- cuja fonte é um comentário antigo ao \textit{Idílio }28, do poeta
Teócrito (séculos \textsc{iv}--\textsc{iii} a.C.) --, temos os respectivos versos inaugurais de
\textit{hinos cléticos}.

\pagebreak
\section{Fragmento 53} 

\begin{gkverse}
Βροδοπάχεες ἄγναι Χάριτες δεῦτε Δίος κόραι
\end{gkverse}

\begin{verse}
Para cá, ó Cárites de róseos braços, meninas de Zeus \ldots{}
\end{verse}\bigskip


\section{Fragmento 128} 

\begin{gkverse}
Δεῦτέ νυν ἄβραι Χάριτες καλλίκομοί τε Μοῖσαι
\end{gkverse}

\begin{verse}
Para cá, vós, delicadas Cárites e Musas de belas comas, \ldots{}
\end{verse}


\chapter{Eos, a Aurora}

\section{Fragmento 123}

\begin{gkverse}
ἀρτίως μὲν ἀ χρυσοπέδιλος Αὔως
\end{gkverse}

\begin{verse}
\ldots{} recentemente Eos, de áurea sandália, \ldots{}
\end{verse}

\medskip

{\paragraph{Comentário} O destaque ao adorno dos pés enfatiza o erotismo da imagem; pés, braços,
cabelos, colo/\,seio e olhos são do corpo feminino as partes mais enfocadas na
linguagem erotizante da poesia grega. Isso se intensifica na menção ao ouro;
afinal, só uma deusa é dita simplesmente ``áurea'' nos poemas homéricos
e na maior parte da poesia antiga: Afrodite. O fragmento tem por fonte o
tratado \textit{Sobre palavras similares, mas diferentes} (75), do gramático Amônio
(século \textsc{ii} d.C.).
Nele, como no Fr.\,157, em que lemos apenas ``soberana Eos'', e na ``Canção sobre a velhice'', a deusa da Aurora se faz presente.}



\chapter{Hera}

\section{Fragmento 17}

\begin{gkverse}
Πλάσιον δη μ[\\
πότνι’ Ἦρα σὰ χ[\\
τὰν ἀράταν Ἀτ[ρέιδαι\qquad        κλῆ-]\\
τοι βασίληες·\\

ἐκτελέσσαντες μ[\\
πρῶτα μὲν περι̣.[\\
τυίδ’ ἀπορμάθεν[τες\\
οὐκ ἐδύναντο

πρὶν σὲ καὶ Δί\,’ ἀντ[\\
καὶ Θυώνας ἰμε̣[\\
νῦν δὲ κ[\\
κὰτ τὸ πάλ̣[

ἄγνα καὶ κα̣[\\
π]αρθ[εν\\
ἀ]μφι.[\\

\textnormal{[\textit{versos 16--8: ilegíveis e lacunares}]}

ἔμμενα̣[ι\\
{[}?{]}ρ̣ ̣ἀπίκε[σθαι.
\end{gkverse}

\chapter*{}
\section*{}

\begin{verse}
Perto, cá \ldots{}\\
veneranda Hera, teu \ldots{}\\
a prece \ldots{} os Atridas,\\*
os reis,

e perfizeram \ldots{}\\
primeiro em redor \ldots{}\\
para cá tendo partido \ldots{}\\*
não conseguiam,

e, antes de a ti, Zeus dos suplicantes \ldots{}\\
e ao adorável \ldots{} filho de Tione;\\
mas agora \ldots{}\\*
tal qual no passado,

sacro e \ldots{}\\
de moças \ldots{}\\
em torno \ldots{}\\*
\ldots{}

ser \ldots{}\\*
Hera \ldots{}, vir.
\end{verse}

\pagebreak
{\paragraph{Comentário} O fragmento, uma prece a Hera, é bastante precário, e suas principais fontes são o \textit{Papiro de Oxirrinco} 1231, rolo em que também se preservaram os Frs.~15 e 22, e o \textit{Papiro de Milão} \textsc{ii} 123. São nomeados Hera,\footnote{Versos 2 e talvez 20.} Zeus e Dioniso -- ``filho de Tione'',\footnote{Verso 10.} epíteto de Sêmele, a princesa tebana com quem Zeus o gerou e que foi imortalizada, após a morte pela visão do deus em seu terrível brilho e esplendor. Essa junção faz pensar no santuário das três divindades em Messa, em Lesbos. A referência aos Atridas, Agamêmenon e Menelau, por sua vez, traz à prece no presente (11) o mundo do passado mítico de Troia; aos Atridas remontaria a fundação do culto lésbio à deusa invocada na canção, quando de seu regresso de Troia a Argos. E as referências a algo sagrado e a moças virgens
fazem pensar em algo relativo ao culto no contexto de um exílio, considerando que talvez Zeus receba no verso 9 o epíteto cultual \textit{antaîos}, ``dos suplicantes'', que Alceu, o poeta lésbio contemporâneo a Safo, e por duas vezes exilado, confere ao deus no Fr.\,129; e no templo de Hera ele vê um concurso de beleza no Fr.\,130\,\textsc{b}. Diga"-se, por fim, que o nome da deusa\footnote{Verso 4.} e ``festividade'', \textit{eórtan},\footnote{Verso 3.} são as únicas palavras legíveis do Fr.\,9, indicativas de culto a Hera, como um pouco melhor vemos neste Fr.\,17.}



\chapter{Musas}

\section{Fragmento 32}

\begin{gkverse}
αἴ με τιμίαν ἐπόησαν ἔργα\\
τὰ σφὰ δοῖσαι
\end{gkverse}

\begin{verse}
\ldots{} elas [as Musas] me fizeram honrada, suas próprias\\
obras doando-me \ldots{}
\end{verse}

\medskip

{\paragraph{Comentário} A fonte do fragmento é \textit{Sobre os pronomes} (144\textsc{a}), de Apolônio
Díscolo, gramático que, noutro tratado, preservou também o já visto Fr.\,33. O
gênero feminino do \textit{eu} e a menção de seus \textit{érga }, ``trabalhos'', como
``dons'' que a tornam honrada levam a pensar que o sujeito feminino
coletivo no pronome \textit{aí}\footnote{Verso 1.} deve ser divino -- deve ser o conjunto das Musas. Logo, tais dons
seriam o da poesia, no entendimento mais comum da expressão.}


\section{Fragmento 124}

\begin{gkverse}
αὔτα δὲ σὺ Καλλιόπα
\end{gkverse}

\begin{verse}
\ldots{} e tu mesma, ó Calíope, \ldots{}
\end{verse}

\medskip

{\paragraph{Comentário} Preservado em Heféstion, como o 102, o fragmento nomeia a líder das Musas, ``Bela Voz'', como a retrata Hesíodo no catálogo de seus nomes.\footnote{\textit{Teogonia}, versos 77--80.}}



\pagebreak
\section{Fragmento 127}

\begin{gkverse}
Δεῦρο δηὖτε Μοῖσαι χρύσιον λίποισαι \ldots{}
\end{gkverse}

\begin{verse}
Para cá, de novo, ó Musas, deixando a áurea [casa de Zeus] \ldots{}
\end{verse}

\medskip

{\paragraph{Comentário} O fragmento, citado em Heféstion (15.25), como o 102, consiste, pode"-se imaginar, no
verso inaugural de um \textit{hino clético }às Musas, invocadas a deixarem,
talvez, a casa de Zeus, no Olimpo, regularmente ``áurea''
na poesia grega antiga. O chamamento, repetido no presente da prece, deve ter
por pedido algo relativo a essas deusas, como a habilidade poética.}

\pagebreak
\section{Fragmento 150}

\begin{gkverse}
οὐ γὰρ θέμις ἐν μοισοπόλων <δόμωι>\\
θρῆνον ἔμμεν’ <...> οὔ κ’ ἄμμι πρέποι τάδε
\end{gkverse}

\begin{verse}
\ldots{} pois não é correto na casa dos servos das Musas\\*
haver o treno; \ldots{} isso não nos seria adequado \ldots{}
\end{verse}

\medskip

{\paragraph{Comentário} De acordo com a fonte, Máximo de Tiro,\footnote{\textit{Oração} 18.} que também preservou o Fr.\,47, anteriormente traduzido, ``Sócrates ficou bravo com Xantipe que se
lamentava enquanto ele morria'', em referência ao final do \textit{Fédon}, de
Platão (séculos \textsc{v--iv} a.C.), ``e Safo ficou brava com sua filha'', segundo os testemunhos, de
nome Cleis, conforme ainda se verá. A leitura é
claramente biografizante; de seguro, vemos a ênfase na adequação: na casa de quem serve
deidades luminosas como as Musas, não se admite o canto lamentoso, ou seja, o
treno, a nênia. Mas se Safo é serva dessas deusas, vale lembrar que sua poesia
escapa à restrição, pois há em seu \textit{corpus} ao menos um canto lamentoso
fúnebre, o Fr.\,140, que, portanto, problematiza o biografismo da interpretação
do fragmento cuja cena perdemos. Ademais, o canto da Musa não tem, na visão grega antiga,
do tipo que se percebe no Fr.\,150. Lembre"-se que a \textit{Ilíada},
poema aberto por uma invocação à Musa, é de caráter
profundamente trágico.}


\chapter{Deuses vários em inícios frustrados}


\section{Fragmento 103}

\begin{gkverse}
\textnormal{[\textit{verso 1: ilegível e lacunar}]}

].ατε τὰν εὔποδα νύμφαν [\\[8pt]
].τ̣α παῖδα Κρονίδα τὰν ἰόκ[ολπ]ον [\\[8pt]
].ς̣ ὄργαν θεμένα τὰν ἰόκ[ολ]πος α[\\[8pt]
 ].. ἄγναι Χάριτες Πιέριδέ\,[ς τε] Μοῖ\,[σαι\\[8pt]
     ].[. ὄ]πποτ’ ἀοιδαι φρέ̣ν[...]αν.[\\[8pt]
             ]σ̣αιοισα λιγύραν [ἀοί]δαν\\[8pt]
         γά]μβρον, ἄσαροι γ̣ὰρ̣ ὐμαλικ[\\[8pt]
  ]σε φόβαισι θεμένα λύρα .[\\[8pt]
  ]..η χρυσοπέδιλ̣[ο]ς Αὔως
\end{gkverse}

%\chapter*{}
%\section*{}
\begin{verse}
\ldots{} a noiva de belos pés \ldots{}\\[8pt]
\ldots{} a filha de colo violáceo do Cronida \ldots{}\\[8pt]
\ldots{} a raiva pondo \ldots{} a de violáceo colo \ldots{}\\[8pt]
\ldots{} sacras Cárites e Piérias Musas \ldots{}\\[8pt]
\ldots{} quando \ldots{} canções \ldots{}\\[8pt]
\ldots{} clara canção \ldots{}\\[8pt]
\ldots{} noivo, pois atrevidos coevos \ldots{}\\[8pt]
\ldots{} cachos, pondo a lira \ldots{}\\[8pt]
\ldots{} Eos de áurea sandália \ldots{}\\[8pt]
\end{verse}

\medskip

{\paragraph{Comentário} Preservado  no \textit{Papiro de Oxirrinco} 2294 (século \textsc{ii} d.C.), o fragmento traz, explica a fonte, dez versos de abertura de dez distintas canções, mencionando vários deuses: uma não identificável filha de Zeus, o filho de Crono; as Cárites; as Musas Piérias e Eos. Traduzo os mais legíveis, como é sempre o caso nesta antologia, dos quais o primeiro e o sétimo podem ser de epitalâmios, espécie mélica de que vimos o Fr.\,112, e outros ainda veremos.
Noto, quanto ao verso inicial, que traz os mesmos termos (\textit{eúpoda nýmphan}), na mesma sequência e declinação em que se acham no Fr.\,103B, no qual só se lê ainda uma palavra, ``quarto nupcial'', \textit{thalámō}.}


\chapter{Cenas míticas}

\section[A saga troiana: as bodas de Heitor e Andrômaca (Fr.\,44)]{A saga troiana: as bodas de Heitor\\e Andrômaca (Fr.\,44)\protect\footnote{\MakeUppercase{V}er
artigo \MakeUppercase{R}agusa, ``\MakeUppercase{H}eitor e \MakeUppercase{A}ndrômaca, da festa de bodas à celebração fúnebre:
imagens épicas e líricas do casal na \textit{\MakeUppercase{I}líada} e em \MakeUppercase{S}afo (\MakeUppercase{F}r.~44 Voigt)''.
\textit{\MakeUppercase{C}alíope} 15, 2006, pp. 36--63.}}

\begin{gkverse}
Κυπρο̣.[\qquad     - 22 -\qquad       ] ας̣·\\
κάρυξ ἦλθε̣ θε̣[\qquad - 10 -\qquad  ]ελε̣[\ldots{}].θεις\\
Ἴδαος ταδεκα\ldots{}φ[\ldots{}].ις τάχυς ἄγγελος\\
< ``\qquad			>\\
τάς τ’ ἄλλας Ἀσίας .[.]δε.αν κλέος ἄφθιτον·\\
Ἔκτωρ καὶ συνέταιρ̣[ο]ι ἄγ̣οισ’ ἐλικώπιδα\\
Θήβας ἐξ ἰέρας Πλακίας τ’ ἀπ’ [ἀϊ]ν<ν>άω\\
ἄβραν Ἀνδρομάχαν ἐνὶ ναῦσιν ἐπ’ ἄλμυρον\\
πόντον· πόλλα δ’ [ἐλίγματα χρύσια κἄμματα\\
πορφύρ[α] καταΰτ[με]να, ποίκ̣ιλ’ ἀθύρματα,\\
ἀργύρα̣ τ̣’ ἀνά̣ριθ̣μα ποτήρια κἀλέφαις.''\\
ὢς εἶπ’· ὀτραλέως δ’ ἀνόρουσε πάτ[η]ρ̣ φίλος·\\
φάμα δ’ ἦλθε κατὰ πτ̣όλιν εὐρύχο̣ρ̣ο̣ν φίλοις.\\
αὔτικ’ Ἰλίαδαι σατίναι[ς] ὐπ’ ἐυτρόχοις\\
ἆγον αἰμιόνοις, ἐ̣π̣[έ]βαινε δὲ παῖς ὄχλος\\
γυναίκων τ’ ἄμα παρθενίκα[ν] τ\ldots{}[.].σφύρων,\\
χῶρις δ’ αὖ Περάμοιο θυγ[α]τρες[\\
ἴππ[οις] δ’ ἄνδρες ὔπαγον ὐπ’ ἀρ̣[ματα\\
π[\qquad  ]ες ἠίθ̣εοι, μεγάλω[σ]τι δ̣[\\
δ[\qquad  ]. ἀνίοχοι φ[\ldots{}].[\\
π̣[\qquad  ]ξα.o[

\textnormal{[\textit{alguns versos perdidos}]}\\

\vinphantom{ὄ̣ρ̣ματ̣α̣ι̣ [\qquad\qquad\quad}    ἴ]κελοι θέοι[ς\\
\vinphantom{ὄ̣ρ̣ματ̣α̣ι̣ [\qquad\qquad\quad}     ] ἄγνον ἀολ[λε\\
ὄ̣ρ̣ματ̣α̣ι̣ [\qquad\qquad\quad			     ]νον ἐς Ἴλιο[ν\\
αὖλος δ’ ἀδυ[μ]έλης̣ [\qquad	    ]τ’ ὀνεμίγνυ[το\\
καὶ ψ[ό]φο[ς κροτάλ[ων\qquad    ]ως δ’ ἄρα πάρ[θενοι\\
ἄειδον μέλος ἄγν̣[ον ἴκα]νε δ’ ἐς α̣ἴ̣θ̣[ερα\\
ἄχω θεσπεσία̣ γελ̣ [\\
πάνται δ’ ἦς κὰτ ὄδο[ις\\
κράτηρες φίαλαί τ’ ὀ[...]υεδε[..]..εακ[.].[\\
μύρρα καὶ κασία λίβανός τ’ ὀνεμείχνυτο\\
γύναικες δ’ ἐλέλυσδον ὄσαι προγενέστερα[ι\\
πάντες δ’ ἄνδρες ἐπήρατον ἴαχον ὄρθιον\\
πάον’ ὀνκαλέοντες Ἐκάβολον εὐλύραν,\\
ὔμνην δ’ Ἔκτορα κ’ Ἀνδρομάχαν θεο<ε>ικέλο[ις.	
\end{gkverse}

\begin{verse}
\ldots{} Veio o arauto \ldots{}\\
Ideu\ldots{}, veloz mensageiro:\\
``\ldots{} e do resto da Ásia \ldots{} glória imperecível.\\
Heitor e os companheiros a de vivos olhos trazem\\
de Tebas sacra e da Plácia de fontes perenes -- ela,\\
delicada Andrômaca --, nas naus, sobre o salso\\
mar. E muitos áureos braceletes e vestes\\
de púrpura fragrantes, adornos furta-cor,\\
incontáveis cálices prateados e marfins''.\\
Assim ele falou; e rápido ergueu-se o pai querido;\\
e a nova, cruzando a ampla cidade, chegou aos amigos.\\
De pronto os troianos às carruagens de boas rodas\\
atrelaram as mulas, e nelas subiu toda a multidão \\
de mulheres e junto as virgens \ldots{} tornozelos\\
mas apartadas as filhas de Príamo\\
e cavalos os homens atrelaram aos carros\\
\ldots{} moços solteiros, e por um largo espaço \\
\ldots{} os condutores das carruagens \\
\ldots{} símeis aos deuses\\
\ldots{} sacro, em multidões\\
rumou \ldots{} em direção a Ílio\\
e o aulo de doce som \ldots{} se misturou\\
e o som dos crótalos \ldots{} e então as virgens\\
cantaram uma canção sacra e chegou aos céus\\
eco divino \ldots{}\\
e em toda parte estava ao longo das ruas\\
crateras e cálices \ldots{}\\
mirra e cássia e incenso se misturavam,\\
e as mulheres soltavam alto brado, as mais velhas,\\
e todos os homens entoavam adorável e alto\\
peã invocando o Arqueiro hábil na lira,\\*
e hineavam Heitor e Andrômaca, aos deuses símeis.
\end{verse}

\medskip

{\paragraph{Comentário} Esse fragmento, cujas fontes são os \textit{Papiros de Oxirrinco} 1232 e 2076\footnote{Primeiras metades dos séculos \textsc{iii} e \textsc{ii} d.C., respectivamente.} leva"-nos à saga
de Troia e a personagens que vemos principalmente na \textit{Ilíada}: o arauto
Ideu, o herói troiano Heitor, irmão de Páris, e sua esposa, Andrômaca, o rei
troiano Príamo, o deus Apolo, protetor dos troianos na guerra. Além disso,
mostra"-nos o único exemplar de narrativa mítica epicizante
em Safo, cujos versos ditos por um narrador distanciado dividem"-se na
chegada do arauto e anúncio da vinda dos noivos Heitor e Andrômaca,\footnote{Versos
1--10.} e, depois, no espalhar dessa mensagem cidade afora, iniciando"-se com
isso a procissão festiva em honra do casal.\footnote{Versos 11--34.} A linguagem, a
atmosfera e a métrica que constroem essas etapas estão permeadas pela tradição
épico"-homérica; note"-se, por exemplo, a inescapável expressão do verso 4,
``glória imperecível'', \textit{kléos áphthiton}, síntese do ideal heroico a alcançar, a
ser preservado no canto épico, recorda a \textit{Ilíada },\footnote{Canto \textsc{ix}, verso 413.} 
pela boca de seu grande herói, Aquiles.

No discurso do arauto, repare"-se no
elogio dos noivos e no pequeno catálogo do valioso e belo dote da princesa da
Tebas asiática, Andrômaca, foco da celebração do casamento, como deve ser a noiva.
A importância da
referência ao dote é melhor
compreendida quando se pensa o casamento na vida das jovens gregas, momento
crucial que lhes atribui novo \textit{status} social -- o de esposa -- e gera
muitas mudanças: a saída da casa paterna e às vezes também da pátria; a
inserção na casa do marido e/\,ou numa realidade geográfica e cultural diversa; a
transformação da virgem em mulher que tem vida sexual e que deverá garantir a
continuidade das linhagens e administrar o espaço doméstico. A questão do dote
relaciona"-se intimamente à da legitimidade da união na Grécia antiga. Vale
lembrar que o dote da esposa se destina à prole do casal, que precisava ser
protegida no caso de perda do(s) pai(s) ou de separação. A procissão (verso
13) é etapa típica das cerimônias de casamento, e a mais retratada na iconografia grega, dada sua relevância de sanção pública à união celebrada naquelas festas, em geral muito elaboradas e
estendidas ao longo de vários dias, buscando propiciar, de todas as maneiras
possíveis, o enlace sexual que tornava legítima e consumada a união.
Vale observar o tom em crescendo da festa, com mistura de cantos, instrumentos de sopro, \textit{aulo}, e percussão, \textit{crótalo}, de líquidos -- nas ``crateras'',\footnote{\textit{Krátēres}, verso 29.} grandes jarros de larga boca, vinho misturado à água -- e arômatas  -- entre eles, o incenso,\footnote{\textit{Líbanos}, verso 30.} mencionado pela primeira vez, produto semítico --, que termina no apogeu da celebração dos noivos sob os auspícios de um canto a Apolo, o \textit{peâ} em que se integra a cidade e com o qual louva o deus do arco e flecha -- da guerra que devastará a cidade e destruirá essa boda tão alegremente celebrada --, e o deus da lira que favorece a música, os cantos, a festa do tempo presente. 
Há, pois, no fragmento, um caráter epitalâmico, mas não há elementos consistentes para
qualificá"-lo como exemplar do subgênero mélico dos epitalâmios. Por fim, não se
pode deixar de lembrar que o casal Heitor e Andrômaca protagoniza no canto \textsc{vi}\footnote{Versos 369--502.} 
da \textit{Ilíada} uma das cenas mais comoventes de despedida
entre cônjuges afetuosamente próximos um do outro e do filho, o bebê Astiánax.
Essa cena empresta ao poema épico grande densidade trágica, pois, ao acenar
para a morte certa de Heitor, acena para a ruína de Troia e de seus elos mais
frágeis: mulheres e crianças, em geral mortas ou escravizadas pelos vitoriosos.
A carga dramática intensifica"-se pela ironia trágica de que Heitor perecerá
pelas mãos de Aquiles, que já haviam matado o pai e os irmãos de Andrômaca; Aquiles cuja sombra parece pairar sobre a canção, na expressão ``glória imperecível'', que profere na \textit{Ilíada} e que o fragmento sáfico enuncia pela boca de Ideu, e na palavra final da canção que termina, no texto grego e na tradução, com o adjetivo ``aos deuses símeis'', \textit{theoeikélois}, dado aos noivos em Safo, mas ao herói grego na \textit{Ilíada}.\footnote{\textsc{i}, verso 131; \textsc{xix}, verso 155.}
Essa lembrança confere ao fragmento, tão alegre e festivo, grande tristeza e
melancolia, e entrelaça casamento e morte, duas etapas de transição na trajetória
humana, cujas cerimônias guardam muitos paralelos no imaginário grego ao qual são característicos.\footnote{Redfield, 1982, p. 188.}}



\pagebreak
\section{Fragmento 141}

\begin{gkverse}
κῆ δ’ ἀμβροσίας μὲν\\
κράτηρ ἐκέκρατ’\\
Ἔρμαις δ’ ἔλων ὄλπιν θέοισ’ ἐοινοχόαισε.\\
κῆνοι δ’ ἄρα πάντες\\
καρχάσι’ ἦχον\\
κἄλειβον· ἀράσαντο δὲ πάμπαν ἔσλα γάμβρωι
\end{gkverse}

\begin{verse}
\ldots{} e depois que uma cratera\\
de ambrosia foi misturada à água,\\
Hermes, tomando o jarro, vinhoverteu aos deuses.\\
E todos eles\\
seguravam cálices,\\
e libavam, e aguraram bons votos ao noivo \ldots{}
\end{verse}

\medskip

\paragraph{Comentário}
Ateneu (séculos \textsc{ii--iii} d.C.), \textit{Banquete dos eruditos} (10.425\,\textsc{cd}, 475\,\textsc{a}) é a principal fonte do fragmento. A cena  do  banquete nupcial pode ser da boda da Nereida Tétis e do mortal Peleu, pais de Aquiles, o grande herói grego da Guerra de Troia e da epopeia homérica \textit{Ilíada}. Ateneu afirma, antes da citação, que ``Alceu introduz Hermes como um servidor de vinho dos deuses, exatamente como Safo''. Note"-se a referência ao alimento dos deuses, ambrosia, \textit{ambrosías}, e a fusão do divino ao mortal na imagem do banquete em que o alimento dos deuses é objeto do verbo ``vinhoverteu'', \textit{eoinokhóēse},\footnote{Verso 3.} que encerra em si o líquido mais nobre dos mortais -- verbo visto já na festividade em espaço sacroerótico do Fr.\,2, em que Afrodite e néctar se combinam no \textit{vinhoverter} nas taças.


\pagebreak
\section{Fragmento 142}

\begin{gkverse}
Λάτω καὶ Νιόβα μάλα μὲν φίλαι ἦσαν ἔταιραι
\end{gkverse}

\begin{verse}
\ldots{} Leto e Níobe eram as mais caras \rlap{companheiras \ldots{}}
\end{verse}

\medskip

{\paragraph{Comentário} Preservado em Ateneu, o verso abaixo traz novamente,
agora no plural, o termo ``companheira'', \textit{étaira}, já visto no
126, desta vez ligando o \textit{eu} poético aos nomes de duas figuras míticas: a mãe de
Ártemis e Apolo, Leto, e Níobe, rainha que teve 14 filhos\footnote{Sete homens, sete
mulheres.} -- números que variam nas tradições --, e que, comparando"-se a Leto, insultou a deusa que tinha apenas dois.
Mas estes então puniram a mortal, matando"-lhe toda a prole, e ela, sucumbindo à dor que dela tudo drenou, metamorfoseou"-se em rocha. Não é possível afirmar
com certeza, mas o fragmento parece mítico, e nele as três figuras aparecem
estreitamente associadas, tanto pelo termo \textit{étairai}, quanto pelo
adjetivo que o precede.}



\pagebreak
\section{Leda e Zeus (Fr.\,166)}

\begin{gkverse}
φαῖσι δή ποτα Λήδαν ὐακίνθινον\\
<...> ὤιον εὔρην πεπυκάδμενον
\end{gkverse}

\begin{verse}
\ldots{} dizem que um dia Leda achou um ovo jacintino\\
na cor, coberto \ldots{}
\end{verse}

\medskip

{\paragraph{Comentário} Em certa tradição mítica, Zeus seduziu Leda, esposa de Tíndaro, disfarçando"-se
de cisne; o ovo por ela descoberto traria os gêmeos Castor e Polideuces, os
Dióscuros -- o primeiro mortal, e o segundo, imortal --, sendo a eles
permitido desfrutar de ambas as condições alternadamente. Na canção de Safo,
esse mito parece aludido, mas a cor do ovo contrasta de modo drástico com a usual,
e nos leva ao universo erótico das paisagens de enlace sexual, das quais o
jacinto é flor integrante. Cabe lembrar que os Dióscuros, noutras tradições,
são simplesmente filhos de Tíndaro e Leda, ou, noutras ainda, heróis de dupla
ascendência. A fonte do fragmento é Ateneu (2.57d).}




\chapter{Canções de recordação}

%\section{Memória e desejo}

\section{«\,Ode a Anactória\,» (Fr.\,16)}

\begin{gkverse}
Ο]ἰ μὲν ἰππήων στρότον οἰ δὲ πέσδων\\
οἰ δὲ νάων φαῖσ’ ἐπ[ὶ] γᾶν μέλαι[ν]αν\\
ἔ]μμεναι κάλλιστον, ἔγω δὲ κῆν’ ὄτ-\\
τω τις ἔραται·

πά]γχυ δ’ εὔμαρες σύνετον πόησαι\\
π]άντι τ[ο]ῦ̣τ’, ἀ γὰρ πόλυ περσκέ̣θ̣ο̣ι̣σ̣α\\
κ̣άλ̣λο̣ς̣ [ἀνθ]ρ̣ώπων Ἐλένα [τὸ]ν ἄνδρα\\
τ̣ὸν̣ [    αρ̣]ι̣στον

κ̣αλλ[ίποι]σ̣’ ἔβα ’ς Τροΐαν πλέοι̣[σα\\
κωὐδ[ὲ πα]ῖδος οὐδὲ φίλων το[κ]ήων\\
π̣ά[μπαν] ἐμνάσθ<η>, ἀλλὰ παράγ̣α̣γ̣’ α̣ὔταν\\

\textnormal{[\textit{versos 12--4: ilegíveis e lacunares}]}

..]μ̣ε̣ νῦν Ἀνακτορί[ας ὀ]ν̣έ̣μναι-\\
σ’ οὐ] παρεοίσας,

τᾶ]ς <κ>ε βολλοίμαν ἔρατόν τε βᾶμα\\
κἀμάρυχμα λάμπρον ἴδην προσώπω\\
ἢ τὰ Λύδων ἄρματα κἀν ὄπλοισι\\
πεσδομ]άχεντας.

			   ].μεν οὑ δύνατον γένεσθαι\\
		   ].ν ἀνθρωπ[..(.) π]εδέχην δ’ ἅρασθαι\\

\textnormal{[\textit{versos 23--31: 23--7 perdidos, 28--31 ilegíveis e lacunares}]}


     τ’ ἐξ ἀδοκή[τω.
\end{gkverse}

\chapter*{}
\section*{}
\section*{}

\begin{verse}
Uns, renque de cavalos, outros, de soldados,\\
outros, de naus, dizem ser sobre a terra negra\\
a coisa mais bela, mas eu: o que quer\\*
que se ame.

De todo fácil fazer ver a\\
todos isso, pois a que muito superou\\			
em beleza os homens, Helena, o marido, \\
o mais nobre,

tendo deixado, foi para Troia navegando,\\
até mesmo da filha e dos queridos pais\\*
de todo esquecida, mas desencaminhou-a \ldots{}

agora traz-me Anactória à lembrança,\\*
a que está ausente, \ldots{}

Seu adorável caminhar quisera ver,\\
e o brilho luminoso de seu rosto,\\
a ver dos lídios as carruagens e a armada\\*
infantaria.

\ldots{} impossível vir a ser\\
\ldots{} humano \ldots{} partilhar e rezar\\
\ldots{}\\*
e inesperadamente.
\end{verse}

\pagebreak

\paragraph{Comentário} Ao lado dos Frs. 1 e 31, o 16 é seguramente um dos mais estudados da mélica
sáfica, sobretudo por conta da exemplificação mítica da afirmação feita no
\textit{priamel} -- como
se designa a estrofe inicial, em que a uma série de
negativas coloca"-se uma afirmativa, do ponto de vista da voz poética -- e da
alegada facilidade de sua compreensão,\footnote{Versos 5--6.} que soa aos nossos ouvidos
como ironia. Isso porque não está claro se o mito recordado, da fuga de Helena, esposa 
de Menelau (não nomeado), e Páris, o príncipe troiano não nomeado, é exemplo
negativo, o que se alinharia à imagem prevalecente na poesia grega antiga, após
os poemas homéricos -- nos quais, todavia, não há a condenação de Helena senão por ela mesma --, ou se positivo, o que tornaria singular a imagem do fragmento. A
ambiguidade que já caracteriza o olhar para a personagem naqueles poemas épicos também se verifica no Fr.\,16 de Safo: Helena é o \textit{exemplum} da afirmação da estrofe inicial sobre o \textit{tò kálliston}, ``a coisa mais bela'', mas é aquela que quebrou os contratos sociais mais relevantes de seu papel de filha e esposa, pois abandonou pais e marido ``o mais nobre'', \textit{tòn\ldots{} áriston},\footnote{Verso 8.} talvez sob coação divina, se Afrodite é personagem da terceira estrofe. Tais crimes são lembrados por Alceu, o poeta lésbio contemporâneo a Safo, em similar dicção no Fr.\,283, que sobretudo condena Helena pela Guerra de Troia e seus incontáveis mortos, tal qual faz no Fr.\,42, mas enfocando outros elementos. Merece atenção, ainda, a
oposição \textit{éros}--guerra, sobre a qual se elaboram os versos, e a
referência ao rico reino oriental da Lídia, na Ásia Menor, com que estava
ligada a aristocracia da arcaica Mitilene. E a possibilidade de que haja um
viés metalinguístico relativo às escolhas temáticas da poeta da mélica arcaica.
A fonte do fragmento, \textit{Papiro de Oxirrinco} 1231, é a mesma dos Frs.~15, 17 e 22.
A jovem distante, não mais em Lesbos, parece ser uma das coreutas da associação coral de Safo, uma das \textit{parthénoi}, das moças, cuja partida pode coincidir com sua transição ao mundo do casamento. Resta a memória de sua beleza e graça, canta a canção que a celebra.




\pagebreak
\section{Fragmento 96}

\begin{gkverse}
] σαρδ.[\ldots{}]\\
     πόλ]λακι τυίδε̣ [.]ων ἔχοισα

ὠσπ.[\ldots{}].ώομεν, .[\ldots{}]..χ[\ldots{}]\\
    σε \dagger{}θεασικελαν ἀρι-\\
      γνωτα\dagger{}, σᾶι δὲ μάλιστ’ ἔχαιρε μόλπαι·̣

νῦν δὲ Λύδαισιν ἐμπρέπεται γυναί-\\
    κεσσιν ὤς ποτ’ ἀελίω\\
      δύντος ἀ βροδοδάκτυλος <σελάννα>

πάντα περ<ρ>έχοισ’ ἄστρα· φάος δ’ ἐπί-\\
    σχει θάλασσαν ἐπ’ ἀλμύραν\\
      ἴσως καὶ πολυανθέμοις ἀρούραις·

ἀ δ’ <ἐ>έρσα κάλα κέχυται τεθά-\\
    λαισι δὲ βρόδα κἄπαλ’ ἄν-\\
      θρυσκα καὶ μελίλωτος ἀνθεμώδης 

πόλλα δὲ ζαφοίταισ’ ἀγάνας ἐπι-\\
    μνάσθεισ’ Ἄτθιδος ἰμέρωι\\
      λέπταν ποι φρένα κ[.]ρ\ldots{} βόρηται·

κῆθι δ’ ἔλθην ἀμμ.[\ldots{}].ισα τό̣δ’ οὐ\\
    νωντα[\ldots{}]υστονυμ[̣..(.)] πόλυς\\
      γαρύει̣[\ldots{}(.)]αλον[̣\ldots{}(.)]τ̣ο ̣μέσσον·

ε]ὔ̣μαρ[ες μ]ὲ̣ν οὐ.α.μι θέαισι μόρ-\\
    φαν ἐπή[ρατ]ον ἐξίσω-\\
      σθ̣αι συ[..]ρ̣ο̣ς ἔχη<ι>σθα[\ldots{}].νίδηον

\textnormal{[\textit{versos 24--5: ilegíveis e lacunares}]}

      καὶ δ[.]μ{[}̣	\qquad		{]}ος Ἀφροδίτα

καμ[̣		\qquad	] νέκταρ ἔχευ’ ἀπὺ\\
    χρυσίας [		       ]ν̣αν\\
      \ldots{}.(.)]απουρ̣[\qquad		] χέρσι Πείθω

\textnormal{[\textit{versos 30--2: ilegíveis e lacunares}]}

[\qquad			]ες τὸ Γεραίστιον\\
    {[}\qquad			   {]}ν ̣φίλαι

\textnormal{[\textit{versos 35--6: ilegíveis e lacunares}]}
\end{gkverse}


\begin{verse}
\ldots{} Sárdis \ldots{}\\
muita vez para cá \ldots{} ela tendo\\
\ldots{}\\
\ldots{} qual deusa manifesta,\\*
e [ela] muito se deleitava com tua dança-canção.

Mas agora ela se sobressai entre Lídias\\
mulheres como, depois do sol\\*
posto, a dedirrósea lua

supera todas as estrelas; e sua luz se esparrama\\
por sobre o salso mar \\*
e igualmente sobre multifloridos campos.

E o orvalho é vertido em beleza, e brotam\\
as rosas e o macio \\*
cerefólio e o trevo-mel em flor.

E [ela] muito agitada de lá para cá a \\
recordar a gentil Átis com desejo;\\*
decerto frágil peito \ldots{} se consome \ldots{}

\ldots{} canta \ldots{}

\ldots{} Fácil não \ldots{} com deusas quanto à forma\\
amável rivalizar \ldots{}\\
\ldots{}\\
\ldots{} ó Afrodite\\
\ldots{}

\ldots{} o néctar vertia da\\
áurea \ldots{}\\
\ldots{} mãos \ldots{} Peitó, deusa Persuasão\\
\ldots{} o Geraístio\\
\ldots{} queridas \ldots{}
\end{verse}

\medskip

{\paragraph{Comentário} O texto, preservado no \textit{Papiro de Berlim} 9722 (século \textsc{vi}
d.C.), está bastante danificado, sendo legíveis os versos 4--17.
A cena parece envolver uma 3\textsuperscript{a} pessoa do
singular, feminina, que no passado se relacionou à 2\textsuperscript{a} do
singular em chave erótica, como indicam a referência ao prazer e à canção,\footnote{Versos 4--5.} e à beleza e ao desejo.\footnote{Versos 6--17.} No tempo presente, 
em que estão separadas essas duas personagens, \textit{ela} encontra"-se na Lídia -- Sárdis, sua cidade mais importante, talvez no v.\,1 já seja nomeada --, onde sua beleza é proeminente, como provavelmente o fora em Lesbos, quando lá viveu junto ao grupo coral de Safo. No entanto, \textit{ela} sofre, saudosa de ``Átis''\footnote{Verso 17.} -- talvez o \textit{tu} da canção --, cuja figura a recordação lhe traz, fazendo
com que se consuma seu peito em desassossego erótico. A alternância temporal do
passado ao presente introduz o longo símile central,\footnote{Versos 6--14.} alavancado
na natureza, como é característico da linguagem erótica da poesia grega antiga.
Em Safo, essa linguagem é em especial elaborada; e no fragmento, o principal
elemento natural enfocado para o canto da beleza feminina é a feminina lua\footnote{Verso 8.} que, como Eos, a Aurora, nos poemas homéricos, é inesperadamente
chamada ``dedirrósea''. Ao erguer"-se, posto o sol, tinge"-se a lua
de seus tons laranja"-avermelhados ou rosados que esquentam seu branco prateado;
estamos, pois, no início do anoitecer. No verso 9, ganha relevo a luz da lua,
vibrante, já em plena noite; e no 12, o orvalho que vem ao findar"-se a noite,
avançada já a madrugada.

Não será esta a única vez em que Safo cantará a lua,
como mostram os Fr.\,34 e 154, inseridos nesta antologia. Sua escolha é
eloquente: a lua é elemento feminino, mesmo na língua grega, e tem a função de
regular os ciclos biológicos e o ritmo das marés, interferindo na fertilidade;
aproxima"-se, pois, da esfera da mulher, percebida na Antiguidade como ligada
aos líquidos, à umidade e à natureza, já que lhe cabe a crucial
reprodução no ciclo biológico humano. Chamo a atenção, por fim, para o cenário
florido e pulsante do símile, integrado, inclusive, pelas diletas flores de
Afrodite, as rosas, marcadas na própria adjetivação da lua. O símile conecta"-se
ao restante do fragmento, na medida em que ambos tratam da beleza feminina e
estão perpassados pelo erotismo, o que é frisado na estrofe dos versos 15--17.
Portanto, pode"-se dizer que a voz do fragmento insere sua canção no universo de
Afrodite, como no Fr.\,2.
E, referindo a própria coralidade, talvez cante a memória de uma coreuta que se foi, como seria o caso da Anactória do Fr.\,16, e da não nomeada jovem cuja despedida é recordada adiante, no Fr.\,94.
Os versos finais ficam cada vez mais precários, mas interessa notar que trazem o tema da beleza e do desejável, da festividade -- que remete ao Fr.\,2 -- e da presença divina de deusas da sedução, e a menção de um santuário de Posêidon na Eubeia (continente grego), obscura para nós. Salvo por este elemento, dos demais emana o mundo mais caracteristicamente sáfico, em que estão imersas a poeta, seu coro e as canções.}

\pagebreak
%\section{Separação (e solidão)}
\section{Fragmento 23}

\begin{gkverse}
]ἕρωτος  ἠλπ[\\
		       ]\\
		   αν]τιον εἰσίδωσ[\\
		       ] ’Ερμιόνα τεαυ[τα\\
	   ] ξάνθαι δ’ ’Ελέναι σ’ ἑίσ[κ]ην\\
	]κες \\
	].ις θνάταις, τόδε ἴσ[θι] τὰι σᾶι\\
	]παίσαν κέ με τὰν μερίμναν\\
	]λαισ’ ἀντιδ[..][.]α̣θοις δὲ̣\\
     ]\\
				]τας ὄχθοις\\
			          ]ταιν\\
			      παν]νυχίσ[δ]ην
\end{gkverse}

\begin{verse}
\ldots{} do desejo \ldots{}\\
\ldots{}\\
\ldots{} face a face contemplo \ldots{}\\
\ldots{} Hermíone tal \ldots{}\\
e comparar-te à loira Helena \ldots{}\\
\ldots{}\\
\ldots{} às mortais; e isto sabe, em tua\\
\ldots{} de todos os meus anseios\\
\ldots{}\\
\ldots{} nas margens \ldots{}\\
\ldots{}\\*
\ldots{} celebrar um festival noturno \ldots{} 
\end{verse}

\pagebreak

{\paragraph{Comentário} Preservado no rolo \textit{Papiro de Oxirrinco} 1231, que contém o Fr.\,15 e outros vistos, o Fr.\,23 traz elementos notáveis: a bela Helena -- que antes surge no Fr.\,16 -- e sua filha Hermíone, e o erotismo sobretudo nas menções ao desejo, ao olhar, à beleza de Helena -- projetada pelo loiro de seus cabelos, mesmo tom do ouro, o metal mais valioso, e do sol, astro essencial à vida -- e às ansiedades. Na última linha, \textit{pannykhísdēn} refere o celebrar de festa noturna, tal como outra forma do mesmo verbo no Fr.\,30, adiante. Não há propriamente separação, aqui, mas a fala à \textit{persona} feminina, comparada às personagens míticas decerto pela beleza compartilhada, desenha"-se como íntima, pessoal, e talvez em separado de um conjunto. Este seria o de celebrantes da boda, contexto provável também do Fr.\,30, em vista do rito noturno ritualístico, \textit{pannykhís}. Como veremos no Fr.\,111, a comparação dos noivos a figuras divinas ou míticas é frequente na dicção das canções relacionadas ao mundo do casamento, no qual, aliás, vimos Safo inserir a cena mítica troiana do Fr.\,44, do enlace de Heitor e Andrômaca, e do qual já algo disse no comentário ao Fr.\,112, e mais direi, chegando aos epitalâmios.}

\pagebreak
\section{Fragmento 49}

\begin{gkverse}
Ἠράμαν μὲν ἔγω σέθεν, Ἄτθι, πάλαι ποτά

\hspace*{27mm}\adforn{47}

σμίκρα μοι πάις ἔμμεν’ ἐφαίνεο κἄχαρις.
\end{gkverse}

\begin{verse}
Eu te desejei, Átis, há tempos, um dia \ldots{}

\hspace*{27mm}\adforn{47}

\ldots{} criança mirrada e sem graça me parecias ser \ldots{}
\end{verse}

\medskip

{\paragraph{Comentário} De novo, Átis é personagem de uma canção de erótica tonalidade, da qual temos o verso inicial preservado em Heféstion (7.7), fonte do Fr.\,102, entre outros. Recorda a voz poética sua paixão por Átis, exaurida no presente, razão pela qual estão ambas as personagens separadas. Depois, num outro verso, que não sabemos em que ponto da mesma canção seria entoado, Plutarco (séculos \textsc{i}--\textsc{ii} d.C., \textit{Diálogos sobre o amor} 751\,\textsc{d}),
sua fonte, conta que, ``falando a uma menina demasiado nova para o
casamento, Safo lhe diz'' palavras de clara reprovação, por conta de sua forma
física. }



\pagebreak
\section{Fragmento 88}

\begin{gkverse}
\textnormal{[\textit{versos 1--4 e 28: ilegíveis e lacunares}]}

].θέλοις· οὐδυ̣[\\
].άσδοισ’ ὀλιγα[\\
].ένα φέρεσθα[ι

  ].φι̣α̣ τισ̣...[\\
  ].δ’ ἄδιον εἰσορ[\\
ο]ἶσθα καὔτα·

λέ]λ̣αθ’ ἀλλονιά[\\
     ].αν τι̣ραδ[\\
     ]α̣ί̣ τις εἴποι

].σαν· ἔγω τε γαρ[\\
]μ̣’ ἆς κεν ἔνηι μ’[\\
  ]α̣ι μελήσην·

]φίλα φαῖμ’ ἐχύρα γέ[νεσθαι\\
   ]ενα[.]αις· ἀτ̣[\\
          ]δ’ ὀνίαρ̣[ο]ς̣ [

        ]. πίκρος ὔμ[\\
        ]τα.θᾶδ̣[\\
        ].α τόδε δ’ ἴσ̣[θ

       ]ὤττι σ’ ἐ.[\\
]α φιλήσω[\\
]τ̣ω τι̣ λο[

]σσον γὰρ .[\\
          ]σ̣θαι βελέω[ν
\end{gkverse}

\pagebreak
\begin{verse}
\ldots{} eu desejaria \ldots{}\\
\ldots{}\\
\ldots{} carregar \ldots{}\\
\ldots{}\\
\ldots{} mais doce de ver \ldots{}\\
\ldots{} sabes tu mesma;

\ldots{} esqueceu \ldots{}\\ 
\ldots{}\\
\ldots{} alguém diria

\ldots{} pois eu \ldots{}\\
amarei \ldots{} até que haja em mim \ldots{}\\
\ldots{} será objeto de cuidado;\\
\ldots{} digo ser amiga confiável \\
\ldots{}\\
\ldots{} doloroso \ldots{}\\
\ldots{} amargo \ldots{}\\
\ldots{}\\
\ldots{} isto sabe\\
\ldots{} amarei \ldots{}\\
\ldots{}\\
\ldots{} de flechas\ldots{}
\end{verse}

\medskip

{\paragraph{Comentário} Tendo por fonte o \textit{Papiro de Oxirrinco} 2290 (fim do século \textsc{ii} ou início do \textsc{iii} d.C.), o fragmento traz a \textit{persona} a expressar volição e falar a um \textit{tu} feminino, afirmando que tipo de amizade é capaz de oferecer, no reiterado uso do verbo \textit{phileîn}, ``amar'', e do termo ``amiga'', \textit{phíla}. Algo de prazeroso se mescla a algo amargo e dolorido; e no último verso pode haver 
referência à caçadora deusa virgem, Ártemis.\footnote{Campbell, 1994, p. 113.} Há insistência, por fim, em que a destinatária saiba de algo que não se revela, e a menção do esquecimento. Difícil somar as partes, mas há na base dos versos  uma relação de amizade entre figuras femininas -- uma das quais, a \textit{persona} --, e talvez uma ruptura.}




\pagebreak
\section{Fragmento 94}

\begin{gkverse}
τεθνάκην δ’ ἀδόλως θέλω·\\
ἄ με ψισδομένα κατελίμπανεν

πόλλα καὶ τόδ’ ἔειπέ̣  [μοι\\
ὤιμ’ ὠς δεῖνα πεπ[όνθ]αμεν,\\
Ψάπφ’, ἦ μάν σ’ ἀέκοισ’ ἀπυλιμπάνω.

τὰν δ’ ἔγω τάδ’ ἀμειβόμαν·\\
χαίροισ’ ἔρχεο κἄμεθεν\\
μέμναισ’, οἶσθα γὰρ ὤς σε πεδήπομεν·

αἰ δὲ μή, ἀλλά σ’ ἔγω θέλω\\
ὄμναισαι [...(.)].[...(.)].ε̣αι\\
ὀ̣σ̣[\qquad     - 10 -\qquad   ] καὶ κάλ’ ἐπάσχομεν·

πό̣[λλοις γὰρ στεφάν]οις ἴων\\
καὶ βρ[όδων ...]κ̣ίων τ’ ὔμοι\\
κα..[ - 7 -\quad   ] πὰρ ἔμοι π<ε>ρεθήκα<ο>

καὶ πό̣λλαις ὐπαθύμιδας\\
πλέκ[ταις ἀμφ’ ἀπάλαι δέραι\\
ἀνθέων ἐ[  -  6  -  ] πεποημμέναις

καὶ π\ldots{}[\qquad      ]. μύρωι\\
βρενθείωι ̣[\qquad      ]ρ̣υ[...]ν\\
ἐξαλ<ε>ίψαο κα̣[ὶ βασ]ι̣ληίωι

καὶ στρώμν[αν ἐ] πὶ μολθάκαν\\
ἀπάλαν παρ[̣ \quad   ]ο̣ν̣ων\\ 
ἐξίης πόθο̣[ν \quad    ]νίδων

κωὔτε τισ[\quad       οὔ]τ̣ε̣ τι\\
ἶρον οὐδ’ ὐ[ \quad        ]\\
ἔπλετ’ ὄππ̣[οθεν ἄμ]μες ἀπέσκομεν,

οὐκ ἄλσος .[\qquad	       ].ρος

\textnormal{[\textit{versos 28--9: ilegíveis e lacunares}]}
\end{gkverse}

\pagebreak

\section*{}
\begin{verse}
\ldots{} morta, honestamente, quero estar;\\*
ela me deixava chorando

muito, e isto me disse:\\*
``Ah!, coisas terríveis sofremos,\\*
Ó Safo, e, em verdade, contrariada te deixo''.

E a ela isto respondi:\\*
``Alegra-te, vai, e de mim\\*
te recorda, pois sabes quanto cuidamos de ti;

mas se não, quero te\\*
lembrar \ldots{}\\*
\ldots{} e coisas belas experimentamos;

pois com muitas guirlandas de violetas\\*
e de rosas \ldots{} juntas\\*
\ldots{} ao meu lado puseste,			

e muitas olentes grinaldas\\*
trançadas em torno do tenro colo, \\*
de flores \ldots{} feitas;

e\ldots{} com perfume\\*
de flores \ldots{}\\*
digno de rainha, te ungiste,

e sobre o leito macio\\*
tenra \ldots{}\\*
saciavas [teu] desejo \ldots{}

Não havia \ldots{} nem algum\\*
santuário, nem \ldots{}\\*
do qual estivéssemos ausentes,

nem bosque \ldots{}
\end{verse}

\medskip

{\paragraph{Comentário} Preservado no mesmo rolo papiráceo do Fr.\,96, o 94 é fortemente dramático ao
reencenar a separação entre o \textit{eu} -- ``Safo'' -- e \textit{ela}, revivida em
recordação detalhada, de tons eróticos cada vez mais intensos na
gradação corpo, perfumes, adornos, leito, e com participação no rito,
ao fim ainda legível do fragmento em que se destaca a coralidade própria à natureza do grupo de \textit{parthénoi}, de meninas virgens, liderado por Safo. Grupo em que dança, canto, atividades de culto e corais, a preparação para o \textit{gámos}, ``casamento'', e para a atuação no mundo do sexo faziam parte do que podemos chamar de \textit{paideía}, ``formação'' feminina. Grupo do qual já se acham distantes, na Lídia, as \textit{parthénoi} dos Frs.\,16 (Anactória) e 96, e a \textit{parthénos} cuja despedida a canção relata em chave consolatória -- o consolo a quem partiu e a quem ficou na memória feliz da coralidade compartilhada, da beleza, do prazer e do convívio com as amigas, de uma vida que, ao casar"-se e tornar"-se mulher, \textit{gynḗ}, deixa para trás.

Merece atenção a frequência do convite ao passado em Safo nesses Frs. 16, 94 e 96, o qual ``tem valor ideológico: nos momentos de fratura'', quando uma das jovens se separa do grupo ao mudar de \textit{status}, ``a poeta exorta a criar uma ponte entre passado e futuro, que mantenha viva a relação de companheirismo através da recordação. Esta operação tem um duplo significado: de um lado, consola quem parte e quem fica; de outro, reafirma os princípios constitutivos da comunidade, recordando"-lhe os valores e as atividades''.\footnote{Caciagli, 2009, p. 78.}
Ressalto, enfim, a força do tema da memória (e do esquecimento), marcado já no Fr.\,1, e que mesmo noutros fragmentos aqui não incluídos, como o 24 e 25, estaria presente, se aceitarmos a possibilidade de que, em seus respectivos e ilegíveis textos, haja formas verbais relativas ao lembrar ou esquecer.
}





\pagebreak
%\section{Imortalidade}
\section{Fragmento 55}

\begin{gkverse}
κατθάνοισα δὲ κείσηι οὐδέ ποτα μναμοσύνα σέθεν\\
ἔσσετ’ οὐδὲ \dagger{}ποκ’\dagger{} ὔστερον· οὐ γὰρ πεδέχηις βρόδων\\
τὼν ἐκ Πιερίας, ἀλλ’ ἀφάνης κἀν Ἀίδα δόμωι\\
φοιτάσηις πεδ’ ἀμαύρων νεκύων ἐκπεποταμένα.
\end{gkverse}

\begin{verse}
Morta jazerás, nem memória alguma futura\\
de ti haverá, nem desejo, pois não partilhas das rosas\\
de Piéria; mas invisível na casa de Hades\\*
vaguearás esvoaçada entre vagos corpos \ldots{}
\end{verse}

\medskip

{\paragraph{Comentário} Preservado na \textit{Antologia} (3.4.12) de Estobeu (século \textsc{v} d.C.), o fragmento traz um dos muitos momentos em que os poetas refletem em seus versos
sobre o poetar e a própria poesia, que vai se configurar no imaginário grego
como caminho para a imortalidade do nome.\footnote{Explorei o tema na poesia arcaica (Ragusa, 2018, pp. 143--152).} A linguagem é de ataque, de
invectiva, e dirigida a um \textit{tu} feminino que pensa ser (hábil) poeta, mas
não o é. Por isso, sua dupla morte ao descer ao mundo, reino de
Hades, tornando"-se não mais alcançável sua figura aos olhos dos vivos, e
apagada sua existência da memória destes.}



\pagebreak
\section{Fragmento 129}

\begin{gkverse}
ἔμεθεν δ’ ἔχηισθα λάθαν

\hspace*{17mm}\adforn{47}

ἤ τιν’ ἄλλον ἀνθρώπων ἔμεθεν φίληισθα

\end{gkverse}

\begin{verse}
\ldots{} e de mim memória não há \ldots{}

\hspace*{17mm}\adforn{47}

\ldots{} ou algum outro dos homens ame, além de mim \ldots{}
\end{verse}

\medskip

\paragraph{Comentário}
Citado no já referido tratado sobre pronomes de Apolônio, fonte do Fr.\,32, o 129 traz dois versos cuja conexão nos escapa, com a negação da memória e do desejo.


\section{Fragmento 147}

\begin{gkverse}
μνάσεσθαί τινα φα<ῖ>μι \dagger{}καὶ ἔτερον\dagger{} ἀμμέων
\end{gkverse}

\begin{verse}
\ldots{} digo, lembrar-se-á de nós alguém no porvir \ldots{}
\end{verse}

\medskip

{\paragraph{Comentário} Preservado em Dio Crisóstemo (séculos \textsc{i"-ii} d.C),\footnote{\textit{Oração} (37.47).} o fragmento talvez expresse a confiança na imortalidade por parte da poeta, em sua \textit{persona} dramática, em canção.}


\chapter{Desejos}

\section{Fragmento 95}

\begin{gkverse}
\textnormal{[\textit{versos 1--3 e 14--6: ilegíveis e lacunares}]}

Γογγυλα.[

ἦ τι σᾶμ’ ἐθε.[\\
παισι μάλιστα.[\\
μας γ’ εἴσηλθ’ ἐπ.[

εἶπον· ὦ δέσποτ’, ἐπ.\\
ο]ὐ μὰ γὰρ μάκαιραν̣ [\\
ο]ὐδὲν ἄδομ’ ἔπαρθ’ ἀγα[

κατθάνην δ’ ἴμερός τις [ἔχει με καὶ\\
λωτίνοις δροσόεντας [ὄ-\\
χ[θ]οις ἴδην Ἀχερ[
\end{gkverse}

\pagebreak
\begin{verse}
\ldots{} Gongila \ldots{}

\ldots{} decerto um sinal \ldots{}

a todos e sobretudo \ldots{}\\
\ldots{} veio \ldots{}

\ldots{} disse: ``Ó senhor \ldots{}\\*
pois pela venturosa \ldots{}\\*
não me deleito em estar agitada \ldots{}

um desejo de morrer me toma, e\\*
com lótus orvalhadas as\\*
margens do Aqueronte ver \ldots{}
\end{verse}

\medskip

{\paragraph{Comentário} Um nome feminino e um forte desejo de rendição à morte marcam esse fragmento do
\textit{Papiro de Berlim }9722, também fonte dos Frs.~94 e 96. Ressalto a imagem do
rio do mundo dos mortos, o Aqueronte mencionado no Fr.\,65, cujas margens são usualmente pintadas
como densas de flores de lótus, flores desde a tradição egípcia associadas à morte. O \textit{eu} feminino parece dirigir"-se a um deus,
talvez Hermes, condutor que guia os mortos ao Hades, mensageiro que atravessa todas as fronteiras.}

\pagebreak
\section{Fragmento 121}

\begin{gkverse}
ἀλλ’ ἔων φίλος ἄμμι λέχος ἄρνυσο νεώτερον·\\
οὐ γὰρ τλάσομ’ ἔγω σύν <τ’> οἴκην ἔσσα γεραιτέρα
\end{gkverse}

\begin{verse}
\ldots{} mas grato me sendo, toma o leito de [outra] mais nova,\\
pois não suportarei ser a mais velha numa \rlap{parceria \ldots{}}
\end{verse}

\medskip

{\paragraph{Comentário} Preservado em Estobeu (4.22.112), como o Fr.\,55, o 121 trabalha com a ideia da
reciprocidade, importante ao imaginário grego; o \textit{eu} feminino fala a um \textit{tu}
masculino, reclamando a necessidade do gesto de gratidão e de equiparação de
idade, em se tratando de alianças, talvez a do casamento.}

\section{Fragmento 126}

\begin{gkverse}
δαύοισ(’) ἀπάλας ἐτα<ί>ρας ἐν στήθεσιν
\end{gkverse}

\begin{verse}
\ldots{} que adormeças no peito macio de tua companheira \ldots{} 
\end{verse}

\medskip

{\paragraph{Comentário} A fonte é o \textit{Etimológico genuíno }(século \textsc{ix}). O termo empregado para
``companheira'', \textit{etaíras}, usado no Fr.\,142, marca uma amizade e aliança muito estreita entre o \textit{tu}
e \textit{ela}, e indica serem coetâneas essas personagens. O modo como o desejo é expresso parece acrescentar a essa relação o
ingrediente erótico, que de modo algum seria estranho ao convívio na associação coral feminina das \textit{parthénoi}, as virgens. Se o \textit{tu} referir uma delas, o íntimo convívio compartilhado pode estar em cena, como está no Fr.\,94 e no catálogo de belezas desfrutadas na coralidade, que lá se desenha.}




\pagebreak
\section{Fragmento 138}

\begin{gkverse}
στᾶθι \dagger{}κἄντα\dagger{} φίλος\\
καὶ τὰν ἐπ’ ὄσσοισ’ ὀμπέτασον χάριν
\end{gkverse}

\begin{verse}
\ldots{} posta-te [diante de mim], se és meu amigo,\\
e estende afora a graça de teus olhos \ldots{}
\end{verse}

\medskip

{\paragraph{Comentário} O foco nos olhos é comum quando se canta a beleza; nesse canto, o \textit{eu} chama o
\textit{tu} de \textit{phílos }, ``amigo'', termo que pode ou não ter conotação
erótica -- a primeira possibilidade sendo, creio, mais provável, inclusive
porque a fonte do fragmento, Ateneu (13.564\,\textsc{d}), que preservou também o Fr.\,141, afirma:
``Safo diz ao homem que é excessivamente admirado por sua forma e tido como belo''.}



\chapter{Dores de amor}

\section[«\,Phaínetaí moi\ldots{}\,» (Fr.\,31)]{«\,Phaínetaí moi\ldots{}\,» (Fr.\,31)}

\begin{gkverse}
Φαίνεταί μοι κῆνος ἴσος θέοισιν\\
ἔμμεν’ ὤνηρ, ὄττις ἐνάντιός τοι\\
ἰσδάνει καὶ πλάσιον ἆδυ φωνεί-\\
σας ὐπακούει

καὶ γελαίσας ἰμέροεν, τό μ’ ἦ μὰν\\
καρδίαν ἐν στήθεσιν ἐπτόαισεν·\\
ὠς γὰρ <ἔς> σ’ ἴδω βρόχε’ ὤς με φώναι-\\
σ’ οὐδὲν ἔτ’ εἴκει,

ἀλλα \dagger{}καμ\dagger{} μὲν γλῶσσα \dagger{}ἔαγε\dagger{}, λέπτον\\
δ’ αὔτικα χρῶι πῦρ ὐπαδεδρόμηκεν,\\
ὀππάτεσσι δ’ οὐδὲν ὄρημμ’, ἐπιβρόμ-\\
μεισι δ’ ἄκουαι,

\dagger{}έκαδε\dagger{} μ’ ἴδρως ψῦχρος κακχέεται, τρόμος δὲ\\
παῖσαν ἄγρει, χλωροτέρα δὲ ποίας\\
ἔμμι, τεθνάκην δ’ ὀλίγω ’πιδεύης\\
φαίνομ’ ἔμ’ αὔτ[̣αι.

ἀλλὰ πὰν τόλματον, ἐπεὶ \dagger{}καὶ πένητα\dagger{}
\end{gkverse}

\pagebreak
\begin{verse}
Parece-me ser par dos deuses ele,\\
o homem, que oposto a ti\\
senta e de perto tua doce\\*
fala escuta,

e tua risada atraente. Isso, certo,\\
no peito atordoa meu coração;\\
pois quando te vejo por um instante, então\\*
falar não posso mais,

mas se quebra minha língua, e ligeiro\\
fogo de pronto corre sob minha pele,\\
e nada veem meus olhos, e\\*
zumbem meus ouvidos,

e água escorre de mim, e um tremor\\
de todo me toma, e mais verde que a relva\\
estou, e bem perto de estar morta\\
pareço eu mesma.

Mas tudo é suportável, se mesmo um pobre \rlap{homem \ldots{}}
\end{verse}

\medskip

{\paragraph{Comentário} O Fr.\,31 ou \textit{Phaínetaí moi}, ``Parece"-me\ldots{}'', tem por fonte
principal o famoso tratado \textit{Do sublime}.\footnote{10.1--3, século \textsc{i} d.C.?, \textit{Longino}.} Ao
explicar as maneiras de um texto alcançar a grandeza, seu autor de incerta identidade menciona os
``pensamentos elevados'';\footnote{ Cito as traduções do volume Hirata,
F. (trad., introdução, notas). \textit{Longino.} \textit{Do Sublime.} São
Paulo: Martins Fontes, 1996.} no quadro destes, cita o fragmento de Safo,
em que louva a ``escolha dos motivos'' e a ``concentração dos
motivos escolhidos''. ``Por exemplo Safo: as afecções consecutivas ao delírio
amoroso, a cada vez, ela as apreende como elas se apresentam sucessivamente e
na sua própria verdade. Mas onde mostra ela sua força? Quando ela é capaz, a
uma vez, de escolher e de ligar o que há de mais agudo e de mais
intenso nessas afecções''. A citação é sucedida por palavras que equiparam Safo
a Homero, ``o Poeta'': 
\pagebreak
\begin{quote}
Não admiras como, no mesmo momento, ela
procura a alma, o corpo, o ouvido, a língua, a visão, a pele, como se tudo isso
não lhe pertencesse e fugisse dela; e, sob efeitos opostos, ao mesmo tempo ela
tem frio e calor, ela delira e raciocina (e ela está, de fato, seja
aterrorizada, seja quase morta)? Se bem que não é uma paixão que se mostra
nela, mas um concurso de paixões! Todo esse gênero de acontecimentos cerca os
amantes, mas, como eu disse, a maneira de agrupá"-los, para relacioná"-los num
mesmo lugar, realiza a obra de arte. Da mesma maneira, a meu ver, para as
tempestades o Poeta escolhe as mais terríveis das consequências.
\end{quote}

A cena retratada eroticamente produz uma triangulação na qual o olhar do \textit{eu}
feminino contempla primeiramente um homem que se porta como audiência de um
\textit{tu} feminino, para depois concentrar"-se na contemplação dessa
personagem, que provoca sua excitação -- na imagem recorrente do peito que se agita ou,
literalmente, voa, como se vê no Fr.\,22 -- e crescente dominação erótica, ou
seja, fragmentação do corpo e da mente que leva à morte. Há um notável eco
metalinguístico no fragmento, em que a perda da voz, central na patologia
erótica, é crucial para poetas de tradição oral como Safo, que não existe sem
ela, e para a representação dramática da resposta à cena, que só se realiza a
partir justamente dessa perda. Temos o início do fragmento, mas não estamos
seguros de seu fim. Seu texto foi reelaborado por Catulo (século \textsc{i}
a.C.) no \textit{Poema 51},\footnote{ Cito"-o, em tradução de Oliva (1996):
\textit{Ele parece"-me ser par de um deus,/ Ele, se é fás dizer, supera os
deuses,/ Esse que todo atento o tempo todo/ Contempla e ouve"-te/ doce rir, o
que pobre de mim todo/ sentido rouba"-me, pois uma vez/ que te vi, Lésbia,
nada em mim sobrou/ De voz na boca/ Mas torpece"-me a língua e leve os membros
/ Uma chama percorre e de seu som/ Os ouvidos tintinam, gêmea noite/ cega"-me
os olhos./ O ócio, Catulo, te faz tanto mal./ No ócio tu exultas, tu vibras
demais./ Ócio já reis e já ricas cidades/ Antes perdeu.}} praticamente uma
tradução da canção sáfica.
Nos versos do Fr.\,31, impressiona com que habilidade Safo cria uma dimensão de intimidade; neles, como noutros, a poeta revela"-se ``a maior mestra em pseudointimidade''.\footnote{Scodel, 1996, p. 77.}}



\pagebreak
\section{Fragmento 36}

\begin{gkverse}
καὶ ποθήω καὶ μάομαι\ldots{}
\end{gkverse}

\begin{verse}
\ldots{} e desejo e enlouqueço \ldots{}
\end{verse}

\medskip

{\paragraph{Comentário} Nesse parco fragmento preservado, como o Fr.\,126, no \textit{Etimológico genuíno}, combina"-se um binômio recorrente na poesia erótica grega:
paixão e loucura. Vale anotar: a loucura erótica, provocada por Afrodite, será
tratada como um dos tipos de loucura definidos por Platão
no diálogo \textit{Fedro}, por exemplo.}


\section{Fragmento 48}

\begin{gkverse}
ἦλθες, \dagger{}καὶ\dagger{} ἐπόησας, ἔγω δέ σ’ ἐμαιόμαν,
ὂν δ’ ἔψυξας ἔμαν φρένα καιομέναν πόθωι.
\end{gkverse}

\begin{verse}
\ldots{} vieste, e eu ansiava por ti -- \\*
me esfriaste o peito que queimava com desejo \ldots{}
\end{verse}

\medskip

{\paragraph{Comentário} Preservado na \textit{Carta a Jâmblico} (183), nome do filósofo neoplatonista do
século \textsc{iv} d.C., de autoria do imperador romano Juliano, seu contemporâneo, o
breve fragmento elabora"-se a partir de outro motivo comum na concepção de
\textit{éros}: o de que paixão é fogo; logo, sua satisfação é como o apagar de
um incêndio, como diz o \textit{eu} feminino ao \textit{tu}, abaixo.}

\section{Fragmento 51}

\begin{gkverse}
οὐκ οἶδ’ ὄττι θέω· δύο μοι τὰ νοήματα
\end{gkverse}

\begin{verse}
\ldots{} não sei que faço: duas as minhas mentes \ldots{}
\end{verse}

\medskip

{\paragraph{Comentário} Como no Fr.\,36, temos aqui, possivelmente, o binômio paixão"-loucura, que fratura o amador. A
fonte do fragmento é o tratado \textit{Das negativas} (23), de Crísipo (filósofo
estoico, século \textsc{iii} a.C.).}   


\chapter{Sono}

\section{Fragmento 46}

\begin{gkverse}
ἔγω δ’ ἐπὶ μολθάκαν\\
τύλαν <κασ>πολέω \dagger{}μέλεα· κἂν μὲν τετύλαγκας ἀσπόλεα\dagger{}
\end{gkverse}

\begin{verse}
\ldots{} mas eu, sobre macias\\
almofadas quero deitar meus membros \ldots{}
\end{verse}

\medskip

{\paragraph{Comentário} O que de legível há no texto corrompido do fragmento citado por Herodiano (fins do século \textsc{ii} d.C.), no tratado \textit{Sobre palavras anômalas} (\textit{beta} 39), que o atribui ao Livro \textsc{ii} da compilação alexandrina da obra de Safo, e com ele ilustra a palavra \textit{týlē}, ``almofada'', em que a \textit{persona} talvez repouse o corpo. Mesmo em tão exíguo texto vemos a dimensão profundamente sensorial, tátil aqui, da dicção sáfica, cuja força está evidenciada nas canções.}



\pagebreak
\section{Fragmento 63}

\begin{gkverse}
Ὄνοιρε μελαινα[\\
φ[ο]ίταις ὄτα τ’ ὔπνος [

γλύκυς̣ θ̣[έ]ο̣ς, ἦ δεῖν’ ὀνίας μ[\\
ζὰ χῶρις ἔχην τὰν δυναμ[

ἔλπις δέ μ’ ἔχει μὴ πεδέχη[ν\\
μηδὲν μακάρων ἐλ̣[

ο̣ὐ̣ γάρ κ’ ἔον οὔτω[..́   \\
ἀθύρματα κα.[

γένοιτο δέ μοι[
\end{gkverse}

\begin{verse}
Ó Ôneiros, negra \ldots{}\\
vagueias, quando quer que o sono \ldots{}

doce deus, de fato terrivelmente angústias \ldots{}\\
manter \ldots{} apartada [tua] potência \ldots{}

mas a expectação me sustém, de não partilhar \ldots{}\\
nem \ldots{}  dos venturosos \ldots{}

pois eu não seria assim \ldots{}\\
berloques \ldots{}

viria a ser a mim \ldots{}
\end{verse}

\medskip

{\paragraph{Comentário} Tendo por fonte o \textit{Papiro de Oxirrinco} 1787, o mesmo do Fr.\,65, o 63 tem seu início preservado, com invocação em prece a Ôneiros, o deus Sonho, que move o sonho, \textit{ónar}, referido antes no Fr.\,134, e se associa ao sono ou ao deus que é seu soberano -- \textit{Hýpnos} é o nome de ambos. Este deus ou aquele deus é qualificado como \textit{glýkys}. ``doce'',\footnote{Verso 3.} e perturbações emocionais e expectativa compõem a cena que não podemos recompor.}



\pagebreak
\section{Fragmento 149}

\begin{gkverse}
ὄτα πάννυχος ἄσφι κατάγρει
\end{gkverse}

\begin{verse}
\ldots{} quando [o sono] de noite afora captura os seus [olhos] \ldots{}
\end{verse}

\medskip

{\paragraph{Comentário} Citado pelo gramático Apolônio Díscolo na discussão sobre pronomes, mesma fonte do Fr.\,32, o 149 traz apenas um verso que de novo canta o que se desenrola no correr da noite, \textit{pánnykhos}, como antes o Fr.\,23, e outros adiante.}


\section{Fragmento 151}

\begin{gkverse}
ὀφθάλμοις δὲ μέλαις νύκτος ἄωρος
\end{gkverse}

\begin{verse}
\ldots{} sobre os olhos sono da negra noite \ldots{}
\end{verse}

\medskip

 {\paragraph{Comentário} Citado no \textit{Etimológico Magno} (século \textsc{xii}), no verbete do termo sinônimo a \textit{hýpnos}, \textit{áōros}, o fragmento tem só um verso, sobre o sono, como o anterior.}


\section{Fragmento 168\textsc{b}}

\begin{gkverse}
Δέδυκε μὲν ἀ σελάννα\\
καὶ Πληΐαδες· μέσαι δὲ\\
νύκτες, παρὰ δ’ ἔρχετ’ ὤρα,\\
ἔγω δὲ μόνα κατεύδω.
\end{gkverse}

\begin{verse}
Imergiu a lua,\\
também as Plêiades; é\\
meia-noite, vai-se o tempo,\\
e eu sozinha durmo \ldots{}
\end{verse}

\medskip

{\paragraph{Comentário} Heféstion (11.5), referido já no comentário ao Fr.\,102, é a fonte principal do 168\,\textsc{b}, em que a natureza enquadra a imagem da solidão da voz poética, talvez devido a uma separação erótica.}


\chapter{Viagem}

\section{Fragmento 20}

\begin{gkverse}
\textnormal{[\textit{versos 1 e 22--5: ilegíveis e lacunares}]}

] ε, γάνος δὲ και̣..[\\
τ]ύχαι σὺν ἔσλαι\\
         λί]μενος κρέτησαι\\
  γ]ᾶς μελαίνας
    ]\\
    ]έλοισι ναῦται\\
  ]μ̣εγάλαις ἀήται[ς\\
  ]α κἀπὶ χέρσω\\
    ]\\
  .́ ]μοθεν πλέοι.\\
  ]δε τὰ φόρτι’ εἰκ[

\textnormal{[\textit{versos 14--9: ilegíveis e lacunares}]}

]ι̣ν ἔργα\\
] χέρσω [
\end{gkverse}

\chapter*{}
\section*{}


\begin{verse}
\ldots{} brilho\\
\ldots{}\\
\ldots{} com boa sorte \ldots{}\\
\ldots{} porto alcançar \ldots{} \\
\ldots{} da terra negra\\
\ldots{} \\
\ldots{} nautas\\
\ldots{} grandes ventos \ldots{}\\
e em terra firme\\
\ldots{}\\
\ldots{} navegar \ldots{}\\
\ldots{} carga \ldots{}\\
\ldots{}\\
\ldots{} trabalhos \ldots{}\\
\ldots{}\\
\ldots{} terra firme \ldots{}
\end{verse}

\medskip

{\paragraph{Comentário} Tendo por fonte o \textit{Papiro de Oxirrinco} 1231, o mesmo do Fr.\,15, o 20 traz uma viagem marinha difícil, ou usa essa imagem para falar das perturbações políticas da pólis, no motivo conhecido como da ``nau do Estado''. Mas tão precário texto não permite navegar num ou noutro sentido com segurança.}

\chapter{Imagens da natureza}

\paragraph{\textsc{nota introdutória}}
Como observei ao comentar o Fr.\,96 de Safo, é notável em sua mélica o trabalho
da linguagem com imagens da natureza, de grande presença e força não raro
metafóricas.

\section{Fragmento 34}

\begin{gkverse}
ἄστερες μὲν ἀμφὶ κάλαν σελάνναν\\
ἂψ ἀπυκρύπτοισι φάεννον εἶδος,\\
ὄπποτα πλήθοισα μάλιστα λάμπη\\
γᾶν\ldots{}\\
\hspace*{23mm}\adforn{47}\\
ἀργυρία
\end{gkverse}

\begin{verse}
\ldots{} e as estrelas, em volta da bela lua,\\
de novo ocultam sua luzidia forma,\\
quando plena ao máximo ilumina\\
a terra \ldots{}\\
\hspace*{23mm}\adforn{47}\\
\ldots{} argêntea
\end{verse}

\medskip

{\paragraph{Comentário} A principal fonte desse fragmento concentrado no corpo celeste feminino da lua, antes cantada no Fr.\,96, é
um comentário de Eustácio (bispo de Tessalônica, século \textsc{xii} d.C.) à
\textit{Ilíada},\footnote{\textsc{viii}, 555.} em passo sobre o brilho dos astros no enredor da destacada lua.} 


\pagebreak
\section{Fragmento 42}

\begin{gkverse}
ταῖσι <--> ψῦχρος μὲν ἔγεντο θῦμος\\
πὰρ δ’ ἴεισι τὰ πτέρα
\end{gkverse}

\begin{verse}
\ldots{} e deles se tornou frio o peito,\\*
e suas asas se afrouxaram \ldots{}
\end{verse}

\medskip

{\paragraph{Comentário} Segundo a fonte do fragmento, um escólio (comentário antigo) a Píndaro
(\textit{Ode pítica} 1), poeta mélico da virada dos séculos \textsc{vi}--\textsc{v} a.C.,
``Safo diz dos pombos'' o que lemos nos versos que cita.}


\section{Fragmento 101\textsc{a}}

\begin{gkverse}
πτερύγων δ’ ὔπα\\
κακχέει λιγύραν ἀοίδαν,\\
ὄπποτα φλόγιον \dagger{}καθέ-\\
ταν\dagger{} ἐπιπτάμενον \dagger{}καταυδείη\dagger{}
\end{gkverse}

\begin{verse}
\ldots{} e de sob suas asas\\*
verte clara canção,\\*
quando o flamejante [verão] \ldots{}\\
\ldots{} voando \ldots{}
\end{verse}

\medskip

{\paragraph{Comentário} A fonte do fragmento, que também cita o 101, é Demétrio; ele afirma que seus versos tratam da cigarra. O texto apresenta"-se bastante corrompido da última palavra do verso 3 ao 4; traduzo, pois, os versos compreensíveis. Mas há interessante possibilidade de emenda a um dos termos problemáticos, o último, \textit{kataudeíē}, que lhe daria o sentido de ``tocar o aulo'', instrumento de sopro símil ao oboé, referido no Fr.\,44.}



\pagebreak
\section{Fragmento 135}

\begin{gkverse}
Τί με Πανδίονις, ὦ Εἴρανα, χελίδων\ldots{};
\end{gkverse}

\begin{verse}
Por que, ó Irene, a Pandionida, a andorinha a mim \ldots{}?
\end{verse}

\medskip

{\paragraph{Comentário} Citado em Heféstion, fonte do Fr.\,102, o 135 dirige"-se a Irene, também mencionada no Fr.\,91, para falar ``da filha de Pandião'', mítico rei de Atenas, chamada Procne, que se transformou em andorinha. Esposa de Tereu, rei da Trácia, e irmã de Filomela, que se transformou em rouxinol. Na tradição, conta"-se que Tereu, desejando"-a, violou a cunhada e por isso as irmãs dele se vingaram, matando Ítis, o próprio filho de Procne com o rei. Daí o canto triste da andorinha, ave que a persona canta.}


\section{Fragmento 136}

\begin{gkverse}
ἦρος ἄγγελος ἰμερόφωνος ἀήδων
\end{gkverse}

\begin{verse}
\ldots{} mensageiro da primavera, o rouxinol de desejável voz \ldots{}
\end{verse}

\medskip

{\paragraph{Comentário} Segundo a fonte do fragmento, um escólio a Sófocles (século \textsc{v} a.C.) e à \textit{Electra},\footnote{Verso 149.} Safo já falava como na tragédia dos
rouxinóis.}

\section{Fragmento 143}

\begin{gkverse}
χρύσειοι <δ’> ἐρέβινθοι ἐπ’ ἀϊόνων ἐφύοντο
\end{gkverse}

\begin{verse}
\ldots{} e áureos grãos"-de"-bico cresciam nas margens \ldots{}
\end{verse}

\medskip

{\paragraph{Comentário} Ateneu (2.54\,\textsc{f}), fonte do Fr.\,141 já aqui visto, atribui a Safo a descrição.}


\pagebreak
\section{Fragmento 146}

\begin{gkverse}
μήτε μοι μέλι μήτε μέλισσα
\end{gkverse}

\begin{verse}
\ldots{} para mim, nem mel, nem abelha \ldots{}
\end{verse}

\medskip

{\paragraph{Comentário} O gramático Trifo, da era romana de Augusto (séculos \textsc{i} a.C.--\textsc{i} d.C.), 
cita como proverbiais estas palavras de Safo, no tratado \textit{Figuras de expressão} (25).
Notáveis são as reiterada assonâncias e aliterações na sequência de palavras, que costuram as sílabas pelo sentido e pelo som; translitero o verso: \textit{mḗte moi méli mḗte mélissa}.}


\section{Fragmento 167}

\begin{gkverse}
ὠίω πόλυ λευκότερον
\end{gkverse}

\begin{verse}
\ldots{} muito mais branco do que um ovo \ldots{}
\end{verse}

\medskip

{\paragraph{Comentário} Tendo por fonte Ateneu (2, 57\,\textsc{d}), que também preservou o Fr.\,141 e vários outros, as poucas palavras do 167 são dadas na sequência da citação do Fr.\,166, sobre Leda e o ovo de que nasce Helena, em certa tradição mítica. Aqui, o enfoque é na cor, e o ovo dá o padrão comparativo ao superlativo cujo uso é digno de atenção em Safo.}


\section{Fragmento 168\textsc{c}}

\begin{gkverse}
ποικίλλεται\\
    γαῖα πολυστέφανος 
\end{gkverse}

\begin{verse}
\ldots{} ela adornou,\\
a terra de multiguirlandas \ldots{}
\end{verse}

\medskip

{\paragraph{Comentário} A fonte do fragmento é, como no caso do 101\textsc{a}, Demétrio, que trata do charme na expressão.}



\chapter[O cantar, as canções e as companheiras]{O cantar, as canções e as companheiras}



\paragraph{\textsc{nota introdutória}}
É importante na mélica sáfica a metalinguagem -- o poeta a falar de seu
poetar, a criticar o poetar de outros, como no Fr.\,55, a tratar de sua
audiência, a tratar de seus temas. Nesse item, insiro alguns fragmentos que
ilustram isso -- aqueles que estão mais legíveis no \textit{corpus} de Safo.


\section{Fragmento 70}

\begin{gkverse}
\textnormal{[\textit{versos 1--2 e 14: ilegíveis e lacunares}]}

]ν ̣δ’ εἶμ’ ε[\\

\textnormal{[\textit{versos 4--8: ilegíveis e lacunares}]}\\

]αρμονίας δ̣[\\
         ]αθην χόρον, ἄα[\\
   ]δ̣ε λίγηα.[\\
   ]ατόν σφι̣[\\
  ]παντεσσι[

\end{gkverse}

\begin{verse}
\ldots{} e vou \ldots{}\\
\ldots{}\\
\ldots{} de harmonia \ldots{}\\
\ldots{} dança \ldots{}\\
\ldots{} clara [canção] \ldots{}\\
\ldots{}\\
\ldots{} a todos \ldots{} 
\end{verse}

\medskip

{\paragraph{Comentário} Preservado no \textit{Papiro de Oxirrinco} 1787, fonte do Fr.\,65, o 70 traz poucas palavras nos versos precários. Destaco \textit{harmonía} e \textit{khorón}, o coro que dança e, por extensão, canta, possivelmente referente ao coro de Safo que faz a \textit{performance}.}

\pagebreak
\section{Fragmento 118}

\begin{gkverse}
ἄγι δὴ χέλυ δῖα \dagger{}μοι λέγε\dagger{}\\
φωνάεσσα \dagger{}δὲ γίνεο\dagger{}
\end{gkverse}

\begin{verse}
\ldots{} vem, divina lira, fala-me\\*
e torna-te dotada de voz \ldots{}
\end{verse}

\medskip

{\paragraph{Comentário} Preservado no \textit{Sobre os tipos de estilo} (2.4), de Hermógenes (século \textsc{ii} d.C.),
o fragmento, segundo essa fonte, traz"-nos o momento em que ``Safo
questiona a sua lira, e a lira assim lhe responde''.}


\section{Fragmento 153}

\begin{gkverse}
πάρθενον ἀδύφωνον
\end{gkverse}

\begin{verse}
\ldots{} virgem dulcífona \ldots{}
\end{verse}

\medskip

{\paragraph{Comentário} Citado por Atílio Fortunato (metricista, século \textsc{iv} d.C.), na sua \textit{Arte} (28), o fragmento tem apenas duas palavras que trazem aos olhos e ouvidos a \textit{performance} do coro de meninas de Safo, ao destacar a habilidade de uma \textit{parthénos}, uma das moças e o prazer erótico, sensorial de sua voz \textit{adýphōnon}.}



\section{Fragmento 154}

\begin{gkverse}
Πλήρης μὲν ἐφαίνετ’ ἀ σελάν<ν>α\\
αἰ δ’ ὠς περὶ βῶμον ἐστάθησαν
\end{gkverse}

\begin{verse}
Em plenitude brilhava a lua, \\*
quando elas em volta do altar se postaram \ldots{}
\end{verse}

\medskip

{\paragraph{Comentário} Heféstion (11.3), fonte do Fr.\,102, citou os versos abaixo, provavelmente de abertura
de uma canção que giraria em torno de um rito sacrificial realizado
por mulheres, à noite.}


\section{Fragmento 156}

\begin{gkverse}
πόλυ πάκτιδος ἀδυμελεστέρα\ldots{}\\
χρύσω χρυσοτέρα\ldots{}
\end{gkverse}

\begin{verse}
\ldots{} muito mais dulcissonante que a harpa \ldots{}\\*
\ldots{} mais áurea que o ouro \ldots{}
\end{verse}

\medskip

{\paragraph{Comentário} A fonte desse fragmento é, como do 101\textsc{a}, Demétrio (161\,\textsc{s}), que recorda o uso da
hipérbole.}


\section{Fragmento 160}

\begin{gkverse}
\vinphantom{xxxxxxxxxxxxxx}τάδε νῦν ἐταίραις\\
ταὶς ἔμαις \dagger{}τέρπνα\dagger{} κάλως ἀείσω
\end{gkverse}

\begin{verse}
\ldots{} agora às minhas\\*
companheiras estas coisas aprazíveis belamente cantarei \ldots{}
\end{verse}

\medskip

{\paragraph{Comentário} A fonte do fragmento, como do 141, é Ateneu (13.571\,\textsc{d}). Nele ocorre o termo ``companheira'', \textit{étaira}, no plural.
Para alguns, esse termo em Safo levou à leitura de que a poeta estaria associada a uma \textit{hetaireía} feminina. Tratar"-se"-ia de um grupo aristocrático de iguais, ligados por laços de amizade e políticos. Porém, conforme enfatizei na introdução, não há nenhuma evidência de que existiria esse tipo de confraria típica do universo masculino no feminino; deslocá"-la a este é movimento de espelhamento carente de fundamentação. Logo, o que de mais seguro podemos dizer é que o termo, marcadamente afetivo, denota uma relação amistosa entre coevos, e é usado também nos Frs.\,126 e 142. Neste Fr.\,160, como no 126, indicaria uma fala dirigida às coreutas do grupo de Safo; e a \textit{persona} é decerto uma delas falando às amigas, suas iguais. Fica anunciada, em chave metalinguística, a bela \textit{performance}
de uma canção prazerosa às suas audiências interna e externa.}


\pagebreak
\section{Elogio}

\paragraph{\textsc{nota introdutória}}
O elogio e a censura consistem num dos pares opostos mais trabalhados na poesia
grega antiga, podendo mesmo ser tomados como categorias de organização dos seus gêneros.
Em Safo, há fragmentos que se concentram no
elogio, outros na
censura. Mesmo alguns textos demasiado mutilados e lacunares, com poucas palavras legíveis, parecem caminhar no sentido do binômio e de seus polos, como os Frs. 3 e 4 -- naquele, há referências à nobreza e beleza, à vileza, dores e ``censura'', \textit{óneidos};\footnote{Verso 5.} neste, talvez à beleza do ``rosto''\footnote{Verso 7.} e ao brilho.\footnote{Verso 6.} O poder de um e de outro só pode ser
minimamente compreendido se lembrarmos que a cultura da Grécia arcaica e
clássica pauta"-se pela ideia da vergonha; ou seja, o olhar público é
fundamental para a definição do lugar social ocupado pelo indivíduo; essa
definição repercute na sua linhagem e na sua comunidade. 

\section{Fragmento 41}

\begin{gkverse}
ταὶς κάλαισ’ ὔμμιν <τὸ> νόημμα τὦμον\\
       οὐ διάμειπτον
\end{gkverse}

\begin{verse}
\ldots{} para vós, as belas, meus pensamentos\\*
não mudaram \ldots{}
\end{verse}

\medskip

{\paragraph{Comentário} Apolônio Díscolo,\footnote{\textit{Sobre os pronomes} 124\,\textsc{c}.} fonte dos Frs.~32 e 33, cita os versos acima.
As destinatárias são elogiadas, e provavelmente os pensamentos são positivos e talvez expressem como a \textit{persona} as vê.}


\pagebreak
\section{Fragmento 82\textsc{a}}

\begin{gkverse}
Εὐμορφοτέρα Μνασιδίκα τὰς ἀπάλας Γυρίννως
\end{gkverse}

\begin{verse}
Mnasidica, mais formosa que a tenra Girino \ldots{}
\end{verse}

\medskip

{\paragraph{Comentário} Heféstion (11.5), citado aqui como fonte do Fr.\,102, preservou o 82\,\textsc{a}, que traz
um elogio à beleza física de uma moça, realçada pela comparação com outra
também louvada por um adjetivo, ``tenra'', \textit{apálas}, de marcada
carga erótica e empregado reiteradamente nos fragmentos da mélica sáfica que
adentram a esfera do erotismo, como o 94.}

\section{Fragmento 122}

\begin{gkverse}
ἄνθε’ ἀμέργοισαν παῖδ’ ἄγαν ἀπάλαν
\end{gkverse}

\begin{verse}
\ldots{} tenra virgem, colhendo flores \ldots{}
\end{verse}

\medskip

{\paragraph{Comentário} A fonte desse fragmento, como do 141 e de vários outros, Ateneu (12.554\,\textsc{b}), comenta: 

\begin{quote}
Pois é natural
que os que pensam ser belos e maduros colham flores. Por isso, dizem que
Perséfone e suas companheiras colhem flores, e Safo diz ter visto 
[citação do fragmento].
\end{quote}

 Ou seja, Ateneu sugere a aproximação entre as imagens
da menina virgem, filha da deusa Deméter, e da virgem ``tenra'', \textit{apálan}, de modo a ressaltar nesta a beleza e a
sensualidade. Vale lembrar que Perséfone, como conta o \textit{Hino homérico a Deméter} 
(início do século \textsc{vi} a.C), foi abduzida por Hades exatamente
quando brincava a colher flores num prado primaveril, ignorante dos perigos que
espreitam as belas e sensuais meninas virgens, \textit{parthénoi}. Cabe frisar, como fiz na introdução, que a virgindade delas,
na concepção grega, implica
apenas que estão na fase transitória entre infância e idade
adulta, a caminho do casamento, quando passam a ter vida sexual e, portanto, integram"-se
de vez à sociedade à qual pertencem, deixando para trás a dimensão algo
selvagem e indomada da fase virginal.}



\section{Censura}

\paragraph{\textsc{nota introdutória}}
Não temos no \textit{corpus }da mélica sáfica textos pertencentes ao gênero
poético do jambo, que no correr dos tempos acaba se configurando em essência como poesia de
ataque, de vituperação, de invectiva. Há, porém,
fragmentos de caráter jâmbico mais ou menos evidente; afinal, se um gênero
acabou por ter na vituperação seu elemento definidor, este não lhe é exclusivo,
mas integra vários da poesia grega antiga, desde a \textit{Ilíada}. Eis
alguns exemplos sáficos.

\section{Fragmento 3} 

\begin{gkverse}
]δώσην\\
κλ]ύτων μέντ’ ἐπ[\\
κ]άλων κἄσλων, σ[\\
.΄]οις, λύπης τέμ[\\
 ]μ’ ὄνειδος\\
]οιδήσαισ. ἐπιτα̣[\\
].΄αν, ἄσαιο. τὸ γὰρ [\\
]μον οὐκοὔτω μ[\\
] διάκηται,\\
  ] μη̣δ̣[\qquad    ]αζε,\\
  ]χις, συνίημ[\\ 
].ης κακότατο[ς\\
  ]μεν\\
  ]ν ἀτέραις με[\\
]η φρένας, εὔ[
\end{gkverse}
\pagebreak
\begin{verse}
\ldots{} dar \ldots{}\\
\ldots{} de ínclitos \ldots{}\\
\ldots{} de belos e também nobres \ldots{}\\
\ldots{} me vexas \ldots{}\\
\ldots{} censura \ldots{}\\
\ldots{} pois \ldots{}\\
\ldots{} mão desse modo \ldots{}\\
\ldots{} ele/a disposto/a \ldots{}\\
\ldots{}\\
\ldots{} compreendo \ldots{}\\
\ldots{} o mais vil \ldots{}\\
\ldots{}\\
\ldots{} as outras \ldots{}\\
\ldots{} sensos, \ldots{}
\end{verse}

\medskip

\paragraph{Comentário}
Preservado no precário \textit{Papiro de Berlim} (século \textsc{vii} d.C.), somado ao \textit{Papiro de Oxirrinco} 424 (século \textsc{iii} d.C.), o fragmento parece falar de amizade, de censura, \textit{óneidos}, em vexar -- e nessa atmosfera se inserem a \textit{persona} que é"-nos opaca tanto quanto o \textit{tu} indicado na  forma verbal do verso 4 (\textit{lýpēs}), e reflete sobre o bom/\,belo, \textit{kalós}, o nobre, \textit{esthlós}, e o feio/\,vil, \textit{kakós}, noções ético"-estéticas muito caras ao pensamento arcaico e seus valores tradicionais.



\pagebreak
\section{Fragmento 56}

\begin{gkverse}
οὐδ’ ἴαν δοκίμωμι προσίδοισαν φάος ἀλίω\\
ἔσσεσθαι σοφίαν πάρθενον εἰς οὐδένα πω χρόνον\\
τεαύταν
\end{gkverse}

\begin{verse}
\ldots{} não imagino que uma virgem, após ver a luz do sol,\\
terá em algum tempo futuro tal habilidade \ldots{}
\end{verse}

\medskip

{\paragraph{Comentário} Preservado em Crísipo (13), como o 51, o Fr.\,56 parece referir"-se à habilidade
poética de uma virgem, \textit{parthénos}, que em vida estaria condenada à
mediocridade.}


\section{Fragmento 57}

\begin{gkverse}
τίς δ’ ἀγροΐωτις θέλγει νόον\ldots{}\\
ἀγροΐωτιν ἐπεμμένα στόλαν\ldots{}\\
οὐκ ἐπισταμένα τὰ βράκε’ ἔλκην ἐπὶ τὼν σφύρων;
\end{gkverse}

\begin{verse}
\ldots{} e quem é a rusticona que encantou a mente \ldots{}\\*
vestida em rústica veste \ldots{}\\*
sem saber como erguer seus trapos aos tornozelos?
\end{verse}

\medskip

{\paragraph{Comentário} Ao citar os versos acima, em que roupa, atitude e modos de uma figura feminina são
duramente censurados, Ateneu (1.21\,\textsc{b}\,\textsc{c}), fonte também do Fr.\,141, declara: ``Safo
escarnece Andrômeda'', personagem abordada nos comentários aos Frs.\,130 e 133, talvez uma líder de coro de meninas, rival do de Safo.}


\pagebreak
\section{Fragmento 68} 

\begin{gkverse}
]ι ̣γάρ μ’ ἀπὺ τὰς ἐ[\\
ὔ]μως δ’ ἔγεν[το\\
   ] ἴσαν θέοισιν\\
 ]ασαν ἀλίτρα[\\
        Ἀν]δρομέδαν[.].αξ[\\
   ]αρ[...]α μά̣κα̣[ιρ] α\\
 ]ε̣ον δὲ τρόπον α[.].ύνη[\\
      ] κ̣όρο̣ν οὐ κατισχε.[\\
]κ̣α[...]. Τυνδαρίδαι[ς\\
]ασυ[.]...κα[.] χαρίεντ’ ἀ[\\
]κ’ ἄδολον [μ]ηκέτι συν[\\
] Μεγάρα.[...]ν̣α[...]α[

\textnormal{[\textit{versos 1--2 e 4--7: ilegíveis e lacunares}]}

\end{gkverse}

\begin{verse}
\ldots{} pois a mim desta \ldots{}\\
\ldots{}contudo tornou-se \ldots{}\\
\ldots{} par dos deuses \ldots{}\\
\ldots{} culpada \ldots{}\\
\ldots{} Andrômeda \ldots{}\\
\ldots{} venturosa \ldots{}\\
\ldots{} modo \ldots{}\\
\ldots{} ganância não foi refreada \ldots{}\\
\ldots{} Tindaridas \ldots{}\\
\ldots{} graciosa \ldots{}\\
\ldots{} sem dolo não mais \ldots{}
\end{verse}

\pagebreak
{\paragraph{Comentário} Preservado no duramente danificado rolo \textit{Papiro de Oxirrinco} 1787, em que se acham o Fr.\,65 e outros, o 68 traz, entre versos que mal se discernem, de novo a figura de Andrômeda, além de uma virgem associada ao grupo e Safo nos testemunhos antigos, Megara.  A linguagem parece lidar com a censura, e refere deuses ao nomeá"-los, \textit{théoisin},\footnote{Verso 3.} e pelo adjetivo \textit{mákaira}, ``venturosa'',\footnote{Verso 6.} que noutros fragmentos traduzi, e que só aos imortais pode ser atribuído -- no Fr.\,1, é dado a Afrodite. No verso 8, há algo talvez sobre o conter da ganância,\footnote{Ferrari, 2011, p. 158.} a qual é sempre condenada. Noto ainda a presença das Tindaridas, Helena e Clitemnestra, as filhas do rei de Esparta, Tíndaro, e Leda, casal de que falei no comentário ao Fr.\,166. Tal presença reforça a possibilidade de censura, pois as duas figuras míticas femininas sob tal signo se acham, ambas adúlteras, desertora -- aquela -- e assassina -- esta -- de seus respectivos maridos, os filhos de Atreu, Menelau e Agamêmnon. E por fim, veja"-se o adjetivo \textit{ádolon}, ``sem dolo'',\footnote{Verso 11.} que traz para a canção algo que se relacione justamente a um engano, um ardil.}



\pagebreak
\section{Fragmento 71}

\begin{gkverse}
]μισσε Μίκα\\
]ελα[..ἀλ]λά σ’ ἔγωὐκ ἐάσω\\
         ]ν̣ φιλότ[ατ’] ἤλεο Πενθιλήαν̣[\\
]δα κα̣[κό]τροπ’, ἄμμα[\\
    ] μέλ̣[ος] τι γλύκερον .[\\
    ]α μελλιχόφων[ος\\
    ]δει, λίγυραι δ’ ἄη[\\
        ]δροσ[ό]εσσα[
\end{gkverse}

\begin{verse}
\ldots{} Mica \ldots{}\\
\ldots{} mas eu não permitirei a ti \ldots{}\\
\ldots{} escolheste a amizade de uma Pentilida \ldots{}\\
\ldots{} ó maligna \ldots{}\\
\ldots{} uma doce canção \ldots{}\\
\ldots{} voz de mel \ldots{}\\
\ldots{} e claras [brisas] \ldots{}\\*
\ldots{} orvalhada \ldots{}
\end{verse}

\medskip

{\paragraph{Comentário} Em dado momento das convulsões políticas internas e da sucessão de regimes
tirânicos pós"-queda da tradicional aristocracia dos Pentilidas, surge Pítaco, que se tornou governante de
Mitilene, mas fez aliança com aquela linhagem por meio do casamento,
canta Alceu no Fr.\,70, envolvido nos conflitos,
rompendo com as facções aristocráticas rivais aos Pentilidas, que tinham apoiado, entre as quais, a
do grupo do poeta e de Safo. Acima, é notável como a traição de 
Mica, figura feminina, também se firma com a aliança da amizade com os Pentilidas. Termos
pesados são seguidos por referências luminosas, prazerosas, eróticas, mas o
fragmento, cuja precária fonte é o \textit{Papiro de Oxirrinco} 1787, o mesmo do Fr.\,65 e de outros, quase nada revela da canção perdida.}


\pagebreak
\section{Fragmento 91}

\begin{gkverse}
ἀσαροτέρας οὐδάμα πω Εἴρανα, σέθεν τύχοισαν
\end{gkverse}

\begin{verse}
\ldots{} nunca tendo encontrado outra mais abjeta que tu, Irena \ldots{}
\end{verse}

\medskip

{\paragraph{Comentário} A fonte do fragmento, como do 102, é Heféstion (11.5).
A mesma figura feminina é nomeada no Fr.\,135.}


\section{Fragmento 137}

\begin{gkverse}
θέλω τί τ’ εἴπην, ἀλλά με κωλύει\\
αἴδως\ldots{}
\end{gkverse}

\begin{verse}
\ldots{} quero algo te dizer, mas me impede\\
o pudor \ldots{}
\end{verse}

\medskip

{\paragraph{Comentário} Aristóteles (século \textsc{iv} a.C.), na \textit{Retórica} (1367\,\textsc{a}), afirma: ``Os homens se envergonham de dizer, de fazer ou de pretender fazer vilezas, exatamente
como Safo em sua resposta, quando Alceu falou: [citação do fragmento].''
O filósofo pressupõe um poema de Alceu respondido por um de Safo, esclarece um anônimo comentário antigo (escólio). Os
dois poetas, comentei na introdução, podem ter se conhecido na Mitilene arcaica, e
os antigos sempre os aproximavam, inclusive na iconografia.
Em outros entendimentos, os versos acima, que a edição adotada nesta antologia, a de Voigt, dá como de Safo, são atribuídos a Alceu, à poeta são atribuídos os versos sequenciais, que Voigt indica, mas sob suspeita: ``\ldots{} mas se ansiasses pelo honrado e pelo belo,/ e tua língua não se animasse a dizer algo vil,/ a vergonha não tomaria teus olhos,/ mas dirias tua fala\ldots{}''. Assim é em edições e traduções do tratado aristotélico, e em Ferrari.\footnote{2010, p. 75.} Na edição Lobel"-Page\footnote{De 1955.} de Safo, que era a de autoridade antes de Voigt, tais versos suspeitos sequer são transcritos. A questão é, portanto, bastante espinhosa. Fiquemos, pois, com os versos que Voigt autoriza como sáficos.}

\pagebreak
\section{Fragmento 144}

\begin{gkverse}
μάλα δὴ κεκορημένοις\\
Γόργως
\end{gkverse}

\begin{verse}
\ldots{} bem saturada de Gorgo \ldots{}
\end{verse}

\medskip

{\paragraph{Comentário} A fonte desse fragmento é o autor que também preservou o já visto Fr.\,46, Herodiano, no tratado
\textit{Sobre a declinação dos substantivos}. Nele, a personagem feminina é
criticada por uma voz sem qualquer identificação. Mas nos testemunhos, além de Andrômeda, circula também o nome de Gorgo, alvo de censura na canção, como poeta rival de Safo. Não temos evidência dessas poetas rivais que não sejam os testemunhos antigos. Não parece, contudo, implausível que Mitilene tivesse outras poetas líderes de grupos corais, como Safo, dadas as evidências sobre os numerosos  grupos corais femininos disseminados pelas \textit{póleis} gregas.}

\section{Fragmento 155}

\begin{gkverse}
πόλλα μοι τὰν Πωλυανάκτιδα παῖδα χαίρην
\end{gkverse}

\begin{verse}
\ldots{} minhas muitas saudações à filha de \rlap{Polianactides \ldots{}}
\end{verse}

\medskip

{\paragraph{Comentário} Para Máximo de Tiro (18), fonte desse fragmento, como do 47, Safo fala com ironia, tal qual
Sócrates fala a Íon na abertura do diálogo de Platão que leva o nome desse
rapsodo, \textit{Íon}, ou \textit{Sobre a inspiração poética}. Ora, a ironia
liga"-se à censura, tanto mais no paralelo socrático sugerido.
E o alvo da censura seria novamente Gorgo,\footnote{Ferrari, 2010, p. 81.} considerados os testemunhos antigos.}



\chapter[Epitalâmios: canções de casamento]{Epitalâmios: canções de casamento}

\paragraph{\textsc{nota introdutória}}
``Em Homero, a canção é um acompanhamento sempre presente no casamento, e a frequente menção de canções nupciais por autores de todos os períodos na literatura grega antiga atesta a continuada importância da canção para a cerimônia de casamento na Grécia antiga''.\footnote{Hague, 1983, p. 131.} Na era arcaica, são sáficas as canções que nos chegaram. Assim como o 112, os fragmentos arrolados neste tópico compõem o pequeno grupo
de representantes do subgênero mélico que teria sido compilado no nono livro de
Safo na Biblioteca de Alexandria, como ressaltei na introdução desta antologia.
Naquele fragmento, o elogio aos noivos se destaca; nestes, outros dos aspectos
da canção de casamento, inclusive o da linguagem e temática jocosas, que
revelam seus elos genéticos com a tradição popular.
Esses traços são muito próprios aos epitalâmios, pois não apenas é continuada a tradição grega de seu canto nos casamentos, mas há neles  ``um marcante nível de continuidade''.\footnote{Swift, 2010, p. 245.} Diga"-se ainda que os epitalâmios de Safo aqui apresentados ``exibem, com relação ao rito da boda, uma função pragmática precisa, por articular suas fases e realçar suas principais características''.\footnote{Ferrari, 2010, p. 119.} 

\pagebreak
\section{Fragmento 104 (\textsc{a--b})}

\begin{gkverse}
Ἔσπερε πάντα φέρηις ὄσα φαίνολις ἐσκέδασ’ Αὔως,\\
      φέρηις ὄιν, φέρηις αἶγα, φέρηις ἄπυ μάτερι παῖδα.

\hspace*{35mm}\adforn{47}

ἀστέρων πάντων ὀ κάλλιστος
\end{gkverse}

\begin{verse}
Ó Vésper, trazes tudo que a luzidia Eos espalhou:\\
trazes ovelha, trazes cabra, arrebatas da mãe a filha \ldots{}

\hspace*{35mm}\adforn{47}

\ldots{} dos astros todos o mais belo \ldots{} 
\end{verse}

\medskip

{\paragraph{Comentário} O fragmento ``a'', composto pelo par de versos abaixo, tem por fonte Demétrio (141),
como o Fr.\,101; já o fragmento ``b'', o verso traduzido em separado daquele par,
foi preservado na \textit{Oração 46}, de Himério, retórico antes citado no comentário ao Fr.\,112.
Ambos os fragmentos integram uma canção de Safo a Vésper, a estrela da tarde,
segundo Himério, e parecem aludir à procissão que sucede o banquete na casa da
noiva, iniciada no começo do anoitecer, para levá"-la à casa do noivo.
Noto que Demétrio destaca o uso que faz Safo da repetição -- característica da estilística epitalâmica, diga"-se -- da forma verbal \textit{phérēis}, da qual Vésper é sempre sujeito; uso que considera gracioso, \textit{kharientízetai}. O verbo ocorre uma vez no verso 1 e três vezes no 2, modificado, na última ocorrência, seu sentido modificado pelo termo sucessivo \textit{ápy} -- indicativo de afastamento, \textit{ápo}.\footnote{Em tradição anterior, segui o entendimento de que o verbo ao final diria ``trazes de volta à mãe a criança''. Todavia, adoto novo entendimento aqui, considerando, junto a Ferrari (2010, pp. 120--1) o que sabemos das etapas da boda e uma passagem no poeta latino Catulo (\textit{Ode} 62, versos 20--23), cujas odes epitalâmicas têm Safo como referente: ``\textit{Héspero, qual no céu é brilho mais cruel/ que tu, que a filha arrancas dos braços da mãe?/ Dos braços da mãe a arrancas, que se agarra,/ e a dás, menina casta, dom ao moço ardente?/ Rasa a urbe, o que mais cruel inimigos fazem?/ Hímen, ó Himeneu, vem Himeneu, Himeneu!}'', trad. Oliva 1996.} }




\pagebreak
\section{Fragmento 105 (\textsc{a--b})}

\begin{gkverse}
οἶον τὸ γλυκύμαλον ἐρεύθεται ἄκρωι ἐπ’ ὔσδωι,\\
ἄκρον ἐπ’ ἀκροτάτωι, λελάθοντο δὲ μαλοδρόπηες·\\
οὐ μὰν ἐκλελάθοντ’, ἀλλ’ οὐκ ἐδύναντ’ ἐπίκεσθαι

\hspace*{35mm}\adforn{47}

οἴαν τὰν ὐάκινθον ἐν ὤρεσι ποίμενες ἄνδρες\\
πόσσι καταστείβοισι, χάμαι δέ τε πόρφυρον ἄνθος\ldots{}

\end{gkverse}

\begin{verse}
\ldots{} como o mais doce pomo enrubesce no ramo ao alto,\\
alto no mais alto ramo, e os colhedores o esquecem;\\*
não, não o esquecem -- mas não o podem alcançar \ldots{}

\hspace*{35mm}\adforn{47}

\ldots{} como o jacinto que nas montanhas homens, pastores,\\
esmagam com os pés, e na terra a flor purpúrea \ldots{}
\end{verse}

\medskip

{\paragraph{Comentário} Citados por Siriano, comentador bizantino do tratado de estilística de
Hermógenes (1.1), e por Demétrio (106), ambos já aqui mencionados respectivamente nos Frs.~118 e 101\textsc{a}, os fragmentos trazem duas imagens metafóricas distintas: no trio de versos (\textbf{a}), a da menina fruta madura, a maçã, sensual, desejada por todos,
mas não a todos alcançável; na dupla (\textbf{b}), a da perda da virgindade nas bodas.
Em ambos, é a natureza que alavanca a linguagem erótica.
E na segunda, o casamento faz"-se, como de fato é, uma espécie de morte,\footnote{Swift, 2010, p. 250.} porque provoca a transição irreversível de uma vida, a de \textit{parthénos}, ``moça'', para a outra, a de \textit{gynḗ}, ``mulher''. Como observei no comentário ao Fr.\,44, cujo tema é a boda mítica de Heitor e Andrômaca, são muitos e especificamente gregos os paralelos entre as cerimônias nupcial e fúnebre.\footnote{Redfield 1982, p. 188.}


\pagebreak
\section{Fragmento 106}

\begin{gkverse}
πέρροχος, ὠς ὄτ’ ἄοιδος ὀ Λέσβιος ἀλλοδάποισιν
\end{gkverse}

\begin{verse}
\ldots{} superior, como o cantor lésbio aos de outras terras \ldots{}
\end{verse}

\medskip

{\paragraph{Comentário} O fragmento tem por fonte Demétrio (146), como o 101\textsc{a}, cujo tratado diz que
``do homem excepcional assim fala Safo''. Se tal homem for o
noivo, há seu elogio em chave comparativa com o elevado \textit{status}
entre os antigos da tradição poético"-musical lésbio"-eólica.}


\section{Fragmento 107}

\begin{gkverse}
ἦρ’ ἔτι παρθενίας ἐπιβάλλομαι;
\end{gkverse}

\begin{verse}
\ldots{} será que ainda anseio pela virgindade?
\end{verse}

\medskip

{\paragraph{Comentário} O gramático Apolônio Díscolo, fonte do Fr.\,33, preservou o 107, no tratado
\textit{Sobre as conjunções} (1.223.24\,\textsc{ss}.). A virgindade, tema recorrente no epitalâmio, uma
vez que está em evidência sua ruptura no enlace sexual que deve consumar o
casamento, é pensada pelo \textit{eu} que nos fala.}


\section{Fragmento 108}

\begin{gkverse}
ὦ κάλα, ὦ χαρίεσσα
\end{gkverse}

\begin{verse}
Ó bela, ó graciosa \ldots{}
\end{verse}

\medskip

{\paragraph{Comentário} Citado em Himério,\footnote{\textit{Oração} 9.} fonte do Fr.\,104, este fragmento traz as palavras de condução do noivo à noiva, no aposento nupcial, informa a fonte, ao qual realçam a beleza dela.}


\pagebreak
\section{Fragmento 109 }

\begin{gkverse}
δώσομεν, ἦσι πάτηρ
\end{gkverse}

\begin{verse}
\ldots{} dá-la-emos, diz o pai \ldots{}
\end{verse}

\medskip

\paragraph{Comentário} Citado em anônimo comentário reunido junto a outros dispersos na coletânea \textit{Anedota Oxfordiana},\footnote{\textsc{i} 190, século \textsc{xix}.} o fragmento surge a propósito de um verso da \textit{Ilíada},\footnote{\textsc{i}, 528.} que com o termo equivalente (\textit{ē̂}) ao usado por Safo, \textsc{ē̂si}, observa o comentador, expressa o dizer em 3ª pessoa do singular. Nele, há apenas três palavras, ligadas decerto à passagem da noiva de sua própria família à do noivo, pelas mãos do pai.


\section{Fragmento 110}

\begin{gkverse}
Θυρώρωι πόδες ἐπτορόγυιοι,\\
τὰ δὲ σάμβαλα πεμπεβόεια,\\
πίσσυγγοι δὲ δέκ’ ἐξεπόναισαν
\end{gkverse}

\begin{verse}
Os pés do porteiro têm sete braças,\\*
e as sandálias, couro de cinco bois -- \\*
e dez sapateiros nelas labutaram \ldots{}
\end{verse}

\medskip

{\paragraph{Comentário} Assim como o Fr.\,102, este é citado em Heféstion (7.6); e, segundo Pólux, fonte do Fr.\,54, a figura nele enfocada jocosamente, o guardião do leito nupcial ou tálamo, 
tinha por função impedir que os amigos da noiva acorressem a resgatá"-la do
quarto que a guardava junto ao noivo.
Tal figura tem contraparte real na cerimônia, o escolhido amigo do noivo, o real \textit{thyrōrós}, como é chamado, ressalta Caciagli,\footnote{2009, p. 68.} que se mantém em vigilância, enquanto os coros e danças prosseguem, do lado de fora do tálamo. Hesíquio (século \textsc{v} d.C.), no verbete ao termo em seu \textit{Léxico}, dá \textit{thyrōrós} o sinônimo \textit{paránymphos}, ``o amigo do noivo''. As amigas da noiva começam seu canto jocoso por ele, na figura do monstrinho, zombeteiramente, sublinha\footnote{Ferrari, 2010, pp. 124--5.} -- a zombaria, a jocosidade sendo da tradição popular das canções de casamento.}


\section{Fragmento 111}

\begin{gkverse}
Ἴψοι δὴ τὸ μέλαθρον\\
ὐμήναον,\\
ἀέρρετε, τέκτονες ἄνδρες·\\
ὐμήναον,\\
γάμβρος \dagger{}(εἰσ)έρχεται ἴσος Ἄρευι\dagger{}\\
<ὐμήναον,>\\
ἄνδρος μεγάλω πόλυ μέζων\\
<ὐμήναον.>
\end{gkverse}

\begin{verse}
Ao alto o teto -- \\
Himeneu! -- \\
levantai, vós, varões carpinteiros! -- \\
Himeneu! -- \\
o noivo chega, qual Ares -- \\
Himeneu! -- \\
muito maior do que um varão grande --\\*
Himeneu!
\end{verse}

\medskip

{\paragraph{Comentário} Também citado em Heféstion (7.1), o fragmento louva a beleza do noivo símil a Ares,
deus da guerra, e altíssimo, canta em tom de brincadeira a descrição
hiperbólica. Na cena, alude"-se à entrada do noivo no aposento nupcial -- daí a
referência ao teto e a Himeneu, deus da boda, cuja presença na festa, por isso
solicitada aos gritos, garantiria o sucesso da união sexual para a qual o noivo está superlativamente equipado em sua virilidade. Da tradição popular das canções nupciais vem esse elemento da jocosidade e da malícia à beira da vulgaridade, que pode ter dois propósitos: aliviar a tensão de todo o processo do enlace sexual dos noivos em dado momento da cerimônia; afastar mau agouro, \textit{aprotopaica}, da festa, logo, da união nela celebrada e consumada. Mais: daquela tradição vem o refrão e a repetição de palavras, típicos da oralidade, que aqui e em outros epitalâmios de Safo encontramos.}


\section{Fragmento 113}

\begin{gkverse}
  οὐ γὰρ\\
ἀτέρα νῦν πάις, ὦ γάμβρε, τεαύτα
\end{gkverse}

\begin{verse}
Pois, ó noivo, jamais como agora outra menina como esta \ldots{}
\end{verse}

\medskip

{\paragraph{Comentário} Citado em Dionísio de Halicarnasso (25), mesma fonte do Fr.\,1, o fragmento traz canção em que, a certa altura, o coro se dirige ao noivo, celebrando a beleza ímpar da noiva.}


\pagebreak
\section{Fragmento 114}

\begin{gkverse}
[νύμφη) παρθενία, παρθενία, ποῖ με λίποισ’ α<π>οίχηι;\\
(παρθενία)  \dagger{}οὐκέτι ἤξω πρὸς σέ, οὐκέτι ἤξω\dagger{}.
\end{gkverse}

\begin{verse}
(\textsc{noiva}) ``Virgindade, virgindade, aonde vais, me abandonando?''

(\textsc{virgindade}) ``Nunca mais a ti voltarei, nunca mais voltarei''
\end{verse}

\medskip

{\paragraph{Comentário} Este, como o 101\textsc{a}, tem em Demétrio (140) sua fonte. O canto dialogado, forma já vista
nos Frs.~133 e 140, é de \textit{performance} coral. Novamente, a canção jocosa
centra"-se na incontornável perda da virgindade, usando o humor decerto como
meio de aliviar a tensão e o impacto de todo o processo que conduz a virgem à
idade adulta, em que passa a atuar na esfera do sexo, deixando para trás sua casa, a mãe e as amigas -- a existência de \textit{parthénos}.
``O casamento é um evento importante e que muda a vida; logo, é potencialmente assustador. O trauma ritualizado elabora os temores reais, apresentando"-os de um modo regulado. Articular essas ansiedades serve, pois, como forma de mitigá"-las, retratando"-as como parte normal e necessária da transição e, assim, uma resposta saudável, em vez de destrutiva, à mudança''.\footnote{Swift 2010, p. 248.}
Os termos do Fr.\,114 configuram uma ``linguagem reminiscente do lamento'',\footnote{Lardinois 2011, p. 164.} da \textit{moirología} -- canto de lamento pela perda (morte, casamento, partida da casa); e sua perspectiva é feminina, a refletir a ``ansiedade que a noiva sente em antecipação à noite nupcial e à nova vida com seu marido''.\footnote{Lardinois, pp. 164--5.}}



\pagebreak
\section{Fragmento 115}

\begin{gkverse}
Τίῳ σ᾿, ὦ φίλε γάμβρε, κάλως ἐικάσδω;\\
ὄρπακι βραδίνῳ σε μάλιστ᾿ ἐικάσδω.
\end{gkverse}

\begin{verse}
A que, ó caro noivo, belamente te comparo?\\*
A um ramo esguio sobretudo te comparo \ldots{}
\end{verse}

\medskip

{\paragraph{Comentário} Também preservado em Heféstion (7.6), esse fragmento faz o elogio do noivo jovem,\footnote{Ragusa e Rosenmeyer 2019, pp. 62--75.} porque comparado a jovem planta, de sua
graça física marcada no adjetivo à vergôntea, \textit{órpaki bradínōi}, chamando a atenção da noiva e tornando"-o atraente sexualmente aos seus olhos.
A comparação é feita pela técnica da \textit{eikasía},\footnote{Hague, 1983, pp. 132--9.} pela qual se exprime elogio, como é o caso, ou jocosidade; daí o uso do verbo \textit{eikázein} nos dois versos do fragmento. Tal recurso, bem como a própria repetição, são característicos da tradição popular das canções de casamento. e tradicional é igualmente o conjunto de imagens ``para expressar beleza e vigor juvenis: flores e primavera se enlaçam ao topos da assimilação da vida humana à fertilidade natural, enquanto o uso de um ramo para representar o vigor jovem se acha em Homero tanto para moços quanto para moças".\footnote{Swift, 2010, p. 246.}}

\pagebreak
\section{Fragmento 116}

\begin{gkverse}
χαῖρε, νύμφα, χαῖρε, τίμιε γάμβρε, πόλλα
\end{gkverse}

\begin{verse}
Salve, ó noiva, salve, ó digno noivo, muitas \ldots{}
\end{verse}

\medskip

{\paragraph{Comentário} Preservado por Sérvio (século \textsc{iv} d.C.), em seu comentário às \textit{Geórgicas} (1.31) de Virgílio (século \textsc{i} a.C.), o fragmento saúda os noivos, elogiando"-os.}


\section{Fragmento 117}

\begin{gkverse}
\dagger{}χαίροις ἀ νύμφα\dagger{}, χαιρέτω δ’ ὀ γάμβρος
\end{gkverse}

\begin{verse}
Saudações, ó noiva, saudações, ó noivo \ldots{}
\end{verse}

\medskip

{\paragraph{Comentário} Citado por Heféstion (4.2), o fragmento traz conteúdo similar ao do 116.}


\chapter{Festividades}

\paragraph{\textsc{nota introdutória}}
O Fr.\,2 fala em festividades que não podemos precisar; os epitalâmios,
caracterizados por estruturas métricas próprias, têm por texto/contexto a
cerimônia do casamento e seus festejos, os quais estão presentes em fragmentos não epitalâmicos: 23, 44, 141. No grupo a seguir, destaca"-se a festa,
ora ligada ao casamento, ora a ritos de celebração aos deuses, ora ao banquete,
evento da vida cotidiana masculina, largamente celebrado na poesia grega
antiga, em todos os seus gêneros.

\section{Fragmento 27}

\begin{gkverse}
\textnormal{[\textit{versos 1--3: ilegíveis e lacunares}]}

\ldots{}]. καὶ γὰρ δὴ σὺ πάις ποτ[\\
\ldots{}]ικ̣ης μέλπεσθ’ ἄγι ταῦτα[\\
\ldots{}] ζάλεξαι, κἄμμ’ ἀπὺ τωδεκ[\\
ἄ]δρα χάρισσαι·

σ]τείχομεν γὰρ ἐς γάμον· εὖ δε[\\
κα]ὶ σὺ τοῦτ’, ἀλλ’ ὄττι τάχιστα[\\
πα]ρ̣[θ]ένοις ἄπ[π]εμπε, θέοι[\\
]εν ἔχοιεν\\
      ] ὄδος̣ μ[έ]γαν εἰς Ὄλ[υμπον\\
       ἀ]νθρω[π\qquad]αίκ.[
\end{gkverse}
\pagebreak
\begin{verse}
\ldots{} e certa vez tu também menina \ldots{}\\
\ldots{} cantar-dançar vem! -- estas coisas \ldots{}\\
\ldots{} discutir, e para nós \ldots{}\\
\ldots{} abundantes deleites;\\
\ldots{} pois apressamo-nos à boda; bem \ldots{}\\
\ldots{} isso também tu, mas então rápido\\
as virgens envia, deuses \ldots{}\\
\ldots{} tivesse \ldots{}\\
\ldots{} caminho ao grande Olimpo\\*
\ldots{} mortais \ldots{}
\end{verse}

\medskip

{\paragraph{Comentário} Preservado no \textit{Papiro de Oxirrinco} 1231, como o Fr.\,15 e tantos outros, o 27 traz uma voz que busca convencer alguém  -- o \textit{tu}
pode ser a noiva ou a mãe da noiva -- a tomar parte na procissão de
ida ao casamento, liderando virgens certamente ligadas à noiva; os últimos versos
legíveis enveredam para uma sentença moralizante.
Caciagli\footnote{2009, p. 63.} nota que no fragmento ``afloram quase todos os aspectos que caracterizam a obra da poeta'', como os entrelaçados canto"-dança, \textit{mélpesthai}, e o desejo, \textit{érōs}, entrelaçados, como nos Frs. 22, 94, 96; ``termos conexos à juventude'' das personagens, recorrentes nas canções de Safo; o temo da memória, destacado nesta antologia, indicado no verso 1 (\textit{pote}); e ``a referência à cerimônia nupcial''. Nesse sentido, a linguagem faz"-se reflexo da canção da poeta, e a \textit{persona} faz"-se a líder das meninas do coro -- a \textit{khorodidáskalos}.\footnote{Frisam"-no Caciagli (2009, p.\,64) e Ferrari (2010, p.\,34).}}



\pagebreak
\section{Fragmento 30}

\begin{gkverse}
νύκτ[...].[

πάρθενοι δ[\\
παννυχίσδοι̣[σ]α̣ι[̣\\
σὰν ἀείδοισ̣[ι]ν φ[ιλότατα καὶ νύμ-\\
φας ἰοκόλπω.\\
ἀλλ’ ἐγέρθε̣ι̣ς, ἠϊθ[ε\\
στεῖχε σοὶς ὐμάλικ̣[ας\\
ἤπερ ὄσσον ἀ λιγ̣ύφω̣[νος\\
ὔπνον [ἴ]δωμεν.
\end{gkverse}

\begin{verse}
\ldots{} virgens \ldots{}\\
\ldots{} celebrando um festival noturno \ldots{}\\
\ldots{} teus amores cantariam e os da noiva\\*
de violáceo colo \ldots{}

mas, despertos,\footnote{A tradução busca enfatizar o sentido do acordar com o dia, mas é válida ainda a tradução por ``tendo se erguido'' (Ragusa, 2019a, p. 95). A opção depende do entendimento da celebração noturna (\textit{pannykhís}) do verso 3, que é a ocasião de \textit{performance}:  se sucede o banquete do casamento (Ferrari, 2010, p. 114), com o coro de moças a chamar o noivo e seus coevos a se erguerem da mesa; ou se é o momento em que o coro vem para o despertar celebrativo matinal dos noivos, ao fim da noite e raiar do dia (Stehle, 1997, pp.\,279--80), como pensar prefiro aqui.} os moços solteiros\ldots{}\\*
traz teus coevos \ldots{}\\
para que do que a clarissonante \ldots{}\\*
um sono vejamos.
\end{verse}
\pagebreak
{\paragraph{Comentário} Tendo por fonte a mesma do Fr.\,27, o 30 igualmente revela uma linguagem muito característica do universo sáfico: a menção de moças virgens, o cantar e a festividade, destacados como temas dos fragmentos nesta antologia. Desse modo, pode"-se pensar como sendo da persona da líder do coro, Safo, a
voz que fala ao noivo, como se nota no início e no fim, e da celebração
festiva e ritual noturna, \textit{pannykhís} -- como a do Fr.\,23 --, nomeada no verso 2, no verbo \textit{pannykhísdoisai}. Em seguida, a \textit{persona} parece concentrar"-se no grupo masculino que do
banquete, antes da procissão nupcial, tomava parte separadamente do feminino,
ligado à noiva e também presente. O fragmento, como o 27, tem caráter
epitalâmico, mas sua estrutura métrica não condiz com as desse subgênero
mélico cujos fragmentos acham"-se adiante, em tópico próprio.}

\pagebreak
\section{Fragmento 43} 

\begin{gkverse}
\textnormal{[\textit{versos 1--3: ilegíveis e lacunares}]}

] [κ]αλος\\
         ]. ἄκαλα κλόνει\\
         ] κάματος φρένα\\
       ]ε̣ κ̣ατισδάνε[ι]\\
       ] ἀλλ’ ἄγιτ’, ὦ φίλαι,\\
       ], ἄγχι γὰρ ἀμέρα.
\end{gkverse}

\begin{verse}
\ldots{} belo \ldots{}\\
\ldots{} serenidade \ldots{} agita\\
\ldots{} fadiga \ldots{} sensos\\
\ldots{} senta-se \ldots{}\\
\ldots{} mas vamos, ó minhas queridas,\\
\ldots{} pois perto o dia.
\end{verse}

\medskip

{\paragraph{Comentário} Tendo por fonte o \textit{Papiro de Oxirrinco} 1232, fonte do Fr, 44, o 43 talvez seja destinado a uma  ritualístico \textit{pannykhís}, como o anterior e o Fr.\,23. Isso segundo a indicação do último de seus parcos versos que é o final da canção perdida, que fala da aproximação do amanhecer. Ressalto a presença de \textit{phílai}, designando possivelmente as meninas do coro de Safo, responsáveis pela \textit{performance}. A atividade ritual, como enfatizei na introdução e no comentário do Fr.\,94, por exemplo, é de todo compatível e própria às atividades de grupos corais femininos, como atestam os partênios -- as canções para virgens de Álcman, o poeta mélico da Esparta mais ou menos contemporânea a Safo.}


\pagebreak
\section{Fragmento 81}

\begin{gkverse}
\textnormal{[\textit{versos 1--3: ilegíveis e lacunares}]}

σὺ δὲ στεφάνοις, ὦ Δίκα, πέρθεσθ’ ἐράτοις φόβαισιν\\
ὄρπακας ἀνήτω συν<α>έρραισ̣’ ἀπάλαισι χέρσιν·\\
εὐάνθεα \dagger{}γὰρ πέλεται\dagger{} καὶ Χάριτες μάκαιρα<ι>\\
μᾶλλον \dagger{}προτερην\dagger{}, ἀστεφανώτοισι δ’ ἀπυστρέφονται.
\end{gkverse}

\begin{verse}
\ldots{} e tu, ó Dica, cinge teus cachos com amáveis guirlandas,\\
tramando raminhos de aneto com mãos macias;\\
pois mesmo as Cárites venturosas voltam-se ao florido,\\*
sobretudo, mas ao não-coroado dão as costas.
\end{verse}

\medskip

{\paragraph{Comentário} Ateneu, fonte também do Fr.\,141, declara sobre a tradição de adornar"-se:
``Safo expressa com mais simplicidade a razão de nossa prática
de usar guirlandas, dizendo isto'', em cena talvez de adorno da noiva, a jovem em evidência nos versos, Dica, em busca do favor das deusas Graças do charme sedutor.
Dica, preparada algo ritualisticamente, é tornada ``ainda mais amável, mais bela, e, portanto, mais desejável''\footnote{Bartol, 1997, p. 79.} para seu \textit{gámos}, sua boda. A intimidade -- ilusão da dicção sáfica -- da fala à jovem insere a cena possivelmente na preparação da noiva para o banquete nupcial.}


\chapter{Vestes e adornos}

\paragraph{\textsc{nota introdutória}}
No universo feminino tão privilegiado na mélica sáfica, recebem atenção vestes e
adornos, que tornam mais belas e atraentes as figuras contempladas. Isso se
nota no Fr.\,22, em que o vestido integra a apreensão erótica do objeto do
desejo de quem com os olhos o detém; e no Fr.\,44 em que a beleza da noiva que chega a
Troia se reflete no dote cheio de tecidos e ornamentos. Eis mais dois fragmentos.


\section{Fragmento 39}

\begin{gkverse}
πόδα<ς> δὲ\\
ποίκιλος μάσλης ἐκάλυπτε, Λύδι-\\
ον κάλον ἔργον.
\end{gkverse}

\begin{verse}
\ldots{} e cobria\\*
seus pés sandália furta-cor, belo\\*
trabalho lídio \ldots{}
\end{verse}

\medskip

{\paragraph{Comentário} Citado num escólio à comédia \textit{A paz},\footnote{Verso 1174.} 
de Aristófanes (séculos \textsc{v}--\textsc{iv} a.C.),
o fragmento faz menção à Lídia que para a Grécia exportava variados e altamente
elaborados produtos de luxo, como a sandália de uma figura feminina, nos versos acima.}


\pagebreak
\section{Fragmento 62}

\begin{gkverse}
’Επτάξατε̣ [\\
δάφνας ὄτα̣[

πὰν δ’ ἄδιον[\\
ἢ κῆνον ἐλο[

καὶ ταῖσι μὲν ἀ̣[\\
ὀδοίπορος ἄν[\ldots{}]\ldots{}[

μύγις δέ ποτ’ εἰσάιον· ἐκλ̣[\\
ψύχα δ’ ἀγαπάτασυ.[.́

τέαυτα δὲ νῦν ἔμμ̣[\\
ἴκεσθ’ ἀγανα[

ἔφθατε· κάλαν[\\
τά τ’ ἔμματα κα̣[
\end{gkverse}

\begin{verse}
Vós vos agachastes \ldots{}\\
de louro \ldots{}

e tudo mais doce \ldots{}\\
do que aquele \ldots{}

e a essas \ldots{}\\
viajante \ldots{}

e a custo certa vez eu ouvi; \ldots{}\\
e o ânimo dileto \ldots{}

e agora tais vestes \ldots{}\\ 
chegar \ldots{} gentil \ldots{}

primeiro viestes; bela \ldots{}\\ \EP[1]
e as vestes \ldots{}
\end{verse}

\medskip

\textls[-10]{\paragraph{Comentário} Preservado no \textit{Papiro de Oxirrinco} 1787, fonte do Fr.\,65 e de outros, o 62, cujo início e final são marcados, conjuga elementos que colocam em cena figuras femininas, como parece. Seria o coro o \textit{vós}, e sua líder, a voz em 1ª pessoa do singular? Seriam vestes e a memória de uma jovem já ida o foco?}


\section{Fragmento 92} 

\begin{gkverse}
\textnormal{[\textit{versos 1--2 e 14--6: ilegíveis e lacunares}]}

πέπλον[...]π̣υ̣σ̣χ[\\
καὶ κλ̣ε̣[...]σαω[\\
κροκοεντα[\\
πέπλον πορφυ[ρ......]δ̣εξω̣[\\
χλαιναι περσ.[\\
στέφανοι περ[\\
καλ[.]ο̣σ̣σ̣α̣μ̣[\\
φρυ[\\
πορφ[υρ
\end{gkverse}

\begin{verse}
\ldots{} peplo \ldots{}\\ 
\ldots{}\\
açafroado \ldots{}\\
peplo purpúreo \ldots{}\\
mantos persas \ldots{}\\
guirlandas \ldots{}\\
\ldots{}\\
purpúreo \ldots{}
\end{verse}

\medskip

{\paragraph{Comentário} Tendo por fonte o \textit{Papiro de Berlim 9722}, que nos trouxe o Fr.\,96, entre outros, o 92 contém algumas palavras em inícios de versos, mas todas ligadas a vestes e adornos que são objeto de cuidado nos grupos corais femininos e tema de suas canções, como mostram Álcman e seus partênios, e as muitas canções de Safo que já vimos. Seja pelo luxo, expresso no colorido, seja pela forma ou pela maciez -- qualidade que, criando um contraste, marca pelo adjetivo \textit{ábrois'(i)} o pano rústico, \textit{lasíois'(i)}, que bem recobre alguém na linha única e lacunar do Fr.\,100.}



\pagebreak
\section{Fragmento 125}

\begin{gkverse}
\dagger{}αυταόρα\dagger{} ἐστεφαναπλόκην
\end{gkverse}

\begin{verse}
\ldots{} eu mesma em meu tempo tecia \rlap{guirlandas \ldots{}}
\end{verse}

\medskip

{\paragraph{Comentário} Citado num escólio à comédia \textit{As tesmoforiantes},\footnote{Verso 401.} de Aristófanes, o fragmento faz menção a uma das atividades corais mais típicas das associações femininas, mencionada nos Frs.~81, 94, e a seguir, no 98. O escólio comenta que a prática liga"-se à juventude de mulheres de tempos antigos, como Safo.}



\chapter{Cleis}

\section{Fragmento 98 (\textsc{a--b})}

\begin{gkverse}
\ldots{}]θος· ἀ γάρ με ἐγέννα̣[τ

σ]φᾶς ἐπ’ ἀλικίας μέγ[αν\\
κ]όσμον αἴ τις ἔχη φόβα<ι>σ̣[\\
π̣ορφύρωι κατελιξαμέ [να

ἔ̣μμεναι μά̣λα τοῦτο .[\\
ἀ̣λλα ξανθοτέρα<ι>ς ἔχη[\\
τ̣α<ὶ>ς κόμα<ι>ς δάϊδος προφ[

σ]τεφάνοισιν ἐπαρτία[ις\\
ἀ̣νθέων ἐριθαλέων· [ \\
μ]ι̣τράναν δ’ ἀρτίως κλ[

π̣οικίλαν ἀπὺ Σαρδίω[ν\\
\ldots{}]. αονίας πόλ{ε}ις [

\hspace*{16mm}\adforn{47}
%\ast\quad\ast\quad\ast

-- σοὶ δ’ ἔγω Κλέι ποικίλαν [\\
-- οὐκ ἔχω -- πόθεν ἔσσεται; -- [\\
-- μιτράν<αν>· ἀλλὰ τὼι Μυτιληνάωι [
\end{gkverse}

\pagebreak

\begin{verse}
\ldots{} pois ela, a que me gerou \ldots{}


em sua época, era grande \\*
adorno, se alguém tinha os cachos\\*
atados em nó purpúreo;


era isso mesmo \ldots{} \\*
mas se alguém tinha a coma\\*
mais fulva que a tocha \ldots{},

com guirlandas [ornadas]\\*
de flores em flor \ldots{}\\*
Há pouco, [Cleis], uma fita

furta-cor de Sárdis \ldots{}\\*
\ldots{} cidades \ldots{}

Mas eu, a ti, Cleis, uma fita furta-cor \ldots{} --\\*
não tenho meios de tê-la; \\*
mas com o mitilênio \ldots{}

\hspace*{16mm}\adforn{47}

\ldots{} ter \ldots{}\\
\ldots{} furta-cor \ldots{}\\
estas coisas dos Cleanatidas \ldots{}\\
o exílio \ldots{}\\*
memoriais \ldots{} pois terrivelmente \rlap{devastado(a) \ldots{}}
\end{verse}

\pagebreak

{\paragraph{Comentário} Preservado em fontes do século \textsc{iii} a.C., o \textit{Papiro de Copenhagen} 301 (\textbf{a}) e o \textit{Papiro de Milão} 32 (\textbf{b}),
o fragmento é lido em chave biográfica, por conta do
nome de Cleis que, segundo fontes antigas, era filha da poeta; logo, o \textit{eu} seria
``Safo'', que se referiria, ainda, à sua mãe, a avó de Cleis, na
abertura. Há nele uma contraposição entre os tradicionais adornos, tidos como
elegantes no passado, e o adorno mais sofisticado e desejado no presente, a
\textit{mítra}, fita ou faixa adornada que cobria o cabelo, mas não as orelhas,
feita na Lídia, como a sandália do Fr.\,39 -- provas do influxo oriental na
cultura da ilha de Lesbos, ambas ditas ``furta"-cor'', na tradução do
adjetivo derivado do substantivo \textit{poikilía}, que carrega as ideias do
cintilar, da múltipla cor, do variegado, ou seja, da mistura de luz, formas ou
cor, que dificulta a direta e clara apreensão do objeto que se contempla, e que
justamente por isso associa"-se à sedução erótica e à astúcia. O \textit{eu} lamenta a
incapacidade de dar a Cleis a \textit{mítra}, devido à austeridade projetada no tempo presente, que contrasta com o passado. Que uma fita de cabelo Lídia ``possa chamar tanta atenção como um objeto capaz de perturbar os sonhos de uma jovem menina aristocrática é um dos sinais de um processo de aculturação que se espalhava pelas cidades eólias e jônicas da Ásia Menor e das ilhas próximas''.\footnote{Ferrari, 2010, p.\,5.} Mais: bem pode ser símbolo de \textit{status} e/ou de afinidades políticas. No fragmento, talvez por restrições políticas a \textit{mítra} esteja fora de alcance,  impostas pelo governo do ``mitilênio'', possivelmente
Pítaco, ou o adjetivo de um substantivo identificado a outra personagem. Há
ainda a menção ao exílio ``dos Cleanactidas'' ou ``de Cleanactides'', de memória
ainda viva em Lesbos, mas não sabemos como isso se ligaria ao
``mitilênio'', a Cleis, a ``Safo''. Cabe recordar que fontes antigas falam
de um período de exílio vivido por Safo em Siracusa, na Sicília (Magna Grécia),
por conflitos de seu grupo aristocrático com Pítaco, de quem falei no comentário do Fr.\,71, anteriormente.}



\pagebreak
\section{Fragmento 132}

\begin{gkverse}
Ἔστι μοι κάλα πάις χρυσίοισιν ἀνθέμοισιν\\
ἐμφέρη<ν> ἔχοισα μόρφαν Κλέις <  > ἀγαπάτα,\\
ἀντὶ τᾶς ἔγωὐδὲ Λυδίαν παῖσαν οὐδ’ ἐράνναν \ldots{}
\end{gkverse}

\begin{verse}
Tenho bela criança, portando forma símil\\*
à das áureas flores, Cleis, filha amada,\\*
por quem eu não [trocaria] toda a Lídia, nem a amável \ldots{}
\end{verse}

\medskip

{\paragraph{Comentário} Citado em Heféstion (15, 18s.), fonte do Fr.\,102, o 132 é o segundo e último em que vemos claramente o nome de Cleis, que o \textit{eu} introduz como sua filha, louvando-lhe a
beleza da juventude, na imagem das flores de ouro, e afirmando seu valor único no termo \textit{agapáta}, ``amada, querida'' -- que denota o que deve bastar ao contentamento. Por ela, nenhuma troca poderia ser feita, indica a elaboração do verso 3, nem pelo reino mais rico em ouro (a Lídia), nem, talvez, pela terra"-mãe, Lesbos.}



\chapter{Reflexões ético-morais}

\section{Fragmento 26} 

\begin{gkverse}
\textnormal{[\textit{versos 1 e 13--5: ilegíveis e lacunares}]}

ὄ]ττινα[ς γὰρ\\
εὖ θέω, κῆνοί με μά]λ̣ιστα πά[ντων\\
σίνονται̣

\textnormal{[\textit{versos 5--6: ilegíveis e lacunares}]}

           ].ιμ’ οὐ πρ[\\
  ]αι\\
  ] σέ, θέλω[\\
  ]το πάθη[\\
         ]αν, ἔγω δ’ ἔμ’ αὔται\\
τοῦτο συνοίδα
\end{gkverse}

\begin{verse}
\ldots{} pois aqueles a quem trato bem\\
são os que dentre todos sobretudo me\\
machucam \ldots{}\\
\ldots{} não \ldots{}\\
\ldots{} a ti, quero \ldots{}\\
\ldots{} sofrimentos \ldots{}\\
\ldots{} mas eu própria\\*
disso tenho consciência \ldots{}
\end{verse}

\medskip

{\paragraph{Comentário} A fonte do fragmento é o \textit{Papiro de Oxirrinco} 1231, como também do Fr.\,15. O que se lê parece centrar"-se numa discussão de caráter ético"-moral
a partir da experiência do \textit{eu} feminino que se mostra consciente da dolorosa
ingratidão, e não ingenuamente iludida.}



\pagebreak
\section{Fragmento 37}

\begin{gkverse}
κὰτ ἔμον στάλαγμον

\hspace*{16mm}\adforn{47}

τὸν δ’ ἐπιπλάζοντ’ ἄνεμοι φέροιεν\\
      καὶ μελέδωναι
\end{gkverse}

\begin{verse}
\ldots{} em minha dor \ldots{}

\hspace*{16mm}\adforn{47}

\ldots{} a ele que me censura, que o carreguem ventos
e anseios \ldots{}
\end{verse}

\medskip

{\paragraph{Comentário} A fonte é o tardio \textit{Etimológico genuíno}, como é o caso do 126. Dor e
rejeição à censura recebida e talvez indevida: eis o que canta o fragmento.}


\section{Fragmento 50}

\begin{gkverse}
ὀ μὲν γὰρ κάλος ὄσσον ἴδην πέλεται <κάλος>,\\
ὀ δὲ κἄγαθος αὔτικα καὶ κάλος ἔσ<σε>ται.
\end{gkverse}

\begin{verse}
\ldots{} pois o belo é belo enquanto se vê,\\*
mas o bom será de pronto também belo \ldots{}
\end{verse}

\medskip

{\paragraph{Comentário} Na \textit{Exortação à aprendizagem} (8.16), Galeno (século \textsc{ii} d.C.) cita esses dois versos
em que Safo contrapõe a beleza à bondade: ``Portanto, já que o auge da
juventude é como as flores primaveris, trazendo prazer de curta vida, é melhor
louvar a lésbia, quando ela diz''.}

\pagebreak
\section{Fragmento 52}

\begin{gkverse}
ψαύην δ’ οὐ δοκίμωμ’ ὀράνω \dagger{}δυσπαχέα\dagger{}
\end{gkverse}

\begin{verse}
\ldots{} não espero tocar o céu com meus dois braços \ldots{}
\end{verse}

\medskip

{\paragraph{Comentário} A fonte desse fragmento, como do 144, é Herodiano, no tratado \textit{Sobre
palavras anômalas} (2.912). Nele, o \textit{eu} afirma"-se consciente dos limites próprios da
mortalidade, que têm no alcance dos céus e no voo uma de suas imagens mais
fortes e recorrentes na tradição mítico"-poética grega.}


\section{Fragmento 58}

\begin{gkverse}
]ιμέναν νομίσδει\\
    ]αις ὀπάσδοι\\
ἔγω δὲ φίλημμ’ ἀβροσύναν,\qquad       ] τοῦτο καί μοι\\
τὸ λάμπρον ἔρος ἀελίω καὶ τὸ κά]λον λέλογχε.
\end{gkverse}

\begin{verse}
\ldots{} considera \ldots{}\\
\ldots{} concederia;\\
mas eu amo a delicadeza \ldots{} isso, e a mim\\
o desejo do sol deu por parte a luz e a beleza também.
\end{verse}

\medskip

{\paragraph{Comentário} Os dois últimos e mais legíveis versos\footnote{Versos 25--26.} do Fr.\,58, cuja fonte é o
\textit{Papiro de Oxirrinco} 1787, que preservou o 65, constam de outra, Ateneu (15.687\,\textsc{b}), que também cita o já visto Fr.\,141. Após a publicação em 2004 do
novo papiro de Safo -- que revelou uma nova canção fragmentária, em texto que se
sobrepõe a boa parte do que até então era editado como Fr.\,58 e que chamamos ``Canção sobre a velhice'', traduzida adiante --, concluiu"-se que aqueles
versos são independentes dos que os precedem e abririam uma nova canção. Eis acima, pois, a tradução do que seriam os versos 23--26 daquele papiro de Oxirrinco, que
trazem uma noção muito cara à mélica sáfica, da \textit{habrosýnē}, ``delicadeza'', de entranhada sensualidade em dimensão de sentido
estético, que alude a certo modo de vida aristocrático, marcado pelo luxo e pelo
refinamento que estimulavam a importação de objetos e costumes orientais
sob o signo de tais marcas. Note"-se a imagem do \textit{éros} do sol, que denota o desejo de viver, de vida, na solução para a tradução que me parece mais interessante ao contexto de citação.}



\section{Fragmento 120}

\begin{gkverse}
ἀλλά τις οὐκ ἔμμι παλιγκότων\\
ὄργαν, ἀλλ’ ἀβάκην τὰν φρέν’ ἔχω\ldots{}
\end{gkverse}

\begin{verse}
\ldots{} mas não sou das de têmpera rancorosa,\\*
mas tenho a mente serena \ldots{}
\end{verse}

\medskip

{\paragraph{Comentário} É fonte do fragmento o \textit{Etimológico magno} (2.43); nele, o \textit{eu} explica sua disposição.}


\section{Fragmento 148}

\begin{gkverse}
ὀ πλοῦτος ἄνευ ἀρέτας οὐκ ἀσίνης πάροικος\\
(ἀ δ’ ἀμφοτέρων κρᾶσις \dagger{}εὐδαιμονίας ἔχει τὸ ἄκρον\dagger{})
\end{gkverse}

\begin{verse}
\ldots{} a riqueza sem a excelência não é vizinha inofensiva,\\
mas a mistura de ambas traz a mais alta ventura \ldots{}
\end{verse}

\medskip

{\paragraph{Comentário} Um escólio a Píndaro\footnote{\textit{Ode olímpica} 2, verso 96\,\textsc{b}.} cita o fragmento abaixo,
que condena a separação entre a prosperidade material e a virtude ético"-moral.}

\pagebreak
\section{Fragmento 158}

\begin{gkverse}
σκιδναμένας ἐν στήθεσιν ὄργας\\
μαψυλάκαν γλῶσσαν πεφύλαχθαι
\end{gkverse}

\begin{verse}
\ldots{} a raiva espalhando-se \\*
no peito, proteger-se da língua tagarela \ldots{}
\end{verse}

\medskip

{\paragraph{Comentário} No tratado \textit{Sobre o refrear da cólera} (456\,\textsc{e}), Plutarco, fonte do Fr.\,49, cita o 158 como um conselho de Safo diante da seguinte situação:

\begin{quote}
Quando as pessoas estão bebendo, o que permanece silencioso é um peso
cansativo a seus companheiros; mas quando alguém está com raiva, nada é mais
digno do que a quietude, como exorta (\textit{paraineî}) Safo. Esse contexto realça o elemento parenético, muito disseminado na poesia grega arcaica e clássica, por sua natureza oral e pragmática, em essência, de estreitos laços com a vida cotidiana e função paidêutica -- no caso, junto ao coro de meninas da poeta.
\end{quote}}


\chapter{«\,Canção sobre a velhice\,»}

\paragraph{\textsc{nota introdutória}}
Como observei na anotação ao Fr.\,58, este foi quase totalmente reeditado com a 
publicação, em 2004, do \textit{Papiro de Colônia} 21351 (início do século \textsc{iii} a.C.), 
cujo texto se sobrepunha a boa parte daquele fragmento.\footnote{Tradução e comentários embasados na edição do texto grego dada em Buzzi \textit{et alii} (2008, p.~14) e em Greene e Skinner (2009 pp.~11 e 14--15), com a possibilidade de alguns suplementos sugeridos por Martin L.\,West (``The new Sappho''. \textit{\textsc{zpe}} 151, 2005, pp. 1--9), indicados no texto traduzido entre parênteses, como sempre nos fragmentos desta antologia. Tais suplementos são: o nome das Musas no primeiro verso preservado, cujo colo é qualificado; os adjetivos \textit{hápalos} e \textit{lêukai}, para adjetivar como ``tenra'' (\textit{ápalon}) a ``pele'' (\textit{khróa}) e ``brancos'' os cabelos de quem envelhece. As canções do novo papiro foram publicadas pela primeira vez por M.\,Gronewald e R.\,W.\,Daniel (``Ein neuer Sappho-Papyrus''. \textit{\textsc{zpe}} 147, 2004, pp. 1--8).} No início, a 
linguagem metapoética desenha uma cena de canto junto a ``meninas'', como há que entender o termo \textit{paîdes},\footnote{Verso 1.} considerada a associação liderada por Safo, 
integrada pelo favorecimento de divindades que devem ser as Musas,
dado o tema dos versos da abertura preservada. Em seguida, o tema da velhice
vem à tona e, com ele, a reiterada imagem dos cabelos que se tornam grisalhos,
e a provável referência à perda do frescor do corpo ou da pele, como parece cantar o precário Fr.\,21; depois, os
tormentos e preocupações, e a dificuldade de movimentação, que contrasta
drasticamente com a leveza de joelhos dançantes na juventude. Esse quadro é
motivo de dor, mas é inexorável e inelutável; daí a expressão do sentimento
agudo de impotência resignada e consolada contra o que é próprio da natureza mortal: envelhecer. Para
ilustrar tal verdade, em passagem gnômica, o \textit{eu} que dramatiza a líder do coro de meninas recorda o
mito da paixão de Eos, a Aurora, pelo jovem mortal troiano Titono, para o qual é nossa fonte
principal o \textit{Hino homérico \textsc{v}, a Afrodite },\footnote{Versos 218--238.} datado do
século \textsc{vii} a.C., e de autoria anônima. Em síntese, esse mito conta como a
deusa, tomada de paixão, pediu a Zeus que tornasse imortal seu amado por ela abduzido e levado ao Olimpo; a deusa,
porém, esqueceu"-se de que a mortalidade do homem não se concretiza apenas na
morte, mas na velhice. Assim, Titono tornou"-se imortal, mas imortalmente,
eternamente velho. Ora, sendo a paixão suscitada e sustentada pela
beleza do corpo e pela sua capacidade de atração, a esfera da velhice é
inadequada aos dons de Afrodite. Eos, então, acaba por se desinteressar por
completo do mortal, a quem encerra num quarto, do qual ressoa sem cessar sua
voz -- em certas tradições, esta é uma referência à metamorfose de Titono em
cigarra --, a ecoar de seu débil e velho corpo sempre a minguar. Merece nota o fato de
que Safo, com esse fragmento, ganha relevo na série de textos poéticos da
Grécia arcaica que trata da velhice, em gêneros variados, com ênfase na
decadência física que, no célebre Fr.\,1 -- uma elegia talvez completa -- de
Mimnermo, ganha conotação ético"-moral, e é descrita como obstáculo
intransponível à participação na esfera de Afrodite.
Dado o ingrediente erótico do mito que também esse poeta de meados do século
\textsc{vii} a.C. recorda noutra elegia -- o Fr.\,4, em que o presente de Titono é
julgado pior do que a própria morte, dada a primazia da paixão na sua perspectiva
--, é possível que a soma do canto e dança à
velhice e à referência mítica no novo fragmento de Safo apontem para um cenário
em que também Afrodite e/ou seu universo tomassem parte. Digno de nota, ainda,
é o modo como Eos se caracteriza, recordando a imagem homérica da Aurora
\textit{rhododáktylos}, ``dedirrósea'', no
epíteto composto \textit{brodópakhyn}, no Fr.\,53 atribuído às Cárites.
Ampliando o epíteto homérico, Safo amplia os rasgos
róseos"-avermelhados"-alaranjados do amanhecer, traçados não por dedos, mas pelos
braços que tanto desejaram enlaçar Titono, enquanto jovem e belo foi seu
corpo. Digna de nota, por fim, é a ressonância do Fr.\,26 de Álcman no novo de Safo, em que a voz autodramatizada do poeta -- como o é a voz de Safo na canção que traduzo -- canta às \textit{parthénoi}, as virgens de seu coro, a dor de seu próprio envelhecer que o impede de seguir acompanhando"-as nas \textit{performances} das canções.

\begin{gkverse}
ἰ]ο̣κ[ό]λ̣πων κάλα δῶρα, παῖδες,\\
        τὰ]ν̣ φιλάοιδον λιγύραν χελύνναν·

] π̣οτ̣’ [ἔ]ο̣ντα χρόα γῆρας ἤδη\\
         ἐγ]ένοντο τρίχες ἐκ μελαίναν·

βάρυς δέ μ’ ὀ [θ]ῦμο̣ς ̣πεπόηται, γόνα δ’ [ο]ὐ φέροισι,\\
τὰ δή ποτα λαίψηρ’ ἔον ὄρχησθ’ ἴσα νεβρίοισι.

τὰ <μὲν> στεναχίσδω θαμέως· ἀλλὰ τί κεν ποείην;\\
ἀγήραον ἄνθρωπον ἔοντ’ οὐ δύνατον γένεσθαι.

καὶ γάρ π̣[ο]τ̣α̣ Τίθωνον ἔφαντο βροδόπαχυν Αὔων\\
ἔρωι φ̣.. ̣ ̣α̣θ̣ε̣ισαν βάμεν’ εἰς ἔσχατα γᾶς φέροισα[ν,

ἔοντα̣ [κ]ά̣λ̣ο̣ν καὶ νέον, ἀλλ’ αὖτον ὔμως ἔμαρψε\\
χρόνωι π̣ό̣λ̣ι̣ο̣ν̣ γῆρας, ἔχ[ο]ν̣τ̣’ ἀθανάταν ἄκοιτιν.
\end{gkverse}

\begin{verse}
\ldots{} [das Musas] de violáceo colo os belos dons, meninas,\\
\ldots{} a melodiosa lira, amante do canto;

[tenra] outrora, agora é a pele da velhice,\\*
\ldots{} os cabelos, de negros (brancos) se tornaram.


Pesado se me fez o peito, e os joelhos não me suportam -- \\
os que um dia foram lépidos no dançar, quais os da corça.

Isso lamento sem cessar, mas que posso fazer?\\
O não-envelhecer não é possível ao ser humano.

Pois, certa vez, dizem que Eos de róseos braços,\\
com paixão \ldots{} carregando Titono aos confins da terra,

belo e jovem que era; mas mesmo a ele alcançou similmente\\
em tempo a grisalha velhice -- a ele que tinha imortal esposa.
\end{verse}

\chapter{Canto, velhice: um convite}


\section{Fragmento 21}

\begin{gkverse}
\textnormal{[\textit{versos 1--2: ilegíveis e lacunares}]}

]α̣νδ’ ὄλοφυν [....]ε̣.\\
] τρομέροις π.[..]α.λλα\\
]\\
  ] χρόα γῆρας ἤδη\\
   ]ν ἀμφιβάσκει\\
     ]ς πέταται διώκων\\
     ]\\
     ]τας ἀγαύας\\
   ]ε̣α, λάβοισα\\
   ] ἄεισον ἄμμι\\
τὰν ἰόκολπον\qquad         ]\\
        ]ρ̣ων μάλιστα\\
         ]ας π[λ]άναται
\end{gkverse}


\begin{verse}
\ldots{} lamento \ldots{}\\
\ldots{} trêmulos \ldots{}\\
\ldots{} a pele da velhice \ldots{}\\
\ldots{} em redor \ldots{}\\
\ldots{} voa, perseguindo\\
\ldots{} brilhante\\
\ldots{} ele pegando\\
\ldots{} canta tu a nós\\
a de violáceo colo \ldots{}\\
\ldots{} sobretudo\\
\ldots{} vagueia \ldots{}
\end{verse}

\medskip

{\paragraph{Comentário} Preservado no \textit{Papiro de Oxirrinco} 1231, em que estão vários
outros fragmentos vistos desde o Fr.\,15, o precário texto traz uma
cena em que, como podemos sugerir, a \textit{persona} da líder do grupo
faz um convite ao cantar, dirigindo"-se a uma das coreutas
imperativamente, após lamentar a própria velhice. Ressoam aqui os versos
da canção do novo fragmento que recorda o mito de Titono e Eos,
justamente a propósito do inevitável envelhecer aos seres humanos.
Naquele fragmento, o envelhecimento é temível à \textit{persona} da líder que canta lamentosamente às
coreutas a chegada do que será impeditivo ao dançar, ressaltando as
marcas no corpo -- pele, cabelos, joelhos -- e o peso das
preocupações e ansiedades no peito. Neste Fr.\,21, é possível que a
velhice seja obstáculo ao cantar que honra a figura de ``colo violáceo''
-- talvez uma das Musas pela mesma qualidade provavelmente referidas na
``Canção sobre a velhice'', ou Afrodite, ou uma coreuta
ou, como no Fr.\,30, uma noiva cujo colo é qualificado pelo adjetivo \textit{iókolpon}, ``de violáceo colo''. Se, porém, não pode a líder cantar, que
cante a ela, às demais \textit{parthénoi} do coro e à audiência uma das
coreutas, uma das jovens de seu grupo. Afinal, parece dizer a poeta, a
\textit{performance} não deve parar.}



\chapter{De cantos, cordas, prêmio: imortalidade?}

\paragraph{\textsc{nota introdutória}}
Do mesmo papiro que preservou a ``Canção sobre a velhice'' vem, de mais legível, um outro novo fragmento breve, do qual temos os versos finais. Neles, estão presentes festa, o canto da \textit{persona} e instrumentos de corda. Precisamente, a ``harpa'', \textit{pâktin}, que referida no Fr.\,22, e talvez a \textit{khelýnna}, um tipo de lira nomeado pela casca da tartaruga que lhe servia de caixa de ressonância e que designa o instrumento na ``Canção sobre a velhice''.
Tais cordófonos se combinam, ressoando as canções que já aqui ouvimos -- incluindo aquelas que cantam o próprio cantar. O ``prêmio'', \textit{géras}, talvez seja a ``grande fama das Musas'', \textit{kléos méga Moíseion}. No presente da canção, tempo enfatizado por duas vezes, emergem as ideias da morte, do estar sob a terra, e do estar vivo, sobre ela. Haveria no precário texto algo relativo à contemplação de Safo no Hades, morta, e a sua imortalização pela poesia, como vimos nos Frs.~55, 65 e 147? A resposta seria sim, e decerto, neste caso, ``prêmio'' maior e mais caro não há à poeta que continuamos a cantar:\footnote{Texto grego nas edições dadas em Buzzi \textit{et alii} (2008, pp.~21 e 57) e Greene e Skinner (2009, p.~10). As sugestões do suplemento que dá sentido ao ``prêmio'', aceita naquela, e do que traz a \textit{khelýnna} ao verso final, aceita nesta, baseiam-se nos estudos de Gronewald e Daniel, e de West, referidos à nota à ``Canção sobre a velhice''.}

\pagebreak

\begin{gkverse}
\textnormal{[\textit{versos 1--4: ilegíveis e lacunares}]}

      ] ε̣ὔχο̣μ[\\
]. νῦν θ̣αλ̣[ί]α γε̣[\\
] ν̣έρθε δὲ γᾶς γε̣[νοίμα]ν̣·\\
]..ν̣ ἔχο̣ισαν γέρας ὠς [ἔ]οικεν,\\
]ζοεν ὠς νῦν ἐπὶ γᾶς ἔοισαν.\\
] λιγύραν, [α]ἴ ̣κεν ἔλοισα πᾶκτιν\\
        χε]λύν̣ν̣αν̣.αλαμοις ἀείδω. 
\end{gkverse}

\begin{verse}
\ldots{} rezo \ldots{}\\
\ldots{} agora festividade \ldots{}\\
\ldots{} sob a terra viria a ser;\\
\ldots{} tendo prêmio como é justo\\
\ldots{} como agora sobre a terra estando\\ 
\ldots{} se clara harpa agarrasse\\
\ldots{} lira \ldots{} eu canto.
\end{verse}



%\input{\printindex}


\printindex

\end{document}
