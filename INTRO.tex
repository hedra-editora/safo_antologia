\chapter*{Introdução \subtitulo{Safo revisitada: viagem pela\break poesia grega
antiga}}
\addcontentsline{toc}{chapter}{Introdução, \textit{por Giuliana Ragusa}}
%\protect\footnote{\MakeUppercase{A}mparam estas páginas os seguintes estudos:
%\MakeUppercase{R}agusa (2005, pp.~23--53, 55--78; 2010, pp.~23--53, 55--97).}}

%\chapter[Prefácio, \textit{por Celso Lafer}]{Prefácio \subtitulo{O próximo e o distante}}


\begin{flushright}
\textsc{giuliana ragusa}
\end{flushright}

% \section*{Safo revisitada: viagem pela\break poesia grega
% antiga\protect\footnote{\MakeUppercase{A}mparam estas páginas os seguintes estudos:
% \MakeUppercase{R}agusa (2005, pp.~23--53, 55--78; 2010, pp.~23--53, 55--97).}}

{\setlength{\epigraphwidth}{.7\textwidth}
\epigraph{Estas mulheres, divas línguas, o Hélicon nutriu -- e o  \\
rochedo macedônio de Piéria -- com hinos: \\
Praxila, Mero, Anite eloquente, feminino Homero, \\
Safo, adorno das lésbias de belos cachos, \\
Erina, Telesila mui gloriosa e tu, Corina, \\
o impetuoso escudo de Atena cantando, \\
Nóssis de feminina língua, e Mirtes, doce de ouvir --  \\
todas fazedoras de eternos escritos. \\
Nove Musas do grande Urano, e nove mesmas \\
Gaia pariu, para a imperecível alegria dos mortais.}
{\textsc{antologia palatina}\footnotemark{}\\\textit{Livro \textsc{ix}, epigrama 26, de Antípatro de Tessalônica}\footnotemark{}}
\addtocounter{footnote}{-2}
\stepcounter{footnote}\footnotetext{Daqui para frente, \textit{\textsc{ap}}.}
\stepcounter{footnote}\footnotetext{Séculos \textsc{i} a.C. a \textsc{i} d.C. Tradução de Ragusa (2005, p.\,57; 2020, p.\,116), com pequenas alterações. Todas as traduções, salvo quando indicado, são minhas. Os textos gregos dos epigramas são da edição de Paton (1916--1918), em cinco volumes. Embasam esta introdução meus trabalhos prévios listados na bibliografia.}
}

\noindent{}Ao contrário do que faz supor o mito -- e quando se trata da poeta a que se
dedica este livro, mito e realidade se confundem sem cessar, mal se
distinguindo entre si --, Safo não é o único nome feminino da poesia da Grécia
	antiga, mas de sua primeira fase histórica, a arcaica\footnote{ O adjetivo é
	usado no sentido de “antiga, remota”; a era arcaica é, por assim dizer, a mais
	antiga da Grécia antiga, e divide"-se em duas etapas: a arcaica, até \textit{c.}
	550 a.C., e a tardo"-arcaica, \textit{c.} 550--450 a.C.; ver Shapiro (2007, pp.\,1--3) a respeito.} (\textit{c.} 800--480 a.C.). Nascida em 630~a.C., de família
aristocrática, na costeira Ereso, oeste da ilha de Lesbos, ela viveu na
proeminente Mitilene, costa oriental, contemporaneamente ao poeta e guerreiro
Alceu. Ambos são os primeiros poetas lésbios dos quais sobreviveram, para cada
um, corpos de obra substanciosos; suas práticas, porém, se beneficiaram,
ressalta Angus \textsc{m}.\,Bowie, de uma forte e bem reputada tradição
poética lésbio"-eólica, em que se inserem nomes como os dos célebres citaredos
Terpandro (séculos \textsc{viii}--\textsc{vii} a.C.) e Árion (séculos
\textsc{vii}--\textsc{vi} a.C.), que levaram a
outras geografias do mundo grego, e a dois polos culturais da era arcaica --
	Esparta e Corinto --, suas práticas métrico"-musicais.\footnote{ Bowie (1984,
	pp.\,7--10).} Mais não podemos dizer,
pois do primeiro há só dois fragmentos de autoria duvidosa, e do segundo, nada
resta. De todo modo, a relevância dessas figuras e o peso que conferiram a uma
tradição lésbio"-eólica de canção bem conhecida e firmada se
fazem sentir na imagem que os antigos projetaram de Terpandro, tido como
inovador da música grega num século \textsc{vii} a.C. de ricas experimentações, e
inventor da lira de sete cordas, algo que a arqueologia prova insustentável,
uma vez que o instrumento era já conhecido no mundo minoico"-micênico, que
antecede o que chamamos “Grécia histórica”. E Árion é dado como o
poeta do ditirambo, canção de forte aspecto narrativo.

Safo e Alceu são, ainda, dois dos nomes notáveis de um gênero poético, a
\textit{mélica} ou a lírica propriamente dita -- a canção destinada à
\textit{performance} em solo ou coral,
com o acompanhamento da lira (e de outros instrumentos e da dança, na
modalidade em coro). Se falo em \textit{performance} é porque, recorde"-se
desde já, sobretudo no período arcaico e depois no clássico (\textit{c.}
480--323 a.C.), pelo menos até \textit{c}. 400 a.C., a poesia grega é eminentemente de tradição oral e inserida no
que John Herington chama de ``cultura da canção'',\footnote{ Herington (1985, p.\,3).} na
qual, recitada ou cantada numa ocasião de \textit{performance}, disseminava
“ideias morais, políticas e sociais”. A oralidade, portanto, marca
a composição e a circulação dessa poesia em \textit{performances} e
\textit{reperformances} profissionais e/ou amadoras a determinada
audiência, de certo modo, em dada ocasião, colocada assim em ligação estreita
com a vida cotidiana da comunidade em que se fazia e pela qual passava, ligação
esta que lhe confere um caráter em essência pragmático. A mélica grega,
como bem ressalta Bruno Gentili, “não foi intimista, no senso
moderno”,\footnote{ Gentili (1990b, p.\,9).} uma vez que existia integrada na
vida da comunidade em meio à qual
circulava oralmente. Não por acaso, a voz poética, apresentada numa
situação de diálogo entre o \textit{performer} e sua audiência, está sempre em
diálogo: em vez de falar consigo mesmo ou a ninguém, o \textit{eu/\,nós} sempre
se dirige ao outro, ao \textit{tu/\,vós} com que estabelece a interlocução. Se por
vezes esta não nos é de todo discernível, isso se deve aos problemas
materiais de preservação dos textos. Ora, o diálogo, dimensão viva da
comunicação verbal humana, é um elemento crucial da oralidade, incorporado com
grande força aos gêneros poéticos da Grécia antiga desde a épica homérica e
seus poemas monumentais, a \textit{Ilíada} e a \textit{Odisseia}, em que há uma
divisão quase equivalente entre narrativa de ação e interação verbal dialogada entre as personagens. 

A oralidade se evidencia na composição da mélica, que se vale regularmente de
estruturas e procedimentos estilísticos de caráter mnemônico, que,
de maneira mais flagrante na era arcaica, refletem a tradição poética oral,
mesmo que já possamos pensar, naquele momento, no uso da escrita -- o alfabeto
grego, adaptação do fenício, se disseminava desde fins do século \textsc{ix} a.C. --
pelos poetas nos processos e técnicas de construção de seus
versos.\footnote{ Ver a respeito Gentili (1990a, pp.\,14--23) e Svenbro (1993, pp.\,27--30).} Pensando o caso de Safo, Jesper Svenbro acredita
que ela teve seus textos escritos à sua época e com sua interferência direta,
ela mesma os escrevendo -- algo que pode ser excessivo e que não podemos
comprovar; de todo modo, afirma ele:

\begin{quote}
Um grego que vivesse por volta de 600 a.C., se refletisse sobre o problema de
registrar o poema sob a forma escrita, provavelmente consideraria a questão em
termos de uma \textit{transcrição} de algo que já tinha uma existência
socialmente reconhecida e que tenha sido tecnicamente controlado num estado
oral ou memorizado. Considerar a transcrição como uma operação que tornava o
poema duradouro e famoso não seria necessário; a tradição oral era bastante
capaz de fazer isso, sem o auxílio da escrita.\footnote{ Svenbro (1993, pp.\,145--59).}
\end{quote}

Em outras palavras, ainda que aceitemos a possibilidade de que Safo e outros
poetas arcaicos, principalmente, tenham feito uso da escrita, o estudo atento
aos elementos estruturais e estilísticos de suas obras dá a perceber que a oralidade as gera e
sustenta.

\section*{O problemático nome «\,lírica\,»}

Decorre do modo de \textit{performance}, justamente, o nome tardio para a mélica,
\textit{lírica}, que prevalecerá na referência moderna a tal gênero somado a outros, elegia e jambo, que compõem, juntos, um variado corpo de
textos de vários autores, metros, dialetos, tradições culturais, temas,
\textit{performances}, tons -- que nem são poesia hexamétrica, como a
épica --, nem dramática -- tragédia e comédia. Esse uso moderno do nome é decerto
prático, mas acaba por vestir, com um mesmo manto,
gêneros poéticos autônomos e distintos que os antigos jamais unificaram ou confundiram, o elegíaco, o mélico e o jâmbico, por vezes tratados como subgêneros da lírica -- um grave equívoco de importantes consequências para seu entendimento. \textit{Mélica}, essa palavra não
dicionarizada em nosso vernáculo, é o termo que os antigos identificavam à
\textit{lírica}, rigorosamente, o gênero da canção para a lira e tão"-somente ele. 

Fica, então, exposta a primeira armadilha de “lírica”, que por isso grafo
entre aspas: suas acepções antiga e moderna não são correspondentes uma
à outra. A segunda reside no fato de que essa mesma designação é empregada para
um gênero de
poesia moderna -- fruto de uma cultura da escrita. Isso cria uma falsa impressão
de familiaridade no que tange à poesia antiga, que, se não revertida, acarreta
para sua leitura um olhar modernizante potencialmente equivocado, sobretudo se
guiado pelas expectativas de uma noção comum -- não menos errônea até para certa
poesia moderna -- sobre a “lírica” e o “lírico”. Tal noção agrega as ideias da
brevidade, da subjetividade -- amparada na abundante e intensa constância da
primeira pessoa do singular --, da explosão dos sentimentos. Pautar"-se por
ela, todavia, é esquecer o filtro da dimensão estética, que faz com que
experiências e sentimentos, ainda que pessoais, conhecidos, vividos pelo poeta,
passem pelo processo da elaboração artística. Nele, a linguagem é trabalhada
estilisticamente, formalmente, transformando a experiência ou a emoção -- pouco
importa, a rigor, se vividas ou não pelo poeta -- em experiência
\textit{representada} no presente da composição. Há, portanto, uma filtragem
que impõe um distanciamento que precisa ser observado, mesmo que a voz poética
assuma o nome do próprio poeta, pois, ainda assim, estamos diante de sua
\textit{persona}, elaborada em linguagem artística diversa -- não importa
quão próxima ela se pretenda -- da cotidiana.

Se é preciso atentar para tudo isso já na poesia moderna, mais ainda o é na antiga, em que, por sua composição genérica, devem
estar articuladas as escolhas que faz o poeta da matéria, do metro, da ocasião
e do modo de \textit{performance}, bem como da linguagem, do caráter, do tom, e
assim por diante. Noutras palavras, a composição dessa poesia -- que, à diferença
da moderna, tem caráter pragmático -- centra"-se no gênero, cujas regras,
na época arcaica, são tradicionalmente preservadas e praticadas, sem que
estejam escritas, o que nos obriga a pensar os gêneros de poesia arcaica e
clássica menos como identidades de severas leis fixas e mais como
“tendências firmadas o suficiente para permitir que afinidades e influências
sejam discerníveis”, observa Chris Carey, gerando
“expectativas na audiência”, mas sem prejuízo da flexibilidade que dá margem “à
frustração e à redefinição de tais expectativas” pelo poeta e pela sua
audiência.\footnote{ Carey (2009, p.\,22).}

Claro está, a esta altura, que há uma defasagem considerável, para não dizer
enorme, entre nossas prática e cultura literárias e as antigas, às quais não
podem ser associados, sem grande prejuízo para a compreensão e leitura dos
textos -- na letra fria, artificial e estática de uma poesia feita para a
\textit{performance} em viva voz -- e de seus universos, conceitos relativos à
ideia moderna de “literatura”, tais como originalidade e criatividade. A poesia
elegíaca, jâmbica e mélica da Grécia arcaica, sobretudo, e clássica, resume
Anne \textsc{p}.~\,Burnett: 

\begin{quote}
{[}\ldots{}{]} é mais engenhosa e menos apaixonada,
mais convencional e menos individual do que desejariam os que advogam essa noção {[}da explosão da individualidade{]}.\footnote{ Burnett (1983, p.\,2).} 
\end{quote}

Essa poesia antiga, oral e de
ocasião, “fincada no sistema social de uma \textit{pólis} grega arcaica”,
recorda Wolfgang Rösler, é essencialmente discurso.\footnote{ Ver
Rösler (1985, p.\,139), Johnson (1982, p.\,72) e Clay \mbox{(1998, p.\,11)}.}

Em conclusão a essas palavras, cabe esclarecer que não se trata, aqui, de negar
qualquer medida de identificação entre o poeta e seu \textit{eu} poético, mas de
afirmar que, ao lidarmos com os antigos, o somatório do modo de composição, da
falta de conhecimento sobre a biografia dos poetas e seus contextos
histórico"-sociais, e da precariedade mais ou menos intensa dos textos
sobreviventes, torna essa identificação ainda mais complexa do que é.
E o terceiro ingrediente dessa soma não é de importância menor, como
bem mostra este aviso aos que buscam a obra de Safo que, reunida com rigor,

\begin{quote}
contém apenas um poema completo, aproximadamente dez fragmentos substanciais,
uma centena de citações breves de autores antigos e cerca de 50 peças de textos
em papiro, que emergiram das areias do deserto egípcio.
\end{quote}

Daí ser “mais exato
falar em fragmentos de Safo”,\footnote{ Lardinois (1995, p.\,29).} e não em
poemas. Essa síntese citada de André Lardinois é acurada, e de modo algum restrita ao
caso da poeta de Lesbos.
Antes, o cenário da preservação das obras de outros poetas da poesia jâmbica, da elegíaca e da mélica é similar ao ou pior do que
o da obra de Safo. A propósito disso, cito a fala marcante de Walter \textsc{r}.\,Johnson:

\begin{quote}
Nós todos, quando lemos a lírica grega, ficamos desapontados em certo sentido:
não porque a poesia não impressione -- antes, é supremamente bela --, mas
porque existe para nós apenas em cacos e farrapos. {[}\ldots{}{]} E quando a comparamos
aos outros remanescentes da literatura e da cultura gregas, essas ruínas são de
machucar o coração. Nenhuma experiência de leitura, talvez, é mais deprimente e
mais frustrante do que a de abrir um volume dos fragmentos de Safo e
reconhecer, ainda uma vez -- pois sempre se espera que desta vez seja diferente
--, que essa poesia está perdida para nós.

Esse é um fato que escolhemos não encarar -- não de frente e constantemente.
Logo, divisamos uma ficção -- {[}\ldots{}{]} que chamamos poesia lírica grega --,
mesmo que saibamos que ela está em meros fragmentos. Na verdade,
\textit{porque} sabemos que está em meros fragmentos, agimos, falamos
e escrevemos como se o impensável não tivesse acontecido, como se bispos pios,
monges descuidados e ratos famintos não tivessem consignado Safo e seus colegas
líricos ao esquecimento irremediável. {[}\ldots{}{]} Naturalmente, qualquer helenista a
quem você perguntar admitirá o fato da fragmentação. Pode até ter prazer em
descrever o estado verdadeiro dos textos e a incerteza fascinante de
restaurações e conjecturas. Mas se você persistir em seu escrutínio e conversar
sobre a poesia, sobre os poemas não existentes, à medida que a conversa
esquentar, o esqueleto ganhará carne e cor, e a ruína se esvaecerá. Isso não é
prevaricação ou enganação, isso é a natureza humana: nós queremos aqueles
poemas e, nos momentos em que nos desarmamos, nós os imaginamos de volta à
existência. {[}\ldots{}{]}

É o leitor, então, que deve se lembrar, quando eu me esquecer, que a lírica
grega {[}\ldots{}{]} nos é essencialmente inacessível.\footnote{ Johnson (1982, pp.\,25--6).}
\end{quote}

Nesta introdução e na tradução que se lhe seguirá --
para não esquecer essa realidade e para que o leitor dela tenha consciência,
acrescento à síntese de Lardinois esta outra: aos que procuram a poesia de
Safo, restarão seus fragmentos vivos, desafiadores, pulsantes há séculos, a
despeito de sua fragilidade material; aos que procuram Safo, a mulher, ou Safo,
a \textit{lésbica}, restará pouco mais do que ficções e anedotas. Penso que,
entre a substância não obstante precária dos fragmentos e a névoa tão sedutora
quanto impalpável da biografia, mais profícuo será privilegiar a primeira
opção. Mas falemos um pouco da névoa sáfica.


\section*{Em busca de Safo, poeta de Lesbos}

Um papiro encontrado na antiga cidade egípcia de Oxirrinco, próxima a
Alexandria, isto nos conta da vida de Safo:\footnote{\textit{Papiro de Oxirrinco} 1800,
fr.\,1, século \textsc{ii} ou início do \textsc{iii} d.C.} seu pai seria Escamandro ou
Escamandrônimo; seus irmãos, Cáraxo, o mais velho, Erígio e Lárico, o mais
novo; sua filha, Cleis, que levaria o nome da mãe de Safo ``foi acusada
por alguns de ser irregular e amante de mulheres”; era feia, mirrada e de
compleição escura. Já o léxico bizantino \textit{Suda}, compilado no século \textsc{x},
no verbete à poeta repete o segundo nome para o pai de Safo, acrescentando à
lista outros sete; reitera o nome Cleis como sendo de sua mãe e de sua filha, e
os de seus três irmãos; diz ainda que Safo teria sido casada; que teria mantido
amizades impuras com jovens meninas, como Átis, e, por isso, adquirido má
reputação; e teria tido pupilas, como Gongila. Alguns desses nomes se registram
na obra de Safo, ou nas fontes que as preservaram. Mas, como se vê, os
testemunhos antigos se retomam uns aos outros, discordando aqui e ali, tornando
mais intricada a rede de inconsistências, e embasando"-se, claramente, na
leitura dos fragmentos de Safo, em circularidade viciosa. O pequeno painel
biográfico composto pelo papiro e pelo léxico é, pois, indigno de confiança,
para dizer o mínimo, mesmo que associado a outros testemunhos.\footnote{ Na
edição bilíngue de Campbell (1994; 1ª ed.: 1982), são arrolados 61 testemunhos
sobre a poeta e sua obra. O volume é, \mbox{comparado} aos testemunhos de outros
poetas, considerável; de seu contemporâneo Alceu, há 27 testemunhos na mesma edição. 
Quando me referir a testemunhos sobre Safo, uso
sempre a compilação de Campbell. A propósito dos problemas da biografia de Safo e de suas evidências, ver Ragusa (2019b, pp.\,211--239).}

O mistério, então, persiste e as perguntas que estimula -- e que com frequência
ganham dimensão desproporcional à própria inviabilidade de solução
corroborada em sólidas evidências -- dificilmente podem ser respondidas: quem
foi Safo e que figura portava? Como se fez a poeta? Como construiu e fez
circular suas canções? Como viveu na Lesbos arcaica? Sustenta"-se a imagem da
Safo \textit{lésbica}? Vejamos um pouco do muito que se diz, e do
pouco que podemos dizer.

Embora seja a mais velha das poetas mulheres listadas no epigrama que abre esta
introdução, atuante numa Grécia da oralidade, em quem a escrita ainda estava
por se consolidar como veículo principal de produção artística, Safo é aquela
cujo \textit{corpus} é o mais extenso\footnote{Cerca de duzentos fragmentos, um dos quais é,
na verdade, uma canção completa, o “Hino a Afrodite”.} e cujos testemunhos são
os mais numerosos e espalhados pelos séculos. Será a poeta
privilegiada pela sorte, ou sua reconhecida superioridade garantiu sua
preservação mais generosa do que a de outros?
Impossível simplificar assim a explicação:
nem a mera sorte, nem a mera fama podem ser responsabilizadas exclusivamente por
nossa boa ou má fortuna quanto ao estado e volume do \textit{corpus}
remanescente de cada um dos antigos poetas.

Assim sendo, o fato de que Safo é a única poeta do período
arcaico não deve ser superestimado de nenhuma maneira, inclusive por não sabermos se houve outras poetas em seu tempo e em sua Lesbos. Afinal, estamos no mundo da prevalente oralidade que seria ainda mais acentuada em épocas anteriores à de Safo, dificultando o rastreio de suas existências. Interessa antes observar como Safo e as demais poetas mulheres valeram"-se da ``forma -- gêneros, metros, estilos e abordagens -- do \textit{mainstream}, isto é, da tradição poética pública de autoria primordialmente masculina do período em que estavam escrevendo. Isso é evidência poderosa de que elas não estavam isoladas da poesia de seus pares na maioria masculinos e de seus predecessores na grande tradição, mas, ao contrário, estavam muito interessadas nessa poesia e eram por ela influenciadas'', como bem argumenta Laurel Bowman.\footnote{Bowman (2004, p.\,10). A helenista discute as poetas mulheres, sobretudo aquelas que sucedem Safo nas eras clássica e helenística, nomeadas no epigrama em epígrafe a esta introdução, e seu lugar na tradição da poesia grega. Ver ainda, sobre o tema, e por vezes com visão algo distinta, De Martino (1991, pp.\,17--75), Klinck (2008) e Ragusa (2020, pp.\,113--136).}
Considerados essa tradição e o elenco de poetas mulheres, porém,
a reputação de Safo, atestam os
testemunhos, não encontrou em nenhuma um nome que a
superasse; Estrabão (\textsc{i} a.C.--\textsc{i} d.C.) ressalta isso: 

\begin{quote}
Neste tempo que se tem na memória, mulher alguma nem de longe se revela páreo à graça daquela poeta.\footnote{ \textit{Geografia} (13. 2. 3).}
\end{quote}

No desenho geográfico composto pelos nomes de poetas mulheres
lembrados no epigrama, Albin Lesky atenta para um detalhe
notável: há poetas mulheres de Lesbos, da Beócia e do Peloponeso, mas não da
Ática. “Assim se manifesta uma posição diferente, mais livre, da mulher”
nessas outras regiões, do que aquela “que conhecemos no mundo de Atenas”.\footnote{ Lesky (1995, p.\,210).} A
conclusão do helenista, em princípio possível, leva a outra: nas sociedades
que geraram poetas mulheres, deve ter havido acesso a uma forma de educação
feminina e abertura para a participação maior das mulheres na vida da
comunidade, incluindo as de origem aristocrática, como Safo; daí a provável
aceitação do fazer poético exercido por uma mulher.

A circulação da obra de Safo, e de outras poetas mulheres mais tarde, parece
apontar nessa direção, bem como o silêncio que ressoa da Ática, de Atenas, onde
prevalecia, ao menos na era clássica, o confinamento das mulheres ao \textit{oîkos}, “casa”, espaço
feminino por excelência na Grécia; salvo em ocasiões específicas, como um rito
religioso"-cultual ou grandes celebrações de transição -- casamentos e funerais --, as mulheres atenienses não deveriam ser vistas, nem suas
vozes ouvidas.\footnote{ Para mais sobre a mulher ateniense, ver os estudos de
Mossé (1991, pp.\,49--61 e pp.\,152--3) e Murray (1993, p.\,41).} Na casa, talvez elas
tivessem acesso à educação e mesmo à escrita, mas, à diferença do que teria
ocorrido noutras partes do diversificado mundo grego, não haveria em Atenas
condições, segundo o que se sabe, para que suas manifestações artísticas
pudessem circular.

Nada disso pode ser afirmado com segurança, mas não é disparate imaginar que o acesso
à educação aristocrática e a maior liberdade de ação e inserção social
contribuíram e muito para que poetas mulheres tivessem condições de existir,
como Safo,\footnote{ Diz Bennett (1994, p.\,346):
“Para tornar"-se uma poeta, Safo teve de ser treinada, em expressão
e composição, e nós naturalmente suporíamos que tal treino era aquele de outras
meninas aristocráticas de Mitilene”.} cujas composições, como as dos demais
poetas arcaicos e clássicos, tinham que ser, necessariamente, apresentadas em
determinados modo e ocasião de \textit{performance} -- e dadas a celebridade e a variedade da mélica sáfica, não pode ter sido esta uma só, e nem limitada
a grupos secretos de mulheres ou a segregada poesia feminina não atestados em parte alguma da Grécia, e contrariados pela influência de Safo na tradição poética em que beberam homens e mulheres poetas gregos.\footnote{Ver crítica de Bowman (2004, pp.\,5--6) a tal ideia da segregação, influente em certa crítica moderna, mas sem respaldo em qualquer evidência.}

Lamentavelmente, não é fácil avaliar a condição feminina na Grécia
antiga, menos ainda na era arcaica, e menos ainda na lésbia Mitilene, cuja
especificidade, em qualquer de suas dimensões, escapa"-nos quase que
de todo. As evidências escritas ou iconográficas são muito
escassas, mas não podemos deixar de mencionar que nos vasos atenienses que
retratam figuras femininas associadas à escrita e/ou à leitura, estas são Safo
-- porque é a poeta de grande fama -- e as Musas -- porque são deusas.
Essas imagens, em última análise, nada provam quanto à poeta -- e outras mulheres
aristocratas -- no que concerne às habilidades da escrita e da leitura -- as
quais são separadas na Antiguidade, o domínio de uma não implicando
automaticamente o da outra, lembra Susan \textsc{g}.\,Cole, limitando"-se a
leitura, sempre em voz alta, a poucos textos.\footnote{ Cole (1992, p.\,220).} Na representação iconográfica,
prossegue ela, os livros -- na verdade, rolos de papiros -- amparam a
recitação, e não a “leitura solitária” e silenciosa; e nas imagens de mulheres,
é típico o desenho de uma que lê para uma moça ou para um grupo feminino.
Quando a leitora é Safo, diz Cole, a imagem quer antes celebrar a poeta, do que
retratar “uma cena familiar ou tipicamente doméstica”, até porque, ressalta,
mesmo em vasos onde figuram cenas domésticas, além dos que trazem a poeta ou as
deusas da poesia, jamais vemos mulheres escrevendo.\footnote{ Cole (1992, p.\,224).}
%Dúvida: Musas é mesmo "Musas", em maiúscula?

Da busca de Safo, a mulher, e da formação da poeta, voltamos de mãos
praticamente vazias -- praticamente, porque podem carregar ideias plausíveis,
não obstante inverificáveis. Da busca da Safo \textit{lésbica}, à qual
aludem os testemunhos tardios do papiro e do \textit{Suda}, o que traremos nas
mãos será um resultado semelhante, ou até mais esvaziado.

A condição de mulher poeta cujas canções têm na temática erótico"-amorosa e no
universo feminino suas linhas de força -- canções estas lidas em chave
\textit{biografizante} já pelos antigos que nelas buscavam a substância da figura
histórica para eles, como para nós, esvaecida -- faz emergir a questão da
sexualidade feminina e de seu exercício a partir da figura de Safo,
amiúde reelaborada em antigas e modernas tramas tecidas no correr dos
tempos, desprovidas quase de historicidade atestada, mas
inseridas na rede que se pretende explicativa da poeta e de seus textos. Judith
\textsc{p}.~\,Hallett\footnote{ Hallett (1996, pp.\,125--7). Ao tratar das
imagens da poeta, ela fornece as indicações das fontes antigas.} comenta as
imagens que se formaram, sobretudo, do século \textsc{iv} a.C. em diante; e assim resume
Glenn \textsc{w}.~\,Most as díspares e desencontradas -- não raro,
extremadas -- imagens de Safo: 

\begin{quote}
As várias fontes que fluíram juntas para criá"-la creditaram"-na com um marido,
uma filha, muitos irmãos, numerosas amigas e companheiras (com as quais, ao
menos segundo alguns relatos, ela teve relações sexuais), numerosos amantes, um
homem que rejeitou as investidas de Safo, e um salto suicida de um penhasco. Em
princípio, decerto, não há razão para que uma vida social tão variada e rica
não tenha sido possível -- embora se pudesse cogitar como, entre um e outro
compromisso, Safo teria encontrado tempo para compor sua poesia {[}\ldots{}{]} Mas tanta
complexidade apresenta um desafio a qualquer um que tente imaginar um retrato
coerente da vida de Safo, pois requer que elementos potencialmente divergentes
sejam trazidos a uma relação plausível uns com os outros. Mais
fundamentalmente, a recepção de Safo pode ser interpretada como uma série de
tentativas de chegar a termo com a complexidade dessa gama de informações.\footnote{ Most (1996, p.\,14).}
\end{quote}

Ora, a leitura dos modernos dessa intricada trama biográfica não foi, no mais
das vezes, menos infeliz que a dos antigos. Most comenta, por exemplo,
a dos românticos:

\begin{quote}
\textit{Condensando} numa única pessoa as muitas contradições com as quais a
tradição tinha suprido Safo, inventaram uma figura intensamente paradoxal
{[}\ldots{}{]} A Safo romântica é a primeira que é, essencialmente, uma poeta -- mas uma
poeta romântica, insatisfeita com a realidade banal e lutando para alcançar a
perfeição espiritual incompatível com a vida e somente alcançável às custas da
morte.\footnote{ Most (1996, p.\,20).}
\end{quote}

Em verdade, uma revisão das leituras modernas de Safo mostra que cada época,
cada contexto histórico"-social e cultural criou para si a imagem desejada da
poeta; tal liberdade explica"-se pela carência de conhecimento consistente sobre
ela e sua vida na arcaica Mitilene -- liberdade, diga"-se ainda, criativa, que
em sedutoras e intrigantes projeções pseudobiográficas, acabaram por
ganhar mais atenção do que sua arte
muitas vezes usada para alavancar tais projeções. 

A imagem da Safo \textit{lésbica} construiu"-se nesse movimento, e ganhou fôlego
na esteira de vogas na crítica literária -- \textit{gay studies}, \textit{women
studies} -- que valorizam aspectos do entorno dos textos que passam a ser
estudados de modo secundário, e não no primeiro plano, para sustentar a análise
de tais aspectos. Nessa linha, certa leitura de Safo e de certa fatia de
testemunhos sobre a poeta amparam a afirmação de que trata"-se da primeira poeta
\textit{lésbica} do Ocidente, não porque seja filha da ilha de Lesbos, mas
porque, segundo um olhar modernizante e romântico, Safo dividia seu leito com mulheres e por
elas era tomada de paixão.
Tal olhar desconsidera ou minimiza o fato de que o uso da poesia como história é, no mínimo, imprudente, e mais: de que a defasagem entre nós e os clássicos reside também no modo como percebemos a sexualidade, como bem sublinham os dizeres de Maria Fernanda Brasete:

\begin{quote}
Na antiga cultura grega, efectivamente, o relacionamento erótico entre pessoas do mesmo sexo – que hoje apenas conhecemos de forma fragmentária e indirecta, através de escassos documentos literários e iconográficos – parecia integrar, naturalmente, a vida dos membros de uma sociedade, que, talvez por essa razão, nunca criou um léxico específico para o designar, nem nunca o entendeu de um modo discriminatório ou em termos psico-morais. Mas falar de erotismo grego não é o mesmo que falar de sexualidade, porque, em primeiro lugar, não se trata de fenómenos atemporais e, por conseguinte, não podem ser descontextualizados das práticas sociais institucionalizadas numa determinada comunidade histórica.\footnote{Brasete (2009, p.\,292).}
\end{quote}


\noindent{}Digo certa leitura, porque há outros modos, mais
seguros e coerentes com o que realmente sabemos do mundo antigo grego, de
compreender a prevalência de meninas -- e não mulheres --  na poesia erótica de Safo; e digo
certa fatia de testemunhos antigos, porque naqueles que enfocam sua
sexualidade, Safo é retratada no mais das vezes como promíscua e engajada em relações com homens. Vejamos. 

A partir de Aristófanes (séculos \textsc{v}--\textsc{iv} a.C.) e da comédia clássica, pelo menos, a
referência às mulheres de Lesbos e o uso de verbos como \textit{lesbiázein} e
\textit{lesbízein}\footnote{``Agir como uma mulher de Lesbos''.} conotavam luxúria e
lascívia; em particular, diz Hallett, a prática da
felação, que as lésbias teriam inventado\footnote{ Hallett (1996, p.\,129).} -- algo sem qualquer respaldo
histórico. A esse respeito, Gentili observa: 

\begin{quote}
Já na segunda metade do século \textsc{v} a.C. -- e seu uso é certamente bem mais antigo
-- as palavras \textit{Lésbia} ou \textit{moça de Lesbos} tinham a típica
conotação de \textit{fellatrix}, e não de \textit{lésbica} no sentido moderno
do termo. \textit{Lesbís} e \textit{lesbiázein} eram essencialmente peças de
terminologia sociológica com um significado específico e inequivocamente
erótico.\footnote{ Gentili (1990a, p.\,95).}
\end{quote}

Não há como precisar a razão dessa ligação das mulheres de Lesbos a práticas
sexuais específicas, mas talvez isso se deva à fama da beleza incomparável e da
sensualidade das mulheres da ilha, já atestada na \textit{Ilíada},\footnote{Canto \textsc{ix}, versos 128--30.}
e/ou à intensidade erótica da poesia de Safo. Seja como for, a habilidade
sexual atribuída às lésbias e a imagem da poeta parecem
emanar, confundidas, no uso do adjetivo “lésbia” mesmo tarde, em Catulo
(século \textsc{i} a.C.) -- repare"-se, sempre em contextos heteroeróticos, frisa Hallett.\footnote{ Hallet 
(1996, pp.\,129--30). Ver ainda os estudos de Brasete (2003, pp.\,17--26; 2009, pp.\,289--303) e Ragusa (2019b, pp.\,211--39).} 

A despeito dessas ressalvas, \textit{Safo de Lesbos} é designação que
recorrentemente projeta em nosso imaginário certa Safo, a poeta
\textit{lésbica}. Essa projeção, cabe entender, vale"-se do peso de Safo como referente. 
Jogada para a poeta, porém, não só carece de respaldo em evidências antigas,
confiáveis e consistentes, mas resulta de uma percepção da
sexualidade que em muito se distingue da dos antigos, como ressaltado na citação de Brasete anteriormente.

Repare"-se, a propósito, que o adjetivo \textit{lésbica} não existia na
Antiguidade, tampouco o \textit{lesbianismo} em sua concepção moderna.
Volto a Brasete:

\begin{quote}
O sentido homoerótico que os adjectivos ``sáfico'' e ``lésbico'' hoje veiculam testemunham a ressonância que o nome da poetisa e o da sua terra natal produziram na cultura universal, onde a homossexualidade feminina passou a ser tradicionalmente designada pelo termo ``lesbianismo'', se bem que a custa de uma deformação semântica do termo originário. É que na Grécia antiga, Lesbos era uma ilha conotada com práticas sexuais promíscuas que, contrariamente ao que se poderia supor, não se circunscreviam a práticas homossexuais femininas que, por sua vez, também não eram exclusivas dessa zona. Apesar de a fama das mulheres lésbias estar muito mais relacionada com práticas heterossexuais, não foi fácil, para os autores antigos e modernos, dissociar a figura de Safo do tipo de relações amorosas que emergem da sua poesia, conotada, desde a época helenística, com o homoerotismo feminino.\footnote{Brasete (2003, p.\,17).}
\end{quote}

\noindent{}Não há,
ademais, qualquer base com lastro mínimo na afirmação de que teria sido esta
uma opção sexual da poeta que, enquanto sujeito histórico, é pouco mais, se
tanto, do que uma neblina para nossos olhos, como já para os dos antigos. O
termo \textit{lésbica}, enfim, frisa Sue Blundell, “é uma
invenção moderna” que passa a nomear “uma mulher homossexual no final do século
\textsc{xix}, como resultado da publicidade criada por uma controvérsia acadêmica em
torno da sexualidade da própria Safo”\footnote{ Blundell (1995, p.\,83).} -- polêmica esta 
estéril nos seus resultados.
Lardinois precisa as datas em que, em
língua inglesa, o adjetivo se verifica, a partir de 1890, e o substantivo
\textit{lesbianismo} -- homossexualismo feminino --, primeiro em 1870, escrito com
inicial maiúscula, em clara alusão à ilha de Lesbos; e ele indaga:

\begin{quote}
Será justificada a relação entre a ilha de Lesbos e o homossexualismo das
mulheres? Existiriam razões para crer que Safo de Lesbos fosse uma ‘lésbica’?

É essa a \textit{grande questão sáfica} {[}\ldots{}{]} que já era debatida na Antiguidade, mas
os estudiosos ainda não foram capazes de chegar a um consenso. Provavelmente,
jamais o consigam, não apenas porque as evidências sejam muito escassas, mas
porque existe algo intrinsecamente errôneo na forma de colocar a questão.\footnote{ Lardinois (1995, pp.\,27--8).}
\end{quote}

Concluindo seu estudo, afirma Lardinois:

\begin{quote}
Podemos concluir que, no caso de Safo, estamos, no máximo, diante de
relacionamentos breves entre uma mulher adulta e uma jovem prestes a se casar.
Chamar de \textit{lésbicas} essas relações é um anacronismo. É impossível avaliar se a
palavra se aplica à própria Safo ou à sua vida íntima. Na verdade, essa é uma
questão sem sentido. Mesmo se, pelos padrões modernos, Safo devesse ser
considerada lésbica, sua experiência deve ter sido muito diferente, vivendo,
como viveu, em uma era diferente com diferentes noções e tipos de sexualidade.\footnote{ Lardinois (1995, p.\,50).}
\end{quote}

O grau elevado de complexidade da chamada \textit{questão sáfica} é um alerta
crucial a afirmações simplistas, como as que anunciam em Safo a
primeira poeta (engajadamente) \textit{lésbica} da literatura ocidental,
servindo"-se de modo igualmente simplista e redutor de sua poesia para provar a alegada
verdade sobre a Safo histórica, quando é antes parte de sua recepção que, ao adotá-la 
para os anseios do dia, ajudam a manter Safo viva entre nós. Holt Parker declara: 

\begin{quote}
O texto de Safo está em fragmentos {[}\ldots{}{]} A linguagem é difícil, a sociedade,
obscura. Voltamo"-nos a manuais e comentários em busca de auxílio. Isso
significa, porém, que chegamos a Safo já cegos pelas assunções largamente não
examinadas de gerações prévias de estudiosos; e no caso de Safo, o acúmulo de
assunções é profundamente milenar e inclui comédias gregas, romances italianos
e pornografia francesa. O caso é pior com Safo do que com qualquer outro autor,
incluindo Homero. Pois aqui não lidamos apenas com a literatura arcaica, mas
com a sexualidade; os comentários são pesadamente carregados de emoção e de
nossos preconceitos. Mais importante, estamos lidando com homossexualidade (ou
melhor, o que construímos como homossexualidade) e sexualidade feminina.\footnote{ Parker (1996, p.\,149).}
\end{quote}

As considerações de Parker e Lardinois são de válidas, embora raras, lucidez e
prudência, tanto mais se considerarmos que, segundo Blundell, em
textos antigos, a única “clara alusão a um comportamento homossexual de uma
mulher de Lesbos ocorre num diálogo entre duas prostitutas escrito pelo
satirista Luciano, no século \textsc{ii} d.C.”, o \textit{Diálogo das cortesãs}.\footnote{Blundell (1995, p.\,82).} A
alusão é clara, mas sua historicidade está comprometida pela natureza
literária da fonte, mais do que isso, satírica e, portanto, distorcida pelos objetivos difamatórios desse gênero discursivo. Gêneros como a
comédia e a sátira, nunca é demais frisar, valem"-se com frequência da liberdade
para tratar do sexo para fazer rir e, no caso do segundo, que visa ao riso dos
cúmplices do satirista e à destruição do alvo de seu texto, da linguagem do
vitupério. 

Deve estar evidente, agora, o porquê do uso aqui de lésbia para adjetivar a origem geográfica de Safo, em vez de lésbica; igualmente, do uso de homoerotismo, em vez de homossexualismo e outros, dado que muito mal compreendemos a sexualidade entre as mulheres da Grécia antiga. Para esse assunto, as fontes são demasiado escassas, e problemáticas são as analogias
com o abundante material do atestado homossexualismo masculino -- que,
diga"-se logo, também não funciona entre os antigos como funciona no imaginário
que com olhos modernizantes o contempla. Isso tudo está bastante
bem explorado, documentado e analisado no estudo fundamental de Kenneth \textsc{j}.\,Dover, 
originalmente publicado em 1978.\footnote{ Dover (1994).}

Reitero: as canções de Safo não
são registros biográficos ou documentos de sua sexualidade, mas discursos
poéticos em condição fragmentária, que compromete o alcance de nossas leituras,
filiados a determinado gênero poético, o mélico, inseridos em certa tradição
poética de linguagem erótica, e em geral plasmados no universo feminino. Estamos fadados a fracassar e a nos perder em
especulações, se tentarmos explicar Safo e sua poesia a partir das
características da vida cotidiana das mulheres em Lesbos, ou pela sua
biografia, mesmo porque extraíam"-na os antigos das canções de Safo, e
sem qualquer constrangimento preenchiam as lacunas com narrativas
que criavam segundo a verossimilhança, que é categoria discursiva.

Tendo a exposição feita até este ponto em mente, e passeando por versos de poetas que ou precederam Safo ou
a sucederam, melhor entenderemos a similaridade tremenda entre as imagens, a
linguagem e o tom de sua poesia por natureza tradicional, genérica e de \textit{performance}. Bem a ilustram seus fragmentos eróticos e os versos marcados pelo
erotismo em Hesíodo (ativo em \textit{c.} 700 a.C.), Arquíloco (\textit{c.} 680--640
a.C.), Íbico (ativo em \textit{c.} 550 a.C.) e Eurípides (\textit{c.} 482--406 a.C.). Eis
uma pequena amostragem de como, no registro erótico de gêneros, poetas e
tradições distintos, é semelhante a linguagem para falar da paixão erótica, do
desejo nas traduções possíveis para o grego \textit{éros} que, longe de nomear
o amor romântico, designa a força controlada por Afrodite -- mesmo na forma do
deus Eros, sempre a ela subordinado. Essa força, segundo seus desígnios, toma de
assalto sua vítima e se apodera de seu corpo e de sua mente, como “uma invasão,
uma doença, uma insanidade, um animal selvagem, um desastre natural”, que vem a
“provocar o colapso, consumir, queimar, devorar, exaurir, entontecer, picar,
aguilhoar, {[}\ldots{}{]}”, resume Anne Carson.\footnote{ Carson (1998, p.\,148).}

%Dúvida: como resolver essa parte? O que dá pra jogar como nota? Lembrando que já joguei um \textsc{vv} . como nota antes e, caso esteja errado, preciso voltar.

\begin{verse}
\small{\textsc{«\,teogonia\,», hesíodo}
\textit{(Poesia didática)}
\smallskip
{[}\ldots{}{]} Eros: o mais belo entre imortais deuses,\\
deslassa-membros, de todos os deuses e de todos os mortais\\
ele doma no peito a mente e o prudente desígnio.\footnote{Tradução minha. Edição: West (1988), versos 120--2.}
\smallskip
\hspace*{35mm}\adforn{47}
\smallskip

{[}\ldots{}{]} Eurínome gerou as três Cárites de belas faces -- ela,\\
\hspace*{2em}a menina de Oceano, de mui amável forma:\\
\hspace*{2em}Aglaia, Eufrosine e Talía amorável:\\
\hspace*{2em}de seus olhos, quando fitam, goteja o desejo\\
\hspace*{2em}deslassa-membros, e de sob sobrancelhas belamente fitam.\footnote{Versos 907--11.}}
\end{verse}

\smallskip

\begin{verse}
\small{\textsc{fragmentos 193 e 196 w, arquíloco}\footnote{Tradução minha. Edição: West (1988).}
\textit{(Poesia jâmbica)}
\smallskip
\hspace*{4.5em}mísero jazo com desejo,\\
\hspace*{2em}exânime, por dores atrozes -- vontade dos deuses --\\
\hspace*{4.5em}trespassado até os ossos.\\

Mas o desejo deslassa-membros, ó companheiro, doma-me}
\end{verse}

\smallskip
\begin{verse}
\small{\textsc{fragmento 287, íbico}\footnote{Tradução: Ragusa (2010, p.\,650; ver estudo às pp.\,480--507). Edição: Davies (1991). Cípris é um dos outros nomes de Afrodite.}
\textit{(Poesia mélica)}
\smallskip
Eros, de novo, de sob escuras\\
pálpebras, com olhos me fitando derretidamente,\\ 
%[F020?][F020?][F020?][F020?][F020?] 
com encantos de toda sorte, às inextricáveis\\
redes de Cípris me atira.\\ %[F020?][F020?][F020?][F020?] 

Sim, tremo quando ele ataca, \num{5}\\
tal qual atrelado cavalo vencedor, perto da velhice,\\
contrariado vai para a corrida com carros velozes.}
\end{verse}

\smallskip
\begin{verse}
\small{\textsc{«\,hipólito\,», eurípides}\footnote{Tradução: Oliveira (2010), em edição bilíngue que segue o texto grego de Barrett (1992). Os versos são a parte inicial do canto do coro de mulheres conhecido como ``Ode a Eros''.}
\textit{(Tragédia)}
\smallskip
Eros, Eros que nos olhos \num{525}\\
destilas desejo, incutindo doce\\ 
prazer n’alma dos que atacas,\\
que jamais me apareças com dano\\
nem venhas desmedido.\\
Pois nem dardo de fogo e nem dos astros é forte \num{530}\\
como o de Afrodite, que atira das mãos\\
Eros, filho de Zeus.\\

Em vão, em vão às margens do Alfeu\num{535}\\[5pt]
e na morada pítica de Febo\\
bovinos sacrifícios prodigaliza a terra heládica,\\
se não veneramos Eros,\\
o soberano dos homens,\\
claviculário da deleitosa alcova de Afrodite \num{540}\\
exterminador que atira os mortais\\
em todos os desastres quando vem.}
\end{verse}

Nos fragmentos de Safo, o leitor decerto perceberá os ecos intensos
dessas linguagem e imagens sobre a paixão erótica, da qual lançam mão
reiteradamente os poetas gregos, como mostra a seleção citada. Esse movimento é
característico da prática genérica de composição dessa poesia de tradição oral, que lida sobretudo com motivos consolidados e embasados em
percepções mantidas pela repetição, pelo seu retomar por vezes incrementado em
escolhas estilísticas que parecem singulares a um ou outro poeta -- parecem, é
prudente dizer, em vista de nosso magro \textit{corpus} de textos, a partir do
qual afirmar a inovação é um risco.



Confrontados com os versos que reproduzi de Hesíodo, de Arquíloco, de Eurípides,
poetas homens de gêneros poéticos distintos, os fragmentos de Safo que envolvem
a temática da ação da paixão sobre sua vítima e a concepção sobre \textit{éros}
não se distinguem deles a ponto de tornar sustentável o argumento questionável
de uma “literatura feminina”; ao contrário, ecoam versos
dos dois primeiros poetas, e ressoam nos do terceiro, pois trabalham com
suas imagens e, se lhes acrescentam outras, aparentemente singulares à poeta
lésbia, fazem"-no calcados num tratamento da \pagebreak temática referida que não se pode
dizer masculino, nem feminino, mas tradicional. Também por esse, além dos demais motivos antes levantados, é tão complicado
inferir, a partir da leitura das canções em que o \textit{eu} -- nem sempre
identificável para além de uma voz em primeira pessoa do singular -- acontece de ser
feminino e de se relacionar a figuras femininas eroticamente, que Safo seja
\textit{lésbica}, que represente a “literatura homossexual” -- algo
demasiado modernizante e anacrônico, em se tratando dos poetas antigos
(homens ou mulheres).

O recorte de um universo que é distintamente feminino no \textit{corpus} preservado de sua poesia, inclusive
a de caráter intensamente erótico, se pensado no contexto da Grécia arcaica, leva"-nos para mais perto da Safo que naquele mundo produziu suas canções, e pode nos dar a vislumbrar, em meio às brumas, a poeta que tanto marcou o imaginário, em vez das ficções que nele a partir dela se criaram, e que nos são mais próximas, falando a língua de nossos tempos. Estas são importantes, porque ajudam a manter viva a memória daquela. Como estudiosa dedicada àquela Grécia, porém, é aquela Safo que busco fazer emergir desta antologia, nos limites inerentes ao objeto, que tenho ressaltado no andamento destas linhas. No mundo
de Safo, como na Grécia antiga em geral, estavam cindidos o universo masculino
e o feminino, e Jan \textsc{n}.\,Bremmer observa que mesmo as relações
heterossexuais no casamento eram distanciadas. Conclui o helenista:
“Dificilmente foi por acaso que o homossexualismo moderno”, que, em largos termos, impõe
a rejeição ao heterossexualismo, “se desenvolveu na mesma época em que a
relação heterossexual no casamento adquiria um caráter muito mais íntimo”.\footnote{ Bremmer (1995, p.\,24).} Na
Grécia, prossegue Bremmer, evidências literárias e iconográficas -- vasos
atenienses da segunda metade do século \textsc{vi} a.C., “que circulavam nos banquetes
aristocráticos”\footnote{ Bremmer (1995, pp.\,20--21).} -- “mostram que as relações homossexuais normalmente
ocorriam só entre adultos e rapazes”,\footnote{ Bremmer (1995, p.\,12).} sendo “um aspecto estabelecido do
caminho de um rapaz rumo à idade adulta”.\footnote{ Bremmer (1995, p.\,12). Ao menos, na visão
externa dessas relações; da interna, pouco se pode dizer, lembra Bremmer (p.\,20).} E frisa Bremmer que tais relações não implicavam a rejeição “do contato
heterossexual”, necessário à procriação e à continuidade das linhagens e das
comunidades, mas eram vivenciadas no mundo dos homens, nitidamente distinto do
mundo das mulheres.\footnote{ Bremmer (1995, p.\,26) ainda observa que era,
porém, a “relação pederasta que transformava o rapaz em um verdadeiro homem” --
pois se voltava ao benefício do ensinamento intelectual e social do jovem, e
não apenas ao benefício do prazer sexual ao adulto -- e abria as portas para o
universo da “elite social”.}

%Dúvida: o que é essa "epístola"?
Ao falar de \textit{éros} numa linguagem tradicional e a partir de uma concepção
reafirmada na poesia de temática erótica, Safo segue as práticas poéticas de
seu tempo e lugar histórico, viabilizando assim a circulação de sua poesia
que por todos poderia ser fruída. É decerto nesse sentido que
devemos tomar uma conhecida frase -- na verdade, um verso hexamétrico -- do poeta
latino Horácio (século \textsc{i} a.C.), numa \textit{Epístola}:\footnote{1.\,19.\,28.}
``Regula a Musa pelo pé de Arquíloco a máscula Safo”,\footnote{ Agradeço
ao colega latinista Prof.~Dr.~Marcos Martinho dos Santos (\textsc{usp}) a discussão sobre esse verso e
sua tradução.} ou seja, a poeta lésbia regulou seus ritmos pelos de Arquíloco,
numa compreensão possível de um verso controverso. A rigor, porém,
estudos métricos mostram que a tradição lésbio"-eólica é mais antiga que a
jônica, de que se vale Arquíloco, e que a dórica.\footnote{ Ver o estudo de West
(1973, pp.\,179--92).} Mas talvez possamos, apoiados nos versos dos dois poetas
arcaicos, pensar a ideia do ritmo em termos mais amplos: a linguagem e
as imagens que Safo emprega no tratamento da paixão são as mesmas que
encontramos em Arquíloco e em outros poetas homens, antes ou depois dela. E
quanto à conhecida expressão ``máscula Safo”, a primeira das duas
explicações oferecidas por Porfírio (século \textsc{iii} d.C.), comentador de Horácio,
de que a poeta foi excelente na poesia em que homens mais amiúde se
destacam, é, decerto, aquela que pode ser entendida coerentemente e com
respaldo vasto e sólido das práticas poéticas antigas testemunhadas na
composição dos textos sobreviventes. A segunda, de que Safo foi difamada como
dissoluta tríbade -- termo grego de que se vale o comentador, que significa
mulher homossexual ou \textit{lésbica} -- é de pouca valia, inclusive para a
compreensão do verso epistolar horaciano, conduzindo"-nos para longe de sua
poesia e do fazer poético de que resultam suas canções, e para perto das
ficções de Safo e de ansiedades de nosso tempo. 


\section*{A mélica de Safo}

Melhor é, portanto, ir para perto da obra da poeta lésbia, como espero que esta
antologia evidencie aos que creem conhecer Safo, mas talvez tenham se perdido
nas tramas lendárias em que foi enredada, antes mesmo de chegar às ruínas da
matéria viva que resta da poeta, de um sopro quente e fragrante que os séculos
não abafaram.

No começo desta introdução, observei o fato de que o gênero poético praticado
por Safo é o que os antigos chamavam simplesmente “canção”
(\textit{mélos},\footnote{ Termo mais usado nos períodos arcaico e clássico.
O sentido primeiro de \textit{mélos} é “membro” do corpo, daí “membro
musical, frase” e “canção” (palavra, melodia e ritmo), na compreensão
explicitada na \textit{República} (398c), de Platão (séculos \textsc{v}--\textsc{iv} a.C.).
Budelmann (2009, p.\,2) lembra que \textit{mélos} é bastante usado pelos próprios poetas arcaicos ``com referência a suas composições”.} \textit{âisma, oidé}) -- daí a
designação \textit{mélica} -- e que se destinava à \textit{performance} ao som
da lira, daí \textit{lírica} em acepção específica, termo mais tardio, em
circulação da era helenística (323--31 a.C.) em diante. A propósito
desses nomes, Rudolf Pfeiffer afirma: “Em tempos
modernos, toda a poesia não épica e não dramática é usualmente chamada lírica.
Mas os antigos teóricos e editores faziam a distinção entre, de um lado, poemas
elegíacos e jâmbicos, e, de outro, mélicos”.\footnote{ Pfeiffer (1998, pp.\,182--3).} Mélica, prossegue,
designava o “verso que era cantado para a música e, muito frequentemente, a
dança, e era composto de elementos de ritmos e tamanhos variados”:
\textit{mélos} era, na literatura grega arcaica, o poema; o poeta,
\textit{melopoiós} -- o “fazedor de canções”, literalmente -- ou mélico; o
gênero, mélica ou poesia mélica. Ainda segundo Pfeiffer, tais nomes
“permaneceram usuais em mais tardias pesquisas sobre a teoria poética e a
classificação da poesia”, mas “líricos” era o termo que designava os autores
“em referências a edições de textos e em listas de ‘fazedores’\,”; e do século \textsc{i}
a.C. em diante, \textit{lírica} sobrepõe"-se a \textit{mélica} para designar a canção
cantada ao som da lira, o instrumento mais importante de seu acompanhamento, e
os latinos acabaram por adotar o primeiro, a despeito do uso ocasional do
segundo.

Como se vê, o modo de \textit{performance} nomeia o gênero;
mas o que mais o caracteriza? O canto solo,
ou em coro, com o acréscimo de
outros instrumentos e da dança, a configurarem um espetáculo. Na métrica,
em estrofes mais breves e menos
complexas na modalidade monódica do que na coral. No conteúdo daquela, grande variedade de temas -- principalmente os vinculados à
vida cotidiana na \textit{pólis}, a eventos de um passado recente, à
experiência humana, sempre em relação direta com a voz poética --, formas e
linguagem, enquanto na coral e seus subgêneros são comuns a celebração, a
narrativa mítica e o canto de autorreferência à \textit{performance} em
execução pelo coro. Não são mais estas palavras do que uma síntese para dar uma ideia de um cenário mais complexo. 

Considerando que as composições da mélica grega arcaica “estavam destinadas,
desde o início, à execução pública ou privada, e constituíam por definição uma
poesia \textit{de} e \textit{para} a voz”, Gustavo Guerrero
afirma decorrer daí que

\begin{quote}
apareça dominada por um maior traço -- o caráter circunstancial do discurso -- de
literatura oral, que reflete a relação direta do texto com um local e um
momento precisos, um espaço e um tempo ritualizados {[}\ldots{}{]}; daí os índices
textuais de um discurso situacional, que se expressam através do uso de certas
figuras pronominais e de marcas do presente, signos que traduzem a interação
geral entre o sujeito da enunciação e seus destinatários.\footnote{ Guerrero (1998, pp.\,20--1).}
\end{quote}

Inserida e movida “culturalmente no seio de um sistema de comunicação oral”, a
mélica se concretizava, “portanto, numa prática artística performática”,
conclui Guerrero.\footnote{ Guerrero (1998, p.\,21).} Carecemos, porém, de informações que permitam uma
reconstrução clara e precisa de sua \textit{performance}, entre as quais, as
relativas à música, que logo se perdeu. No que se refere à mélica monódica,
lembra Giovan \textsc{b}.\,D’Alessio que se destinava à apresentação “em
contextos mais próximos aos da comunicação espontânea, face a face”; logo, nela
“espera"-se grau maior de imediatismo”.\footnote{ D’Alessio (2004, p.\,270).} 

Recompondo minimamente o quadro, haveria para a canção solo várias audiências e
ocasiões de \textit{performance} possíveis, com destaque para o simpósio,
central também para gêneros como a elegia e o jambo; como anota Massimo Vetta,
no mundo grego de até meados do século \textsc{v} a.C., em que não
estava previsto “um público de leitores”, o simpósio 

\begin{quote}
é o lugar de conservação
e evolução da cultura ‘literária’ relativa a todos os temas que resultam
alternativos ao interesse ecumênico do \textit{epos} e à ambientação
exclusivamente pública do canto religioso oficial e da lírica agonística
{[}temas estes muito trabalhados na elegia, no jambo e na mélica monódica{]}.\footnote{ Vetta (1995, p.\,\textsc{xiii}).}
\end{quote}

Entenda"-se por simpósio o que Pauline Schimitt"-Pantel define, em
sentido restrito e etimológico, como o “momento, após a refeição, em que todos
passam a beber”, e, em sentido mais amplo e corrente, “uma prática, de beber
junto, e uma instituição” que “é a expressão do modo de vida aristocrático” dos
homens na \textit{pólis}.\footnote{ Schimitt"-Pantel (1990, p.\,15).} O simpósio
é, portanto, coletivo do ponto de vista do
evento, mas restrito do ponto de vista da classe e do gênero;\footnote{A presença
feminina limitava"-se às tocadoras de \textit{aulo}, instrumento de sopro, às
dançarinas -- normalmente \textit{heteras}, espécie de cortesãs --, e às servidoras de bebida.} e era pautado por um
“código rígido e próprio de honra”, ressalta Oswyn Murray, visando
garantir a moderação, caminho para a harmonia essencial à atmosfera
\textit{simposiástica}.\footnote{ Murray (1990, p.\,7).}

No andamento do evento, o beber era privilegiado e, por isso, tornou"-se
altamente ritualizado. A \textit{(re)performance} amadora ou profissional da
poesia, pelo poeta ou não, tinha lugar em meio a essa fase, e o simpósio acaba
por exercer um papel fundamental na sua preservação e difusão. Relaxados,
sorvendo o vinho, os gregos ouviam e cantavam e/ou recitavam elegias, trechos
dos poemas homéricos e, claro, exemplares da mélica, competindo uns com os
outros, e demonstrando a habilidade e desenvoltura esperadas de sua formação
aristocrática. E poderia haver
versos em que o \textit{eu} poético fosse de uma mulher, admitindo"-se, no contexto, a
representação de um papel feminino.\footnote{ Ver Bowie (1986, pp.\,16--7).}
Nesse sentido, mais uma vez, não haveria obstáculo às canções de Safo em que a primeira pessoa do singular é feminina. E não por acaso se diz ser o simpósio uma
ocasião bastante adequada ao caráter mais informal e privado que, nela construído discursivamente, emana da mélica monódica; mas Most ressalta:

\begin{quote}
{[}\ldots{}{]} a aparente privacidade da canção monódica não é do individual espontâneo,
introspectivo, mas, antes, do pequeno grupo fora do qual o sujeito grego
arcaico mal pode ser concebido. {[}\ldots{}{]} Por sua própria natureza, portanto,
centra"-se nas relações pessoais entre um poeta individual e outro membro de seu
grupo de amigos, ou entre ele e o grupo como um todo, ou ainda entre ele e
indivíduos de fora desse grupo. {[}\ldots{}{]} Logo, em geral, a poesia monódica tem
dois modos principais: erótico para com os de dentro do mesmo grupo, de
invectiva, contra os de fora.\footnote{ Most (1982, p.\,90).}
\end{quote}

O simpósio é mais voltado ao “mundo do privado”, frisa Schimitt"-Pantel, menos
formal e oficial do que o festival próprio à \textit{performance}
da canção coral; nem por isso, todavia, deixa de ser público e
ritualizado.\footnote{ Schimitt"-Pantel (1990, p.\,25).} Na
“tentativa de generalização”, Most diz sobre a \textit{performance} e função da
mélica monódica, que essa poesia 

\begin{quote}
era apresentada em ocasiões informais, para
pequenos grupos ligados por laços de amizade e interesse comum, e cumpria a
função social de unir esses grupos em todos coesos e separá"-los ou colocá"-los
em oposição a outros grupos numa mesma cidade.
\end{quote}

A mélica nas modalidades solo e coral está representada no \textit{corpus} sobrevivente
de Safo, que conta ainda com ao menos um fragmento de narrativa de
caráter \textit{epicizante}.\footnote{Não sabemos se coral ou solo na \textit{performance}} 
Falemos um pouco mais da canção coral.

Feita para a apresentação no festival cívico"-religioso, patrocinado
pelos governos e aristocracias das \textit{póleis}, no qual se
desenvolviam muitas atividades -- como o \textit{agón} ou ``competição'' 
para cada gênero poético, para a música e para o corpo (os jogos) \mbox{--,} a canção coral tinha no poeta o compositor das palavras e
da música, e o mestre do coro de cidadãos, que, liderado por um de seus
membros, cantava e dançava ao som dos instrumentos.


Nessa atmosfera festiva e de celebração pública e coletiva, o canto tinha por
“tônica dominante”, frisa Herington, “o \textit{prazer}, humano e
divino”, pois eram homens e deuses homenageados a um só tempo.\footnote{ Herington (1985, p.\,6).} Mas, para além
do festival, também podiam servir de ocasião à
\textit{performance} da mélica coral os grandes funerais e as grandes bodas.
São notáveis nessas três ocasiões a ideia da
solenidade, o caráter cultual e a comemoração em chave de elogio; e esses
três elementos permeiam a construção das canções corais, refletindo"-se
em sua linguagem altamente elaborada em registro elevado, em seu forte
componente mítico, no canto autodramático do coro, e nas máximas de tom
ético"-moral que pontuam seus versos, a (re)validar e reiterar valores e
ensinamentos compartilhados pela comunidade, pela audiência, pelos
\textit{performers}, pelo poeta cuja voz ganha dimensão mais
pública do que privada, na medida em que sua poesia, diz Most,
“tem papel vital na autoconsciência pública da cidade”.\footnote{Most (1982, p.\,94).}
Lembremos que, no universo grego, “participar de uma \textit{performance} coral ou assisti"-la desempenhavam um papel central na vida cultural e musical”, frisa Laura Swift.\footnote{Swift, (2010, p.\,1).} E a canção coral era apresentada em múltiplas ocasiões de distintas naturezas e amplitudes, por toda a Grécia, em cujas cidades demarcava “os momentos mais significativos da vida de indivíduos e da comunidade, de casamentos, a funerais, a celebrações religiosas cívicas”.\footnote{Swift, (2010, p.\,2).} Por conta disso, desde a infância os gregos (homens e mulheres) eram expostos ao treinamento para a \textit{performance} coral -- canto, dança e o tocar de instrumentos --, que se inscreve nas dimensões política, cultural, ritual, social e cultual da vida cotidiana, e na dimensão da \textit{paideía} grega, “a educação em amplo sentido”, como ressalta Anton Bierl.\footnote{Bierl (2011, p.\,417).}

As composições de Safo, volto a afirmar, são canções, mais
precisamente, fragmentos de algo mais próximo do que chamamos “canção” do que
daquilo que chamamos “poema”, retirada do primeiro termo a ideia moderna da
prevalência da música sobre o texto, ou de uma relação de paridade entre ambos,
diz a advertência de Guerrero,\footnote{ Guerrero (1998, p.\,18).} uma vez que, na era arcaica,
completa Eric \textsc{a}.\,Havelock, 

\begin{quote}
a melodia permaneceu serva
das palavras, e seus ritmos foram moldados para obedecer à pronúncia
quantitativa da fala.\footnote{ Havelock (1996, p.\,132).}
\end{quote}

 A poesia grega antiga, cabe recordar, não se
escandia por acento, mas por duração breve ou longa de pronúncia da sílaba, o
que já lhe confere, pela natureza de seus metros, uma sonoridade bem ritmada. Em Safo, até onde permite afirmar seu \textit{corpus} preservado,
há canções majoritariamente de estrofes breves, de variados temas,
linguagens, tons e metros -- estes, um elemento de destaque em sua mélica, nortearam a organização de sua obra na Biblioteca de Alexandria.
Permanecem, contudo, problemáticas
sua audiência, sua ocasião e seu modo de \textit{performance} presumivelmente com participação da própria poeta, na Mitilene arcaica, e de seu grupo de \textit{parthénoi}, em festivais cívico"-cultuais públicos e em cerimônias de casamento, ocasiões valorizadas no estudo de Franco Ferrari.\footnote{Ferrari (2010) se baseia nas novas reflexões a que levaram os avanços nos estudos de \textit{performance} e a descoberta do novo fragmento da poeta (``Canção sobre a velhice''), em 2004, que nos fez reavaliar a coralidade em Safo e a natureza coral de sua associação de \textit{parthénoi}. Ferrari traz um olhar fresco ao aproximá"-la de Álcman, o poeta dos partênios -- canções para coros de virgens -- atuante na Esparta de c.\,620 a.C. Ver ainda Ragusa (2019a, pp.\,85--111; 2019b, pp.\,211--39).}
Diga"-se, nas palavras de Giulia Sissa,\footnote{Sissa (1990, p.\,76).} que a \textit{parthénos} se define “por idade e estatuto marital”, e é um “estágio pelo qual toda mulher passava na rota rumo à completa integração social” pelo \textit{gámos} (casamento). A \textit{parthénos} é, pois, aquela que chegou ao amadurecimento sexual indicado pela menarca, ao florescer da feminilidade. Como diz Mary Lefkowitz,\footnote{Lefkowitz (1995, p.\,32).} “o principal papel da mulher na sociedade antiga” era desempenhado pela \textit{gyné}, a ``mulher adulta'': o de ser esposa e mãe; daí o interesse e a atratividade da \textit{parthénos}, cuja energia vital deve convergir para o mundo do \textit{gámos} que a institucionaliza socialmente e que lhe confere, uma vez consumado, um lugar social definido na condição de \textit{gyné.} Daí o fascínio da figura da \textit{parthénos}, cujos principais cantores são decerto os mélicos Álcman, poeta ativo na Esparta de c.\,620, famoso pelas canções para coros de virgens (\textit{partênios}) e Safo.
No mundo da coralidade e como \textit{khorodidáskalos}, ``mestra do coro'', equiparada à líder das Musas,\footnote{Bela Voz ou Calíope.} vislumbra"-a um epigrama anônimo que a insere no famoso templo lésbio de Hera, em que, segundo Alceu,\footnote{Fr.\,130\textsc{b}, traduzido em Ragusa, 2013.} contemporâneo da poeta em Lesbos, havia concursos anuais de beleza em homenagem à deusa:

\begin{verse}
%\parindent=0em 
\small{Vinde ao sacro recinto esplêndido de Hera de olhos de touro,\\
\hspace*{4.5em} ó lesbias, os delicados passos dos pés rodopiando;\\
lá, belo coro à deusa firmai; a vós liderará\\
\hspace*{4.5em} Safo, nas mãos tendo a áurea lira.\\
Felizes -- vós! -- no regozijo da dança; sim!, doce hino\\
\hspace*{4.5em} pensareis ouvir, da própria Calíope!}\footnote{\textit{\textsc{ap}} \textsc{ix}, 189.}
\end{verse}

Se olharmos para a canção poética, seus temas mais recorrentes e a inserção da
primeira pessoa do singular, e para a canção popular -- algumas do universo grego nos
foram transmitidas --, perceberemos as origens pré"-literárias da mélica. Afinal,
o canto é uma forma de expressão verbal própria do homem, e a canção popular é
“atributo quase universal de sociedades tradicionais”, frisa Bowie.\footnote{
Bowie (1984, p.\,3).} Lesky comenta as formas
pré"-literárias da mélica: cantos de culto aos deuses, de lamento ou de celebração
nos “momentos culminantes da vida e da morte” ou de acompanhamento do trabalho nos teares, na colheita das uvas, ou ainda de queixas pelo sofrimento amoroso.\footnote{ Lesky (1995, pp.\,133--4).} Esses cantos ligam"-se ao ritmo do
cotidiano e às festividades das comunidades envolvidas na sua
\textit{performance}. Note"-se, enfim, que boa parte da música entoada pelas
personagens épicas da \textit{Ilíada} e da \textit{Odisseia} ilustra as formas
lembradas por Lesky, e, por conseguinte, relaciona"-se principalmente à mélica
arcaica e a suas espécies.

Em síntese, a mélica é um gênero de evidentes raízes
encravadas em tradições populares, visíveis em subgêneros
como o epitalâmio ou canção de casamento, que muito pode nos contar, tomada
junto a outras fontes escritas e iconográficas, sobre a cerimônia
da boda. Segundo Cecil \textsc{m}.\,Bowra, com variações
impostas por tradições locais, os epitalâmios apontam características gerais:
o banquete inicial na casa do pai da noiva; os sacrifícios aos
deuses do casamento na festa; a noiva, escondida por um véu, sentada junto às
outras virgens, aguardando a apresentação ao noivo, feita por parentes e amigos
dela; a procissão que acompanhava os noivos a seguirem em carruagem rumo à nova
morada; a condução dos noivos ao aposento nupcial, para consumar a boda,
condução esta feita pelos amigos de ambos, com danças, cantos, brincadeiras e
tochas, visando amenizar a tensão dos noivos -- em geral, estranhos um ao outro
-- e propiciar sua união. Aliás, a temática dos epitalâmios varia entre a
solene, quando centrada na procissão da carruagem dos noivos, jocosa, quando na
condução deles ao leito, ou elogiosa, quando na aparência física dos noivos de
modo a estimular a atração entre eles, passo importante à sua necessária união
sexual.\footnote{ Bowra (1961, pp.\,214--8).}

Como se percebe, a execução dos epitalâmios tende a ser sobretudo coral, dado
que reforça seus elos com a tradição popular; e no caso dos fragmentos sáficos
dessas canções, anota Lesky, 

\begin{quote}
{[}\ldots{}{]} vemos como a poesia popular
tradicional é captada em toda a sua frescura e espontaneidade, por uma grande
poetisa que, no âmbito de sua arte, a modela em composições que alcançam uma
forma perfeita, sem perderem o encanto daquilo que surgiu do povo.\footnote{ Lesky (1995, p.\,168).}
\end{quote}

Considerados os epitalâmios sáficos, bem como o já referido fragmento de
narrativa mítica,\footnote{``As bodas de Heitor e Andrômaca''.} neles
encontramos uma poesia menos “pessoal”, que mais ``pública'' se afigura e, por isso mesmo, bem menos propícia à tão frequente leitura romântico"-biografista do \textit{corpus} de
Safo, a qual, voltando"-se aos outros fragmentos, sobretudo aos mais famosos,\footnote{Fragmentos 1, 2, 16, 31, 94, 96.} insiste em tomar por íntima, pessoal e privada a voz de seus versos, nos quais ela se revela “a maior mestra em pseudointimidade”, nas palavras de Ruth Scodel, dando"-nos amiúde os seus poemas “uma impressão extraordinariamente exitosa de terem sido compostos para ela e suas amigas” -- impressão que “por vezes ela manipula”.\footnote{Scodel (1996, p.\,77).}
A temática
erótico"-amorosa -- uma de suas linhas de força --, a autonomeação
pela qual se faz a autodramatização da poeta como \textit{persona}, a força da primeira pessoa do singular, o caráter
aparentemente intimista, as numerosas figuras
femininas contempladas em chave erótica --, tudo isso estimula aquela referida.
Mas tais elementos, incluindo o homoerotismo, mais coerentemente devem ser pensados, como se tem argumentado após a descoberta da ``Canção sobre a velhice'', no âmbito da coralidade e das associações corais femininas, atestadas com evidências consistentes pelo mundo arcaico afora.\footnote{Ver estudos indicados à nota 75 e ainda Lardinois (1994, pp.\,57--84; 1996, pp.\,150--72). Todos tratam da relação das meninas e de sua líder, Safo, a partir da ideia da coralidade e das associações corais em que se dava a \textit{paideía} feminina, atestadas no mundo arcaico. Tal formação, sentido do termo grego, era conduzida por uma mulher adulta, e tinha no casamento uma de suas orientações, bem como na formação ético"-moral e nas atividades femininas, no cantar e no dançar, nos ritos e em outros âmbitos.} 

Nelas, as \textit{parthénoi}, as moças não casadas, recebiam a formação -- \textit{paideía}-- feminina específica, que incluía a \textit{khroreía} ou ``atividade coral'' -- canto, dança, música --, a reafirmação de valores ético"-morais e o conhecimento da tradição mítica -- elementos de força nas composições da mélica coral em geral --, a preparação para o \textit{gámos} no exercício da sensualidade, do erotismo, internamente ao universo feminino. A formação do grupo de meninas, sublinha Christina Clark,\footnote{Clark (1996, p.\,144).} era parte integral de um “processo que inculcava responsabilidade cívica, valores sociais e tradições {[}\ldots{}{]} codificados na \textit{performance} que servia para integrar o indivíduo do sexo feminino em seu contexto social”.


No que tange ao biografismo e à simplista equiparação poeta"-\textit{persona} aos quais tão bem se prestam as canções da mélica sáfica forjadas, pela habilidade retórica, para serem percebidas como intimistas, pessoais e privadas, há que reafirmar: se a identificação é problemática,
a separação radical também o é; na verdade, há que reconhecer certa medida de
proximidade e outra de distância entre o \textit{eu} do poeta e o de seus versos. Qual
medida? Eis o enigma próprio da natureza poética dos textos; arriscar
responder à esfinge, em se tratando dos poetas antigos, cujas figuras nossos
olhos bem mal alcançam, é deixar devorar"-se a frágil matéria que deles nos foi
transmitida pela especulação e pela simplificação.

Vale reiterar que a relação na poesia entre a primeira pessoa do
singular de um poema e o sujeito empírico do poeta é, desde o século \textsc{xix}, um
dos grandes nós no estudo da elegia, do jambo e da mélica arcaicos,\footnote{
Ver Slings (1990, pp.\,1--30). Corrêa (2009, pp.\,31--93) discute, com indicação de
bibliografia pertinente, os trabalhos de Fränkel (1975; 1ª edição
1951) e Snell (2001; 1ª edição 1955), da “escola Fränkel"-Snell”, de
caráter romântico"-hegeliano; também Ragusa (2005, pp.\,23--44), com o foco em
Safo.} mas que o biografismo é corrente entre os antigos, que alimentavam a
imagem dos poetas tão admirados quanto desconhecidos a partir de seus poemas,
agindo esses estudiosos como “poetas da ficção biográfica”, sintetiza Diskin
Clay.\footnote{ Clay (1998, p.\,10). Não por acaso “a separação entre poeta e
\textit{persona} chegou tarde e com grande dificuldade”, prossegue ele (p.\,16),
tendo sido o latino Catulo (século \textsc{i} a.C.) o primeiro poeta a clamar por tal
separação, num protesto poético em chave de virulenta vituperação de tradição
jâmbica, o que, portanto, compromete a argumentação. Cito o Poema 16 (tradução
Oliva, 1996): ``\textit{Meu pau no cu, na boca, eu vou meter"-vos,/ Aurélio bicha
e Fúrio chupador,/ Que por meus versos breves, delicados,/ Me julgastes não
ter nenhum pudor./ A um poeta pio convém ser casto/ Ele mesmo, aos seus
versos não há lei./ Estes só têm sabor e graça quando/ São delicados, sem
nenhum pudor,/ E quando incitam o que excite não/ Digo aos meninos, mas esses
peludos/ Que jogo de cintura já não têm/ E vós, que muitos beijos (aos
milhares!)/ Já lestes, me julgais não ser viril?/ Meu pau no cu, na boca, eu
vou meter"-vos}''. As questões que se impõem, insolúveis, são: quem fala
aqui? Catulo ou seu \textit{eu} poético? Quão identificados estão um e outro? E quanto
pesa na linguagem o gênero com o qual dialoga o poema, o jambo, de vituperação,
de ataque aos inimigos, de sarcasmo e de sátira?} Como bem resume Simon \textsc{r}.~Slings, 
todavia, as modernas teorias sobre a poesia “lírica”
pressupõem a leitura; mas a “lírica” grega arcaica e clássica destinava"-se à
\textit{performance} perante uma audiência, em dada situação e de certo modo
que se relacionavam ao gênero -- por conseguinte, à linguagem, à matéria e ao
metro -- do poema apresentado de viva voz. Assim, o poeta grego arcaico,
“precisamente porque é o causador de uma experiência estética, é, em certa
medida, despersonalizado”:

\begin{quote}
Se olharmos para o problema desse ângulo, tornar"-se"-á claro, de imediato, que
a oposição Eu ficcional x Eu biográfico é, na verdade, uma simplificação
irresponsável. O Eu é o Eu do \textit{performer}, que se move através de um
\textit{continuum} no qual o Eu biográfico e o Eu ficcional são os dois
extremos: na maior parte do tempo, ele não é nenhum deles.\footnote{ Slings (1990, pp.\,11--12).}
\end{quote}


\section*{em síntese: safo, sua mélica e o coro de virgens}

Resumo o cenário para Safo e sua mélica. Esta, como toda a poesia grega, tem por principais ocasiões de \textit{performance} os festivais cívico"-cultuais públicos e o simpósio. Este, para as canções de Safo, destina"-se à \textit{reperformance}, e não à apresentação original da poeta, porque é, como se disse, eminentemente masculino. Isso significa, porém, que, na dramaticidade inerente à “cultura da canção”, o cantor de sua poesia no simpósio assumia o papel do \textit{eu} feminino, quando feminina fosse a voz da canção. Mais seguras como ocasiões de \textit{performance} da mélica sáfica são o festival público cívico"-cultual e a festividade de casamento. Ferrari sublinha nos fragmentos da poeta as indicações nesse sentido: 

\begin{quote}
{[}\ldots{}{]} as referências a vigílias noturnas, templos, bosques sacros, altares, sacrifícios, cantos, danças, instrumentos musicais, grupos corais, mantos, guirlandas, parecem organicamente reconduzir ao âmbito das festividades cerimoniais nupciais, ritos em honra de Afrodite, Adonias, hinos a divindades no espaço sagrado de um templo (Afrodite, Nereidas, Hera, Ártemis), competições de beleza -- que de vários modos e com várias funções animavam a vida comunitária de Mitilene {[}\ldots{}{]} e da ilha de Lesbos.
	
Internamente à festa, vemos a ativação de duas distintas modalidades comunicativas: de um lado, canções geralmente corais dotadas de uma função especificamente pragmática e portanto destinadas a escandir momentos significativos do evento comunitário; de outro, composições monódicas (mas com possível acompanhamento de dança) ou corais que se correlacionavam a emoções, situações, contatos que interessavam a Safo e suas adeptas e que dos eventos cruciais da relacionada ocasião ritual podiam configurar uma espécie de regência por meio de instrução, comandos e sugestões.\footnote{Ferrari (2003, p.\,89).}
\end{quote}

As canções de Safo são dirigidas a seu grupo feminino de \textit{parthénoi}, as moças ou meninas. Isso que foi reforçado pela “Canção sobre a velhice”, como o foi a coralidade, já se via nos seus fragmentos, na frequência impressionante de termos como \textit{kórai}, \textit{parthénoi}, \textit{paîdes}, que, para além de nomes de figuras femininas, fazem das meninas, “com toda probabilidade, o assunto da maior parte da poesia de Safo”, sublinhava Lardinois.\footnote{Lardinois (1994, p.\,70).} A propósito dessa coralidade feminina, Lardinois afirma:

\begin{quote}
Dado o fato de que as mulheres gregas antigas, e tanto mais as jovens, não eram encorajadas a falar em público, é notável que tantas \textit{performances} de coros de moças jovens sejam atestadas, especialmente no período arcaico.
\end{quote}

Atestadas, sim; sobreviventes, bem poucas. Mas canções como as de Álcman -- os \textit{partênios} -- ``adotam uma perspectiva distintivamente feminina, ambos nos mitos que contam e nas visões que expressam sobre outras mulheres'', prossegue, esclarecendo que usa o termo feminino como designação para ``um conceito ou comportamento que uma dada cultura tipicamente associa às mulheres {[}\ldots{}{]}''.\footnote{Lardinois (2011, p.\,161).}

As canções, observa, não são biologicamente femininas nem feministas -- este termo implicaria que assumem uma “postura crítica com relação à cultura (masculina) dominante”,\footnote{Lardinois (2011, p.\,161).} algo que não fazem, uma vez que, no mundo arcaico e clássico, tradicional, o poeta trabalha em adesão aos valores, práticas, códigos, que sua poesia revalida e reafirma, funcionando, frisei anteriormente, como instrumento de formação de sujeitos sociais e de preservação dos elementos que identificam as comunidades. Essas canções, diz Lardinois,\footnote{Lardinois (2011, pp.\,161-2).} não querem “mudar a cultura dominante, dispensando as tarefas que essa cultura atribui às mulheres, tais como o casamento ou a maternidade”, mas, atente"-se para isto, “elas realmente convocam a uma maior apreciação dos papéis femininos e dos sacrifícios deles advindos”. As canções, conclui, “não são revolucionárias ou subversivas. Elas propagam valores femininos comumente aceitos, como a maternidade, o amor, e o casamento”.\footnote{Lardinois (2011, p.\,172).} Isso tudo é patente em Safo, nos fragmentos, como procura realçar seu arranjo temático nas traduções desta antologia: a importância do mundo feminino, logo, do casamento, do erotismo, da graciosidade, do apuro estético, da noção de beleza. E isso é próprio aos grupos corais de meninas, largamente atestados na Grécia arcaica e clássica: o exercício da feminilidade e da sensualidade no vestir, no adornar e adornar"-se, no cantar e dançar, articula"-se à formação ético"-moral pela reflexão sobre valores não raro ativados pela tradição mítica que deve ser conhecida e que encerra, ainda, as práticas rituais e relativas à dimensão cultual e às etapas significativas da vida. Trata"-se de um dado próprio a uma sociedade tradicional que, anota Bierl,\footnote{Bierl (2016, p.\,311).} “definia"-se a si própria, em medida considerável, pelo mito e pelo rito”. 


\section*{outras poetas}

No epigrama da \textit{Antologia palatina} -- compilação de 15 livros de
epigramas datados dos séculos \textsc{vii} a.C. ao \textsc{v} d.C. --, que abre esta introdução,
nove poetas mulheres são nomeadas -- nove são as Musas, e nove é o número comum
às listas de cânones de poetas, como a do epigrama, que proliferam como reflexo
dos trabalhos na Biblioteca de Alexandria, da qual dentro em pouco falarei.
Somadas, elas compõem um modesto conjunto de obra.
Ofereço, pois, um olhar panorâmico para elas.\footnote{Ver Ragusa (2020, pp.\,113--136), para mais sobre as poetas e sua relação com a tradição poética grega.}

\subsection*{Grécia clássica: Mirtes, Praxila,\break Telesila (e
Corina?)\protect\footnote{\MakeUppercase{P}ara todas essas poetas, suas obras e testemunhos sobre elas, ver a edição bilíngue de \MakeUppercase{C}ampbell (1992).}}

Mirtes (Beócia, fins do século \textsc{vi} a.C.), da qual nada resta, teria sido mestra
de Píndaro (séculos \textsc{vi}--\textsc{v} a.C.) e de Corina -- neste caso, datada do século \textsc{v}
a.C., não do \textsc{iii} a.C., como parece também ser possível. Todos os três são
mélicos predominantemente corais, e estão relacionados no Fr.\,664\textsc{a}
Page (1962) de Corina, cuja compreensão
é bastante nebulosa, como a própria imagem de Mirtes:


\begin{verse}
\small{\ldots{} eu censuro a clarissonante\\
Mirtes eu mesma, porque, sendo mulher,\\
entrou em rivalidade com Píndaro \ldots{}}\footnote{Cito a tradução dada em Ragusa (2020, p.\,121).}
\end{verse}

Praxila, ativa em meados do século \textsc{v} a.C., assim surge num passo do gramático
Ateneu (séculos \textsc{ii}--\textsc{iii} d.C.), no \textit{Banquete dos eruditos}:\footnote{15, 694\textsc{a}.}
``Praxila de Sícion era também admirada pela composição de cantos
convivais” para acompanhar o beber do vinho, nos simpósios. De sua obra,
porém, temos apenas o fragmento de um “Hino a Adônis”,\footnote{747 Page.} outro
de um “Ditirambo: Aquiles”,\footnote{748 Page.} e três de canções convivais. No fragmento
hínico, o mítico jovem amante de Afrodite, Adônis, arrola as mais belas coisas
que deixou no mundo dos vivos ao morrer, no auge de sua juventude e virilidade:

\begin{verse}
\small{\ldots{} o que de mais belo eu deixo: a luz do sol,\\
 depois, as estrelas luzentes, e da lua, sua face,\\
e também maduros pepinos e pomos e peras \ldots{}}\footnote{Cito a tradução dada em Ragusa (2020, p.\,124).}
\end{verse}

Do fragmento de ditirambo -- gênero mélico de difícil classificação, com forte
componente narrativo --, há um verso: ``{[}\ldots{}{]} mas teu coração no peito nunca
eles persuadiram {[}\ldots{}{]}”. Nele, Praxila lembra a intolerância de Aquiles e sua
inabalável recusa ao combate junto aos gregos na luta contra os troianos, em razão da
grave ofensa à honra que lhe fizera o chefe da expedição, o Atrida
Agamêmnon, ao arrebatar"-lhe Briseida, seu \textit{géras}, “prêmio” -- parte do
botim de guerra que é a medida da honra de quem o recebe --, conforme o canto \textsc{i}
da \textit{Ilíada}.

Telesila (Argos, meados do século \textsc{v} a.C.), é associada a um episódio marcial
supostamente biográfico: segundo Plutarco\footnote{Em \textit{As
virtudes das mulheres}, 4. 245c"-\textsc{f}.} (séculos \textsc{i}--\textsc{ii} d.C.) e Pausânias\footnote{Em 
\textit{Descrição da Grécia}, 2. 20. 8--10.} (século \textsc{ii} d.C.), ela teria liderado um conflito
militar contra Cleomenes, de Esparta, o que jogaria sua datação para \textit{c.}
494 a.C. Pausânias descreve, no santuário de Afrodite em Argos, diante da
estátua sentada da deusa, a imagem de Telesila, ``a compositora de
canções”, numa estela:\footnote{ Placa funerária comum nos túmulos gregos, com
inscrições e sobretudo desenhos sobre os mortos e a vida que deixam para trás.} ``seus livros caídos aos seus
pés, ela olha para um elmo, segurando"-o com a mão e prestes a pô"-lo sobre sua
cabeça”. Para o viajante, a poesia deu"-lhe ainda ``maior honra” do
que o mais que realizou. Interessa notar, sobre a imagem de guerreira e poeta,
que a temática marcial deve ter sido relevante nas canções de Telesila, a
confiarmos em Máximo de Tiro\footnote{Em \textit{Oração} 37. 5.} (século \textsc{ii} d.C.), segundo
quem ``os espartanos eram exaltados pelos versos de Tirteu” -- 
poeta elegíaco de meados do século \textsc{vii} a.C. --, ``os argivos, pelas
canções de Telesila, e os lésbios, pelas odes de Alceu”. 

\textls[-15]{A imagem da poeta pode ter se nutrido dos
seus versos, tanto quanto o episódio biográfico, cuja historicidade não
é verificável em nossas evidências.
Infelizmente, não se percebe a temática guerreira na obra
remanescente de Telesila: cinco fragmentos, o maior, com dois versos,}\footnote{Fr.\,717 Page.} 
ao que se vislumbra, sobre o mito da paixão do deus"-rio Alfeu pela irmã de Apolo,
a virgem caçadora Ártemis.\looseness=-1

Por fim, se aceitarmos uma de suas possíveis datações, temos Corina (Tânagra) na
era clássica -- mas ela pode ter vivido na helenística.\footnote{A questão cronológica permanece em disputa, mas fato é que o nome de Corina apenas se registra, para nós, em
fontes de 50 a.C. em diante. Faço a discussão do problema em Ragusa (2020, pp.\,26--7). Ver Ragusa e Delfito (2020, pp.\,3--16) sobre o problema de datação e a obra da poeta, com todos os seus fragmentos legíveis traduzidos.} De sua obra, temos
fragmentos em que se nota a presença bem marcada de mitos da Beócia, sua região.
O mais longo\footnote{654 Page.}\textls[-15]{deles traz a competição poética entre duas montanhas beócias,
Hélicon, o abrigo das Musas, e Citero, no limite com a vizinha região da Ática,
e algo sobre as filhas de Asopo, deus"-rio beócio. Outro fragmento}\footnote{655 Page.}
canta, no que parece ser o proêmio:\footnote{Cito a tradução em Ragusa (2020, p.\,126).}\looseness=-1

\begin{verse}
\small{{[}\ldots{}{]} sobre mim Terpsícore {[}\ldots{}{]}\\
belos contos a cantar\\
às tanagrenses de alvos peplos\\
e grandemente se alegra a cidade\\
com minha clarivívida voz.}
\end{verse}

Há, ainda, fragmentos de alguns versos, uma única linha ou palavra, os quais
formam a maioria dos pequenos e corrompidos textos do \textit{corpus} de
Corina, como o Fr.\,664\textsc{b}. Nele, a \textit{persona} da poeta declara um
tema de sua mélica:

\begin{verse}
\small{{[}\ldots{}{]} canto as excelências dos heróis\\
e das heroínas \ldots{}}\footnote{Cito a tradução em Ragusa e Delfito (2020, p.\,8).}
\end{verse}

\subsection*{Grécia helenística: Mero,\break Erina, Anite, Nóssis}

Esse grupo nos remete à primeira parte da era helenística, cujo centro não é
mais Atenas, mas Alexandria, no Egito ptolomaico, onde o grego foi tornado
língua oficial da administração, do comércio e da educação, e onde proliferaram
escribas, sobretudo na cidade, sede da Biblioteca que era, na
verdade, uma sala entre outras do \textit{Mouseîon} (“a casa das Musas”,
“Museu”) erguido pelo faraó Ptolomeu \textsc{i}, o Sóter, que reinou entre 305--285 a.C.,
e que fora general de Alexandre, o pupilo de Aristóteles. Seu objetivo era um
só: edição e cópia das grandes obras dos antigos, em organização de forte
inspiração aristotélica; e Ptolomeu \textsc{ii}, o Filadelfo, no poder entre 285--246
a.C., ampliou a Biblioteca, de modo a permitir a intensificação desses
trabalhos depois interrompidos por uma catástrofe em 47 a.C., mas retomados
eventualmente e ativos até meados do século \textsc{v} d.C. 

Nesse mundo, “a poesia do passado”, declara Gentili, “passou a
ser lida como literatura pura e simples”, embora fosse ainda recitada -- reflexo
ainda vivo da cultura oral em que se produziu.\footnote{ Gentili (1990, p.\,37).} E em termos da poesia,
destaca"-se entre os gêneros praticados no período o epigrama, “poema curto em
dísticos elegíacos” -- metro próprio da poesia elegíaca -- e de conteúdos
variados, lembra Jane \textsc{m}.\,Snyder, entre os quais, “lamentos,
dedicatórias, casos amorosos, animais de estimação e assim por diante”.\footnote{ Snyder (1989, p.\,66).}
\textls[-10]{Originalmente, o epigrama “limitava"-se a servir de epitáfio”, anota a
helenista; e “a palavra grega \textit{epigramma} significa ‘inscrição’\,”. Os
epigramas das poetas de que passo a me ocupar encontram"-se na \textit{Antologia
palatina}.}

%Dúvida: esse "11, 491b diz respeito ao que?"
Mero, de Bizâncio, é a mais desconhecida; dela só há dois epigramas na \textit{\textsc{ap}}:\footnote{Livro \textsc{vi}, 119 e 189.} uma dedicatória às uvas viníferas,
outra às ninfas das águas. Mas em Ateneu\footnote{11, 491\textsc{b}.} encontram"-se ainda dez
versos hexamétricos -- metro da épica grega e da poesia didático"-sapiencial,
sobretudo --, que tratam da constelação das Plêiades. Esse trânsito por entre os
gêneros não é novidade nem na produção poética dos gregos, nem no período, mas
se atesta desde a era arcaica.
Cito de Mero o epigrama 119:

\begin{verse}
\small{Jazes sob o áureo pórtico de Afrodite,\\
ó cacho d'uvas, pleno da gota de Dioniso.\\
Não mais tua mãe, amável ramo atirando em teu\\
redor, porá sobre tua cabeça pétala nectárea}\footnote{Tradução Ragusa (2020, p.\,130).}
\end{verse}
%Dúvida: padrão "Ragusa" é aceito?

\textls[-10]{Outra poeta, Erina -- não sabemos ao certo sua origem --, também praticou o
epigrama e a poesia épica. Magro, porém, é seu \textit{corpus}: três epigramas, um fragmento com dois versos hexamétricos, pedaços
de “O fuso do tear”, longo poema}\footnote{São cerca de 300 versos.} em hexâmetros e dialetos
lésbio"-eólico e dórico, preservado num papiro muito mutilado que revela somente
algumas palavras. Um epigrama anônimo\footnote{\textit{\textsc{ap}}, \textsc{ix}, 190.}
sobre Erina a dá por lésbia, conta sua morte ainda virgem aos 19 anos, e
afirma que os versos de “O fuso” são \textit{iguais} aos de Homero. Por fim,
declara que ``tanto quanto Safo supera Erina em canções, Erina supera
Safo em hexâmetros”.\footnote{Versos 7--8.}

Sobre os epigramas atribuídos a Erina, Snyder observa
serem todos concernentes às mulheres: o 352\footnote{\textit{\textsc{ap}}, \textsc{vi}.}
retrata uma mulher de nome Agatárquis; o 710 e o 712,\footnote{\textit{\textsc{ap}}, \textsc{vii}} a morte de 
Báucis, sua amiga, que jovem como a poeta morreu -- na noite das suas núpcias.
Cito da poeta o epigrama 710, na tradução do poeta Péricles Eugênio da Silva Ramos:

\begin{verse}
\small{Ó estrelas e Sereias, urna funerária\\ 
\quad que encerras minha pouca cinza,\\
saudai a gente que se acerca de meu túmulo,\\
\quad seja daqui ou forasteira.\\
Contai: mal me casara, a morte me colheu;\\
\quad o nome que meu pai me pôs\\
foi Báucis; Telos, o lugar onde nasci;\\
\quad e Erina, minha amiga,\\
em meu sepulcro estas palavras inscreveu.}\footnote{Tradução Ramos (1964, p.\,161). Noto que as \textit{sereias} são na verdade \textit{sirenas}, mulheres"-aves, no imaginário mítico grego.}
\end{verse}

Já Anite tem 24 
epigramas, informa Snyder.\footnote{ Snyder (1989, p.\,67).} Segundo o \textit{Onomástico}\footnote{5. 48.} de
Pólux (século \textsc{ii} d.C.), ela seria de Tegeia, na Arcádia, dados o dialeto dos
versos, as referências a elementos naturais e a imagem do deus Pã, própria da
mitologia local. Cito dois epigramas da poeta, na tradução de José Paulo Paes:

\begin{verse}
\small{Para o seu gafanhoto, rouxinol dos campos, e a sua\\
cigarra das árvores, fez Miro um duplo túmulo\\
e o regou com lágrimas de menina: pois o cruel Hades\\
levou-lhe embora os dois bichinhos de estimação.\footnote{\textit{\textsc{ap}} \textsc{vii}, 190.}
%\mbox{}\hfill

Vivo, este homem era Manes, um escravo; morto,\\
vale agora o mesmo que o grande Dario.}\footnote{Paes (1995, pp.\,34--5). Dario foi imperador persa entre c. 522--586 a.C.}\footnote{\textit{\textsc{ap}} \textsc{vii}, 538.}
%\mbox{}\hfill 
\end{verse}

Vê"-se acima a diversidade de conteúdos, própria do gênero; e são os temas
pastorais especialmente relevantes nessa poeta da Arcádia -- região configurada
como “o ideal da paisagem bucólica do pastor”, resume Snyder. Eis o epigrama
313:\footnote{\textit{\textsc{ap}} \textsc{ix}.}

\begin{verse}
\small{Senta-te de todo sob as belas folhas vicejantes do loureiro\\
e tira doce porção d’água de beber da graciosa nascente,\\
para que descansem teus membros cansados da labuta\\
do verão, tocados pelo sopro de Zéfiro.}
\end{verse}

\pagebreak

Merecem nota, ainda, os curiosos epitáfios para animais -- dos quais o epigrama
190,\footnote{\textit{\textsc{ap}} \textsc{vii}.} já citado, é exemplo --, o que pode ser visto como uma
brincadeira de Anite com a expectativa de quem ouve um epigrama e espera
“grande solenidade”, observa Snyder.\footnote{ Snyder (1989, p.\,70).} 

Nossa última poeta, Nóssis, nasceu em Lócris, “colônia grega no sul da Itália”,
ressalta Snyder,\footnote{Snyder (1989, p.\,77).} fundada no século \textsc{vii} a.C. Seu \textit{corpus} se
compõe de 12 epigramas, quase todos centrados no
universo feminino e nas deusas Hera e Afrodite. E, como Safo, ela em três deles
se autonomeia, anota a helenista, ``criando a vívida \textit{persona} de uma
mulher que celebra as delícias de Eros e que se proclama, ela própria,
seguidora da tradição poética'' da poeta de Lesbos. Cito dois epigramas,
novamente em tradução de Paes:

\begin{verse}
\small{Nada mais doce que o amor; tudo quanto haja de ditoso\\
lhe fica atrás e eu cuspo da boca até o mel.\\
Eis o que diz Nóssis; aquela a quem a Cípria\footnote{ Outro nome de Afrodite, a deusa do amor erótico, da beleza, da sedução.} não beijou,\\
essa não sabe sequer que flores são as rosas.\footnote{\textit{\textsc{ap}} \textsc{v}, 170.}\medskip
Se fores, estrangeiro, à Mitilene de formosas danças,\\
a qual fez Safo, a flor das Graças, consumir-se,\\ 
diz que a terra locriana produziu, dileta das Musas,\\
alguém que lhe é igual, de nome Nóssis. Vai!}\footnote{\textit{\textsc{ap}} \textsc{vii}, 718. Paes (1995, pp.\,36--7).}
\end{verse}
 %Dúvida: Esse padrão de nota é correto?

Ambos sugerem que o tema amoroso deve ter preenchido alguns de seus textos. No
primeiro epigrama, os dois versos finais parecem referir"-se exatamente à
paixão, pois alinhavam Afrodite, o beijo e as rosas, flores prediletas da
deusa. No segundo, ao igualar"-se a Safo, cujo tema principal, até onde o
\textit{corpus} de sua poesia mélica e a sua reputação na Antiguidade permitem
afirmar, gira em torno de \textit{éros} (paixão, amor, desejo erótico), Nóssis
declara, indiretamente, que sua poesia se afina à mesma temática: Mitilene
gerou uma poeta de \textit{éros}; Lócris, outra que lhe é ``igual”.

Muitos nomes, mas, no geral, escassa substância:\footnote{Alguns nomes mais nebulosos ainda são
os de Megalóstrata, mencionada por um poeta mélico ativo na Esparta de fins do
século \textsc{vii} a.C., Álcman, no Fr.\,59\textsc{b} (edição Davies, 1991;
tradução Ragusa, 2010, p.\,653): ``{[}\ldots{}{]} isto mostrou, das doces
Musas/ o dom, uma das virgens venturosa --/ ela, a loira Megalóstrata {[}\ldots{}{]}'' Há
também o de Cleobulina, filha de Cleóbulo de Lindos, o colecionador de enigmas,
e Carixena, a quem o \textit{Léxico} de Fócio (patriarca de Constantinopla,
século \textsc{ix}) se refere para explicar a expressão ``do tempo de Carixena”:
``Carixena foi uma antiquada tocadora de flauta e compositora de música,
mas alguns a dizem também poeta lírica”. Um provérbio no \textit{Léxico} de
Hesíquio (século \textsc{v} d.C.) buscava já explicar essa expressão depois lembrada em
Fócio: ``Carixena foi famosa por sua estupidez, porque não sabia que era
antiquada. Alguns dizem que ela fazia canções eróticas. Há um provérbio também,
‘o tipo de coisa que é do tempo de Carixena’\,”. Nada resta dessas duas poetas
provavelmente do século \textsc{v} a.C.} é o que se pode espremer do acúmulo dos
séculos que encobrem as obras e as poetas do epigrama de abertura desta
introdução. Distinta e bem mais feliz fortuna teve a obra de Safo -- não sua
figura que, decerto pelo fascínio exercido por sua poesia, ao menos em parte,
tem sido preenchida com múltiplas ficções desde a Antiguidade, as quais a
tornam ainda mais impalpável. 

\section*{A transmissão da mélica de Safo}


{\setlength{\epigraphwidth}{.7\textwidth}
\epigraph{Ó Píndaro, boca sacra das Musas, e loquaz Sirena -- \\
Baquílides! --, e graças eólias de Safo, \\
e escrita de Anacreonte, e quem da fonte homérica \\
extraiu sua própria obra -- Estesícoro! --, \\
e doce página de Simônides, e quem de Peitó\footnotemark{} e dos \\
meninos colheu a doce flor -- Íbico! --, \\
e espada de Alceu, que o sangue de tiranos amiúde \\
derramou, protegendo as leis da pátria, \\
e rouxinóis suaviacantes de Álcman -- sede \qb{}graciosos, vós \\
que fincastes o início e o fim de toda a lírica.\footnotemark{} \\ \bigskip
Gritou alto de Tebas Píndaro; soprou deleites \\
com voz doce"-mel a musa de Simônides; \\
brilha Estesícoro e também Íbico; era doce Álcman; \\
deleitáveis sons dos lábios entoou Baquílides; \\
e Peitó falou junto a Anacreonte; e coisas variegadas canta Alceu, cisne lésbio na Eólida; \\
e dentre os homens Safo não é a nona, mas entre as amáveis Musas, a décima Musa registrada.\footnotemark}{}
}

\setcounter{footnote}{141}
\footnotetext{A deusa Persuasão, crucial na sedução erótica.}

\setcounter{footnote}{142}
\footnotetext{\textsc{ap ix}, 184.}

\setcounter{footnote}{143}
\footnotetext{\textsc{ap ix}, 571.}

\noindent{}Não podemos precisar as razões que favoreceram ou não a
preservação das obras dos poetas gregos. Mas somou"-se aos fatores favoráveis,
em princípio, junto à reputação dos poetas e outros elementos, a edição na
Biblioteca de Alexandria. Aliás, ao tratar do termo \textit{lírica}, disse"-o
tardio, porque seu uso nos remete justamente a esse trabalho de cópia e estudo
que, no caso dos mélicos, teve em Aristófanes de Bizâncio
(\textit{c.} 258--180 a.C.) seu principal executor. Segundo Pfeiffer, desse
erudito pode ser a autoria do cânone dos “nove líricos”,
dado nos dois epigramas declamatórios anônimos que citei
acima.\footnote{Pfeiffer (1998, p.\,205). Traduções: Ragusa (2010, pp.\,27--8), com pequenas modificações.}

A edição dos mélicos listados, além de tardia, seguiu critérios variados e
arbitrários, para nós nem sempre discerníveis: a compilação de Safo pautou"-se
pelo critério métrico, e foi dividida no eloquente número de nove
livros\footnote{ O nono livro seria de epitalâmios, segundo uma hipótese cuja
aceitação não é consensual; ver Lesky (1995, pp.\,168--9). Sobre os critérios
adotados para a edição de Safo, ver ainda Nicosia (1976, pp.\,31--2).}  -- rolos de papiros.

Antes da Biblioteca, a circulação da poesia grega antiga, incluindo a jâmbica,
elegíaca e mélica, foi viabilizada, como vimos, por \textit{performances} e
\textit{reperformances} -- nos mesmos moldes ou não, profissionais ou amadoras --, pela
simples repetição propiciada pela memória,\footnote{ Ver Herington (1985, pp.\,45--8) a respeito.} por inscrições comemorativas em monumentos, por possíveis
e anteriores edições\footnote{ Harvey (1955, p.\,159) afirma que “não há razão para pensar
que as edições alexandrinas foram as primeiras a existir”; e na Atenas clássica
circulavam edições disponíveis dos grandes poetas.} -- termo que deve ser
entendido como cópias de um registro original, em quantidade e difusão muito
restritas e, certamente, custosa.\footnote{ Ver Havelock (1996, p.\,26). Tais
cópias eram feitas sobretudo em papiro, material do qual o Egito, sua fonte,
detinha o monopólio, e que, a partir do século \textsc{vi} a.C., adentra o mundo heleno.
O “livro” é, na verdade, um \textit{bíblos} ou \textit{biblíon}, isto é, rolo
de papiro, sendo tardio o formato do \textit{codex}, do século \textsc{ii} d.C. Sobre
os copistas, muitos devem ter sido escravos, e não necessariamente saberiam ler
o que copiavam.} Tudo isso contribui para a sobrevivência dos textos até os
alexandrinos e para o trabalho destes em suas próprias edições e classificações
numa época em que mudaram demais “as condições fundamentais de produção
poética, assim como a relação entre o poeta e sua audiência”, anota Clay.\footnote{ Clay (1998, p.\,28).} 

Tendo a obra de Safo se inserido nesse cenário geral de circulação e
preservação, como nos foi transmitida? Como chegou até nós? Por dois caminhos
trilhados por toda a literatura produzida na Grécia antiga: por fontes de
transmissão direta -- papiros, manuscritos, inscrições em monumentos, e assim
por diante -- e por fontes de transmissão indireta -- citações. Vejamos. 

Desde a década final do século \textsc{xix} a meados do século \textsc{xx}, sobretudo,
intensos trabalhos de escavações conduzidos no Egito trouxeram à luz uma
incrível massa de papiros literários e não literários, provindos,
majoritariamente, da cidade de Oxirrinco que, para Salvatore Nicosia, “tinha
estreito contato com Alexandria”; diz ele ainda que, “em geral, os
textos lá descobertos reportam à atividade filológica e crítica dos grandes
gramáticos alexandrinos”.\footnote{ Nicosia (1976, p.\,32).}

Com os acréscimos, que hoje ocorrem em ritmo bem mais lento,
muitas obras passaram a ser de fato conhecidas, outras ganharam mais
substância, como a da poeta lésbia. Por outro lado, o volume recuperado demanda
o reconhecimento das pesadas perdas sofridas, com as quais devemos conviver. A
condição do \textit{corpus} da poesia jâmbica, elegíaca e mélica, em termos
quantitativos, melhorou -- o estado material dos papiros, porém, trouxe textos
em geral precários; Frederic \textsc{g}.\,Kenyon frisa:
“É no período lírico, talvez, que as nossas perdas foram maiores; e aqui os
papiros não fizeram muito por nós”.\footnote{ Kenyon (1919, p.\,9).} Animado com os acontecimentos então
recentes, a despeito das frustrações, Kenyon afirmava, ao final de seu
artigo: 

\begin{quote}
Verdadeiramente, para todos aqueles que amam a literatura e reconhecem
na literatura grega a mais alta expressão do pensamento humano, os desertos do
Egito floresceram como uma rosa.\footnote{ Kenyon (1919, p.\,13).}
\end{quote}


Quase 50 anos depois, William \textsc{h}.~\,Willis oferecia ao
leitor um censo dos papiros literários encontrados no Egito, com cerca de 3000
exemplares publicados.\footnote{ Willis (1968, pp.\,205--41).} Eis sua avaliação: 

\begin{quote}
Devemos, é claro, ter em mente as severas limitações de nossa evidência. Quase
todos os nossos papiros vêm de uma única província do mundo greco"-romano; e o
Egito, de muitas maneiras -- na geografia, na tradição e no isolamento político
-- foi uma província atípica. Tampouco podem os nossos textos preservados
derivar uniformemente de todo o Egito. Uma vez que a sobrevivência dos papiros
depende da completa proteção da umidade, as chuvas de Alexandria e a nascente
do Delta, as inundações anuais do Nilo, a irrigação, e o crescimento gradual do
lençol freático ao longo dos séculos -- para não mencionar os inimigos naturais
-- devem, necessariamente, ter"-nos roubado a vasta maioria dos textos antigos.\footnote{ Willis (1968, pp.\,205--6).}
\end{quote}

Além dos fatores relativos ao clima, há que se considerar a sorte, as limitações
relacionadas às próprias escavações e o interesse das equipes quanto ao que
gostariam de ver renascer das areias egípcias. Tudo somado, temos uma medida
dos estragos sofridos pelos papiros: a maior parte desapareceu, e os que
sobreviveram estão corrompidos, mutilados, demasiado escurecidos. Mesmo assim,
tê"-los descoberto foi grande fortuna; e grande foi a sorte dos jâmbicos,
elegíacos e mélicos, que contaram com os esforços de Edgar Lobel, helenista
inglês que trabalhou intensamente com os papiros de Oxirrinco, recorda Willis,
em cujo censo os de Safo concentram"-se nos períodos
romano (31 a.C.--476 d.C.) e bizantino (476--1453).\footnote{ Willis (1968, pp.\,211--3).}

Quanto às fontes de transmissão indireta, paráfrases e citações em escritos
antigos variados, Nicosia observa que devem ter dependido
sobretudo da memória falível e seletiva de quem cita, da versão do texto por
ele conhecida e/ou disponível em cópia escrita, e das suas necessidades para o
uso dos textos citados, as quais influíram no tamanho destes, em geral
reduzido. Tais textos sofreram ainda, lembra ele, alterações decorrentes da
\textit{aticização} dos dialetos nos quais os poemas foram compostos -- no caso de Safo,
o lésbio"-eólico, e não o ático que, em parte pelo impulso de uma Atenas
culturalmente muito poderosa na era clássica, prevaleceu sobre os demais
dialetos gregos. Não obstante os problemas, a maioria dos poetas
arcaicos, notadamente, têm nesse tipo de transmissão uma grande aliada.\footnote{ Nicosia (1976, pp.\,23--5).}

Diante desse quadro, os textos que contam com mais de uma fonte por vezes
apresentam variações, diferenças, que precisam ser resolvidas por escolhas do
editor no trabalho com as obras, ressalta Nicosia.\footnote{ Nicosia (1976, pp.\,28).} Insere"-se na lista
de dificuldades do trabalho com a poesia jâmbica, elegíaca, mélica
ainda isto: o problema do estabelecimento dos textos fragmentários, salvo raras
exceções. 

Vamos, pois, às canções de Safo.

\begin{bibliohedra}
\tit{battistini}, Y. (introd., trad., notas). \textit{Poétesses grecques:
Sapphô, Corinne, Anytè\ldots{}}. Paris: Imprimerie Nationale Éditions, 1998.

\tit{bennett}, C. “Concerning ‘Sappho schoolmistress’\,”. \textit{TAPhA} 124,
1994, pp.\,345--7.

\tit{blundell}, S. \textit{Women in ancient Greece.} London: British Museum
Press, 1995.

\tit{bowie}, A. M. \textit{The poetic dialect of Sappho and Alcaeus.} Salem:
Ayer, 1984. 

\tit{bowie}, E. L. “Early Greek elegy, \textit{symposium} and public festival”.
\textit{JHS} 106, 1986, pp.\,13--35.

\tit{bowra}, C. M. \textit{Greek lyric poetry.} 2ª
ed. Oxford: Clarendon Press, 1961.

\tit{bremmer}, J. N. “Pederastia grega e homossexualismo moderno”. In:
\line(1,0){25}. (org.). \textit{De Safo a Sade: momentos na história da
sexualidade.} Campinas: Papirus, 1995, pp.\,11--26.

\tit{budelmann}, F. “Introducing Greek lyric”. In:
\line(1,0){25}. (ed.). \textit{The Cambridge
Companion to Greek lyric.} Cambridge: Cambridge University Press, 2009, pp.\,1--18.

\tit{burnett}, A. P. \textit{Three archaic poets: Archilochus, Alcaeus,
Sappho.} Cambridge: Harvard University Press, 1983. 

\tit{campbell}, D. A. (ed. e trad.). \textit{Greek lyric \textsc{i}.} Cambridge:
Harvard University Press, 1994. {[}1ª ed.: 1982{]}.

\titidem. (ed. e trad.). \textit{Greek lyric \textsc{iv}.} Cambridge: Harvard
University Press, 1992.

\titidem. (coment.). \textit{Greek lyric poetry.} London: Bristol
Classical Press, 1998. {[}1ª ed.: 1967{]}. 

\tit{carey}, C. “Genre, occasion and performance”. In:
\textsc{budelmann}, F. (ed.). \textit{The Cambridge Companion to Greek
lyric.} Cambridge: Cambridge University Press, 2009, pp.\,21--38.

\tit{carson}, A. \textit{Eros, the bittersweet: an essay.} Chicago:
Dalkey Archive Press, 1998.

\tit{clay}, D. “The theory of the literary \textit{persona} in Antiquity”.
\textit{MD} 40, 1998, pp.\,9--40.

\tit{cole}, S. G. “Could Greek women read and write?”. In: \textsc{foley}, H.
P. (ed.). \textit{Reflections of women in antiquity.} Philadelphia: Gordon and
Breach, 1992, pp.\,219--45.

\tit{corrêa}, P. da C. \textit{Armas e varões: a guerra na
lírica de Arquíloco.} 2ª ed. revista e ampliada. São Paulo: Ed. da Unesp, 2009.

\tit{d’alessio}, G. B. “Past, future and present past: temporal
\textit{deixis} in Greek archaic lyric”. \textit{Arethusa} 37, 2004, pp.\,267--94.

\tit{davies}, M. (ed.). \textit{Poetarum melicorum Graecorum fragmenta \textsc{i}.}
Oxford: Clarendon Press, 1991.

\tit{de martino}, F. “Appunti sulla scrittura al femminile nel mondo antico''.
In: \line(1,0){25}. (ed.). \textit{Rose} \textit{de Pieria.} Bari: Levante Editori,
1991, pp.\,17--75.

\tit{dover}, K. J.  \textit{A homossexualidade na Grécia antiga.}
Trad. L. S. Krausz. São Paulo: Nova Alexandria, 1994. {[}1ª ed. orig.: 1978{]}.

\tit{easterling}, P. E.; \textsc{knox}, B.W. (ed.). \textit{The Cambridge History of
classical literature -- \textsc{i}: Greek literature.} Cambridge: Cambridge University
Press, 1990.

\tit{ferrari}, F. \textit{Sappho’s gift: the poet and her community.} Trad.
B. Acosta"-Hughes e L. Prauscello. Ann Arbor: Michigan University Press, 2010.

\tit{foley}, H. P. (ed.). \textit{Reflections of women in antiquity.}
Philadelphia: Gordon and Breach, 1992.

\tit{fränkel}, H. \textit{Early Greek poetry and philosophy.} Trad. M. Hadas
e J. Willis. Oxford: Basil Blackwell, 1975. {[}1ª ed. orig.: 1951{]}. 

\tit{gentili}, B. \textit{Poetry and its public in ancient Greece.} Trad. A.
T. Cole. Baltimore: The Johns Hopkins University Press, 1990a.
{[}1ª ed. orig.: 1985{]}

\titidem. “Lo ‘io’ nella poesia lirica greca”. \textit{AION (filol)} 12,
1990b, pp.\,9--24.

\tit{guerrero}, G. \textit{Teorías de la lírica.} México: Fondo de
Cultura Económica, 1998.

\tit{johnson}, W. R. \textit{The idea of lyric. Lyric modes in ancient and
modern poetry.} Berkeley: University of California Press, 1982.

\tit{hallett}, J. P. “Sappho and her social context”. In: \textsc{greene}, E.
(ed.). \textit{Reading Sappho: contemporary approaches.} Berkeley: University
of California Press, 1996, pp.\,125--42.

\tit{harvey}, A. E. “The classification of Greek lyric poetry”. \textit{CQ}
5, 1955, pp.\,157--75.

\tit{havelock}, E. \textit{A revolução da escrita na Grécia e suas
consequências culturais.} Trad. O. J. Serra. São Paulo, Rio de Janeiro: Editora
da Unesp, Paz e Terra, 1996.

\tit{henderson}, W. J. ``Received responses: ancient testimony on Greek lyric imagery.'' 
\textit{AClass} 41, 1998, pp.\,5--27. 

\tit{herington}, J. \textit{Poetry into drama. Early tragedy and the Greek
poetic tradition.} Berkeley: University of California Press, 1985.

\tit{kenyon}, F. G. “Greek papyri and classical literature”.
\textit{JHS} 39, 1919, pp.\,1--15.

\tit{lardinois}, A. “Safo lésbica e Safo de Lesbos”. In: \textsc{bremmer}, J.
(org.). \textit{De Safo a Sade: momentos na história da sexualidade.} Campinas:
Papirus, 1995, pp.\,27--50.

\tit{lesky}, A. \textit{História da literatura grega.} Trad. M. Losa. Lisboa:
Fundação Calouste Gulbenkian, 1995. {[}1ª ed. orig.: 1957{]}.

\tit{lourenço}, F. (trad.). \textit{Poesia grega de Álcman a Teócrito.}
Lisboa: Livros Cotovia, 2006.

\tit{mossé}, C. \textit{La femme dans la Grèce antique.} Paris: Éditions
Complexe, 1991.

\tit{most}, G. W. “Greek lyric poets”. In: \textsc{luce}, T. J. (ed.). \textit{Ancient writers -- \textsc{i}: Greece and Rome}. New York: Charles Scribner's Sons, 1982, pp.\,75--98. 

\titidem. ``Reflecting Sappho.'' In: \textsc{greene}, E.~(ed.) 
\textit{Re"-reading Sappho: reception and transmission}. Berkeley: University of California Press,
1996, pp.\,11--35.

\tit{murray}, O. “Sympotic history”. In: \line(1,0){25}. (ed.).
\textit{Sympotica. A symposium on the symposion.} Oxford: Clarendon Press,
1990, pp.\,3--13.

\titidem. \textit{Early Greece.} 2ª ed.
Cambridge: Harvard University Press, 1993.

\tit{nicosia}, S. \textit{Tradizione testuale diretta e indiretta dei poeti
di Lesbo.} Roma: Ateneo, 1976. 

\tit{oliva neto}, J. A. (trad., introd. e notas). \textit{Catulo. O Livro de
Catulo.} São Paulo: Edusp, 1996.

\tit{oliveira}, F. R. (introd., trad. e notas). \textit{Hipólito. Eurípides.}
São Paulo: Odysseus, 2010.

\tit{paes}, J. P. (trad., notas, posfácio). \textit{Poemas da Antologia grega
ou palatina, séculos \textsc{vii} a.C. a \textsc{v} d.C.} São Paulo: Companhia das Letras, 1995.

\tit{page}, D. L. (ed.). \textit{Poetae melici Graeci.} Oxford: Clarendon
Press, 1962.

\titidem. \textit{Sappho and Alcaeus.} Oxford: Clarendon
Press, 2001. {[}1ª ed.: 1955{]}.

\tit{parker}, H. “Sappho schoolmistress”. In: \textsc{greene}, E. (ed.).
\textit{Re"-reading Sappho: reception and transmission.} Berkeley: University of
California Press, 1996, pp.\,146--83.

\tit{pfeiffer}, R. \textit{A history of classical scholarship -- \textsc{i}.} Oxford:
Clarendon Press, 1998. {[}1ª ed.: 1968{]}

\tit{ragusa}, G. \textit{Fragmentos de uma deusa: a representação de Afrodite
na lírica de Safo}. Campinas: Editora da Unicamp, 2005. (Apoio: Fapesp).

\titidem. \textit{Lira, mito e erotismo: Afrodite na poesia mélica grega
arcaica}. Campinas: Editora da Unicamp, 2010. (Apoio: Fapesp).

\tit{robb}, K. \textit{Literacy and paideia in ancient Greece.} New York,
Oxford: Oxford University Press, 1994.

\tit{rösler}, W. “Persona reale o persona poetica?”. \textit{QUCC} 19, 1985,
pp.\,131--44.

\tit{schimitt"-pantel}, P. “Sacrificial meal and symposion: two models of
civic institutions in the archaic city?” In: \textsc{murray}, O. (ed.).
\textit{Sympotica. A symposium on the symposion.} Oxford: Clarendon Press,
1990, pp.\,14--33.

\tit{shapiro}, H. A. “Introduction”. In: \line(1,0){25}.
(ed.). \textit{The Cambridge Companion to archaic Greece.} Cambridge: Cambridge
University Press, 2007, pp.\,1--9.

\tit{skinner}, M. B. “Woman and language in archaic Greece, or, Why is Sappho
a woman?”. In: \textsc{greene}, E. (ed.). \textit{Reading Sappho: contemporary
approaches.} Berkeley: University of California Press, 1996, pp.\,175--92.

\tit{slings}, S. R. “The \textit{I} in personal archaic lyric: an
introduction”. In: \line(1,0){25}. (ed.). \textit{The poet’s I in archaic Greek
lyric.} Amsterdam: VU University Press, 1990, pp.\,1--30.

\tit{snell}, B (ed.). \textit{A cultura grega e as origens do pensamento
europeu.} Trad. P. de Carvalho. São Paulo: Perspectiva, 2001. {[}1ª ed. orig.:
1955{]}.

\tit{snyder}, J. M. \textit{The woman and the lyre: women writers in
classical Greece and Rome}. Carbondale: Southern Illinois University Press,
1989.

\tit{stehle}, E. “Romantic sensuality, poetic sense”. In: \textsc{greene}, E.
(ed.). \textit{Reading Sappho: contemporary approaches.} Berkeley: University
of California Press, 1996, pp.\,143--9.

\tit{svenbro}, J. \textit{Phrasiklea. An anthropology of reading in ancient
Greece.} Trad. J. Lloyd. Ithaca: Cornell University Press, 1993.

\tit{vetta}, M. “Poesia simposiale nella Grecia arcaica e classica”. In:
\line(1,0){25}. (ed.). \textit{Poesia e simposio nella Grecia arcaica.}
Bari: Laterza, 1995, pp.\,xi--lx.

\tit{west}, M. L. “Greek poetry 2000--700 \textsc{b.c.”.} \textit{CQ} 23, 1973,
pp.\,179--92.

\titidem. (ed.). \textit{Iambi et elegi Graeci.} Oxford: Oxford University
Press, 1998. vols. 1--2. {[}1ª ed.: 1971{]}.

\tit{willis}, W. H. “A census of the literary papyri from Egypt”.
\textit{GRBS} 9, 1968, pp.\,205--41.


\pagebreak
\section*{adendo bibliográfico à 2ª edição}

\tit{Bartol}, K. “Saffo e Dika (Sapph. 81 V.)”. \textit{QUCC} 56, 1997, pp.\,75--80.

\tit{Bierl}, A. “Alcman at the end of Aristophanes' \textit{Lysistrata}”. In: \textsc{athanassaki}, L.; \textsc{bowie}, E. (eds.). \textit{Archaic and classical song}. Berlin: De Gruyter, 2011, pp.\,415--436.

\titidem.  “\textit{Symmachos esso}: theatrical role"-playing and mimesis in Sappho fr.\,1\textsc{v}.”. in: \textsc{biggliazzi}, S. (eds.). \textit{Συναγωνίζεσται. Essays in honour of Guido Avezzù}. Verona: Skène, 2018, pp.\,925--951.


\tit{bowman}, L. “The ‘women’s tradition’ in Greek poetry”. \textit{Phoenix} 58, 2004, pp.\,1--27.

\tit{brasete}, M. F. “O amor na poesia de Safo”. In: \textsc{ferreira}, A. M. (ed.). \textit{Percursos de Eros -- representação do erotismo}. Aveiro: Universidade de Aveiro, 2003, pp.\,17--26.

\titidem. “Homoerotismo feminino na lírica grega arcaica: a poesia de Safo”. In: \textsc{fialho}, M. do Céu et alii (eds.). \textit{A sexualidade no mundo antigo}. Lisboa, Coimbra: Centro de História da Universidade de Lisboa/Centro de Estudos Clássicos e Humanísticos U. Coimbra, 2009, pp.\,289--303. 

\tit{brusse}, J. S. “Epigram”. In: \textsc{clauss}, J. J.; \textsc{cuypers}, M. (eds.). \textit{A companion to Hellenistic literature}. Malden: Wiley"-Blackwell, 2010, pp.\,117--35. 

\tit{buzzi}, S. et alii (eds.). \textit{Nuove acquisizioni di Saffo e della lirica greca}. Alessandria: Edizioni dell'Orso, 2008.

\tit{Caciagli}, S. “Sapph. fr.\,27\textsc{v}.: l’unità del pubblico saffico”. \textit{QUCC} 91, 2009, pp.\,63--80.

\tit{cazzato}, V.; \textsc{lardinois}, A. (eds.) \textit{The look of lyric: Greek song and the visual. Studies in archaic and classical Greek song, vol.\,1}. Leiden: Brill, 2016.

\tit{Clark}, C. A. “The gendering of the body in Alcman’s \textit{Partheneion} 1: narrative, sex and social order in archaic Sparta”. \textit{Helios} 23, 1996, pp.\,143--172.

\tit{ferrari}, F. “Il pubblico di Saffo”. \textit{SIFC} 1, 2003, pp.\,42--89.

\titidem. (trad., notas, introd.). \textit{Saffo. Poesie}. Milano: \textsc{bur} Rizzoli, 2011.

\tit{greene}, E.; \textsc{skinner}, M. B. (eds.). \textit{The new Sappho on old age. Textual and philosophical issues}. Washington, \textsc{d.c.}: Center for Hellenic Studies, 2009.

\tit{Hague}, R. H. “Ancient Greek wedding songs: the tradition of praise”. \textit{Journal of Folklore Research} 20, 1983, pp.\,131--43.

\tit{klinck}, A. \textit{Woman’s song in ancient Greece}. Montreal: McGill-Queen’s University Press, 2008. 

\tit{lardinois}, A. “Subject and circumstance in Sappho’s poetry”. \textit{TAPhA} 124, 1994, pp.\,57--84.

\titidem. “Who sang Sappho’s songs?”. In: \textsc{greene}, E. (ed.). \textit{Reading Sappho}. Berkeley: University of California Press, 1996, pp.\,150--72.

\titidem. ``Lesbian Sappho revisited''. In: \textsc{dijkstra}, J. et alii (eds.). \textit{Myths, Martyrs, and Modernity. Studies in the History of Religions in Honour of Jan N. Bremmer}. Leiden: Brill, 2010, pp.\,13--30.

\titidem. “The \textit{parrhesia} of young female choruses in ancient Greece”. In: \textsc{athanassaki}, L.; \textsc{bowie}, E. (eds.). \textit{Archaic and classical song: performance, politics and dissemination}. Berlin: De Gruyter, 2011, pp.\,161--72. 

\tit{Lefkowitz}, M. R. “The last hours of the \textit{parthenos}”. In: \textsc{reeder}, E. D. (ed.). \textit{Pandora’s box. Women in classical Greece}. Baltimore, Princeton: The Walters Art Gallery / University Press, 1995, pp.\,32--39. 

\tit{Paton}, W, R. \textit{The Greek anthology -- \textsc{iii}: book \textsc{ix}}. London, Cambridge: William Heinemann, Harvard University Press, 1916--18. 5 vols.

\tit{ragusa}, G. (org., trad.). \textit{Lira grega: antologia de poesia arcaica}. São Paulo: Hedra, 2013.

\titidem. “Memória, a terra prometida dos poetas: o tema na mélica grega arcaica”. \textit{Forma Breve} 15, 2018, pp.\,143--52. 

\titidem. “A coralidade e o mundo das \textit{parthénoi} na poesia mélica de Safo”. \textit{Revista Aletria} 29.4, 2019a, pp.\,85--111. 

\titidem. “Safo de Lesbos: de liras e neblinas”. In: \textsc{rede}, M. (org.). \textit{Vidas Antigas. Ensaios Biográficos da Antiguidade}. São Paulo: Editora Intermeios, 2019b, pp.\,211--39.

\titidem; \textsc{rosenmeyer}, P. “A delicate bridegroom: \textit{habrosunē} in Sappho, Fr.\,115\textsc{v}”. \textit{CQ} 69, 2019, pp.\,62--75.

\titidem. "Nove Musas mortais: as poetas da Grécia antiga". \textit{Revista do Centro de Pesquisa e Formação} (\textsc{sesc}) 11, 2020, pp.\,113--136.

\tit{Ramos}, P. E. da S. (trad. e notas). \textit{Poesia grega e latina}. São Paulo: Cultrix, 1964.

\tit{Redfield}, J. M. “Notes on the Greek wedding”. \textit{Arethusa} 15, 1982, pp.\,181--201. 

\titidem; \textsc{delfito}, J. S. S. “Corina: uma voz feminina da poesia grega antiga e suas canções”. \textit{Translatio},  18, 2020, pp.\,3--16.

\tit{reeder}, E. D. (ed.). \textit{Pandora’s box. Women in classical Greece}. Baltimore, Princeton: The Walters Art Gallery/University Press, 1995.


\tit{Scodel}, R. “Self-correction, spontaneity, and orality in archaic poetry”. In: \textsc{worthington}, \textsc{i}. (ed.). \textit{Voice into text. Orality and literacy in ancient Greece}. Leiden: Brill, 1996, pp.\,59--79.

\tit{Sissa}, G. \textit{Greek virginity}. Trad. A. Goldhammer. Cambridge: Harvard University Press, 1990.

\tit{Stehle}, E. \textit{Performance and gender in ancient Greece: nondramatic poetry and its setting}. Princeton: University Press, 1997.

\tit{Swift}, L. A. \textit{The hidden chorus. Echoes of genre in tragic lyric}. Oxford: Oxford University Press, 2010.

\tit{Thomas}, B. M. “The rhetoric of prayer in Sappho's `Hymn to Aphrodite'”. \textit{Helios} 26, 1999, pp.\,3--10.

\tit{West}, M. L. (ed. e coment.). \textit{Hesiod, Theogony}. Oxford: Clarendon Press, 1988.


\end{bibliohedra}

