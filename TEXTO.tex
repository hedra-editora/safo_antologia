\SVN $Id: TEXTO.tex 11453 2012-10-22 18:02:24Z isabel $

\pagestyle{plain}



\chapter{Afrodite}

\textsc{Nenhuma} outra divindade grega aparece nas canções de Safo com a mesma
frequência, nem do mesmo modo: Afrodite é a mais presente\footnote{ Afrodite é
personagem dos fragmentos 1, 2, 5, 15, 22, 33, 73a, 86, 96, 102, 112, 133, 134
e 140. Neles, o tema da presença da deusa \textit{corpus} foi estudado em Ragusa (2005), em que se baseiam as traduções de e as notas aos fragmentos legíveis desse corpus}. O fato
não se explica facilmente, mas três das linhas de força da mélica sáfica são a
paixão erótica, a beleza e o universo feminino. Ora, Afrodite, em Safo e nos
demais poetas gregos, para não falar da iconografia e dos cultos, é
multifacetada --- como são em geral os deuses gregos ---, mas é, fundamentalmente,
deidade da beleza física, da feminilidade, da sensualidade, da sedução, da
paixão erótica, do desejo --- características que constituem seus poderes
principais e sua esfera central de atuação, a do erotismo. Há, portanto,
estreita afinidade entre o fazer poético de Safo e a imagem de Afrodite, que se
traduz em notável e inigualável cumplicidade entre a deusa dileta e a voz
poética dos versos.\footnote{ A diferença é já evidente se comparamos a
representação da deusa em Safo à encontrada nos demais poetas arcaicos que, como ela, praticaram a poesia mélica arcaica; para estes e o estudo de Afrodite em seus fragmentos, ver Ragusa (2010).}

\paragraph{Fragmento 1 ou “Hino a Afrodite”}

{\small Esse \textit{hino clético} --- prece que invoca a deidade para instá-la a vir à
presença de quem suplica --- estrutura-se em três etapas convencionais
fundamentais, mostra a tradição: identificação do deus (versos 1--2), essencial
num sistema politeísta; recordação de relação previamente firmada com a
deidade, de modo a nela suscitar o sentido de obrigação para com quem apela no
presente (versos 5--24); explicitação do(s) pedido(s). Essa cuidadosa elaboração
formal explica-se por constituir a própria prece presente à divindade que, com
tal agrado, pode se tornar propícia. No hino, a suplicante, que se autonomeia
“Safo”,\footnote{ Skinner (1996, p. 183) observa, a propósito da autonomeação,
que, “dada a função normal do poeta grego arcaico como falante-representante
indicado por sua --- dele(a) --- comunidade, é bem mais provável que a
autoestilização de Safo enquanto o sujeito (\textit{ego}) que deseja, junto com
seu extenso estoque de temas, formas de versos e melodias, tropos e imagens,
fosse largamente tradicional, um produto de muitas gerações de um esforço
criativo local. [\ldots{}] Abordando as canções de Safo como discursos sociais,
evitamos os difíceis problemas de representação envolvidos no tratamento do
texto como um espelho fiel da subjetividade única de um autor”. Para a
referência completa, ver a bibliografia ao fim da introdução.} invoca a deusa
a vir à sua presença, para junto a ela lutar pela sedução da amada que ora a
rejeita, objetivo em torno do qual giram todos os pedidos (versos 1--5, 25--8).
Atente-se para a fala de Afrodite (versos 21--4), dita no passado, mas
revalidada no presente e a cada nova rodada da vinda da intermitente paixão
erótica; tal fala guarda um valor universal de caráter punitivo-consolatório: o
consolo do amador rejeitado pelo amado está na reversão de papéis que a
experiência erótica em tempo produz. Também vale a pena reparar na visão de
\textit{éros} como patologia de corpo e mente (versos 3--4), que, em princípio,
torna sua vítima impotente; e no modo como a sedução é pensada como uma batalha
e uma caçada, arenas às quais são comuns o ataque violento, a perseguição e a
dominação do outro --- o inimigo, a presa, o objeto de desejo do sedutor. Tudo
isso está muito presente na linguagem erótica da poesia grega antiga, mas se
maximiza no Fr.~1 de Safo, em que a suplicante chama Afrodite a ser sua
“aliada de lutas” (verso 28). Finalmente, nos versos 1--2, repare-se
que os epítetos estabelecem Afrodite não só como poderosa, mas também bela
sedutora ardilosa. A sedução e a deidade caminham, pois, no âmbito do fugidio,
do oblíquo, da dissimulação, da trama. A arte do engano é imprescindível na
esfera da sedução regida por Afrodite; e nessa arte, ninguém superará a deusa,
preciosa aliada.

Não posso deixar de ressaltar que o “Hino a Afrodite” é não só
a mais famosa canção de Safo, mas a única integralmente preservada no tratado
\textit{Sobre o arranjo das palavras}, de Dionísio de Halicarnasso (retórico,
século \textsc{i} a.C.). Mais: é a primeira canção do livro \textsc{i} de Safo,
compilado na célebre Biblioteca de Alexandria, provavelmente na virada dos
séculos \textsc{iii}--\textsc{ii} a.C.}

\begin{verse}
De flóreo manto furta-cor, ó imortal Afrodite,\\
filha de Zeus, tecelã de ardis, suplico-te:\\
não me domes com angústias e náuseas,\\*
veneranda, o coração,

mas para cá vem, se já outrora ---\\
a minha voz ouvindo de longe --- me\\
atendeste, e de teu pai deixando a casa\\*
áurea a carruagem

atrelando vieste. E belos te conduziram\\
velozes pardais em torno da terra negra ---\\
rápidas asas turbilhonando, céu abaixo e\\*
pelo meio do éter.

De pronto chegaram. E tu, ó venturosa,\\
sorrindo em tua imortal face,\\
indagaste por que de novo sofro e por que\\*
de novo te invoco,

e o que mais quero que me aconteça em meu\\
desvairado coração. “Quem de novo devo persuadir\\
(?) ao teu amor? Quem, ó\\*
Safo, te maltrata?

Pois se ela foge, logo perseguirá;\\
e se presentes não aceita, em troca os dará;\\
e se não ama, logo amará,\\*
mesmo que não queira”.

Vem até mim também agora, e liberta-me dos\\
duros pesares, e tudo o que cumprir meu\\
coração deseja, cumpre; e, tu mesma,\\*
sê minha aliada de lutas.
\end{verse}

\paragraph{Fragmento 2 ou “Ode do óstraco”}

{\small Eis outro \textit{hino clético}, em que se destaca o detalhamento do local ao
qual Afrodite é convidada a vir, saindo de Creta. O espaço, porém, não é
definido cartograficamente, mas se desenha como cenário primaveril idealizado
em chave sacroerótica, inerente à visão grega da natureza, suspenso em
temporalidade própria, impregnado de Afrodite, de cujas imagens poéticas e
mítico-religiosas se desprendem seus elementos constitutivos. Desse cenário
emana uma atmosfera carregada de sensualidade e do divino, algo ampliado pela
antecipação da epifania da deusa (verso 13) invocada como ``Cípris'' ---
nome mais frequente na literatura grega antiga, além de “Afrodite” ---, que evoca
seus elos com um de seus locais de culto mais importantes, a ilha de Chipre,
onde são particularmente fortes suas ligações com o mundo vegetal. Atente-se
para o caráter ativo da epifania, que reforça a ideia da fusão num fragmento de
linguagem intensamente sinestésica. Claramente, há o desejo de proximidade
entre a voz poética e a deusa. A fonte principal do texto é um óstraco ou caco
de cerâmica --- material abundante na Grécia antiga e muito utilizado para a
escrita ---, datado do século~\textsc{iii} a.C., o que fazia desta a fonte de transmissão, 
até a descoberta do novo fragmento (último desta antologia), a mais antiga da obra de Safo, e a
única anterior à época das edições alexandrinas.}

\begin{verse}
\ldots{} Para cá, até mim, de Creta \ldots{} templo\\
sagrado onde \ldots{} e agradável bosque\\
de macieiras, e altares nele são esfumeados\\*
com incenso.

E nele água fria murmura por entre ramos\\
de macieiras, e pelas rosas todo o lugar\\
está sombreado, e das trêmulas folhas\\*
torpor divino desce.

E nele o prado pasto de cavalos viceja\\
\ldots{} com flores, e os ventos\\*
docemente sopram \ldots{}

Aqui tu \ldots{} tomando, ó Cípris,\\
nos áureos cálices, delicadamente,\\
néctar, misturado às festividades,\\*
vinho-vertendo \ldots{}
\end{verse}

\section{Fragmentos 5 e 15 (Dórica e Cáraxo)}

\paragraph{Fr.~5 --- Prece a Afrodite e às Nereidas}

{\small No fragmento preservado no \textit{Papiro de Oxirrinco 7} (século \textsc{iii} d.C.),
temos uma canção-prece a Afrodite --- chamada pelo nome que nos remete à sua
geografia mítico-poética e religiosa insular, “Cípris” --- e às Nereidas
--- netas de Oceano e filhas do velho do mar, Nereu. Quem apela às deidades o faz
em benefício de uma 3ª pessoa do singular, a quem se refere como “meu
irmão” (v.2). Uma vez que o “eu” da poesia de Safo é habitualmente
identificado à própria poeta, é leitura corrente que o “ele” é Cáraxo, com base
em Heródoto (\textit{Histórias}, \textsc{ii}, 134--135), segundo quem uma
cortesã de nome Rodopis, feita escrava, foi levada para o Egito do faraó Amásis
(\textit{c.} 570 a.C.) por um homem que, depois, a libertou em troca de vultoso
pagamento efetuado por “Cáraxo de Mitilene, filho de Escamandrônimo e
irmão de Safo, a poeta”; e Heródoto diz: Cáraxo, em seguida, “retornou
a Mitilene, e numa canção Safo muito o atacou, de maneira severa”. Há nesse
relato, porém, sérios problemas de cronologia, difíceis de ajustar de modo a
acomodar todas as personagens; ademais, nada prova que Heródoto se refira ao
nosso Fr.~5; e tampouco se justifica plenamente a identificação automática de
Safo ao “eu” da canção. Na prece, identificadas as deidades, seguem-se os
pedidos. Primeiro, de proteção ao navegante que retorna --- função própria da
atuação de Afrodite e das Nereidas, indicam seus cultos e imagens
mítico-poéticas e iconográficas. Mas nos pedidos seguintes, nada há que os
ligue de maneira especial às deidades invocadas; tais pedidos falam no
cumprimento de todos os desejos do “irmão” --- algo que recorda os
versos 17--18 e 26--27 do Fr.~1 ---, na reparação de seus erros passados, na
atitude para com amigos e inimigos --- “alegria” aos primeiros,
desgraça, aos segundos, de acordo com a ética grega. Falam ainda em algo
relativo à sua ``irmã'', que não sabemos se é outra personagem ou
referência em 3ª pessoa ao próprio “eu” da canção.}

\begin{verse}
Ó Cípris e Nereidas, ileso, a mim,\\
o meu irmão concedei aqui chegar,\\
e o que no coração ele queira que seja ---\\*
tudo cumpri;

e que seus passados erros todos ele repare\\
e que aos amigos uma alegria ele seja,\\*
\ldots{} aos inimigos, e que não nos seja \ldots{}

(\ldots{}) e a irmã --- que ele a queira fazer \ldots{}
\end{verse}

\paragraph{Fr.~15 --- Prece a Afrodite (uma punição para Dórica)}

{\small A única fonte do Fr.15 é o \textit{Papiro de Oxirrinco} 1231 (século \textsc{ii} d.C.).
Nele, acredita-se que o fragmento se ligaria à mesma narrativa que enreda o Fr.5, mas essa
leitura carece de qualquer substância historicamente comprovada, e sofre de
questionável biografismo ficcionalizante no estudo do texto poético. Aqui,
parece ser feita uma prece a Afrodite (“Cípris”): o “eu” pede-lhe a
punição de Dórica. Segundo compreensões usuais, Safo estaria pedindo a punição
de Dórica/Rodopis, em benefício de Cáraxo e de si mesma. Como no caso do Fr.~5,
todavia, além do biografismo, há os dados contraditórios dos testemunhos
antigos, em geral, de caráter francamente anedótico. Note-se, por fim, que se
revela implicitamente no segundo pedido um traço próprio de Afrodite e dos
deuses gregos como um todo, na sua relação com os mortais: gabar-se,
vangloriar-se, é atitude potencialmente perigosa para um mortal que,
deixando-se levar pela arrogância e prepotência, pode bem facilmente esquecer
os limites de sua condição e, por isso, ser punido.}

\begin{verse}
Ó Cípris, e a mais amarga te descubra\\
e não se vanglorie isto contando --- ela,\\
Dórica: como a segunda vez \ldots{}\\*
\ldots{} veio.
\end{verse}


\paragraph{Fragmento 22}

{\small Nessa canção fragmentária, também preservada no \textit{Papiro de Oxirrinco}
1231, dignas de notas são a referência à ``harpa'', instrumento muito
antigo e de procedência oriental, e ao voo do desejo, na imagem poética
recorrente da excitação erótica, ao redor de um “tu” --- voo este que se repete
no presente da canção, como antes, no passado, revela o advérbio ``de
novo”, a marcar, como no Fr.~1, a intermitência de \textit{éros}. A
sensualidade da cena se intensifica com o vestido captado pelos olhos do “eu”,
motivo de sua alegria e gatilho de uma recordação em que se insere Afrodite, a
``Ciprogênia'' ou “nascida em Chipre”. São ligações tradicionais na
poesia erótica de Safo e de outros poetas os binômios beleza/vestuário--desejo e
desejo--olhar.}

\begin{verse}
\ldots{} harpa, enquanto de novo o desejo \ldots{}\\
voa ao redor de ti ---\\
a bela ---; pois o vestido \ldots{}\\
vendo tremeste, e eu me alegro,\\
pois, certa vez, a própria \ldots{} \\*
Ciprogênia \ldots{}
\end{verse}

\paragraph{Fragmento 33}

{\small Estamos uma vez mais diante de uma canção-prece, em prol da obtenção de algo,
com o auxílio da deidade invocada, em imagem dourada. O fragmento tem por fonte
o tratado \textit{Sobre a sintaxe}, do gramático grego Apolônio Díscolo (século
\textsc{ii} d.C.).}

\begin{verse}
\ldots{}se ao menos eu, ó auricoroada Afrodite,\\
este lote \ldots{} obtivesse por parte \ldots{}\footnote{ Nesta tradução, faço uma pequena mudança com relação 
à que se lê em Ragusa (2005).}
\end{verse}

\paragraph{Fragmento 102}

{\small O “eu” feminino desse fragmento, cuja fonte principal é o \textit{Inquérito
sobre os metros}, de Heféstion (século \textsc{ii} d.C.), dirige uma queixa à sua
“doce mãe”: domada pela paixão por um menino, por ação de Afrodite,
acha-se impotente, incapaz de prosseguir com o trabalho no tear. Na dupla de
versos, compõe-se um cenário erótico muito apropriado a Afrodite, tramado no
entrelaçamento da ação de domar, do desejo, da intervenção da deusa
--- a ``tecelã de ardis” no Fr.~1 ---, e do tecer de tramas, tarefa
feminina, que, metaforicamente, alude à ação de enganar, própria da sedução.
Vale reparar ainda em como o fragmento ecoa a canção popular grega, da qual
temos vestígios, tanto na referência ao trabalho, quanto no tema (a queixa da
menina doente de paixão). O “eu” do fragmento é, pois, dramático, como o é
marcadamente a “lírica” grega antiga. Cabe atentar para a figura da mãe: na
casa, espaço das mulheres, a ela cabia supervisionar os trabalhos e os deveres
domésticos da família e de seus servos. E para o contraste entre a violenta
ação de domar, executada por Afrodite, e sua imagem física delicada.}

\begin{verse}
Ó doce mãe, não posso mais tecer a trama --- \\*
domada pelo desejo de um menino, graças à esguia Afrodite \ldots{}
\end{verse}


\paragraph{Fragmento 112}

{\small Esse fragmento é de uma canção de casamento ou epitalâmio, como outros que
adiante traduzo, a ser entoado no decorrer dos eventos da cerimônia.
Fundamentalmente, canta-se a felicidade do ``noivo” e faz-se o elogio
dos noivos. Vale a pena recordar que a fama dos epitalâmios sáficos parece ter
sido ampla e duradoura; afinal, tardiamente, o sofista Corício de Gaza (século
\textsc{vi} d.C.) dizia, na obra \textit{Epitalâmios em Zacarias}: ``Portanto, eu
--- para que a ti novamente agrade --- com um canto sáfico adornarei a noiva \ldots{}”.
Ou seja, não apenas as palavras do texto de um epitalâmio adornam os noivos,
mas a própria canção, uma vez que sua função é louvar a aparência física e
sedutora dos noivos, de modo a estimulá-los ao enlace sexual, que concretiza a
boda. O epitalâmio integra, assim, outros recursos voltados à mesma finalidade,
como o banho ritual. A presença de Afrodite, deusa da beleza e do sexo,
justifica-se plenamente, portanto. As fontes do fragmento são as já indicadas
obras de Heféstion, referido no Fr.~102, e de Corício.}

\begin{verse}
Ó feliz noivo, tua boda, como pediste,\\
se cumpriu, e tens a virgem que pediste.\\
Tua forma é graciosa, e \ldots{}. olhos de\\
mel, e amor se derrama na desejável face\\*
\ldots{} honra-te em especial Afrodite \ldots{}
\end{verse}

\paragraph{Fragmento 133}

{\small O Fr.~133 é um canto coral dialogado, similarmente ao 140. No verso 1, uma voz ---
ou um coro --- faz uma constatação acerca de ``Andrômeda”; no seguinte,
lança uma pergunta a ``Safo”, \textit{persona }da poeta no fragmento de
uma canção de evidente caráter dramático. De acordo com leituras correntes, de
cunho biografizante, ``Andrômeda” seria a poeta-líder de um grupo de
meninas rival ao de Safo. Esse nome aparece também no Fr.~130, em contexto
erótico que envolve outra personagem feminina, “Átis”, referida em
outros fragmentos de Safo. Mas a fragilidade de nosso conhecimento sobre Lesbos
e sua sociedade precisa ser admitida: nada resta de Andrômeda, exceto seu nome
na mélica sáfica materialmente frágil. O fragmento é de difícil leitura:
fala-se em punição ou benefício? E o que encobre a lacuna do verso 2, em que
Afrodite é dita --- como o são os demais deuses gregos --- venturosa, afortunada,
feliz? Sua fonte de transmissão é o tratado do metricista Heféstion, referido
no Fr.~102.}

\begin{verse}
“Tem Andrômeda uma bela paga \ldots{}”

“Ó Safo, por que a multiafortunada Afrodite \ldots{}?”
\end{verse}

\paragraph{Fragmento 134}

{\small Nesse verso, o sonho, como percebido pelos antigos, surge como próprio à
comunicação entre deuses e homens. O fragmento foi preservado em Heféstion, já
referido no Fr.~102.}

\begin{verse}
Em sonho falei à Ciprogênia \ldots{}
\end{verse}

\paragraph{Fragmento 140: Afrodite e Adônis, paixão e morte}

{\small Segundo o viajante grego do século \textsc{ii} d.C., Pausânias, Safo “sobre
Adônis cantou” (\textit{Descrição da Grécia}, \textsc{ix}, 29, 8); prova isso o Fr.~140, 
centrado no contexto mítico da paixão de Afrodite por Adônis, e da morte
lutuosa deste. O belo jovem, associado estreitamente ao universo sacroerótico da
flora, das plantas, dos arômatas, dos jardins --- muito caro a Afrodite ---,
tornado alvo da paixão da deusa, acaba morto violentamente no auge de sua
virilidade; na versão mais retomada, matou-o um javali, fato que deve se ligar
à interdição do porco nos rituais a Afrodite, em muitos cultos gregos. Morto
Adônis, a deusa sofre. Esse mito se acha nas tradições poéticas, iconográficas
e religiosas, o que mostra a influência oriental em terras gregas, uma vez que
a figura de Adônis é fenícia, em termos de sua origem --- e Afrodite associa-se
estreitamente ao Oriente. Ora, repare-se que a Afrodite do Fr.~140 é chamada
``Citereia”, em referência ao seu culto proeminente na ilha de Citera,
de forte influência fenícia. Não é demais ressaltar que é também oriental o
mito da deusa do sexo arrebatada por um jovem deus precocemente morto. O
cenário mítico está plasmado no fragmento em registro fúnebre e dramático, em
estrutura que ecoará mais tarde na tragédia grega: no canto dialogado,
proferido por um coro seguramente feminino, a morte de Adônis é afirmada, e a isso
se segue uma indagação; a resposta vem em seguida, dita pela própria Afrodite.
Pode ter sido cultual a \textit{performance} da canção, se em Lesbos havia um
culto a Adônis, mas não necessariamente: o canto pode ser a representação
poética do ritual. Do ponto de vista das práticas cultuais, o festival das Adonias 
era celebrado no início do verão, por mulheres adultas. Essa festa se atesta 
desde o século \textsc{vi} a.C., em várias partes do
mundo grego, e ainda é feita no Egito do \textsc{iii} a.C., agora de modo público e
oficial. Durante as Adonias, as mulheres pranteavam Adônis, representando o
sofrimento de Afrodite, exposto no fragmento da canção sáfica. A mais antiga
menção ao mito, ao culto e ao rito fúnebre dá-se no Fr.~140, cuja fonte é
Heféstion, como também do 102.}

\begin{verse}
“Morre, Citereia, delicado Adônis. Que podemos fazer?”

“Golpeai, ó virgens, vossos seios, e lacerai vossas vestes \ldots{}”
\end{verse}

\oneside

\chapter{Eros}

\textsc{É forte} em Safo a temática erótica, mas não tem a mesma presença marcante que
Afrodite o deus Eros, que lhe será sempre subordinado, como filho ou servo, na
tradição mítico-poética, iconográfica e cultual da Grécia antiga, como se vê no
Fr.~159, abaixo. A relevância do universo feminino certamente favorece a
proeminência de Afrodite sobre Eros. Na imagem sáfica do deus, projeta-se a
concepção grega sobre o desejo, a paixão erótica: prazerosa e doce, mas
sobretudo violenta, acre, dolorosa, doença de corpo e mente, força dominadora e
intermitente.

\paragraph{Fragmento 47}

{\small A fonte do fragmento é uma \textit{Oração}, do retórico Máximo de Tiro (século
\textsc{ii} d.C.).}

\begin{verse}
\ldots{} Eros sacudiu meus\\*
sensos, qual vento montanha abaixo caindo sobre as árvores \ldots{}
\end{verse}

\paragraph{Fragmento 54}

{\small Segundo a fonte, Pólux (\textit{Onomástico}, século \textsc{ii} d.C.), Safo descreve Eros
quando canta:}

\begin{verse}
\ldots{} vindo do céu, envolto em purpúreo manto \ldots{}
\end{verse}

\paragraph{Fragmento 130}

{\small O primeiro par de versos sintetiza os elementos que marcam a imagem poética de
\textit{éros}, o deus ou a força. O segundo par traz duas figuras femininas
outras vezes mencionadas nas canções de Safo, como mostra esta antologia, entre
as quais se insere, em chave de conflito, a 1ª pessoa do singular, cuja
identidade nos escapa inteiramente. O referido conflito deve ser de caráter
erótico. O fragmento está preservado no tratado do metricista Heféstion, já
referido no Fr.~102.}

\begin{verse}
\ldots{} Eros de novo --- o solta-membros --- me agita,\\*
doce-amarga inelutável criatura \ldots{}.

\ldots{} ó Átis, mas a ti tornou-se odioso meu\\*
pensamento, e para Andrômeda alças voo \ldots{}
\end{verse}

\paragraph{Fragmento 159}

{\small Máximo de Tiro, fonte do fragmento 47 e deste, afirma que, numa canção da poeta,
Afrodite fala a Safo, estabelecendo a hierarquia corrente entre si e o deus que
menciona:}

\begin{verse}
“\ldots{} tu e meu servo, Eros \ldots{}”
\end{verse}


\chapter{Ártemis}

\paragraph{Fragmento 44a}

{\small Apenas neste fragmento, cuja fonte é o \textit{Papiro Fouad }239 (séculos \textsc{ii} ou
\textsc{iii} d.C.), surge Ártemis em seu desenho tradicional de eterna virgem caçadora,
irmã de Apolo, filha de Leto, que bosques percorre, cercada de animais. A
virgindade, desejada pela deusa, concedeu-a Zeus.}

\begin{verse}
\ldots{} a Febo auricomado, a quem (Leto?) gerou de Coiós \ldots{}\\
\ldots{} unida ao Cronida \ldots{}, deus de grande nome.\\
\ldots{} mas Ártemis jurou o grande juramento dos deuses:\\
\ldots{} para sempre serei uma virgem\\
\ldots{} sobre os cimos das montanhas\\
\ldots{} concede-me este favor\\
\ldots{} concedeu-o o pai dos deuses venturosos\\
\ldots{} deia flecha-cervos, a selvagem, os deuses\\
\ldots{} grande título\\*
(Eros) dela nunca se achega \ldots{}
\end{verse}


\chapter{As Cárites ou “Graças”}

\textsc{Quase sempre} nomeadas coletivamente, as deusas integram o séquito de Afrodite.
Esse elo reflete afinidades sobretudo no universo erótico, uma vez que as
Cárites, cujo nome é a forma do plural de \textit{kháris }--- conceito que vai de
“graça, charme, regozijo, prazer”, no sentido físico, a “favor dos deuses”, no
âmbito da reciprocidade das relações ---, favorecem a beleza e a sedução na
esfera do sexo, a festa e a alegria na da vida cotidiana, o vigor, o
crescimento e a renovação, nas esferas humana e vegetal. São as Cárites
filhas de Zeus e integrantes da ordem olímpica, indica Fr.~53, em que
recebem um epíteto dado por Safo também a Eos, a Aurora, no último fragmento desta antologia;
tal epíteto, ao evocar a rosa, flor predileta de Afrodite, joga tintas eróticas
sobre o desenho das deusas. E são as Cárites, segundo a \textit{Teogonia}
(versos 64--67), de Hesíodo, vizinhas das Musas no Olimpo --- imagem suscitada
pelo Fr.~128. Deste --- cuja fonte é Heféstion, como no caso do Fr.~102 --- e do		\EP[]
Fr.~53 --- cuja fonte é um comentário antigo ao \textit{Idílio }28, do poeta
Teócrito (séculos \textsc{iv}--\textsc{iii} a.C.) ---, temos os respectivos versos inaugurais de
\textit{hinos cléticos}.
\pagebreak

\paragraph{Fragmento 53} \

\begin{verse}
Para cá, ó Cárites de róseos braços, meninas de Zeus \ldots{}
\end{verse}

\paragraph{Fragmento 128} \ 

\begin{verse}
Para cá, vós, delicadas Cárites e Musas de belas comas, \ldots{}
\end{verse}


\chapter{Eos, a Aurora}

\paragraph{Fragmento 123}

{\small O destaque ao adorno dos pés enfatiza o erotismo da imagem; pés, braços,
cabelos, colo/seio e olhos são do corpo feminino as partes mais enfocadas na
linguagem erotizante da poesia grega. Isso se intensifica na menção ao ouro;
afinal, só uma deusa é dita simplesmente ``áurea” nos poemas homéricos
e na maior parte da poesia antiga: Afrodite. O fragmento tem por fonte o
tratado \textit{Sobre palavras similares, mas diferentes}, do gramático Amônio
(século \textsc{ii} d.C.).}

\begin{verse}
\ldots{} ultimamente Eos, de áurea sandália, \ldots{}
\end{verse}


\chapter{Cenas míticas}

\textsc{Há,} além da cena do Fr.~140, duas outras cenas míticas na mélica sáfica.

\paragraph{Fragmento 44 e a saga troiana: as bodas de Heitor e Andrômaca\footnoteInSection{ Ver
artigo Ragusa, “Heitor e Andrômaca, da festa de bodas à celebração fúnebre:
imagens épicas e líricas do casal na \textit{Ilíada} e em Safo (Fr.~44 Voigt)”.
\textit{Calíope} 15, 2006, 36--63
(http://www.letras.ufrj.br/pgclassicas/files/upload/caliope15.pdf).}}

{\small Esse fragmento, cuja fonte são os \textit{Papiros de Oxirrinco} 1232 e 2076
(primeiras metades dos séculos \textsc{iii} e \textsc{ii} d.C., respectivamente) leva-nos à saga
de Troia e a personagens que vemos principalmente na \textit{Ilíada}: o arauto
Ideu, o herói troiano Heitor, irmão de Páris, e sua esposa, Andrômaca, o rei
troiano Príamo, o deus Apolo, protetor dos troianos na guerra. Além disso,
mostra-nos o único exemplar de poesia narrativa não épica, ou “lírica”
epicizante, cujos versos ditos por um narrador distanciado dividem-se na
chegada do arauto e anúncio da vinda dos noivos Heitor e Andrômaca (versos
1--10), e, depois, no espalhar dessa mensagem cidade afora, iniciando-se com
isso a procissão festiva em honra dos noivos (versos 11--34). A linguagem, a
atmosfera e a métrica que constroem essas etapas estão permeadas pela tradição
épico-homérica; note-se, por exemplo, a inescapável expressão do verso 4,
``glória imperecível”, síntese do ideal heroico a alcançar, a
ser preservado no canto épico, recorda a \textit{Ilíada }(canto \textsc{ix}, verso 413),
pela boca de seu grande herói, Aquiles.No discurso do arauto, repare-se no
elogio dos noivos e no pequeno catálogo do valioso e belo dote da princesa da
Tebas asiática, Andrômaca, foco da celebração do casamento. A importância da
referência ao dote no fragmento sáfico das bodas de Heitor e Andrômaca é mais
bem compreendida quando se pensa o casamento na vida das jovens gregas, momento
crucial que lhes atribui novo \textit{status} social --- o de esposa --- e gera
muitas mudanças: a saída da casa paterna e às vezes também da pátria; a
inserção na casa do marido e/ou numa realidade geográfica e cultural diversa; a
transformação da virgem em mulher que tem vida sexual e que deverá garantir a
continuidade das linhagens e administrar o espaço doméstico. A questão do dote
relaciona-se intimamente à da legitimidade da união na Grécia antiga. Vale
lembrar que o dote da esposa se destina à prole do casal, que precisava ser
protegida no caso de perda do(s) pai(s) ou de separação. A procissão (verso
13) é etapa típica das cerimônias de casamento, normalmente muito elaboradas e
estendidas ao longo de vários dias, buscando propiciar, de todas as maneiras
possíveis, o enlace sexual que tornava legítima e consumada a união. Há, pois,
no fragmento, um caráter epitalâmico, mas sua estrutura formal não permite
qualificá-lo como exemplar do subgênero mélico dos epitalâmios. Por fim, não se
pode deixar de lembrar que o casal Heitor e Andrômaca protagoniza no canto \textsc{vi}
(versos 369--502) da \textit{Ilíada }uma das cenas mais comoventes de despedida
entre cônjuges afetuosamente próximos um do outro e do filho, o bebê Astiánax.
Essa cena empresta ao poema épico grande densidade trágica, pois, ao acenar
para a morte certa de Heitor, acena para a ruína de Troia e de seus elos mais
frágeis: mulheres e crianças, em geral mortas ou escravizadas pelos vitoriosos.
A carga dramática intensifica-se pela ironia trágica de que Heitor perecerá
pelas mãos de Aquiles, que já haviam matado o pai e os irmãos de Andrômaca. E
pela mistura de lágrimas e risos nos rostos do guerreiro e de sua esposa ---
guerreiro este que, mais humano do que qualquer outro de que me possa lembrar,
tira seu elmo em dado momento, para poder acalmar e tomar nos braços seu bebê,
assustado com o penacho dessa peça da armadura heroica momentaneamente despida.
Essa lembrança empresta ao fragmento, tão alegre e festivo, grande tristeza e
melancolia, e enlaça casamento e morte, duas etapas de transição na trajetória
humana.}

\begin{verse}
\ldots{} Veio o arauto \ldots{}\\
Ideu \ldots{}, veloz mensageiro:\\
“\ldots{} e do resto da Ásia \ldots{} glória imperecível.\\
Heitor e os companheiros a de vivos olhos trazem\\
de Tebas sacra e da Plácia de fontes perenes --- ela,\\
delicada Andrômaca ---, nas naus, sobre o salso\\
mar. E muitos braceletes áureos e vestes\\
de púrpura fragrantes, adornos furta-cor,\\
incontáveis cálices prateados e marfins”.\\
Assim ele falou; e rápido ergueu-se o pai querido;\\
e a nova, cruzando a ampla cidade, chegou aos amigos.\\
De pronto os troianos às carruagens de boas rodas\\
atrelaram as mulas, e nelas subiu toda a multidão \\
de mulheres e junto as virgens \ldots{} tornozelos\\
mas apartadas as filhas de Príamo\\
e cavalos os homens atrelaram aos carros\\
\ldots{} moços solteiros, e por um largo espaço \\
\ldots{} os condutores das carruagens \\
\ldots{} símeis aos deuses\\
\ldots{} sacro, em multidões\\
rumou \ldots{} em direção a Ílio\\
e a flauta de doce som \ldots{} se misturou\\
e o som das castanholas \ldots{} e então as virgens\\
cantaram uma canção sacra e chegou aos céus\\
eco divino \ldots{}\\
e em toda parte estava ao longo das ruas\\
crateras e cálices\\
mirra e cássia e incenso se misturavam,\\
e as mulheres soltavam alto brado, as mais velhas,\\
e todos os homens entoavam adorável e alto\\
peã invocando o Arqueiro hábil na lira,\\*
e hineavam Heitor e Andrômaca, aos deuses símeis.
\end{verse}


\paragraph{Fragmento 166: Leda e Zeus}

{\small Em certa tradição mítica, Zeus seduziu Leda, esposa de Tíndaro, disfarçando-se
de cisne; o ovo por ela descoberto traria os gêmeos Castor e Polideuces, os
Dióscuros --- o primeiro mortal, e o segundo, imortal ---, sendo a eles
permitido desfrutar de ambas as condições alternadamente. Na canção de Safo,
esse mito parece aludido, mas a cor do ovo contrasta drasticamente com a usual,
e nos leva ao universo erótico das paisagens de enlace sexual, das quais o
jacinto é flor integrante. Cabe lembrar que os Dióscuros, noutras tradições,
são simplesmente filhos de Tíndaro e Leda, ou, noutras ainda, heróis de dupla
ascendência. A fonte do fragmento é o \textit{Banquete dos sofistas}, de Ateneu
(séculos \textsc{ii}--\textsc{iii} d.C.).}

\begin{verse}
\ldots{} dizem que um dia Leda achou um ovo jacintino\\
na cor, coberto \ldots{}
\end{verse}


\chapter{Canções de recordação}

\section{Memória e desejo}

\paragraph{Fragmento 16: ``Ode a Anactória''}

{\small Ao lado dos Frs. 1 e 31, o 16 é seguramente um dos mais estudados da mélica
sáfica, sobretudo por conta da exemplificação mítica da afirmação feita no
\textit{priamel}, como se designa a estrofe inicial, em que a uma série de
negativas coloca-se uma afirmativa, do ponto de vista da voz poética, e da
alegada facilidade de sua compreensão (versos 5--6), que soa aos nossos ouvidos
como ironia. Isso porque não está claro se o mito recordado da fuga de Helena, esposa 
de Menelau (não nomeado), e Páris, o príncipe troiano não nomeado, é exemplo
negativo, o que se alinharia à imagem prevalecente na poesia grega antiga, após
os poemas homéricos --- nos quais, todavia, não há a condenação de Helena pelo
narrador ---, ou se positivo, o que tornaria singular a imagem do fragmento. A
ambiguidade existe e, na verdade, não se resolve. Merece atenção, ainda, a
oposição \textit{éros}{}-guerra, sobre a qual se elaboram os versos, e a
referência ao rico reino oriental da Lídia, na Ásia Menor, com que estava
ligada a aristocracia da arcaica Mitilene. E a possibilidade de que haja um
viés metalinguístico relativo às escolhas temáticas da poeta da mélica arcaica.
A fonte do fragmento, \textit{Papiro de Oxirrinco} 1231, é a mesma do Fr.~15.}

\begin{verse}
Uns, renque de cavalos, outros, de soldados,\\
outros, de naus, dizem ser sobre a terra negra\\
a coisa mais bela, mas eu (digo): o que quer\\*
que se ame.

Inteiramente fácil tornar compreensível a\\
todos isso, pois a que muito superou\\			\EP[]
em beleza os homens, Helena, o marido, \\
o mais nobre,

tendo deixado, foi para Troia navegando,\\
até mesmo da filha e dos queridos pais\\*
de todo esquecida, mas desencaminhou-a \ldots{}

agora traz-me Anactória à lembrança,\\*
a que está ausente, \ldots{}\medskip
%
Seu adorável caminhar quisera ver,\\
e o brilho luminoso de seu rosto,\\
a ver dos lídios as carruagens e a armada\\*
infantaria.
\end{verse}


\paragraph{Fragmento 96}

{\small O texto do fragmento, preservado no \textit{Papiro de Berlim} 9722 (século \textsc{vi}
d.C.), está bastante danificado, e apenas são legíveis os versos 4--17, que
abaixo traduzo. A cena parece envolver uma 3\textsuperscript{a} pessoa do
singular, feminina, que no passado se relacionou à 2\textsuperscript{a} do
singular em chave erótica, como indicam a referência ao prazer e à canção		
(versos 4--5), e à beleza e ao desejo (versos 6--17). No tempo presente, 
em que estão separadas essas duas personagens, ``ela” encontra-se na Lídia --- não
mais em Lesbos ---, onde sua beleza é proeminente, como provavelmente o fora na
ilha grega. No entanto, ``ela” sofre, saudosa de ``Átis” (v.
17) --- talvez o ``tu” da canção ---, cuja figura a recordação lhe traz, fazendo
com que se consuma seu peito em desassossego erótico. A alternância temporal do
passado ao presente introduz o longo símile central (versos 6--14), alavancado
na natureza, como é característico da linguagem erótica da poesia grega antiga.
Em Safo, essa linguagem é especialmente elaborada; e no fragmento, o principal
elemento natural enfocado para o canto da beleza feminina é a feminina lua
(verso 8) que, como Eos (a Aurora) nos poemas homéricos, é inesperadamente
chamada ``dedirrósea”. Ao erguer-se a lua, posto o sol, tinge-se a lua
de seus tons laranja-avermelhados ou rosados que esquentam seu branco prateado;
estamos, pois, no início do anoitecer. No verso 9, ganha relevo a luz da lua,
vibrante, já em plena noite; e no 12, o orvalho que vem ao findar-se a noite,
avançada já a madrugada. Não será esta a única vez em que Safo cantará a lua,
como mostram os Fr.~34 e 154, inseridos nesta antologia. Sua escolha é
eloquente: a lua é elemento feminino, mesmo na língua grega, e tem a função de
regular os ciclos biológicos e o ritmo das marés, interferindo na fertilidade;
aproxima-se, pois, da esfera da mulher, percebida na Antiguidade como ligada
estreitamente aos líquidos, à umidade e à natureza, já que lhe cabe a crucial
reprodução no ciclo biológico humano. Chamo a atenção, por fim, para o cenário
florido e pulsante do símile, integrado, inclusive, pelas diletas flores de
Afrodite, as rosas, marcadas na própria adjetivação da lua. O símile conecta-se
ao restante do fragmento, na medida em que ambos tratam da beleza feminina e
estão perpassados pelo erotismo, o que é frisado na estrofe dos versos 15--17.
Portanto, pode-se dizer que a voz do fragmento insere sua canção no universo de
Afrodite, como no Fr.~2. }

\begin{verse}
\ldots{} qual deusa manifesta,\\*
e (ela) muito se deleitava com tua canção.

Mas agora ela se sobressai entre Lídias\\
mulheres como, depois do sol\\*
posto, a dedirrósea lua

supera todas as estrelas; e sua luz se esparrama\\
por sobre o salso mar \\*
e igualmente sobre multifloridos campos.

E o orvalho é derramado em beleza, e brotam\\
as rosas e o macio \\*
cerefólio e o trevo-mel em flor.

E (ela) muito agitada de lá para cá a \\
recordar a gentil Átis com desejo;\\*
decerto frágil peito \ldots{} se consome.
\end{verse}


\section{Separação (e solidão)}

\paragraph{Fragmento 49}

{\small De novo, Átis é personagem de uma canção, em chave erótica. Um “eu” diz num
verso (o inaugural de uma canção hoje perdida) que a paixão sentida no passado
esgotou-se no presente. Isso, decerto, separou as duas personagens. Depois, num
outro verso, Plutarco (séculos \textsc{i}--\textsc{ii} d.C., \textit{Diálogos sobre o amor}),
fonte de ambos, conta que, “falando a uma menina demasiado nova para o
casamento, Safo lhe diz” palavras de clara reprovação, por conta de sua forma
física. }

\begin{verse}
Eu te amei, Átis, no passado, certa vez \ldots{}

\ldots{} me parecias uma criança mirrada e sem graça \ldots{}
\end{verse}


\paragraph{Fragmento 94}

{\small Preservado no mesmo rolo de papiro do Fr.~96, o 94 é fortemente dramático ao
reencenar a separação entre o “eu” --- ``Safo” --- e “ela”, revivida em
recordação detalhada, de tons eróticos inescapáveis e cada vez mais intensos na
gradação corpo, perfumes, adornos, leito, e com participação no sagrado, na
referência ao templo, ao fim ainda legível do fragmento.}

\begin{verse}
\ldots{} morta, honestamente, quero estar;\\*
ela me deixava chorando

muito, e isto me disse:\\*
“Ah!, coisas terríveis sofremos,\\*
Ó Safo, e, em verdade, contrariada te deixo”.

E a ela isto respondi:\\*
“Alegra-te, vai, e de mim\\*
te recorda, pois sabes quanto cuidamos de ti;

se não (sabes) \ldots{} --- mas quero te\\*
lembrar \ldots{}\\*
\ldots{} e coisas belas experimentamos;

pois com muitas guirlandas de violetas\\*
e de rosas \ldots{} juntas\\*
\ldots{} ao meu lado pusestes,			\EP[]

e muitas olentes grinaldas\\*
trançadas em volta do tenro colo, \\*
de flores \ldots{} feitas;

e \ldots{} com perfume\\*
de flores \ldots{}\\*
digno de rainha, te ungiste,

e sobre o leito macio\\*
tenra \ldots{}\\*
saciavas (teu) desejo \ldots{}

Não havia \ldots{} nem \\*
santuário, nem \ldots{}\\*
do qual estivéssemos ausentes,

nem bosque \ldots{}
\end{verse}


\paragraph{Fragmento 168b}

{\small Heféstion, referido já no comentário ao Fr.~102, é a fonte principal do 168B,
nos quais a natureza enquadra a imagem da solidão da voz poética, talvez devido
a uma separação erótica.}

\begin{verse}
Imergiu a lua\\
e também as Plêiades; é\\
meia-noite, caminha o tempo,\\*
e eu sozinha durmo\ldots{}
\end{verse}

\section{Imortalidade}

\paragraph{Fragmento 55}

{\small Preservado na \textit{Antologia }de Estobeu (cristão, século \textsc{v} d.C.), o
fragmento traz um dos muitos momentos em que os poetas refletem em seus versos
sobre o poetar e a própria poesia, que vai se configurar no imaginário grego
como caminho para a imortalidade do nome. A linguagem é de ataque, de
invectiva, e parece dirigida a um “tu” feminino que pensa ser (bom) poeta, mas
não o é. Por isso, sua dupla morte ao descer ao mundo dos mortos, reino de
Hades, tornando-se não mais alcançável sua figura aos olhos dos vivos, e
apagada sua existência da memória destes.}
\pagebreak
\begin{verse}
Morta jazerás, nem memória alguma futura\\
de ti haverá, nem desejo, pois não partilhas das rosas\\
de Piéria; mas invisível na casa de Hades\\*
vaguearás esvoaçada entre vagos corpos \ldots{}
\end{verse}


\chapter{Desejos}

\paragraph{Fragmento 95}

{\small Um nome feminino e um forte desejo de rendição à morte marcam esse fragmento do
\textit{Papiro de Berlim }9722, também fonte do Fr.~96. Ressalto a imagem do
rio do mundo dos mortos, o Aqueronte, cujas margens são usualmente pintadas
como densas de flores de lótus. O “eu” feminino parece dirigir-se a um deus,
talvez Hermes, condutor que guia os mortos ao Hades.}

\begin{verse}
\ldots{} Gongila \ldots{}\\*
\ldots{} decerto um sinal \ldots{}

\ldots{} disse: “Ó senhor \ldots{}\\*
pois pela deusa venturosa \ldots{}\\*
não me deleito em estar erguida \ldots{}

um desejo de morrer me toma, e\\*
com lótus orvalhadas as\\*
margens do Aqueronte \ldots{}
\end{verse}


\paragraph{Fragmento 121}

{\small Preservado em Estobeu, como o Fr.~55, o 121 trabalha com a ideia da
reciprocidade, importante ao imaginário grego; o “eu” feminino fala a um “tu”
masculino, reclamando a necessidade do gesto de gratidão e de equiparação de
idade em se tratando de alianças, talvez a do casamento.}

\pagebreak

\begin{verse}
\ldots{} mas grato me sendo,\\
toma o leito de uma (outra) mais nova, \\
pois não suportarei ser a mais velha numa parceria \ldots{}
\end{verse}


\paragraph{Fragmento 126}

{\small A fonte é o \textit{Etimológico genuíno }(século \textsc{ix}). O termo empregado para
“companheira” marca uma amizade e aliança muito estreita entre o “tu”
e “ela”. O modo como o desejo é expresso parece acrescentar a essa relação o
ingrediente erótico, que de modo algum lhe seria estranho.}

\begin{verse}
\ldots{} que adormeças no peito macio de tua companheira \ldots{} 
\end{verse}


\paragraph{Fragmento 138}

{\small O foco nos olhos é comum quando se canta a beleza; nesse canto, o “eu” chama o
“tu” de \textit{phílos }(``amigo”), termo que pode ou não ter conotação
erótica --- a primeira possibilidade sendo, creio, mais provável, inclusive
porque a fonte do fragmento, Ateneu, que preservou também o Fr.~166, afirma:
``Safo diz ao homem que é excessivamente admirado por sua forma e tido como belo''.}

\begin{verse}
\ldots{} posta-te (diante de mim?), se és meu amigo,\\
e estende afora a graça de teus olhos \ldots{}
\end{verse}


\chapter{Dores de amor}

\paragraph{Fragmento 31: \textit{Phaínetaí moi \ldots{}}}


{\small O Fr.~31 ou \textit{Phaínetaí moi} (“Parece-me \ldots{}”) tem por fonte
principal o famoso tratado \textit{Do sublime} (século \textsc{i} d.C.?, 'Longino'). Ao
explicar as maneiras de um texto alcançar a grandeza, seu autor de incerta identidade menciona os
“pensamentos elevados”;\footnote{ Cito as traduções do volume Hirata,
F. (trad., introdução, notas). \textit{Longino.} \textit{Do Sublime.} São
Paulo: Martins Fontes, 1996.} no quadro destes, cita o fragmento de Safo,
em que louva a “escolha dos motivos” e a “concentração dos
motivos escolhidos”. Por exemplo Safo: as afecções consecutivas ao delírio
amoroso, a cada vez, ela as apreende como elas se apresentam sucessivamente e
na sua própria verdade. Mas onde mostra ela sua força? Quando ela é capaz, a
uma vez, de escolher e de ligar o que há de mais agudo e de mais
intenso nessas afecções”. A citação é sucedida por palavras que equiparam Safo
a Homero, “o Poeta”: 

\begin{hedraquote}
Não admiras como, no mesmo momento, ela
procura a alma, o corpo, o ouvido, a língua, a visão, a pele, como se tudo isso
não lhe pertencesse e fugisse dela; e, sob efeitos opostos, ao mesmo tempo ela
tem frio e calor, ela delira e raciocina (e ela está, de fato, seja
aterrorizada, seja quase morta)? Se bem que não é uma paixão que se mostra
nela, mas um concurso de paixões! Todo esse gênero de acontecimentos cerca os
amantes, mas, como eu disse, a maneira de agrupá-los, para relacioná-los num
mesmo lugar, realiza a obra de arte. Da mesma maneira, a meu ver, para as
tempestades o Poeta escolhe as mais terríveis das consequências.
\end{hedraquote}

A cena retratada eroticamente produz uma triangulação na qual o olhar do “eu”
feminino contempla primeiramente um homem que se porta como audiência de um
“tu” feminino, para depois concentrar-se exatamente na contemplação dessa
personagem, que provoca sua excitação --- na imagem recorrente do peito que se agita ou,
literalmente, voa, como se vê no Fr.~22 --- e crescente dominação erótica, ou
seja, fragmentação do corpo e da mente que leva à morte. Há um notável eco
metalinguístico no fragmento, em que a perda da voz, central na patologia
erótica, é crucial para poetas de tradição oral como Safo, que não existe sem
ela, e para a representação dramática da resposta à cena, que só se realiza a
partir justamente dessa perda. Temos o início do fragmento, mas não estamos
seguros de que tenhamos seu fim. Seu texto foi reelaborado por Catulo (século \textsc{i}
a.C.) no \textit{Poema 51},\footnote{ Cito-o, em tradução de Oliva (1996):
\textit{Ele parece-me ser par de um deus,/ Ele, se é fás dizer, supera os
deuses,/ Esse que todo atento o tempo todo/ Contempla e ouve-te/ doce rir, o
que pobre de mim todo/ sentido rouba-me, pois uma vez/ que te vi, Lésbia,
nada em mim sobrou/ De voz na boca/ Mas torpece-me a língua e leve os membros
/ Uma chama percorre e de seu som/ Os ouvidos tintinam, gêmea noite/ cega-me
os olhos./ O ócio, Catulo, te faz tanto mal./ No ócio tu exultas, tu vibras
demais./ Ócio já reis e já ricas cidades/ Antes perdeu.}} praticamente uma
tradução da canção sáfica.}

\begin{verse}
Parece-me ser par dos deuses ele,\\
o homem, que oposto a ti\\
senta e de perto tua doce\\*
fala escuta,

e tua risada atraente. Isso, certo,\\
no peito atordoa meu coração;\\
pois quando te vejo por um instante, então\\*
falar não posso mais,

mas se quebra minha língua, e ligeiro\\
fogo de pronto corre sob minha pele,\\
e nada veem meus olhos, e\\*
zumbem meus ouvidos,

e água escorre de mim, e um tremor\\
de todo me toma, e mais verde que a relva\\
estou, e bem perto de estar morta\\*
pareço eu mesma.
\end{verse}

\paragraph{Fragmento 36}

{\small Nesse parco fragmento, combina-se um binômio recorrente na poesia erótica grega:
paixão e loucura. Vale anotar: a loucura erótica, provocada por Afrodite, será
tratada como um dos tipos de loucura definidos por Platão (séculos \textsc{v}--\textsc{iv} a.C.),
no diálogo \textit{Fedro}, por exemplo.}

\begin{verse}
\ldots{} e desejo e enlouqueço \ldots{}
\end{verse}

%\vfill\pagebreak

\paragraph{Fragmento 48}

{\small Preservado na \textit{Carta a Jâmblico}, nome do filósofo neoplatonista do
século \textsc{iv} d.C., de autoria do imperador romano Juliano, seu contemporâneo, o
breve fragmento elabora-se a partir de outro motivo comum na concepção de
\textit{éros}: o de que paixão é fogo; logo, sua satisfação é como o apagar de
um incêndio, como diz o “eu” feminino ao “tu”, abaixo.}

\begin{verse}
\ldots{} vieste, e eu ansiava por ti --- \\*
me esfriaste o peito que queimava com desejo \ldots{}
\end{verse}
\pagebreak

\paragraph{Fragmento 51}

{\small Como no Fr.~36, temos aqui o binômio paixão-loucura, que fratura o amador. A
fonte do fragmento é o tratado \textit{Das negativas}, de Crísipo (filósofo
estoico, século \textsc{iii} a.C.).}   \EP[]

\begin{verse}
\ldots{} não sei o que faço: duas são as minhas mentes \ldots{}
\end{verse}

\chapter{Imagens da natureza}

\textsc{Como observei} ao comentar o Fr.~96 de Safo, é notável em sua mélica o trabalho
da linguagem com imagens da natureza, de grande presença e força não raro
metafóricas. Os fragmentos aqui agrupados complementam a ideia que disso já nos
dá aquele, anteriormente visto.

\paragraph{Fragmento 34}

{\small A principal fonte desse fragmento concentrado no corpo celeste feminino da lua é
um comentário de Eustácio (bispo de Tessalônica, século \textsc{xii} d.C.) à
\textit{Ilíada }(\textsc{viii}, 555).}

\begin{verse}
\ldots{} e as estrelas, ao redor da bela lua,\\
de novo ocultam sua luzidia forma,\\
quando plena ao máximo ela ilumina\\
a terra \ldots{} \\*
\ldots{} argêntea
\end{verse}

\paragraph{Fragmento 42}

{\small Segundo a fonte do fragmento, um escólio (comentário antigo) a Píndaro
(\textit{Ode pítica} 1), poeta mélico da virada dos séculos \textsc{vi}--\textsc{v} a.C.,
``Safo diz dos pombos”:}

\begin{verse}
\ldots{} e deles se tornou frio o peito,\\*
e suas asas se afrouxaram \ldots{}
\end{verse}

\paragraph{Fragmento 101a}

{\small A fonte do fragmento, o tratado \textit{Sobre o estilo}, de Demétrio (século \textsc{iii}
a.C. ou \textsc{i} d.C.?), afirma que esses versos tratam da cigarra:}

\begin{verse}
\ldots{} e de sob suas asas\\*
sem cessar derrama clara canção,\\*
quando o flamejante verão \ldots{}
\end{verse}

\paragraph{Fragmento 136}

{\small Segundo a fonte do fragmento, um escólio a Sófocles (século \textsc{v} a.C.) e à
\textit{Electra }(verso 149), Safo abaixo já falava como na tragédia dos
rouxinóis:}

\begin{verse}
\ldots{} mensageiro da primavera, o rouxinol de desejável voz \ldots{}
\end{verse}

\paragraph{Fragmento 143}

{\small Ateneu, fonte do Fr.~166 já aqui visto, atribui a Safo a descrição abaixo:}

\begin{verse}
\ldots{} e áureos grãos-de-bico cresciam nas margens (do rio?) \ldots{}
\end{verse}

\paragraph{Fragmento 146}

{\small O gramático Trifo, da era romana de Augusto (séculos \textsc{i} a.C.--\textsc{i} d.C.), 
cita como proverbiais estas palavras de Safo, no tratado \textit{Figuras de expressão}:}

\begin{verse}
\ldots{} para mim, nem o mel, nem a abelha \ldots{}
\end{verse}

\paragraph{Fragmento 168c}

{\small A fonte do fragmento é, como no caso do 101A, Demétrio, que trata do charme na
expressão:}

\begin{verse}
\ldots{} a terra adornou de muitas guirlandas \ldots{}
\end{verse}


\chapter[O cantar, as canções e as companheiras]{O cantar, as canções\break e as companheiras}

\textsc{É importante} na mélica sáfica a metalinguagem --- o poeta a falar de seu
poetar, a criticar o poetar de outros, como no Fr.~55, a tratar de sua
audiência, a tratar de seus temas. Nesse item, insiro alguns fragmentos que
ilustram isso --- aqueles que estão mais legíveis no \textit{corpus} de Safo.

\section{O dom das Musas e as amigas}

\paragraph{Fragmento 32}

{\small A fonte do fragmento é o tratado \textit{Sobre os pronomes}, de Apolônio
Díscolo, gramático que, noutro tratado, preservou também o já visto Fr.~33. O
gênero feminino do “eu” e a menção de seus \textit{érga }(trabalhos) como
``dons” que a tornam honrada levam a pensar que o sujeito feminino
coletivo deve ser divino --- deve ser o conjunto das Musas. Logo, tais dons
seriam o da poesia, no entendimento mais comum da expressão.}

\begin{verse}
\ldots{} elas (as Musas?) me fizeram honrada com os dons\\
de seus trabalhos \ldots{}
\end{verse}

\paragraph{Fragmento 118}

{\small Preservado no \textit{Sobre os tipos de estilo}, de Hermógenes (século \textsc{ii} d.C.),
o fragmento, segundo essa fonte, traz-nos o momento em que ``Safo
questiona a sua lira, e a lira assim lhe responde”:}

\begin{verse}
\ldots{} vem, divina lira, fala-me\\*
e torna-te dotada de voz \ldots{}
\end{verse}

\paragraph{Fragmento 127}

{\small O fragmento, preservado em Heféstion, como o 102, consiste, pode-se imaginar, no
verso inaugural de um \textit{hino clético }às Musas, invocadas a deixarem,
provavelmente, a casa de Zeus, no Olimpo, dita regularmente ``áurea”,
na poesia grega antiga. O chamamento, repetido no presente da prece, deve ter
por pedido algo relativo a essas deusas, como a habilidade poética.}

\begin{verse}
Para cá, de novo, ó Musas, deixando a áurea (casa de Zeus?) \ldots{}
\end{verse}

\paragraph{Fragmento 142}

{\small Preservado em Ateneu, como também o Fr.~166, o verso abaixo traz novamente,
agora no plural, o termo ``companheira” (\textit{étaira}), já visto no
126, desta vez ligando o “eu” poético aos nomes de duas figuras míticas: a mãe de
Ártemis e Apolo, Leto, e Níobe, rainha tebana que teve 14 filhos (7 homens, 7
mulheres), e que, comparando-se a Leto, insultou a deusa por ter apenas dois.
Mas estes então puniram a mortal, matando-lhe toda a prole, exceto a filha
caçula, Clóris, que foi por Zeus transformada em rocha. Não é possível afirmar
com certeza, mas o fragmento parece mítico, e nele as três figuras aparecem
estreitamente associadas, tanto pelo termo grego \textit{étairai}, quanto pelo
adjetivo que o precede.}

\begin{verse}
\ldots{} Leto e Níobe eram as mais queridas companheiras \ldots{}
\end{verse}

\paragraph{Fragmento 150}

{\small De acordo com a fonte do fragmento, Máximo de Tiro, que também preservou o Fr.
47, anteriormente traduzido, ``Sócrates ficou bravo com Xantipe que se
lamentava enquanto ele morria”, em referência ao final do \textit{Fédon}, de
Platão, ``e Safo ficou brava com sua filha”, segundo os testemunhos, de
nome Cleis, conforme ainda se verá adiante, nesta antologia. A leitura é
claramente biografizante, e Máximo deve ter conhecido a canção perdida para
nós; de seguro, apenas vemos a ênfase na adequação: na casa de quem serve
deidades luminosas como as Musas, não se admite o canto lamentoso, ou seja, o
treno ou nênia. Mas se Safo é serva dessas deusas, vale lembrar que sua poesia
escapa à restrição, pois há em seu \textit{corpus} ao menos um canto lamentoso
fúnebre, o Fr.~140, que, portanto, problematiza o biografismo da interpretação
do fragmento. Ademais, o canto da Musa não tem, na visão grega antiga,
restrição do tipo que se percebe no Fr.~150. Lembre-se que a \textit{Ilíada},
poema aberto por uma invocação à Musa, como é da tradição épica, é de caráter
profundamente trágico.}

\begin{verse}
\ldots{} pois não é correto na casa dos servos das Musas\\*
haver o treno; \ldots{} isso não nos seria adequado \ldots{}
\end{verse}

\paragraph{Fragmento 154}

{\small Heféstion, fonte do Fr.~102, citou os versos abaixo, provavelmente de abertura
de uma canção que, parece, giraria em torno de um rito sacrificial realizado
por mulheres, à noite.}

\begin{verse}
Em plenitude brilhava a lua, \\*
quando elas em torno do altar se postaram \ldots{}
\end{verse}

\paragraph{Fragmento 156}

{\small A fonte desse fragmento é, como do 101A, Demétrio, que recorda o uso da
hipérbole.}

\begin{verse}
\ldots{} muito mais doce-sonante que a harpa \ldots{}\\*
\ldots{} mais áurea que o ouro \ldots{}
\end{verse}

\paragraph{Fragmento 160}

{\small A fonte do fragmento, como do 166, é Ateneu. Nele, como já nos Frs.
126 e 142, ocorre o termo ``companheira” (\textit{étaira}) no plural. É
possível, pela recorrência desse termo no feminino, que Safo estivesse ligada a
algum círculo similar ao masculino da \textit{hetaireía}, associação
aristocrática fechada formada por um grupo de amigos ou companheiros
(\textit{hetaîroi}) ligados por laços de amizade, afinidades políticas e
alianças marciais. Tal associação celebrava sua própria existência no simpósio,
o banquete em que homens adultos da aristocracia compartilhavam convicções
políticas, lealdades, e os prazeres da comida, do vinho, de \textit{éros}, da música e 
da poesia. Teria Safo uma \textit{hetairía }feminina, em que a guerra era substituída pela
paixão, e o simpósio, por encontros de caráter ritualísticos e/ou sociais
informais? Não estamos em condições de negar ou aceitar a possibilidade,
conforme busquei mostrar na introdução a estas traduções, mas os fatos de que a
ideia se constrói sem qualquer base mais sólida e por espelhamento com algo que
apenas ao universo masculino pertence pedem prudência --- e inspiram ceticismo.
Fica anunciada na canção, em chave metalinguística, a bela \textit{performance}
de uma canção prazerosa a uma plateia feminina e estreitamente ligada à
\textit{performer}.}

\begin{verse}
\ldots{} agora às minhas\\*
companheiras estas coisas prazerosas belamente cantarei \ldots{}
\end{verse}


\section{Elogio}

O elogio e a censura consistem num dos pares opostos mais trabalhados na poesia
grega antiga, podendo mesmo ser tomados como categorias de organização dos
vários gêneros dessa poesia. Em Safo, há fragmentos que se concentram no
elogio, ao menos no que de seus textos restou. Haverá outros centrados na
censura, como dentro em pouco se verá. O poder de um e de outro só pode ser
minimamente compreendido se lembrarmos que a cultura da Grécia arcaica e
clássica pauta-se pela ideia da vergonha; ou seja, o olhar público é
fundamental para a definição do lugar social ocupado pelo indivíduo; essa
definição repercute na sua linhagem e na sua terra. 

\paragraph{Fragmento 41}

{\small Apolônio Díscolo, fonte dos Frs. 32 e 33, nos tratados \textit{Sobre a sintaxe}
(no caso do segundo) e \textit{Sobre os pronomes }(no caso do primeiro e deste
que agora traduzo), cita os versos abaixo:}

\begin{verse}
\ldots{} para vós, as belas, meus pensamentos\\*
não mudaram \ldots{}
\end{verse}

\paragraph{Fragmento 82a}

{\small Heféstion, citado aqui como fonte do Fr.~102, preservou o fragmento abaixo, que traz
um elogio à beleza física de uma moça, realçada pela comparação com outra
também louvada por um adjetivo, ``tenra” (\textit{apálas}), de marcada
carga erótica e empregado reiteradamente nos fragmentos da mélica sáfica que
adentram a esfera do erotismo, como o 94.}

\begin{verse}
Mnasidica, mais formosa que a tenra Girino \ldots{}
\end{verse}


\paragraph{Fragmento 122}

{\small A fonte desse fragmento, como do 166, Ateneu, comenta: 

\begin{hedraquote}
Pois é natural
que os que pensam ser belos e maduros colham flores. Por isso, dizem que
Perséfone e suas companheiras colhem flores, e Safo diz ter visto 
[citação do fragmento].
\end{hedraquote}

 Ou seja, Ateneu sugere a aproximação entre a imagem
mítica da menina virgem, filha da deusa Deméter, e a da virgem ``tenra”
(\textit{apálan}) de que falaria Safo, de modo a ressaltar-lhe a beleza e a
sensualidade. Vale lembrar que Perséfone, como conta o \textit{Hino homérico a Deméter} 
(início do século \textsc{vi} a.C), foi abduzida por Hades exatamente
quando brincava a colher flores num prado primaveril, ignorante dos perigos que
espreitam as belas e sensuais meninas virgens. Cabe frisar que a virgindade das
meninas, na concepção grega, não implica a castidade pura e pretensamente
assexuada da tradição judaico-cristã, mas apenas o fato de que as virgens,
belas, atraentes, sensuais, estão na fase transitória entre infância e idade
adulta, o casamento, quando passam a ter vida sexual e, portanto, integram-se
de vez à sociedade à qual pertencem, deixando para trás a dimensão algo
selvagem e indomada da fase virginal.}

\begin{verse}
\ldots{} tenra virgem, colhendo flores \ldots{}
\end{verse}


\section{Censura}

Não temos no \textit{corpus }da mélica sáfica textos pertencentes ao gênero
poético do jambo, que no correr dos tempos acaba se configurando fundamentalmente como poesia de
ataque, de vituperação, de invectiva. Isso dito, porém, há, sim, em Safo
fragmentos de caráter jâmbico mais ou menos evidente; afinal, se um gênero
acabou por ter na vituperação seu elemento definidor, este não é exclusivo de
gênero algum, mas integra a poesia grega antiga, desde a \textit{Ilíada}. Eis
alguns exemplos sáficos.

\paragraph{Fragmento 56}

{\small Preservado em Crísipo, como o 51, o eu no Fr.~56 parece referir-se à habilidade
poética de uma virgem (\textit{párthenos}) que em vida estaria condenada à
mediocridade.}

\begin{verse}
\ldots{} não imagino que uma virgem, tendo visto a luz do sol,\\
terá em algum tempo futuro tal habilidade \ldots{}
\end{verse}

\paragraph{Fragmento 57}

{\small Ao citar os versos abaixo, em que roupa, atitude e modos de uma moça são
duramente censurados, Ateneu, fonte também do Fr.~166, declara: ``Safo
escarnece Andrômeda”:}

\begin{verse}
\ldots{} e quem é a rusticona que encantou a mente \ldots{}\\*
vestida em rústica veste \ldots{}\\*
sem saber como erguer seus trapos aos tornozelos?
\end{verse}

\paragraph{Fragmento 71}

{\small Em dado momento das convulsões políticas internas e da sucessão de regimes
tirânicos pós-queda da tradicional aristocracia dos Pentilidas, surge Pítaco que se tornou governante de
Mitilene, mas fez aliança com aquela linhagem por meio do casamento,
rompendo com as facções aristocráticas rivais aos Pentilidas, que tinham apoiado, entre as quais, a
do grupo de Alceu e, ao que parece, também a da linhagem de Safo. Abaixo, é notável como a traição de 
Mica, figura feminina, também se firma com a aliança da amizade com os Pentilidas. Termos
pesados são seguidos por referências luminosas, prazerosas, eróticas, mas o
fragmento, cuja fonte é o \textit{Papiro de Oxirrinco }1787 (século \textsc{iii} d.C.),
quase nada revela da canção perdida.}

\begin{verse}
\ldots{} Mica \ldots{}\\
\ldots{} mas eu não permitirei \ldots{}\\
\ldots{} escolheste a amizade da casa de Pentilo \ldots{}\\
\ldots{} ó maligna \ldots{}\\
\ldots{} uma doce canção \ldots{}\\
\ldots{} voz de mel \ldots{}\\
\ldots{} e claras (brisas?) \ldots{}\\*
\ldots{} orvalhada \ldots{}
\end{verse}

\paragraph{Fragmento 91}

{\small A fonte do fragmento, como do 102, é Heféstion.}

\begin{verse}
\ldots{} nunca tendo encontrado uma outra mais repulsiva que tu, Irana \ldots{}
\end{verse}

\paragraph{Fragmento 137}

{\small Aristóteles (século \textsc{iv} a.C.), na \textit{Retórica}, afirma: 

\begin{hedraquote}
Os homens se envergonham de dizer, de fazer ou de pretender fazer vilezas, exatamente
como Safo responde à fala de Alceu [citação do fragmento].
\end{hedraquote}

O filósofo pressupõe um poema de Alceu respondido por um de Safo, esclarece um escólio. Os
dois poetas, comentei na introdução, podem ter se conhecido na Mitilene arcaica
os antigos sempre os aproximavam, inclusive na iconografia. 

\begin{verse}
\ldots{} quero algo te dizer, mas me impede\\
a vergonha \ldots{}\\
\ldots{} mas se ansiasses pelo honrado e pelo belo,\\
e tua língua não se animasse a dizer algo vil,\\
a vergonha não tomaria teus olhos,\\*
mas dirias tua fala \ldots{}
\end{verse}

\paragraph{Fragmento 144}

{\small A fonte desse fragmento é Herodiano (fins do século \textsc{ii} d.C.), no tratado
\textit{Sobre a declinação dos substantivos}. Nele, a personagem feminina é
criticada por uma voz sem qualquer identificação.}

\begin{verse}
\ldots{} bem saturada de Gorgo \ldots{}
\end{verse}

\paragraph{Fragmento 155}

{\small Para Máximo, fonte desse fragmento, como do 47, Safo fala com ironia, tal qual
Sócrates fala a Íon na abertura do diálogo de Platão que leva o nome desse
rapsodo, \textit{Íon}, ou \textit{Sobre a inspiração poética}. Ora, a ironia
liga-se à censura, tanto mais no paralelo socrático sugerido.}

\begin{verse}
\ldots{} minhas muitas saudações à filha de Polianactides \ldots{}
\end{verse}


\chapter[Epitalâmios: canções de casamento]{Epitalâmios:\break canções de casamento}

\textsc{Assim como} o 112, os fragmentos arrolados neste tópico compõem o pequeno grupo
de representantes do subgênero mélico que teria sido compilado no nono livro de
Safo na Biblioteca de Alexandria, como ressaltei na introdução desta antologia.
Naquele fragmento, o elogio aos noivos se destaca; nestes, outros dos aspectos
da canção de casamento, inclusive o da linguagem e temática jocosas, que
revelam seus elos genéticos com a tradição popular.

\paragraph{Fragmento 104 (a-b)}

{\small O fragmento “a”, composto pelo par de versos abaixo, tem por fonte Demétrio,
como o Fr.~101A; já o fragmento “b”, o verso traduzido em separado daquele par,
foi preservado em Himério, retórico do século \textsc{iv} d.C., numa \textit{Oração}.
Ambos os fragmentos integram uma canção de Safo a Vésper, a estrela da tarde,
segundo Himério, e parecem aludir à procissão que sucede o banquete na casa da
noiva, iniciada no começo do anoitecer.}

\begin{verse}
Ó Vésper, trazendo tudo que radiante Eos --- a Aurora --- espallhou:\\
trazes a ovelha, trazes a cabra, trazes a criança de volta à mãe \ldots{}\\
\ldots{} tu, o mais belo de todos os astros \ldots{} 
\end{verse}


\paragraph{Fragmento 105 (a-b)}

{\small Citados por Siriano, comentador bizantino do tratado de estilística de
Hermógenes, os fragmentos abaixo trazem duas imagens metafóricas distintas: na
trinca do “a”, a da menina fruta madura, a maçã, sensual, desejada por todos,
mas não a todos alcançável; na dupla do “b”, a perda da virgindade nas bodas.
Em ambos, é a natureza que alavanca a linguagem erótica.}

\begin{verse}
\ldots{} como o mais doce pomo enrubesce no ramo ao alto,\\
alto no mais alto ramo, e os colhedores o esquecem;\\*
não, não o esquecem --- mas não podem alcançá-lo \ldots{}

\ldots{} como o jacinto que nas montanhas homens, pastores,\\
esmagam com os pés, e na terra a flor purpúrea \ldots{}
\end{verse}

\paragraph{Fragmento 106}

{\small O fragmento tem por fonte Demétrio, como o 101A, cujo tratado diz que
``do homem excepcional assim fala Safo” [citação]. Se tal homem for o
noivo, há seu elogio feito em chave comparativa com o elevado \textit{status}
entre os antigos da tradição poético-musical lésbio-eólica.}

\begin{verse}
\ldots{} superior, como o cantor lésbio aos de outras terras \ldots{}
\end{verse}

\paragraph{Fragmento 107}

{\small O gramático Apolônio Díscolo, fonte do Fr.~33, preservou o 107, no tratado
\textit{Sobre as conjunções}. A virgindade, tema recorrente no epitalâmio, uma
vez que está em evidência sua ruptura no enlace sexual que deve consumar o
casamento, é pensada pelo “eu” que nos fala.}

\begin{verse}
\ldots{} será que ainda anseio pela virgindade?
\end{verse}

\paragraph{Fragmento 110}

{\small Assim como o Fr.~102, este é citado em Heféstion; e, segundo Pólux, fonte do Fr.
54, a figura nele enfocada jocosamente, o guardião do leito nupcial ou tálamo,
tinha por função impedir que os amigos da noiva acorressem a resgatá-la do
quarto que a guardava junto ao noivo.}

\begin{verse}
Os pés do porteiro têm sete braças,\\*
e as sandálias, feitas de cinco bois --- \\*
e dez sapateiros nelas labutaram \ldots{}
\end{verse}

\paragraph{Fragmento 111}

{\small Também citado em Heféstion, o fragmento louva a beleza do noivo símil a Ares,
deus da guerra, e altíssimo, canta, em tom de brincadeira a descrição
hiperbólica. Na cena, alude-se à entrada dos noivos no aposento nupcial --- daí a
referência ao teto e a Himeneu, deus da boda, cuja presença na festa, por isso
solicitada aos gritos, garantiria o sucesso da união.}

\begin{verse}
ao alto o teto --- \\
Himeneu! --- \\
levantai, vós, carpinteiros! --- \\
Himeneu! --- \\
o noivo chega, igual a Ares --- \\
Himeneu! --- \\*
muito maior do que um homem grande \ldots{}
\end{verse}

\paragraph{Fragmento 114}

{\small Este, como o 101A, tem em Demétrio sua fonte. O canto dialogado, forma já vista
nos Frs.~133 e 140, é de \textit{performance} coral. Novamente, a canção jocosa
centra-se na incontornável perda da virgindade, usando o humor decerto como
meio de aliviar a tensão e o impacto de todo o processo que conduz a virgem à
idade adulta, em que passa a atuar na esfera do sexo.}

\begin{verse}
(noiva) --- Virgindade, virgindade, aonde vais, me abandonando?

(virgindade) --- nunca mais a ti voltarei, nunca mais voltarei
\end{verse}

\paragraph{Fragmento 115}

{\small Também preservado em Heféstion, esse fragmento faz o elogio do noivo, de sua
graça física, chamando a atenção da noiva e tornando-o atraente sexualmente aos
seus olhos. }

\begin{verse}
Ao que de belo te comparo, ó caro noivo?\\*
A um ramo esguio mais te comparo \ldots{}
\end{verse}


\chapter{Festividades}

\textsc{O Fr.~2 fala} em festividades que não podemos precisar; os epitalâmios,
caracterizados por estruturas métricas próprias, têm por texto/contexto a
cerimônia do casamento e seus festejos. No grupo a seguir, destaca-se a festa,
ora ligada ao casamento, ora a ritos de celebração aos deuses, ora ao banquete,
evento da vida cotidiana masculina, largamente celebrado na poesia grega
antiga, em todos os seus gêneros.

\paragraph{Fragmento 27}

{\small Preservado no \textit{Papiro de Oxirrinco} 1231, como o Fr.~15, o 27 traz uma
voz não identificável, que busca convencer alguém a tomar parte na procissão de
ida ao casamento, liderando virgens, certamente ligadas à noiva; os últimos versos
legíveis enveredam para uma sentença moralizante.}

\begin{verse}
\ldots{} e certa vez tu também uma menina \ldots{}\\
\ldots{} cantar; vem, estas coisas \ldots{}\\
\ldots{} discutir \ldots{} e para nós \ldots{}\\
concede coisas deleitáveis;\\
\ldots{} pois vamos a um casamento; bem \ldots{}\\
\ldots{} isso também tu, mas então rapidamente\\
as virgens conduz, deuses \ldots{}\\
\ldots{} a estrada para o grande Olimpo\\*
\ldots{} mortais(?)
\end{verse}

\paragraph{Fragmento 30}

{\small Tendo por fonte o mesmo \textit{Papiro de Oxirrinco} 1231, o fragmento traz uma
voz que fala ao noivo, como se nota no início e no fim, e da celebração de
caráter ritual, que, na cena, parece concentrar-se no grupo masculino que do
banquete, antes da procissão nupcial, tomava parte separadamente do feminino,
ligado à noiva e também presente. O fragmento, como o 27, tem caráter
epitalâmico, mas sua estrutura métrica não condiz com as desse subgênero
mélico.}

\begin{verse}
\ldots{} virgens \ldots{}\\
\ldots{} celebrando um festival noturno \ldots{}\\
\ldots{} teus amores cantariam e os da noiva\\*
de violáceo colo \ldots{}

mas, tendo se erguido, os moços solteiros (?)\\*
traz os da tua idade \ldots{}\\*
\ldots{} sono do que de clara voz \ldots{}
\end{verse}

\paragraph{Fragmento 81}

{\small Ateneu, fonte também do Fr.~166, declara sobre a tradição de adornar-se no
simpósio: ``Safo expressa com mais simplicidade a razão de nossa prática
de usar guirlandas, dizendo isto”:}

\begin{verse}
\ldots{} e tu, ó Dica, cinge teus cachos com amáveis guirlandas,\\
tramando raminhos de aneto com tuas mãos macias;\\
pois as Cárites --- as Graças! --- voltam-se às coisas floridas,\\
mas dão as costas ao não coroado \ldots{}
\end{verse}

\paragraph{Fragmento 141 (a-b)}

{\small Ateneu, fonte também deste fragmento --- que pode estar ligado a um banquete
nupcial como o do casamento da Nereida Tétis e do mortal Peleu, pais de Aquiles
---, afirma que ``Alceu introduz Hermes como um servidor de vinho dos
deuses, exatamente como Safo, ao dizer'':}

\begin{verse}
\ldots{} e depois que uma cratera\\*
de ambrósia foi misturada à água,\\*
Hermes, tomando o jarro, vinhoverteu (o líquido) aos deuses \ldots{}

\ldots{} e todos\\
seguravam cálices,\\
e libavam, e rezaram para toda a sorte de coisas boas\\
para o noivo.
\end{verse}

\chapter{Vestes e adornos}

\textsc{No universo feminino} tão privilegiado na mélica sáfica, recebem atenção vestes e
adornos, que tornam mais belas e atraentes as figuras contempladas. Isso já se
nota no Fr.~22, em que o vestido integra a apreensão erótica do objeto do
desejo de quem com os olhos o detém; e no Fr.~44, a beleza da noiva que chega a
Troia com o esposo e os companheiros deste, na narrativa da procissão, se
reflete no dote cheio de tecidos e ornamentos. Eis mais dois fragmentos.


\paragraph{Fragmento 39}

{\small Citado num escólio à comédia \textit{A paz}, de Aristófanes (séculos \textsc{v}--\textsc{iv} a.C.),
o fragmento faz menção à Lídia que para a Grécia exportava variados e altamente
elaborados produtos de luxo.}

\begin{verse}
\ldots{} e cobria\\*
seus pés sandália furta-cor, belo\\*
trabalho lídio \ldots{}
\end{verse}

\paragraph{Fragmento 125}

{\small Citado num escólio à comédia \textit{As tesmoforiantes}, de Aristófanes.}

\begin{verse}
\ldots{} eu mesma na juventude costumada tecer guirlandas \ldots{}
\end{verse}


\chapter{Cleis}


\paragraph{Fragmento 98 (a + b)}

{\small Preservado em fontes do século \textsc{iii} a.C., o \textit{Papiro Hauniense }301 (a) (ou
\textit{Papiro de Copenhagen}) e o \textit{Papiro Mediolanense }32 (b) (ou
\textit{Papiro de Milão}), o fragmento é lido em chave biográfica, por conta do
nome Cleis que, segundo fontes antigas, era filha da poeta; logo, o “eu” seria
``Safo”, que se referiria, ainda, a sua mãe, a avó de Cleis, na
abertura. Há nele uma contraposição entre os tradicionais adornos, tidos como		\EP[]
elegantes no passado, e o adorno mais sofisticado e desejado no presente, a
\textit{mitra}, fita ou faixa adornada que cobria o cabelo, mas não as orelhas,
feita na Lídia, como a sandália do Fr.~39 --- provas do influxo oriental na
cultura da ilha de Lesbos, ambas ditas ``furta-cor”, na tradução do
adjetivo derivado do substantivo \textit{poikilía}, que carrega as ideias do
cintilar, da múltipla cor, do variegado, ou seja, da mistura de luz, formas ou
cor, que dificulta a direta e clara apreensão do objeto que se contempla, e que
justamente por isso associa-se à sedução erótica e à astúcia. O “eu” lamenta a
incapacidade de dar a Cleis a \textit{mitra}, talvez por impedimentos
comerciais ou políticos impostos pelo governo do “mitilênio”, possivelmente
Pítaco, ou o adjetivo de um substantivo identificado a outra personagem. Há
ainda a menção ao exílio “dos Cleanactidas” ou “de Cleanactides”, de memória
ainda viva em Lesbos, mas não sabemos como isso se ligaria ao
“mitilênio”, a Cleis, a ``Safo”. Cabe recordar que fontes antigas falam
de um período de exílio vivido por Safo em Siracusa, na Sicília (Magna Grécia),
por conflitos de seu grupo aristocrático com Pítaco.}

\begin{verse}
\ldots{} pois ela (que me gerou, disse que)


em sua época, era grande \\*
adorno, se alguém tinha os cachos\\*
atados em nó purpúreo;


era isso mesmo, \\*
mas se alguém tinha a coma\\*
mais fulva que a tocha \ldots{},

com guirlandas (ornadas)\\*
de flores em flor.\\*
Recentemente, (Cleis?), uma fita

furta-cor de Sárdis \ldots{}\\*
\ldots{} cidades \ldots{}

Mas eu, a ti, Cleis, uma fita furta-cor \ldots{} ---\\*
não tenho meios de tê-la; \\*
mas com o mitilênio \ldots{}


\ldots{} do exílio dos Cleanatidas\\*
esses memoriais \ldots{} terrivelmente\\*
devastado \ldots{}
\end{verse}

\paragraph{Fragmento 132}

{\small Citado em Heféstion, fonte do Fr.~102, o 132 é o segundo e último em que vemos
claramente o nome de Cleis, que o “eu” introduz como sua filha, louvando-lhe a
beleza.}

\begin{verse}
Tenho bela criança, portando forma símil\\*
à das áureas flores, Cleis, filha única,\\*
por quem eu não (trocaria?) toda a Lídia, nem a amável (Lesbos?) \ldots{}
\end{verse}


\chapter{Reflexões ético-morais}


\paragraph{Fragmento 26}

{\small A fonte do fragmento é o \textit{Papiro de Oxirrinco} 1231, como também do Fr.
15. O que abaixo se lê parece centrar-se numa discussão de caráter ético-moral
a partir da experiência do “eu” feminino que se mostra consciente da dolorosa
ingratidão, e não ingenuamente iludida.}

\begin{verse}
\ldots{} pois aqueles a quem trato bem\\
são os que dentre todos sobretudo me\\
machucam \ldots{}\\
\ldots{} mas eu própria\\*
disso tenho consciência\ldots{}
\end{verse}

\paragraph{Fragmento 37}

{\small A fonte é o tardio \textit{Etimológico genuíno}, como é o caso do 126. Dor e
rejeição à censura recebida --- indevidamente?: eis o que canta o fragmento.}

\begin{verse}
\ldots{} em minha dor \ldots{}\\*
\ldots{} que ventos e mágoas carreguem a ele que me censura \ldots{}
\end{verse}

\paragraph{Fragmento 50}

{\small Na \textit{Exortação à aprendizagem, }Galeno (século \textsc{ii} d.C.) cita dois versos
em que Safo contrapõe a beleza à bondade: ``Portanto, já que o auge da
juventude é como as flores primaveris, trazendo prazer de curta vida, é melhor
louvar a lésbia, quando ela diz'':}

\begin{verse}
\ldots{} pois o belo é belo enquanto se vê,\\*
mas o bom será consequentemente belo \ldots{}
\end{verse}

\paragraph{Fragmento 52}

{\small A fonte desse fragmento, como do 144, é Herodiano, no tratado \textit{Sobre
palavras anômalas}. Nele, o “eu” afirma-se consciente dos limites próprios da
mortalidade, que tem no alcance dos céus e no voo uma de suas imagens mais
fortes e recorrentes na tradição mítico-poética grega.}

\begin{verse}
\ldots{} não espero tocar o céu (com meus dois braços?) \ldots{}
\end{verse}

\paragraph{Fragmento 58}

{\small Os dois últimos e mais legíveis versos (25--26) do Fr.~58, cuja fonte é o
\textit{Papiro de Oxirrinco }1787, que também preservou o 71, constam de outra
fonte, Ateneu, que também cita o já visto Fr.~166. Após a publicação em 2004 do
novo papiro de Safo --- que revelou uma nova canção fragmentária, em texto que se
sobrepõe a boa parte do que até então era editado como Fr.~58 ---, concluiu-se que aqueles
versos são independentes dos que os precedem. Eis, pois, os versos 25--26, que
trazem uma noção muito cara à mélica sáfica, da \textit{habrosýne}
(``delicadeza”), de entranhada sensualidade em dimensão de sentido
estético, que alude a certo modo de vida aristocrático marcado pelo luxo e pelo
refinamento, que estimulavam a importação de objetos e costumes orientais que
estavam sob o signo de tais marcas.}

\begin{verse}
\ldots{} mas eu amo a delicadeza \ldots{} isso, e a mim \\*
a luz e a beleza o amor do sol deu por parte.
\end{verse}

\paragraph{Fragmento 120}

{\small É fonte do fragmento o \textit{Etimológico magno} (século \textsc{xii}); nele, o “eu”
explica sua disposição.}

\begin{verse}
\ldots{} mas não sou das de têmpera vingativa:\\*
tenho o ânimo sereno \ldots{}
\end{verse}

\paragraph{Fragmento 148}

{\small Um escólio a Píndaro (\textit{Ode olímpica} 2) cita o fragmento abaixo,
que condena a separação entre a prosperidade material e a virtude ético-moral.}

\begin{verse}
\ldots{} a riqueza sem a excelência não é vizinha inofensiva,\\
mas a mistura de ambas traz a mais alta felicidade \ldots{}
\end{verse}

\paragraph{Fragmento 158}

{\small No tratado \textit{Sobre o refrear da cólera}, Plutarco, fonte do Fr.~49, cita o
158 como um ``conselho de Safo” diante da seguinte situação:

\begin{hedraquote}
Quando as pessoas estão bebendo, o que permanece silencioso é um peso
cansativo a seus companheiros; mas quando alguém está com raiva, nada é mais
digno do que a quietude.
\end{hedraquote}}

\begin{verse}
\ldots{} a raiva espalhando-se \\*
no peito, proteger-se da língua tagarela \ldots{}
\end{verse}

\chapter{\mbox{“Canção sobre a velhice”:} novo fragmento}

\textsc{Como observei} na anotação ao Fr.~58, este foi quase totalmente reeditado com a 
publicação, em 2004, do \textit{Papiro de Colônia} 21351 (início do século \textsc{iii} a.C.), 
cujo texto se sobrepunha a boa parte do texto conhecido daquele fragmento. No início, a 
linguagem metalinguística desenha uma cena de canto junto a crianças --- meninas, creem os 
que pensam na suposta associação liderada por Safo, mencionada na introdução desta antologia ---, 
integrada pelo favorecimento de divindades que devem ser as Musas,
dado o tema dos versos da abertura preservada. Em seguida, o tema da velhice
vem à tona e, com ele, a reiterada imagem dos cabelos que se tornam grisalhos,
e a provável referência à perda do frescor do corpo ou da pele; depois, os
tormentos e preocupações, e a dificuldade de movimentação, que contrasta
drasticamente com a leveza de joelhos dançantes na juventude. Esse quadro é
motivo de dor, mas é inexorável e inelutável; daí a expressão do sentimento
agudo de impotência contra o que é próprio da natureza mortal: envelhecer. Para
ilustrar esta verdade, em passagem gnômica, o “eu” não identificado recorda o
mito da paixão de Eos, a Aurora, por Títono, para o qual é nossa fonte
principal o \textit{Hino homérico \textsc{v}, a Afrodite }(versos 218--238), datado do
século \textsc{vii} a.C., e de autoria anônima. Em síntese, esse mito conta como a
deusa, tomada de paixão, pediu a Zeus que tornasse seu amado imortal; a deusa,
porém, esqueceu-se de que a mortalidade do homem não se concretiza apenas na
morte, mas na velhice; assim, Títono tornou-se imortal, mas imortalmente,
indefinidamente velho; uma vez que a paixão é suscitada e sustentada pela
beleza do corpo e pela sua capacidade de atração, a esfera da velhice é
inadequada aos dons de Afrodite; Eos, então, acaba por se desinteressar por
completo do mortal, a quem encerra num quarto, do qual ressoa sem cessar sua
voz --- em certas tradições, esta é uma referência à metamorfose de Títono em
cigarra ---, saindo eternamente de seu débil e velho corpo sempre a minguar. Merece nota o fato de
que Safo, com esse fragmento, ganha relevo na série de textos poéticos da
Grécia arcaica que tratam da velhice, em gêneros variados, com ênfase na
decadência física que, no célebre Fr.~1 --- uma elegia talvez completa --- de
Mimnermo, ganha conotação ético-moral, e é descrita como obstáculo
intransponível à participação na esfera de Afrodite que é a síntese da vida.
Dado o ingrediente erótico do mito que também esse poeta de meados do século
\textsc{vii} a.C. recorda noutra elegia --- o Fr.~4, em que o presente de Títono é
julgado pior do que a própria morte, dada a primazia da paixão na perspectiva
dos versos conhecidos do poeta ---, é possível que a soma do canto e dança à
velhice e à referência mítica no novo fragmento de Safo apontem para um cenário
em que também Afrodite e/ou seu universo tomassem parte. Digno de nota, também,
é o modo como Eos se caracteriza, ecoando as imagens homéricas da Aurora
``dedirrósea” e da Hera ``de alvos braços”, misturadas no
epíteto composto \textit{brodópakhyn}, no Fr.~53 atribuído às Cárites.
Ampliando o epíteto homérico, Safo amplia os rasgos
róseos-avermelhados-alaranjados do amanhecer, traçados não por dedos, mas pelos
braços que tanto desejaram enlaçar Títono, enquanto jovem e belo foi seu
corpo.


\begin{verse}
\ldots{} (das Musas) de colo violáceo os belos dons, crianças,\\
\ldots{} a lira melodiosa, amante do canto;

outrora tenra (a pele), agora da velhice \ldots{}\\*
\ldots{} e os cabelos de negros se tornaram brancos.


Pesado se me fez o peito, e os joelhos não me suportam --- \\
os que um dia foram ágeis no dançar, como os da corça.

Estas coisas lamento sem cansar, mas que posso fazer?\\
Não é possível, sendo homem, ser desprovido da velhice.

Pois, certa vez, dizem que Eos, a Aurora de róseos braços,\\
com paixão \ldots{} carregando Títono aos confins da terra,

belo e jovem que era ele; mas mesmo a ele alcançou similarmente\\
a velhice grisalha em tempo --- ele que tinha por esposa uma imortal \ldots{}
\end{verse}
