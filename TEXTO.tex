\chapter*{}
\thispagestyle{empty}

\vspace*{\fill}
\begin{verse}
\small{Ó Safo, aos jovens que amam o mais doce travesseiro das paixões,\\
a ti, junto às Musas, a Piéria adorna, ou o\\
Hélicon coberto de hera -- a ti que sopras tal qual\\
elas a ti, Musa na Ereso eólia.\\
Ou Hímen Himeneu, portando sua tocha brilhante,\\
contigo fica sobre o tálamo nupcial;\\
ou junto a Afrodite enlutada, lamentando o jovem rebento de\\
Ciniras, contemplas o bosque sacro dos venturosos.\\
Em toda parte, ó soberana, te saúdo como aos deuses, pois tuas\\
canções ainda hoje consideramos filhas dos imortais.}
\end{verse}

\begin{flushright}
\small{\textsc{dioscúrides, séc.\,iii a.c.}}\\
\small{\textit{Antologia palatina}, \textsc{vii}, ep.\,407}
\end{flushright}

\setbeforesecskip{1.2\onelineskip}%
  \setaftersecskip{.35\onelineskip}%
  \setsecheadstyle{\large\baselineskip=.7\baselineskip\bfseries\scshape\MakeTextLowercase}%

%oneside
\openany
\pagestyle{plain}

\chapter{Afrodite}

\paragraph{\textsc{nota introdutória}} 
Nenhuma outra divindade grega aparece nas canções de Safo com a mesma
frequência, nem do mesmo modo: Afrodite é a mais presente\footnote{ Afrodite é
personagem dos fragmentos 1, 2, 5, 15, 22, 33, 73\textsc{a}, 86, 96, 102, 112, 133, 134
e 140. Neles, o tema da presença da deusa \textit{corpus} foi estudado em Ragusa (2005), em que se baseiam as traduções dos e as notas aos fragmentos legíveis desse \textit{corpus}. Tais traduções, como outras publicadas previamente a esta nova edição da antologia, podem estar aqui ligeiramente alteradas.}. O fato
decerto reflete três das linhas de força da mélica sáfica: a
paixão erótica, a beleza e o universo feminino. Ora, Afrodite, em Safo e nos
demais poetas gregos, para não falar da iconografia e dos cultos, é
multifacetada -- como são em geral os deuses gregos --, mas é, fundamentalmente,
deidade da beleza física, da feminilidade, da sensualidade, da sedução, da
paixão erótica, do desejo -- características que constituem seus poderes
principais e sua esfera central de atuação, a do erotismo. Há, portanto,
estreita afinidade entre o fazer poético de Safo e a imagem de Afrodite, que se
traduz em notável e inigualável cumplicidade entre a deusa dileta e a voz
poética dos versos.\footnote{ A diferença é evidente se comparamos a
representação da deusa em Safo à encontrada nos demais poetas arcaicos que, como ela, praticaram a poesia mélica; para estes e o estudo de Afrodite em seus fragmentos, ver Ragusa (2010).}

\pagebreak
\section{«\,Hino a Afrodite\,» (Fr.\,1)} 

\begin{gkverse}
Ποικιλόθρον’ ἀθανάτ’ Ἀφρόδιτα,\\
παῖ Δίος δολόπλοκε, λίσσομαί σε,\\
μή μ’ ἄσαισι μηδ’ ὀνίαισι δάμνα,\\
	πότνια, θῦμον,

ἀλλὰ τυίδ’ ἔλθ’, αἴ ποτα κἀτέρωτα\\
τὰς ἔμας αὔδας ἀίοισα πήλοι\\
ἔκλυες, πάτρος δὲ δόμον λίποισα\\
	χρύσιον ἦλθες

ἄρμ’ ὐπασδεύξαισα κάλοι δέ σ’ ἆγον\\
ὤκεες στροῦθοι περὶ γᾶς μελαίνας\\
πύκνα δίννεντες πτέρ’ ἀπ’ ὠράνω αἴθε-\\
ρος διὰ μέσσω

αἶψα δ’ ἐξίκοντο σὺ δ’, ὦ μάκαιρα,\\
μειδιαίσαισ’ ἀθανάτωι προσώπωι\\
ἤρε’ ὄττι δηὖτε πέπονθα κὤττι\\
δηὖτε κάλημμι

κὤττι μοι μάλιστα θέλω γένεσθαι\\
μαινόλαι θύμωι τίνα δηὖτε πείθω\\
\ldots{} σ̣άγην ἐς σὰν φιλότατα; τίς σ’, ὦ\\
Ψάπφ’, ἀδικήει;

καὶ γὰρ αἰ φεύγει, ταχέως διώξει,\\
αἰ δὲ δῶρα μὴ δέκετ’, ἀλλὰ δώσει,\\
αἰ δὲ μὴ φίλει, ταχέως φιλήσει\\
κωὐκ ἐθέλοισα.

ἔλθε μοι καὶ νῦν, χαλέπαν δὲ λῦσον\\
ἐκ μερίμναν, ὄσσα δέ μοι τέλεσσαι\\
θῦμος ἰμέρρει, τέλεσον, σὺ δ’ αὔτα\\
σύμμαχος ἔσσο. 
\end{gkverse}

\pagebreak
\section*{}

\begin{verse}
De flóreo manto furta-cor, ó imortal Afrodite,\\
filha de Zeus, tecelã de ardis, suplico-te:\\
não me domes com angústias e náuseas,\\*
veneranda, o coração,

mas para cá vem, se já outrora --\\
a minha voz ouvindo de longe -- me\\
atendeste, e de teu pai deixando a casa\\*
áurea a carruagem

atrelando vieste. E belos te conduziram\\
velozes pardais em torno da terra negra --\\
rápidas asas turbilhonando, céu abaixo e\\*
pelo meio do éter.

De pronto chegaram. E tu, ó venturosa,\\
sorrindo em tua imortal face,\\
indagaste por que de novo sofro e por que\\*
de novo te invoco,

e o que mais quero que me aconteça em meu\\
desvairado coração. ``Quem de novo devo persuadir\\
\ldots{} ao teu amor? Quem, ó\\*
Safo, te maltrata?

Pois se ela foge, logo perseguirá;\\
e se presentes não aceita, em troca os dará;\\
e se não ama, logo amará,\\*
mesmo que não queira''.

Vem até mim também agora, e liberta-me dos\\
duros pesares, e tudo o que cumprir meu\\
coração deseja, cumpre; e, tu mesma,\\*
sê minha aliada de lutas.
\end{verse}

\pagebreak
\textls[-24]{\paragraph{Comentário} Esse \textit{hino clético} -- prece que invoca a deidade para instá"-la a vir à \EP[1]
presença de quem suplica -- estrutura"-se em três etapas
fundamentais, mostra a tradição: identificação do deus,\footnote{Versos 1--2.} essencial
num sistema politeísta; recordação de relação previamente firmada com a
deidade, de modo a nela suscitar o sentido de obrigação para com quem apela;\footnote{Versos 5--24.} explicitação do(s) pedido(s). Essa cuidadosa elaboração
formal explica"-se por constituir a própria prece em presente à divindade que, com
tal agrado, pode se tornar propícia. No hino, a suplicante, que se autonomeia
``Safo'',\footnote{Trata"-se de procedimento muito usado em variados gêneros poéticos, por meio dos quais a \textit{persona} dramatizada do próprio poeta torna"-se parte dos versos que podem imortalizá"-lo. A ideia da poesia como instrumento de imortalização de heróis está no cerne da épica, mas logo vemos a ideia associada de que a poesia confere memória e fama não apenas àquilo que canta, mas àquele que a canta. Safo vale"-se de autonomeação em outros fragmentos, bem como Álcman, poeta mélico ativo em c.\,620 a.C., Teógnis, poeta elegíaco de fins de 600 a.C., e mesmo antes, Hesíodo, ativo em fins de 700 a.C., na sua poesia didática"-comosmogônica e didático"-sapiencial, para mencionar apenas esses mais antigos poetas. A prática se perpetua, claro, no correr dos séculos, em várias tradições poéticas, nas quais vai sendo ressignificada. Discuti recentemente o tema da memória e imortalização do poeta pela poesia na Grécia arcaica (Ragusa, 2018, pp. 143--152).} invoca a deusa
a vir à sua presença, para junto a ela lutar pela sedução da amada que ora a
rejeita, objetivo em torno do qual giram todos os pedidos.\footnote{Versos 1--5, 25--8.}
Atente"-se para a fala de Afrodite,\footnote{Versos 21--4.} dita no passado, mas
revalidada no presente e a cada nova rodada da vinda da intermitente paixão
erótica; tal fala guarda um valor universal de caráter punitivo"-consolatório: o
consolo do amador rejeitado pelo amado está na reversão de papéis que a
experiência erótica em tempo produz. Também vale a pena reparar na visão de
\textit{éros} como patologia de corpo e mente,\footnote{Versos 3--4.} que, em princípio,
torna sua vítima impotente; e no modo como a sedução é pensada como uma batalha
e uma caçada, arenas às quais são comuns o ataque violento, a perseguição e a
dominação do outro -- o inimigo, a presa, o objeto de desejo do sedutor. Tudo
isso está muito presente na linguagem erótica da poesia grega antiga, mas se
maximiza no Fr.\,1 de Safo, em que a suplicante chama Afrodite a ser sua
``aliada de lutas'', \textit{sýmmakhos},\footnote{Verso 28.} em estreita parceria. Eis o último pedido da prece, que, ao contrário dos demais que se retomam, só surge nas penúltimas palavras. Tudo no hino à deusa é, pois, arquitetado para culminar neste pedido que, se concedido, como será (já o foi no passado!), garantirá o sucesso da empreitada de sedução à qual quer se lançar a suplicante"-amada. Pedido que é proferido estrategicamente, depois que aos poucos construiu ``uma relação segura e protetora com a poderosa e caprichosa deusa''.\footnote{Thomas, 1999, p.\,9} Finalmente, nos versos 1--2, veja"-se
que os epítetos estabelecem Afrodite não só como poderosa, mas também bela
sedutora ardilosa. A sedução e a deidade caminham, pois, no âmbito do fugidio,
do oblíquo, da dissimulação, da trama. A arte do engano é imprescindível na
esfera da sedução regida por Afrodite; e nessa arte, ninguém superará a deusa,
preciosa aliada.

Não posso deixar de ressaltar que o ``Hino a Afrodite'' é não só
a mais famosa canção de Safo, mas a única integralmente preservada em citação, feita no tratado \textit{Sobre o arranjo das palavras},\footnote{Número 23.} de Dionísio de Halicarnasso (retórico, século \textsc{i} a.C.). Mais: é a primeira canção do livro \textsc{i} de Safo,
compilado na célebre Biblioteca de Alexandria, provavelmente na virada dos
séculos \textsc{iii}--\textsc{ii} a.C.
Todo plasmado como é em Afrodite, a deusa \textit{sýmmakhos} de ``Safo'' que lhe pede e antecipa sua presença, pode"-se pensar o epíteto e o hino como um todo em dimensão metapoética, à semelhança do que se passa com o Fr.\,2, adiante. Nessa dimensão, ``Safo'' dramatiza"-se como ``uma mulher líder de um grupo feminino e uma poeta totalmente dedicada à esfera da beleza, do amor, e de Afrodite'',\footnote{Bierl 2018, p. 929.} e o faz nas derradeiras palavras, ``drasticamente revertendo seu sentimento inicial de devastação e depressão'', para se afirmar qual ``autoconfiante \textit{choregos}, ‘líder do coro’, e cantora poética''.\footnote{Bierl, p. 930.} ``Reencenando essa divina aliada de luta na própria \textit{performance}, a canção faz Safo se fundir a Afrodite. Assim, em qualquer \textit{performance}, Safo se torna Afrodite, como a cantora perfeita, plena de encantamento poético e erótico''.\footnote{Bierl, p. 948.}}

\pagebreak
\section{«\,Ode do óstraco\,» (Fr.\,2)}

\begin{gkverse}
\dagger{}δευρυμμ̣εκρητε̣σιπ̣[.]ρ[̣ \quad].\dagger{} ναῦον\\
ἄγνον ὄππ̣[αι\quad] χάριεν μὲν ἄλσος\\
μαλί[αν], βῶμοι δ’ ἔ<ν>ι θυμιάμε-\\
νοι [λι]β̣ανώτω<ι>

ἐν δ’ ὔδωρ ψῦχρον κελάδει δι’ ὔσδων\\
μαλίνων, βρόδοισι δὲ παῖς ὀ χῶρος\\
ἐσκίαστ’, αἰθυσσομένων δὲ φύλλων\\
κῶμα \dagger{}καταιριον

ἐν δὲ λείμων ἰπ̣π̣όβοτος τέθαλε\\
\dagger{}τω̣τ...(.)ριννοις\dagger{} ἄνθεσιν, αἰ <δ’> ἄηται\\
μέλλιχα πν[έο]ισιν [\\
\quad[\qquad\qquad]

ἔνθα δὴ σὺ \dagger{}συ.αν\dagger{} ἔλοισα Κύπρι\\
χρυσίαισιν ἐν κυλίκεσσιν ἄβρως\\
<ὀ>μ<με>μείχμενον θαλίαισι νέκταρ\\
οἰνοχόεισα
\end{gkverse}

\begin{verse}
\ldots{} Para cá, até mim, de Creta \ldots{} templo\\
sacro onde \ldots{} e agradável bosque\\
de macieiras, e altares nele são esfumeados\\*
com incenso.

E nele água fria murmura por entre ramos\\
de macieiras, e pelas rosas todo o lugar\\
está sombreado, e das trêmulas folhas\\*
torpor divino desce.

E nele o prado pasto de cavalos viceja\\
\ldots{} com flores, e os ventos\\*
docemente sopram \ldots{}

Aqui tu [\ldots{}] pegando, ó Cípris,\\ \EP[1]
nos áureos cálices, delicadamente,\\
néctar, misturado às festividades,\\
vinho-vertendo \ldots{}
\end{verse}


\pagebreak
{\paragraph{Comentário} Eis outro \textit{hino clético}, em que se destaca o detalhamento do local ao
qual Afrodite é convidada a vir, saindo de Creta. O espaço, porém, não é
definido cartograficamente, mas se desenha como cenário primaveril idealizado
em chave sacroerótica, inerente à visão grega da natureza, suspenso em
temporalidade própria, impregnado de Afrodite, de cujas imagens poéticas e
mítico"-religiosas se desprendem seus elementos constitutivos. Desse cenário
emana uma atmosfera carregada de sensualidade e do divino, algo ampliado pela
antecipação da epifania da deusa\footnote{Verso 13.} invocada como ``Cípris'' --
nome mais frequente na literatura grega antiga, além de ``Afrodite'' --, que evoca
seus elos com um de seus locais de culto mais importantes, a ilha de Chipre,
onde são particularmente fortes suas ligações com o mundo vegetal. Atente"-se
para o caráter ativo da epifania, que reforça a ideia da fusão num fragmento de intensa linguagem sinestésica. Há o desejo de proximidade
entre a voz poética e a deusa. Proximidade que assume um caráter metapoético na diluição dos limites do mundo divino e do sagrado -- a mistura do néctar, nutrição divina, e do vinho, nutrição mortal -- na festividade compartilhada da mélica sáfica que celebra o universo de Afrodite. A fonte principal do texto é um \textit{óstraco} (século \textsc{iii} a.C.) ou caco de cerâmica, material abundante na Grécia antiga e muito utilizado para a
escrita.}


\pagebreak
%\section{Dórica e Cáraxo (Fr.\,5 e 15)}
\section{Prece a Afrodite e às Nereidas (Fr.\,5)}

\begin{gkverse}
Κύπρι καὶ] Νηρήϊδες ἀβλάβη[ν μοι\\
		τὸν κασί]γνητον δ[ό]τε τυίδ’ ἴκεσθα[ι\\
κὤσσα ϝ]ο̣ι ̣θύμω<ι> κε θέλη γένεσθαι\\
πάντα τε]λέσθην,

ὄσσα δὲ πρ]όσθ’ ἄμβροτε πάντα λῦσα[\\
καὶ φίλοισ]ι ϝοῖσι χάραν γένεσθαι\\
\ldots{} ἔ]χθροισι, γένοιτο δ’ ἄμμι\\
\ldots{} μ]ηδ’ εἴς

τὰν κασιγ]νήταν δὲ θέλοι πόησθαι\\
 ]τίμας, [ὀν]ίαν δὲ λύγραν\\
 ]οτοισι π[ά]ροιθ’ ἀχεύων\\
 ].να\\
 ].εισαΐω[ν] τὸ κέγχρω\\
  ]λεπαγ̣[..( .́ )]αι πολίταν\\
  ]λλωσ̣[...]νηκε δ’ αὖτ’ οὐ\\
 ]κρω[]\\
 ]οναικ[\quad]εο[\quad]. ι\\
 ]..[.]ν σὺ [δ] ὲ̣ Κύπ̣[ρι]..[..(.)]να\\
       ]θεμ[έν]α κάκαν [\\
	       ]ι.	
\end{gkverse}

\begin{verse}
Ó Cípris e Nereidas, ileso, a mim,\\
o meu irmão concedei aqui chegar,\\
e o que no coração ele queira que seja --\\*
tudo cumpri;

e que seus passados erros todos ele repare\\
e que aos amigos uma alegria ele seja,\\*
\ldots{} aos inimigos, e que não nos seja \ldots{}

[\ldots{}] e a irmã -- que ele a queira fazer
\end{verse}

\pagebreak
{\paragraph{Comentário} No fragmento preservado no \textit{Papiro de Oxirrinco} 7 (século \textsc{iii} d.C.),
temos uma canção"-prece a Afrodite -- chamada pelo nome que nos remete à sua
geografia mítico"-poética e religiosa insular, ``Cípris'' -- e às Nereidas
-- netas de Oceano e filhas do velho do mar, Nereu. Quem apela às deidades o faz
em benefício da 3ª pessoa do singular, a quem se refere como ``meu
irmão''.\footnote{Verso 2.} Uma vez que o \textit{eu} da poesia de Safo é habitualmente
identificado à própria poeta, é leitura corrente que o \textit{ele} é Cáraxo, com base
em Heródoto (século \textsc{v} a.C., \textit{Histórias}, \textsc{ii}, 134--135), segundo quem uma
\textit{hetera} de nome Rodopis, feita escrava, foi levada para o Egito do faraó Amásis
(\textit{c.} 570 a.C.) por um homem que, depois, a libertou em troca de vultoso
pagamento efetuado por ``Cáraxo de Mitilene, filho de Escamandrônimo e
irmão de Safo, a poeta''; e Heródoto diz: Cáraxo, em seguida, ``retornou
a Mitilene, e numa canção Safo muito o atacou, de maneira severa''. Há nesse
relato, porém, sérios problemas de cronologia; ademais, nada prova que Heródoto se refira ao
nosso Fr.\,5; e tampouco se justifica de fato a identificação automática de
Safo ao \textit{eu} da canção, o qual pode ser outra personagem.
Na prece, identificadas as deidades, seguem"-se os
pedidos. Primeiro, de proteção ao navegante que retorna -- função própria da
atuação de Afrodite e das Nereidas, indicam seus cultos e imagens
mítico"-poéticas e iconográficas. Mas nos pedidos seguintes, nada há que os
ligue de maneira especial às deidades invocadas; eles falam no
cumprimento de todos os desejos do ``irmão'' -- algo que recorda os
versos 17--18 e 26--27 do Fr.\,1 --, na reparação de seus erros passados, na
atitude para com amigos e inimigos -- ``alegria'' àqueles,
males, a estes, de acordo com a ética grega tradicional. Falam ainda em algo
relativo à sua ``irmã'', que não sabemos quem é.\footnote{Mantenho para o Fr.\,5 (e outros) a edição Voigt (1971). Preferi não adotar reedições com base em novos papiros envoltos em complexas questões, das quais dá ideia a nota de retratação de Anton Bierl e André Lardinois, editores de \textit{The newest Sappho} (Brill, 2016). Recomenda"-me a prudência, mesmo se excessiva, aguardar a estabilização do cenário.}}


\pagebreak
\section[Prece a Afrodite, uma punição para Dórica (Fr.\,15)]{Prece a Afrodite, uma punição\\para Dórica (Fr.\,15)}

\begin{gkverse}
]α̣ μάκαι̣[ρ

\textnormal{[\textit{versos 2--8: ilegíveis e lacunares}]}

Κύ]πρι κα[ί σ]ε πι[κροτ..́]α̣ν ἐπεύρ[οι\\
μη]δὲ καυχάσ[α]ιτο τόδ’ ἐννέ\,[ποισα\\
Δ]ω̣ρίχα τὸ δεύ[τ]ερον ὠς πόθε[\\
]ερον ἦλθε.

\end{gkverse}

\begin{verse}
\ldots{} [tu] venturosa \ldots{}\\ 
\ldots{}\\
Ó Cípris, e a mais amarga te descubra\\
e não se vanglorie isto contando -- ela,\\
Dórica: como a segunda vez \ldots{}\\*
\ldots{} veio.
\end{verse}

\medskip

{\paragraph{Comentário} A única fonte do Fr.\,15 é o \textit{Papiro de Oxirrinco} 1231 (século \textsc{ii} d.C.). \EP[1]
Acredita"-se que se ligaria à mesma narrativa que enreda o Fr.\,5, mas essa
leitura carece de substância historicamente comprovada, e sofre de
questionável biografismo ficcionalizante. Aqui,
parece ser feita uma prece a Afrodite, ou ``Cípris'': o \textit{eu} pede"-lhe a
punição de Dórica.\footnote{Esse nome talvez esteja presente também no Fr.\,7, se certa a emenda. O fragmento é de resto ilegível.} Segundo compreensões usuais, Safo estaria pedindo a punição
de Dórica/\,Rodopis, em benefício de Cáraxo e de si mesma. Como no caso do Fr.\,5,
todavia, além do biografismo, há os dados contraditórios dos testemunhos
antigos, em geral, de caráter francamente anedótico. Note"-se, por fim, que se
revela de forma implícita no segundo pedido uma ação própria de Afrodite e dos
deuses gregos como um todo, na sua relação com os mortais: gabar"-se,
vangloriar"-se, é atitude em potencial perigosa para um mortal que,
deixando"-se levar pela arrogância e prepotência, pode bem esquecer
os limites de sua condição e, por isso, ser punido.}

\pagebreak


\section{Fragmento 22}

\begin{gkverse}

\textnormal{[\textit{verso 1: ilegível e lacunar}]}

]εργον, ..λ’α..[\\
  ]ν ῤέθος δοκιμ̣[\\
    ]ησθαι\\
]ν̣ ἀυάδην χ.[\\
δ]ὲ μή, χείμων[\\
  ].οισαναλγεα.[\\
  ]δε\\
.]. ε̣.[\ldots{}].[\ldots{}κ]έλομαι σ.[\\
\ldots{}].γυλα.[\ldots{}]α̣νθι λάβοισα.α[\\
πᾶ]κτιν, ἆς̣ σε δηὖτε πόθος τ̣.[\\
ἀμφιπόταται

τὰν κάλαν ἀ γὰρ κατάγωγις αὔτ̣α[\\
ἐπτόαισ’ ἴδοισαν, ἔγω δὲ χαίρω,\\
καὶ γ̣ὰρ αὔτ̣α δήπο̣[τ’] ἐμεμφ[\\
Κ]υπρογέν[ηα

ὠ̣ς ἄραμα̣[ι\\
τοῦτο τῶ[\\
β]όλλομα̣[ι

\end{gkverse}


\begin{verse}
\ldots{} tarefa \ldots{}\\
\ldots{}\\ 
\ldots{}  rosto \ldots{}\\  
\ldots{}  desagradável \ldots{}\\
\ldots{}  e não, inverno \ldots{}\\
\ldots{}  dor \ldots{}\\
\ldots{}\\ 
\ldots{}  peço a ti\\
\ldots{}  [ela], após pegar \ldots{}\\
\ldots{} harpa, enquanto de novo o desejo \ldots{}\\
voa ao redor de ti --\\
a bela --; pois o vestido \ldots{}\\
vendo tremeste, e eu me alegro,\\
pois, certa vez, a própria \ldots{} \\
Ciprogênia \ldots{}\\
como rezo \ldots{}\\
isto \ldots{}\\*
quero \ldots{}\\
\end{verse}

\medskip

{\paragraph{Comentário} Nessa canção fragmentária, também preservada no \textit{Papiro de Oxirrinco}
1231, dignas de notas são a referência à ``harpa'',\footnote{[\textit{Pâ}]\textit{ktin}, verso 11.} instrumento muito
antigo e de procedência oriental, e ao voo do desejo, na imagem poética
recorrente da excitação erótica, ao redor de um \textit{tu} -- voo este que se repete
no presente da canção, como antes, no passado, revela o advérbio ``de
novo'', \textit{dēûte},\footnote{Verso 11.} a marcar, como no Fr.\,1,\footnote{Versos 15, 16, 18.} a intermitência de \textit{éros}. A sensualidade da cena se intensifica com o vestido captado pelos olhos do \textit{eu}, motivo de sua alegria e gatilho de uma recordação em que se insere Afrodite, a
``Ciprogênia'' ou ``nascida em Chipre''. São ligações tradicionais na
poesia erótica de Safo e de outros poetas os binômios beleza--desejo e
desejo--olhar. Afinal, sendo \textit{éros} o desejo sexual, é pelos olhos que entra -- os que contemplam a beleza do corpo físico. Mas o elemento do vestido e o modo como é enfocado na canção é muito característico da mélica de Safo, que amiúde realça peças do vestuário feminino. Sobre isso, um conjunto de fragmentos adiante traduzidos dirá algo mais.}


\pagebreak

\section{Fragmento 33}

\begin{gkverse}
αἴθ’ ἔγω, χρυσοστέφαν’ Ἀφρόδιτα,\\
τόνδε τὸν πάλον <......> λαχοίην
\end{gkverse}

\begin{verse}
\ldots{}se ao menos eu, ó auricoroada Afrodite,\\
este lote \ldots{} obtivesse por parte \ldots{}
\end{verse}

\medskip

{\paragraph{Comentário} Estamos uma vez mais diante de uma canção"-prece, em prol da obtenção de algo,
com o auxílio da deidade invocada, em imagem dourada. O fragmento tem por fonte
o tratado \textit{Sobre a sintaxe} (3.247), do gramático Apolônio Díscolo (século
\textsc{ii} d.C.).}

\pagebreak


\section{Fragmento 65}

\begin{gkverse}

\textnormal{[\textit{versos 1--4 e 11: ilegíveis e lacunares}]}

]Ψάπφοι, σεφίλ[\\
Κύπρωι ̣β̣[α]σίλ[\\
]κ̣αίτοι μέγα δ.[\\
ὄ]σσοις φαέθων̣ [\\
πάνται κλέος [\\
καί σ’ ἐνν Ἀχέρ[οντ 
\end{gkverse}

\begin{verse}
ó Safo \ldots{}\\
[em] Chipre a rainha \ldots{}\\
e deveras grande \ldots{}\\
para tantos [ele] brilhando \ldots{}\\
em toda parte glória \ldots{}

e a ti, no Aqueronte \ldots{}\\
\end{verse}

\medskip

{\paragraph{Comentário} Preservado no \textit{Papiro de Oxirrinco} 1787  (século \textsc{iii} d.C.), cujos textos acham"-se muito mutilados, o fragmento traz uma voz que se dirige a Safo e menciona a ilha de Chipre -- referida junto a duas outras localidades\footnote{Pafos, na ilha, e Pânormos, na Sicília.} na única linha em que consiste o Fr.\,35. Em seguida, talvez mencione sua mais importante deusa, Afrodite, no verso 2, como sua soberana. A ideia do \textit{kléos} -- a glória alcançada por grandes feitos, referida no Fr.\,44, em sua narrativa mítica -- se sobressai e pode estar vinculada à imortalização pela poesia pela fama, talvez o tema do último fragmento traduzido nesta antologia que congrega todas as canções de Safo, exceto pelos ilegíveis. Nesse sentido poderíamos entender a menção de um dos rios do Hades, o mundo dos mortos, no verso final, o Aqueronte, e igualmente a inserção da poeta como personagem dos versos por meio de seu próprio nome, na abertura que resta.}

\pagebreak

\section{Fragmento 73\,\textsc{a}} 

\begin{gkverse}
\textnormal{[\textit{versos 1--2: ilegíveis e lacunares}]}

]αν Ἀφροδι[τα\\
          ἀ]δύλογοι δ’ ἐρ[\\
        ]β̣αλλοι\\
      α]ις̣ ἔχοισα\\
     ].έ̣να θαασ[ς\\
     ]άλλει\\
     ]ας ἐέρσας [

\end{gkverse}

\begin{verse}
\ldots{} ó Afrodite\\
de doce fala \ldots{}\\
\ldots{} lançaria \ldots{}\\
\ldots{} ela tendo\\
\ldots{}\\
\ldots{} orvalho \ldots{}\\
\end{verse}

\medskip

{\paragraph{Comentário} Preservado na mesma fonte do Fr.\,65, o 73\,\textsc{a} mostra a \textit{persona} a se dirigir a Afrodite no primeiro verso minimamente legível,\footnote{Verso 3.} mas nada podemos saber da cena desenhada. O adjetivo \textit{adýlogoi}, ``de doce fala'', no verso 4, talvez qualifique o termo \textit{éros} na forma plural, \textit{érōtes}, que, todavia, não se atesta em Safo.}

\pagebreak

\section{Fragmento 86}

\begin{gkverse}
\textnormal{[\textit{verso 1: ilegível e lacunar}]}

] α̣ἰ̣γιόχω λ̣α̣[\\
]. Κ̣υ̣θέρη’ εὐχομ[\\
  ]ο̣ν ἔχοισα θῦμο̣[ν\\
κλ]ῦθί μ̣’ ἄρας αἴ π[οτα κἀτέρωτα\\
    ]ας π̣ρ̣ολίποισ̣α κ[\\
    ]. πεδ’ ἔμαν ἰώ[\\
        ]ν χα̣λέπαι.[

\end{gkverse}

\begin{verse}
\ldots{} porta-égide \ldots{}\\
\ldots{} ó Citereia, eu rezo \ldots{}\\
\ldots{} [tu], tendo \ldots{} coração \ldots{}\\
\ldots{} ouve-me as preces, se já outrora\\
\ldots{} [tu] tendo deixado \ldots{}\\
\ldots{} para minha \ldots{}\\
\ldots{} pesares \ldots{}
\end{verse}

\medskip

{\paragraph{Comentário} Tendo por fonte a mesma do fragmento anterior, o 86 de novo revela a \textit{persona} a falar a Afrodite, nomeada Citereia -- como no Fr.\,140 e em boa parte da poesia grega antiga, pela sua ligação mítico"-cultual com a ilha de Citera --, e em linguagem muito similar à do Fr.\,1, da segunda estrofe.\footnote{Versos 5--8.} É certo que é uma prece, possivelmente um hino clético à deusa, tal qual os Frs. 1 e 2, que igualmente refere Zeus, a quem o epíteto ``porta"-égide'', \textit{aigiókhō},\footnote{Verso 2.} costuma ser atribuído.}

\pagebreak

\section{Fragmento 101}

\begin{gkverse}
χερρόμακτρα δὲ \dagger{}καγγόνων\dagger{}\\
πορφύραι \dagger{}καταυταμενἀ-\\
τατιμάσεις\dagger{} ἔπεμψ’ ἀπὺ Φωκάας\\
δῶρα τίμια \dagger{}καγγόνων\dagger{}\\
\end{gkverse}

\begin{verse}
\ldots{} e panos de mão \ldots{}\\
purpúreas \ldots{}\\
\ldots{} enviou da Fócea\\
dons preciosos \ldots{}
\end{verse}

\medskip

{\paragraph{Comentário} Citado no tratado \textit{Sobre o estilo} (140\,\textsc{ss}.), de Demétrio (século \textsc{iii} a.C. ou \textsc{i} d.C.), os versos estariam no Livro 5 da compilação da obra de Safo na Biblioteca de Alexandria, e seriam, segundo ele, ditos pela poeta a Afrodite. Vestes e dons, tão importantes em Safo e marcados na deusa no Fr.\,1, estão em cena, mas não podemos recuperar seu sentido. Demétrio explica que o primeiro dos elementos destina"-se a enfeitar a cabeça; os versos estão severamente corrompidos; traduzo os termos compreensíveis.}

\pagebreak

\section{Fragmento 102}

\begin{gkverse}
Γλύκηα μᾶτερ, οὔ τοι δύναμαι κρέκην τὸν ἴστον\\
πόθωι δάμεισα παῖδος βραδίναν δι’ Ἀφροδίταν
\end{gkverse}

\begin{verse}
Ó doce mãe, não posso mais tecer a trama -- \\*
domada pelo desejo de um menino, graças à esguia \mbox{Afrodite \ldots{}}
\end{verse}

\medskip

{\paragraph{Comentário} O \textit{eu} feminino desse fragmento, cuja fonte principal é o \textit{Inquérito
sobre os metros} (10.5), de Heféstion (século \textsc{ii} d.C.), dirige uma queixa à sua
``doce mãe'', \textit{glýkēa mâter}:\footnote{Verso 1.} domada pela paixão por um menino, por ação de Afrodite, acha"-se impotente, incapaz de prosseguir com o trabalho no tear. Na dupla de
versos, compõe"-se um cenário erótico muito apropriado a Afrodite, tramado no
entrelaçamento da ação de domar, do desejo, da intervenção da deusa
-- a ``tecelã de ardis'', \textit{dolóploke},\footnote{Verso 2.} no Fr.\,1 --, e do tecer de tramas, tarefa feminina, que, metaforicamente, alude à ação de enganar, própria da sedução.
Vale reparar ainda em como o fragmento ecoa a canção popular, da qual
temos vestígios, tanto na referência ao trabalho, quanto no tema -- a queixa da
menina doente de paixão. O \textit{eu} do fragmento é, pois, dramático, como o é
na poesia grega antiga. Cabe atentar para a figura da mãe: na
casa, espaço das mulheres, a ela cabia supervisionar os trabalhos e os deveres
domésticos da família e de seus servos. E para o contraste entre a violenta
ação de domar, executada por Afrodite, e sua imagem física delicada.}

\pagebreak

\section{Fragmento 112}

\begin{gkverse}
Ὄλβιε γάμβρε, σοι μὲν δὴ γάμος ὠς ἄραο\\
ἐκτετέλεστ’, ἔχηις δὲ πάρθενον, ἂν ἄραο.\\
σοὶ χάριεν μὲν εἶδος, ὄππατα <δ’ \ldots{}>\\
μέλλιχ’, ἔρος δ’ ἐπ’ ἰμέρτωι κέχυται προσώπωι\\
<...> τετίμακ’ ἔξοχά σ’ Ἀφροδίτα
\end{gkverse}

\begin{verse}
Ó feliz noivo, tua boda, como pediste,\\
se cumpriu, e tens a virgem que pediste.\\
Tua forma é graciosa, e \ldots{} olhos de\\
mel, e desejo se derrama na desejável face\\*
\ldots{} honra-te em especial Afrodite \ldots{}
\end{verse}

\medskip

\paragraph{Comentário} Esse fragmento é de uma canção de casamento ou epitalâmio, como outros no item a essa espécie mélica dedicado nesta antologia, a ser entoado no decorrer dos eventos da cerimônia.
Da importância do epitalâmio no \textit{corpus} de Safo e, mais, no universo em que a poeta e seu coro de \textit{parthénoi} se encontram, falei na introdução e na nota a esta segunda edição deste trabalho, lembrando, entre outras evidências que se somam às próprias canções de Safo, a existência de um livro só de mélica epitalâmica na compilação de sua obra  em Alexandria, na famosa Biblioteca. Recordo ainda testemunhos antigos como o de Himério, retórico do século \textsc{iv} d.C., que diz na \textit{Oração 9},\footnote{33--47.} ou \textit{Epitalâmio para Severo}, seu aluno: ``[\ldots{}] os ritos de Afrodite foram deixados a Safo de Lesbos, para que os cantasse para a lira e os fizesse para o tálamo. Após os concursos, ela seguia para o tálamo, ornamentava o aposento nupcial, preparava o leito, recrutava as virgens, liderava"-as ao quarto da noiva, e também Afrodite sobre o carro das Cárites -- deusas Graças --, e o coro de Erotes, companheiro de folguedos''.

No Fr.\,112, canta"-se a felicidade do noivo qualificado por ela, \textit{ólbios};\footnote{Verso 1.} trata"-se de um exemplo de \textit{makarismós}, a tradicional ``enunciação do casal como feliz e abençoado'',\footnote{Swift, 2010, p. 246.} que vem da tradição popular, tal qual o elogio dos noivos, também feito nos versos.
Vale a pena recordar que a fama dos epitalâmios sáficos parece ter
sido ampla e duradoura; afinal, tardiamente, o sofista Corício de Gaza (século
\textsc{vi} d.C.) dizia, na obra \textit{Epitalâmios em Zacarias} (19): ``Portanto, eu
-- para que a ti de novo agrade -- com um canto sáfico adornarei a noiva\ldots{}''.
Ou seja, não apenas as palavras do texto de um epitalâmio adornam os noivos,
mas a própria canção, uma vez que sua função é louvar a aparência física e
sedutora deles, de modo a estimulá"-los ao enlace sexual, que concretiza a
boda. O epitalâmio integra, assim, outros recursos voltados a essa mesma finalidade,
como o banho ritual. A presença de Afrodite, deusa da beleza e do sexo,
justifica"-se plenamente, portanto. As fontes do fragmento são as já indicadas
obras de Heféstion, referido no Fr.\,102, e de Corício.
Seus versos mostram que os epitalâmios sáficos não são exclusivamente femininos, pois ``o casamento era uma etapa importante de transição na vida do homem, como na da mulher''.\footnote{Swift, 2010, p. 249.} Porém, se a vida masculina se consolidava na esfera política e bélica, na feminina  ``a boda e a geração de prole era o foco final, o objetivo'',\footnote{Swift, p. 250.} razão pela qual a noiva tem proeminência sobre o noivo nas festividades, algo que se reflete nas canções.

\pagebreak

\section{Fragmento 133}

\begin{gkverse}
Ἔχει μὲν Ἀνδρομέδα κάλαν ἀμοίβαν

\hspace*{21mm}\adforn{47}

Ψάπφοι, τί τὰν πολύολβον Ἀφροδίταν\ldots;
\end{gkverse}

\begin{verse}
``Tem Andrômeda bela paga \ldots{}''

\hspace*{21mm}\adforn{47}

``Ó Safo, por que a multiafortunada Afrodite \ldots{}?''
\end{verse}

\medskip

{\paragraph{Comentário} O Fr.\,133 é um canto coral dialogado, similarmente ao 140, à frente. No verso 1, uma voz -- ou um coro -- faz uma constatação acerca de ``Andrômeda''; no seguinte,
lança uma pergunta a ``Safo'', \textit{persona }da poeta no fragmento de
uma canção de evidente caráter dramático. De acordo com leituras correntes,
``Andrômeda'' seria a poeta"-líder de um grupo de
meninas rival ao de Safo. Esse nome aparece também no Fr.\,130, em contexto
erótico que envolve outra personagem feminina, ``Átis'', referida em
outros fragmentos de Safo. Mas a fragilidade de nosso conhecimento sobre Lesbos
e sua sociedade precisa ser admitida: nada resta de Andrômeda, exceto seu nome
na mélica sáfica materialmente frágil. O fragmento é de difícil leitura:
fala"-se em punição ou benefício? E o que encobre a lacuna do verso 2, em que
Afrodite é dita -- como o são os demais deuses gregos -- venturosa, afortunada, 
feliz (\textit{polýolbon})?\footnote{Verso 2.} Sua fonte de transmissão é o tratado do metricista Heféstion (14.7), referido
no Fr.\,102.}

\pagebreak

\section{Fragmento 134}

\begin{gkverse}
Ζὰ <.> ἐλεξάμαν ὄναρ Κυπρογενηα
\end{gkverse}

\begin{verse}
Em sonho falei à Ciprogênia \ldots{}
\end{verse}

\medskip

{\paragraph{Comentário} O sonho, como percebido pelos antigos, surge como uma das formas de 
comunicação entre deuses e homens. O fragmento foi preservado em Heféstion (12.4), já
referido no Fr.\,102. Nele, projeta"-se a intimidade da \textit{persona} de Safo com Afrodite, pois não é a deusa que lhe fala, como tradicionalmente, mas a poeta que a ela se dirigiu no passado do sonho sonhado, que parece recontar a alguém.}


\pagebreak
\section{Afrodite e Adônis, paixão e morte (Fr.\,140)}

\begin{gkverse}
Κατθνάσκει, Κυθέρη’, ἄβρος Ἄδωνις τί κε θεῖμεν;

καττύπτεσθε, κόραι, καὶ κατερείκεσθε χίθωνας.
\end{gkverse}

\begin{verse}
``Morre, Citereia, delicado Adônis. Que podemos fazer?''

``Golpeai, ó virgens, vossos seios, e lacerai vossas vestes \ldots{}''
\end{verse}

\medskip

\paragraph{Comentário} Segundo o viajante grego do século \textsc{ii} d.C., Pausânias, Safo ``sobre
Adônis cantou'';\footnote{\textit{Descrição da Grécia}, \textsc{ix}, 29, 8.} prova isso o Fr.\,140, 
centrado no contexto mítico da paixão de Afrodite por ele, e da sua morte.
O belo jovem, associado estreitamente ao universo sacroerótico da
flora, das plantas, dos arômatas, dos jardins -- muito caro à deusa --,
tornado alvo da paixão dela, acaba morto violentamente no auge de sua
virilidade; na versão mais retomada, matou"-o um javali, fato que deve se ligar
à interdição do porco nos rituais a Afrodite, em muitos cultos gregos. Morto
Adônis, a deusa sofre. Esse mito se acha nas tradições poéticas, iconográficas
e religiosas, o que mostra a influência oriental em terras gregas, uma vez que
a figura de Adônis é fenícia, em termos de sua origem -- e Afrodite associa"-se
estreitamente ao Oriente. Ora, repare"-se que a deidade do Fr.\,140 é chamada
``Citereia'', em referência ao seu culto proeminente na ilha de Citera,
de forte influência fenícia.
Não é demais ressaltar que é também oriental o
mito da deusa do sexo arrebatada por um jovem deus precocemente morto. O
cenário mítico está plasmado no fragmento em registro fúnebre e dramático por um coro feminino, a morte de Adônis é afirmada, e a isso
se segue uma indagação; a resposta vem em seguida, dita pela própria Afrodite.

\pagebreak
Tal resposta, imperativa, demanda das \textit{kórai}, ``meninas'' ou ``virgens'', a gestualidade fúnebre tradicional do lamento feminino, chamada \textit{góos}), atestada na iconografia e na \textit{Ilíada} \textsc{xviii},\footnote{30--1, 50--1.} no pranto das servas e das Nereidas pela morte de Pátroclo.
Do ponto de vista das práticas cultuais relacionadas à morte de Adônis, o festival das Adonias  
era celebrado no início do verão, por mulheres adultas. Essa festa se atesta 
desde o século \textsc{vi} a.C., em várias partes do
mundo grego, e ainda é feita no Egito do \textsc{iii} a.C., agora de modo público e
oficial. Durante as Adonias, as mulheres pranteavam Adônis, representando o
sofrimento de Afrodite, exposto na canção sáfica. A mais antiga
menção ao mito, ao culto e ao rito fúnebre dá"-se no Fr.\,140, cuja fonte é
Heféstion (10.4), a mesma do Fr.\,102. E como Herington\footnote{1985, p. 57.} afirma, a estrutura dramática do Fr.\,140,  como a do 114, ``antecipa'' cenas semelhantemente trabalhadas na tragédia ática: ``Somos colocados num evento mítico [\ldots{}] que não está sendo narrado, mas imediatamente apresentado''; logo,  temos que creditar a Safo, e a mais ninguém, o mais antigo fragmento de drama em verso, preservado em toda a tradição europeia [\ldots{}]''. Canto dialogado coral fúnebre, o fragmento anuncia a própria \textit{performance} pública -- possivelmente ``uma \textit{reperformance} da morte em vida de Adônis por um grupo de suas adoradoras''.\footnote{Herington.} Diga"-se, por fim: ``Um dos gêneros de discurso público associados à mulher tanto na Grécia arcaica, quanto na rural de hoje, é o lamento'',\footnote{Lardinois 1996, p. 172.} justamente o gênero do Fr.\,140. Em conformidade com a lógica
paidêutica do grupo coral liderada por Safo, a dramatização da dor lutuosa de Afrodite que ensina as moças a prantear a morte de Adônis permiti"-lhes vivenciar no mito a experiência fúnebre, parte da vida cotidiana. Talvez a presença das meninas, em vez das mulheres -- como nos cultos das Adonias --, se deva à dimensão da \textit{paideía} das jovens do coro que, desse modo, ``lamenta com ou como Afrodite a perda de Adônis''.\footnote{Stehle 1997, p. 224.}

\section{Fragmento 168}

\begin{gkverse}
ὦ τὸν Ἄδωνιν
\end{gkverse}

\begin{verse}
Ó por Adônis!
\end{verse}

\medskip

{\paragraph{Comentário} Preservado por Mário Plotino Sacerdo (século \textsc{iii} d.C.), em sua
\textit{Gramática} (\textsc{iii}. 3), e atribuído a Safo pela presença de Adônis em
sua mélica -- o anterior Fr.\,140 --, as palavras parecem lamentosas e
talvez também pranteassem a morte do belo amado de Afrodite, lá cantado. Na canção perdida, de que temos só a sequência acima, a deusa poderia estar presente junto a ele.}


\pagebreak
\section{Fragmento 117\,\textsc{b}}

\begin{gkverse}
 Ἔσπερ’ ὐμήναον

\hspace*{10mm}\adforn{47}

ὦ τὸν Ἀδώνιον
\end{gkverse}

\begin{verse}
\ldots{} Ó Vésper! Ó Himeneu!

\hspace*{10mm}\adforn{47}

\ldots{} Ó pelo Adônio!
\end{verse}

\medskip

{\paragraph{Comentário} Preservados pela mesma fonte do Fr.\,168, como exemplares de métrica de
himeneus, as canções de casamento -- adiante as veremos -- que na
Biblioteca de Alexandria passaram a ser chamadas pelo termo epitalâmio,\footnote{Literalmente, ``sobre o leito nupcial, o tálamo''.} as linhas não contíguas (\textbf{a}, \textbf{b}) viriam de uma composição talvez enlaçada ao mito de Adônis, de que falei no
comentário ao Fr.\,140, que canta sua morte e o luto de Afrodite, da qual foi o noivo em boda não consumada, segundo as tradições. Note"-se
a presença de Vésper, a Estrela da Tarde, que veremos na canção
epitalâmica Fr.\,104\textsc{a}. Não necessariamente é lamentosa a segunda linha
(\textbf{b}); em vista do contexto epitalâmico que traz o deus da boda, do
enlace sexual que a sela -- ver o comentário ao Fr.\,111, entre os
epitalâmios --, pode ser celebratória à beleza do noivo que talvez projete, maximizando"-a. No entendimento da edição Voigt,\footnote{De 1971.} as duas linhas de algum modo integrariam o Fr.\,117, epitalâmio de verso único de saudações a noivo e noiva, como se verá na sua tradução, no item dedicado às canções desse tipo.}




\chapter{Eros}

\paragraph{\textsc{nota introdutória}} 
É forte em Safo a temática erótica, mas não tem a mesma presença marcante que
Afrodite o deus Eros, que lhe será sempre subordinado, como filho ou servo, na
tradição mítico"-poética, iconográfica e cultual da Grécia antiga, como se vê no
Fr.\,159, abaixo. A relevância do universo feminino certamente favorece a
proeminência de Afrodite sobre Eros. Na imagem sáfica do deus, projeta"-se a
concepção grega sobre o desejo, a paixão erótica: prazerosa e doce, mas
sobretudo violenta, acre, dolorosa, doença de corpo e mente, força dominadora e
intermitente.

\section{Fragmento 47}

\begin{gkverse}
\vinphantom{φρένας, ὠς ἄνεμος κὰτ}Ἔρος δ’ ἐτίναξέ <μοι>\\
φρένας, ὠς ἄνεμος κὰτ ὄρος δρύσιν ἐμπέτων.
\end{gkverse}

\begin{verse}
\ldots{} Eros sacudiu meus\\*
sensos, qual vento montanha abaixo caindo sobre as \mbox{árvores \ldots{}}
\end{verse}

\medskip

\paragraph{Comentário} A fonte do fragmento é uma \textit{Oração} (18), do retórico Máximo de Tiro (século \textsc{ii} d.C.). \EP[2]
Note"-se como a imagem da tempestade ou ventania violenta é usada para falar de Eros a se abater sobre a \textit{persona}, comprometendo sua estabilidade mental, na medida em que lhe atingi os ``sensos'', \textit{phrénas}.\footnote{Verso 2.} Veremos sobretudo no Fr.\,31 a síntese do sofrimento erótico, mas já no Fr.\,1\footnote{Versos 3--4.} se faz presente.



\pagebreak
\section{Fragmento 54}

\begin{gkverse}
ἔλθοντ’ ἐξ ὀράνω πορφυρίαν περθέμενον χλάμυν
\end{gkverse}

\begin{verse}
\ldots{} vindo do céu, envolto em purpúreo manto \ldots{} 
\end{verse}

\medskip

{\paragraph{Comentário} Segundo a fonte, Pólux,\footnote{\textit{Onomástico} 10.124, século \textsc{ii} d.C.} Safo descreve Eros quando canta o verso acima.}

\section{Fragmento 130}

\begin{gkverse}
Ἔρος δηὖτέ μ’ ὀ λυσιμέλης δόνει,\\
			γλυκύπικρον ἀμάχανον ὄρπετον\\

\hspace*{22mm}\adforn{47}

			Ἄτθι, σοὶ δ’ ἔμεθεν μὲν ἀπήχθετο\\
φροντίσδην, ἐπὶ δ’ Ἀνδρομέδαν πότη<ι>

\end{gkverse}

\begin{verse}
\ldots{} Eros de novo -- deslassa-membros -- me agita,\\*
dulciamara inelutável criatura \ldots{}

\hspace*{22mm}\adforn{47}

\ldots{} ó Átis, mas a ti tornou-se odioso meu\\*
pensamento, e para Andrômeda alças voo \ldots{}
\end{verse}

\pagebreak
{\paragraph{Comentário} O primeiro par de versos sintetiza os elementos que marcam a imagem poética de
\textit{éros}, o deus ou a força. O segundo par traz duas figuras femininas
outras vezes mencionadas nas canções de Safo,\footnote{Frs.\,133, 49, 96.} entre
as quais se insere, em chave de conflito, a 1ª pessoa do singular, cuja
identidade nos escapa. O referido conflito pode ter caráter
erótico, como indica a linguagem dos dois primeiros versos. E se Andrômeda é mesmo poeta rival de Safo, esta pode estar a censurar Átis,\footnote{Além dos fragmentos que veremos aqui, o nome é a única palavra legível no Fr.\,8.} nome de uma das virgens do coro outras vezes presente nas canções, por ter saído de seu grupo para passar ao daquela líder. O fragmento está preservado no tratado do metricista Heféstion (7.7), já
referido no Fr.\,102.
Abaixo, na caracterização de Eros, merecem destaques os adjetivos  compostos \textit{lysimelés} -- que Hesíodo já lhe dá na \textit{Teogonia} (121) --, que marca a fragmentação do sujeito, e \textit{glýkypikron} -- primeiro doce, depois, amargo --, o adjetivo \textit{amákhanon}, que marca a impotência em que se acha o amador diante de deus; e \textit{órpeton}, que nomeia criaturas rastejantes, arrepiantes, como as serpentes que paralisam de temor o sujeito que as vê.}


\pagebreak
\section{Fragmento 159}

\begin{gkverse}
σύ τε κἆμος θεράπων Ἔρος
\end{gkverse}

\begin{verse}
``\ldots{} tu e meu servo, Eros \ldots{}''
\end{verse}

\medskip

{\paragraph{Comentário} Máximo de Tiro, fonte do fragmento 47 e deste, afirma que, numa canção da poeta,
Afrodite fala a Safo, estabelecendo a hierarquia corrente entre si e o deus que
menciona.}

\section{Fragmento 172}

\begin{gkverse}
ἀλγεσίδωρος
\end{gkverse}

\begin{verse}
\ldots{} doa-dores \ldots{} 
\end{verse}

\medskip

{\paragraph{Comentário} A mesma fonte dos Frs.~47 e 159 preservou como sendo atribuída por Safo a Eros -- o deus ou o desejo -- a seguinte qualificação, coerente com sua imagem.}


\section{Fragmento 188}

\begin{gkverse}
μυθόπλοκος
\end{gkverse}

\begin{verse}
\ldots{} tecelão de palavras \ldots{}
\end{verse}

\medskip

{\paragraph{Comentário} Citado por Máximo de Tiro, fonte do Fr.\,47, o 188 traz apenas uma palavra, \textit{mythoplókon}, que teria sido dita sobre Eros por Safo, diz o antigo retórico. O adjetivo pode se referir à eloquência da sedução, potencialmente enganadora, como indicaria o tecer que no mundo erótico tem tal conotação, como vimos no Fr.\,1 e no epíteto \textit{dolóploke} para Afrodite, a ``tecelã de ardis''.\footnote{Verso 2.}}


\chapter{Ártemis}

\section{Fragmento 44\,\textsc{a}}

\begin{gkverse}
\textnormal{[\textit{verso 1: ilegível e lacunar}]}

Φοίβωι χρυσοκό]μαι τὸν ἔτικτε Κόω .[\\
μίγεισ(α)         Κρ]ονίδαι μεγαλωνύμω̣<ι>\\
Ἄρτεμις δὲ θέων] μέγαν ὄρκον ἀπώμοσε\\
    κεφά]λαν ἄϊ πάρθενος ἔσσομαι\\
             ].ων ὀρέων κορύφα̣ι̣σ’ ἔπι\\
    ]δ̣ε νεῦσον ἔμαν χάριν\\
         ένευ]σ̣ε θέων μακάρων πάτηρ\\
ἐλαφάβ]ολον ἀγροτέραν θέο̣ι\\
            ].σιν ἐπωνύμιον μέγα\\
   ]ερος οὐδάμα πίλναται\\
     ].[.]...μα̣φόβε[..]έ̣ρω

\textnormal{[\textit{versos 1--4 e 10: ilegíveis e lacunares}]}

Μοισαν ἀγλα[\\
πόει καὶ Χαρίτων̣[\\
βραδίνοις ἐπεβ.[\\
ὄργας μὴ ’πιλάθε.[\\
θ̣ν̣άτοισιν πεδ.́χ[\\
      ]δ̣αλίω[
\end{gkverse}
\pagebreak
\begin{verse}
\ldots{} a Febo auricomado, a quem [\ldots{}] gerou de \rlap{Coios \ldots{}}\\
\ldots{} unida ao Cronida de grande nome.\\
\ldots{} mas Ártemis jurou a grande jura dos deuses:\\
``\ldots{} para sempre virgem serei\\
\ldots{} sobre os cimos das montanhas\\
\ldots{} concede-me este favor''.\\
\ldots{} concedeu-o o pai dos deuses venturosos\\
\ldots{} flecha-cervos, a selvagem, os deuses\\
\ldots{} grande título\\*
[Eros] dela nunca se achega \ldots{}

\hspace*{16mm}\adforn{47}

\ldots{} Musas esplêndidas \ldots{}\\
faz\ldots{} das Cárites \ldots{}\\
esguios \ldots{}\\
a cólera não esquecer \ldots{}\\
mortais \ldots{}
\end{verse}

\medskip

{\paragraph{Comentário} Apenas neste fragmento, e no de resto ilegível Fr.\,84, cuja fonte é o \textit{Papiro Fouad} 239 (séculos \textsc{ii} ou
\textsc{iii} d.C.), surge Ártemis em seu desenho tradicional de eterna virgem caçadora,
irmã de Apolo, filha de Leto, que bosques percorre, cercada de animais. A
virgindade, desejada pela deusa, concedeu"-a Zeus à filha, conta o \textit{Hino homérico a Afrodite} (século \textsc{iv} a.C.), de autoria incerta.}




\chapter{As Cárites ou «\,Graças\,»}

\paragraph{\textsc{nota introdutória}}
Quase sempre nomeadas coletivamente, as deusas integram o séquito de Afrodite.
Esse elo reflete afinidades sobretudo no universo erótico, uma vez que as
deusas, cujo nome é a forma do plural de \textit{kháris }-- conceito que vai de
``graça, charme, regozijo, prazer'', no sentido físico, a ``favor dos deuses'', no
âmbito da reciprocidade das relações --, favorecem a beleza e a sedução na
esfera do sexo, a festa e a alegria na da vida cotidiana, o vigor, o
crescimento e a renovação, nas esferas humana e vegetal. São as Cárites
filhas de Zeus e integrantes da ordem olímpica, indica o Fr.\,53, em que
recebem um epíteto dado por Safo também a Eos, a Aurora, na ``Canção sobre a velhice'';
tal epíteto, ao evocar a rosa, flor predileta de Afrodite, joga tintas eróticas
sobre o desenho das deusas. E são as Cárites, segundo a \textit{Teogonia},\footnote{Versos 64--67.} de Hesíodo, vizinhas das Musas e de Hímeros (Desejo) no Olimpo -- imagem refletida na presença daquelas deusas e do erotismo no Fr.\,128. Deste -- cuja fonte é Heféstion (9.2), como no caso do Fr.\,102 -- e do
Fr.\,53 -- cuja fonte é um comentário antigo ao \textit{Idílio }28, do poeta
Teócrito (séculos \textsc{iv}--\textsc{iii} a.C.) --, temos os respectivos versos inaugurais de
\textit{hinos cléticos}.

\pagebreak
\section{Fragmento 53} 

\begin{gkverse}
Βροδοπάχεες ἄγναι Χάριτες δεῦτε Δίος κόραι
\end{gkverse}

\begin{verse}
Para cá, ó sacras Cárites de róseos braços, meninas de Zeus \ldots{}
\end{verse}\bigskip


\section{Fragmento 128} 

\begin{gkverse}
Δεῦτέ νυν ἄβραι Χάριτες καλλίκομοί τε Μοῖσαι
\end{gkverse}

\begin{verse}
Para cá, vós, delicadas Cárites e Musas de belas comas, \ldots{}
\end{verse}


\chapter{Eos, a Aurora}

\section{Fragmento 123}

\begin{gkverse}
ἀρτίως μὲν ἀ χρυσοπέδιλος Αὔως
\end{gkverse}

\begin{verse}
\ldots{} recentemente Eos, de áurea sandália, \ldots{}
\end{verse}

\medskip

{\paragraph{Comentário} O destaque ao adorno dos pés enfatiza o erotismo da imagem; pés, braços,
cabelos, colo/\,seio e olhos são do corpo feminino as partes mais enfocadas na
linguagem erotizante da poesia grega. Isso se intensifica na menção ao ouro;
afinal, só uma deusa é dita simplesmente ``áurea'' nos poemas homéricos
e na maior parte da poesia antiga: Afrodite. O fragmento tem por fonte o
tratado \textit{Sobre palavras similares, mas diferentes} (75), do gramático Amônio
(século \textsc{ii} d.C.).
Nele, como no Fr.\,157, em que lemos apenas ``soberana Eos'', e na ``Canção sobre a velhice'', a deusa da Aurora se faz presente.}



\chapter{Hera}

\section{Fragmento 17}

\begin{gkverse}
Πλάσιον δη μ[\\
πότνι’ Ἦρα σὰ χ[\\
τὰν ἀράταν Ἀτ[ρέιδαι\qquad        κλῆ-]\\
τοι βασίληες\\

ἐκτελέσσαντες μ[\\
πρῶτα μὲν περι̣.[\\
τυίδ’ ἀπορμάθεν[τες\\
οὐκ ἐδύναντο

πρὶν σὲ καὶ Δί\,’ ἀντ[\\
καὶ Θυώνας ἰμε̣[\\
νῦν δὲ κ[\\
κὰτ τὸ πάλ̣[

ἄγνα καὶ κα̣[\\
π]αρθ[εν\\
ἀ]μφι.[\\

\textnormal{[\textit{versos 16--8: ilegíveis e lacunares}]}

ἔμμενα̣[ι\\
{[}?{]}ρ̣ ̣ἀπίκε[σθαι.
\end{gkverse}

\chapter*{}
\section*{}

\begin{verse}
Perto, cá \ldots{}\\
veneranda Hera, teu \ldots{}\\
a prece \ldots{} os Atridas,\\*
os reis,

e perfizeram \ldots{}\\
primeiro em redor \ldots{}\\
para cá tendo partido \ldots{}\\*
não conseguiam,

e, antes de a ti, Zeus dos suplicantes \ldots{}\\
e ao adorável \ldots{} filho de Tione;\\
mas agora \ldots{}\\*
tal qual no passado,

sacro e \ldots{}\\
de moças \ldots{}\\
em torno \ldots{}\\*
\ldots{}

ser \ldots{}\\*
Hera \ldots{}, vir.
\end{verse}

\pagebreak
{\paragraph{Comentário} O fragmento, uma prece a Hera, é bastante precário, e suas principais fontes são o \textit{Papiro de Oxirrinco} 1231, rolo em que também se preservaram os Frs.~15 e 22, e o \textit{Papiro de Milão} \textsc{ii} 123. São nomeados Hera,\footnote{Versos 2 e talvez 20.} Zeus e Dioniso -- ``filho de Tione'',\footnote{Verso 10.} epíteto de Sêmele, a princesa tebana com quem Zeus o gerou e que foi imortalizada, após a morte pela visão do deus em seu terrível brilho e esplendor. Essa junção faz pensar no santuário das três divindades em Messa, em Lesbos. A referência aos Atridas, Agamêmenon e Menelau, por sua vez, traz à prece no presente (11) o mundo do passado mítico de Troia; aos Atridas remontaria a fundação do culto lésbio à deusa invocada na canção, quando de seu regresso de Troia a Argos. E as referências a algo sagrado e a moças virgens
fazem pensar em algo relativo ao culto no contexto de um exílio, considerando que talvez Zeus receba no verso 9 o epíteto cultual \textit{antaîos}, ``dos suplicantes'', que Alceu, o poeta lésbio contemporâneo a Safo, e por duas vezes exilado, confere ao deus no Fr.\,129; e no templo de Hera ele vê um concurso de beleza no Fr.\,130\,\textsc{b}. Diga"-se, por fim, que o nome da deusa\footnote{Verso 4.} e ``festividade'', \textit{eórtan},\footnote{Verso 3.} são as únicas palavras legíveis do Fr.\,9, indicativas de culto a Hera, como um pouco melhor vemos neste Fr.\,17.}



\chapter{Musas}

\section{Fragmento 32}

\begin{gkverse}
αἴ με τιμίαν ἐπόησαν ἔργα\\
τὰ σφὰ δοῖσαι
\end{gkverse}

\begin{verse}
\ldots{} elas [as Musas] me fizeram honrada, suas próprias\\
obras doando-me \ldots{}
\end{verse}

\medskip

{\paragraph{Comentário} A fonte do fragmento é \textit{Sobre os pronomes} (144\textsc{a}), de Apolônio
Díscolo, gramático que, noutro tratado, preservou também o já visto Fr.\,33. O
gênero feminino do \textit{eu} e a menção de seus \textit{érga }, ``trabalhos'', como
``dons'' que a tornam honrada levam a pensar que o sujeito feminino
coletivo no pronome \textit{aí}\footnote{Verso 1.} deve ser divino -- deve ser o conjunto das Musas. Logo, tais dons
seriam o da poesia, no entendimento mais comum da expressão.}


\section{Fragmento 124}

\begin{gkverse}
αὔτα δὲ σὺ Καλλιόπα
\end{gkverse}

\begin{verse}
\ldots{} e tu mesma, ó Calíope, \ldots{}
\end{verse}

\medskip

{\paragraph{Comentário} Preservado em Heféstion, como o 102, o fragmento nomeia a líder das Musas, ``Bela Voz'', como a retrata Hesíodo no catálogo de seus nomes.\footnote{\textit{Teogonia}, versos 77--80.}}



\pagebreak
\section{Fragmento 127}

\begin{gkverse}
Δεῦρο δηὖτε Μοῖσαι χρύσιον λίποισαι \ldots{}
\end{gkverse}

\begin{verse}
Para cá, de novo, ó Musas, deixando a áurea [casa de Zeus] \ldots{}
\end{verse}

\medskip

{\paragraph{Comentário} O fragmento, citado em Heféstion (15.25), como o 102, consiste, pode"-se imaginar, no
verso inaugural de um \textit{hino clético }às Musas, invocadas a deixarem,
talvez, a casa de Zeus, no Olimpo, regularmente ``áurea''
na poesia grega antiga. O chamamento, repetido no presente da prece, deve ter
por pedido algo relativo a essas deusas, como a habilidade poética.}

\pagebreak
\section{Fragmento 150}

\begin{gkverse}
οὐ γὰρ θέμις ἐν μοισοπόλων <δόμωι>\\
θρῆνον ἔμμεν’ <...> οὔ κ’ ἄμμι πρέποι τάδε
\end{gkverse}

\begin{verse}
\ldots{} pois não é correto na casa dos servos das Musas\\*
haver o treno; \ldots{} isso não nos seria adequado \ldots{}
\end{verse}

\medskip

{\paragraph{Comentário} De acordo com a fonte, Máximo de Tiro,\footnote{\textit{Oração} 18.} que também preservou o Fr.\,47, anteriormente traduzido, ``Sócrates ficou bravo com Xantipe que se
lamentava enquanto ele morria'', em referência ao final do \textit{Fédon}, de
Platão (séculos \textsc{v--iv} a.C.), ``e Safo ficou brava com sua filha'', segundo os testemunhos, de
nome Cleis, conforme ainda se verá. A leitura é
claramente biografizante; de seguro, vemos a ênfase na adequação: na casa de quem serve
deidades luminosas como as Musas, não se admite o canto lamentoso, ou seja, o
treno, a nênia. Mas se Safo é serva dessas deusas, vale lembrar que sua poesia
escapa à restrição, pois há em seu \textit{corpus} ao menos um canto lamentoso
fúnebre, o Fr.\,140, que, portanto, problematiza o biografismo da interpretação
do fragmento cuja cena perdemos. Ademais, o canto da Musa não tem, na visão grega antiga, restrição 
do tipo que se percebe no Fr.\,150. Lembre"-se que a \textit{Ilíada},
poema aberto por uma invocação à Musa, é de caráter
profundamente trágico.}


\chapter{Deuses vários em inícios frustrados}


\section{Fragmento 103}

\begin{gkverse}
\textnormal{[\textit{verso 1: ilegível e lacunar}]}

].ατε τὰν εὔποδα νύμφαν [\\[8pt]
].τ̣α παῖδα Κρονίδα τὰν ἰόκ[ολπ]ον [\\[8pt]
].ς̣ ὄργαν θεμένα τὰν ἰόκ[ολ]πος α[\\[8pt]
 ].. ἄγναι Χάριτες Πιέριδέ\,[ς τε] Μοῖ\,[σαι\\[8pt]
     ].[. ὄ]πποτ’ ἀοιδαι φρέ̣ν[...]αν.[\\[8pt]
             ]σ̣αιοισα λιγύραν [ἀοί]δαν\\[8pt]
         γά]μβρον, ἄσαροι γ̣ὰρ̣ ὐμαλικ[\\[8pt]
  ]σε φόβαισι θεμένα λύρα .[\\[8pt]
  ]..η χρυσοπέδιλ̣[ο]ς Αὔως
\end{gkverse}

%\chapter*{}
%\section*{}
\begin{verse}
\ldots{} a noiva de belos pés \ldots{}\\[8pt]
\ldots{} a filha de colo violáceo do Cronida \ldots{}\\[8pt]
\ldots{} a raiva pondo \ldots{} a de violáceo colo \ldots{}\\[8pt]
\ldots{} sacras Cárites e Piérias Musas \ldots{}\\[8pt]
\ldots{} quando \ldots{} canções \ldots{}\\[8pt]
\ldots{} clara canção \ldots{}\\[8pt]
\ldots{} noivo, pois atrevidos coevos \ldots{}\\[8pt]
\ldots{} cachos, pondo a lira \ldots{}\\[8pt]
\ldots{} Eos de áurea sandália \ldots{}\\[8pt]
\end{verse}

\medskip

{\paragraph{Comentário} Preservado  no \textit{Papiro de Oxirrinco} 2294 (século \textsc{ii} d.C.), o fragmento traz, explica a fonte, dez versos de abertura de dez distintas canções, mencionando vários deuses: uma não identificável filha de Zeus, o filho de Crono; as Cárites; as Musas Piérias e Eos. Traduzo os mais legíveis, como é sempre o caso nesta antologia, dos quais o primeiro e o sétimo podem ser de epitalâmios, espécie mélica de que vimos o Fr.\,112, e outros ainda veremos.
Noto, quanto ao verso inicial, que traz os mesmos termos (\textit{eúpoda nýmphan}), na mesma sequência e declinação em que se acham no Fr.\,103B, no qual só se lê ainda uma palavra, ``quarto nupcial'', \textit{thalámō}.}


\chapter{Cenas míticas}

\section[A saga troiana: as bodas de Heitor e Andrômaca (Fr.\,44)]{A saga troiana: as bodas de Heitor\\e Andrômaca (Fr.\,44)\protect\footnote{\MakeUppercase{V}er
artigo \MakeUppercase{R}agusa, ``\MakeUppercase{H}eitor e \MakeUppercase{A}ndrômaca, da festa de bodas à celebração fúnebre:
imagens épicas e líricas do casal na \textit{\MakeUppercase{I}líada} e em \MakeUppercase{S}afo (\MakeUppercase{F}r.~44 Voigt)''.
\textit{\MakeUppercase{C}alíope} 15, 2006, pp. 36--63.}}

\begin{gkverse}
Κυπρο̣.[\qquad     - 22 -\qquad       ] ας̣\\
κάρυξ ἦλθε̣ θε̣[\qquad - 10 -\qquad  ]ελε̣[\ldots{}].θεις\\
Ἴδαος ταδεκα\ldots{}φ[\ldots{}].ις τάχυς ἄγγελος\\
< ``\qquad			>\\
τάς τ’ ἄλλας Ἀσίας .[.]δε.αν κλέος ἄφθιτον\\
Ἔκτωρ καὶ συνέταιρ̣[ο]ι ἄγ̣οισ’ ἐλικώπιδα\\
Θήβας ἐξ ἰέρας Πλακίας τ’ ἀπ’ [ἀϊ]ν<ν>άω\\
ἄβραν Ἀνδρομάχαν ἐνὶ ναῦσιν ἐπ’ ἄλμυρον\\
πόντον πόλλα δ’ [ἐλίγματα χρύσια κἄμματα\\
πορφύρ[α] καταΰτ[με]να, ποίκ̣ιλ’ ἀθύρματα,\\
ἀργύρα̣ τ̣’ ἀνά̣ριθ̣μα ποτήρια κἀλέφαις.''\\
ὢς εἶπ’ ὀτραλέως δ’ ἀνόρουσε πάτ[η]ρ̣ φίλος\\
φάμα δ’ ἦλθε κατὰ πτ̣όλιν εὐρύχο̣ρ̣ο̣ν φίλοις.\\
αὔτικ’ Ἰλίαδαι σατίναι[ς] ὐπ’ ἐυτρόχοις\\
ἆγον αἰμιόνοις, ἐ̣π̣[έ]βαινε δὲ παῖς ὄχλος\\
γυναίκων τ’ ἄμα παρθενίκα[ν] τ\ldots{}[.].σφύρων,\\
χῶρις δ’ αὖ Περάμοιο θυγ[α]τρες[\\
ἴππ[οις] δ’ ἄνδρες ὔπαγον ὐπ’ ἀρ̣[ματα\\
π[\qquad  ]ες ἠίθ̣εοι, μεγάλω[σ]τι δ̣[\\
δ[\qquad  ]. ἀνίοχοι φ[\ldots{}].[\\
π̣[\qquad  ]ξα.o[

\textnormal{[\textit{alguns versos perdidos}]}\\

\vinphantom{ὄ̣ρ̣ματ̣α̣ι̣ [\qquad\qquad\quad}    ἴ]κελοι θέοι[ς\\
\vinphantom{ὄ̣ρ̣ματ̣α̣ι̣ [\qquad\qquad\quad}     ] ἄγνον ἀολ[λε\\
ὄ̣ρ̣ματ̣α̣ι̣ [\qquad\qquad\quad			     ]νον ἐς Ἴλιο[ν\\
αὖλος δ’ ἀδυ[μ]έλης̣ [\qquad	    ]τ’ ὀνεμίγνυ[το\\
καὶ ψ[ό]φο[ς κροτάλ[ων\qquad    ]ως δ’ ἄρα πάρ[θενοι\\
ἄειδον μέλος ἄγν̣[ον ἴκα]νε δ’ ἐς α̣ἴ̣θ̣[ερα\\
ἄχω θεσπεσία̣ γελ̣ [\\
πάνται δ’ ἦς κὰτ ὄδο[ις\\
κράτηρες φίαλαί τ’ ὀ[...]υεδε[..]..εακ[.].[\\
μύρρα καὶ κασία λίβανός τ’ ὀνεμείχνυτο\\
γύναικες δ’ ἐλέλυσδον ὄσαι προγενέστερα[ι\\
πάντες δ’ ἄνδρες ἐπήρατον ἴαχον ὄρθιον\\
πάον’ ὀνκαλέοντες Ἐκάβολον εὐλύραν,\\
ὔμνην δ’ Ἔκτορα κ’ Ἀνδρομάχαν θεο<ε>ικέλο[ις.	
\end{gkverse}

\begin{verse}
\ldots{} Veio o arauto \ldots{}\\
Ideu\ldots{}, veloz mensageiro:\\
``\ldots{} e do resto da Ásia \ldots{} glória imperecível.\\
Heitor e os companheiros a de vivos olhos trazem\\
de Tebas sacra e da Plácia de fontes perenes -- ela,\\
delicada Andrômaca --, nas naus, sobre o salso\\
mar. E muitos áureos braceletes e vestes\\
de púrpura fragrantes, adornos furta-cor,\\
incontáveis cálices prateados e marfins''.\\
Assim ele falou; e rápido ergueu-se o pai querido;\\
e a nova, cruzando a ampla cidade, chegou aos amigos.\\
De pronto os troianos às carruagens de boas rodas\\
atrelaram as mulas, e nelas subiu toda a multidão \\
de mulheres e junto as virgens \ldots{} tornozelos\\
mas apartadas as filhas de Príamo\\
e cavalos os homens atrelaram aos carros\\
\ldots{} moços solteiros, e por um largo espaço \\
\ldots{} os condutores das carruagens \\
\ldots{} símeis aos deuses\\
\ldots{} sacro, em multidões\\
rumou \ldots{} em direção a Ílio\\
e o aulo de doce som \ldots{} se misturou\\
e o som dos crótalos \ldots{} e então as virgens\\
cantaram uma canção sacra e chegou aos céus\\
eco divino \ldots{}\\
e em toda parte estava ao longo das ruas\\
crateras e cálices \ldots{}\\
mirra e cássia e incenso se misturavam,\\
e as mulheres soltavam alto brado, as mais velhas,\\
e todos os homens entoavam adorável e alto\\
peã invocando o Arqueiro hábil na lira,\\*
e hineavam Heitor e Andrômaca, aos deuses símeis.
\end{verse}

\medskip

{\paragraph{Comentário} Esse fragmento, cujas fontes são os \textit{Papiros de Oxirrinco} 1232 e 2076,\footnote{Primeiras metades dos séculos \textsc{iii} e \textsc{ii} d.C., respectivamente.} leva"-nos à saga
de Troia e a personagens que vemos principalmente na \textit{Ilíada}: o arauto
Ideu, o herói troiano Heitor, irmão de Páris, e sua esposa, Andrômaca, o rei
troiano Príamo, o deus Apolo, protetor dos troianos na guerra. Além disso,
mostra"-nos o único exemplar de narrativa mítica epicizante
em Safo, cujos versos ditos por um narrador distanciado dividem"-se na
chegada do arauto e anúncio da vinda dos noivos Heitor e Andrômaca,\footnote{Versos
1--10.} e, depois, no espalhar dessa mensagem cidade afora, iniciando"-se com
isso a procissão festiva em honra do casal.\footnote{Versos 11--34.} A linguagem, a
atmosfera e a métrica que constroem essas etapas estão permeadas pela tradição
épico"-homérica; note"-se, por exemplo, a inescapável expressão do verso 4,
``glória imperecível'', \textit{kléos áphthiton}, síntese do ideal heroico a alcançar, a
ser preservado no canto épico, recorda a \textit{Ilíada },\footnote{Canto \textsc{ix}, verso 413.} 
pela boca de seu grande herói, Aquiles.

No discurso do arauto, repare"-se no
elogio dos noivos e no pequeno catálogo do valioso e belo dote da princesa da
Tebas asiática, Andrômaca, foco da celebração do casamento, como deve ser a noiva.
A importância da
referência ao dote é melhor
compreendida quando se pensa o casamento na vida das jovens gregas, momento
crucial que lhes atribui novo \textit{status} social -- o de esposa -- e gera
muitas mudanças: a saída da casa paterna e às vezes também da pátria; a
inserção na casa do marido e/\,ou numa realidade geográfica e cultural diversa; a
transformação da virgem em mulher que tem vida sexual e que deverá garantir a
continuidade das linhagens e administrar o espaço doméstico. A questão do dote
relaciona"-se intimamente à da legitimidade da união na Grécia antiga. Vale
lembrar que o dote da esposa se destina à prole do casal, que precisava ser
protegida no caso de perda do(s) pai(s) ou de separação. A procissão (verso
13) é etapa típica das cerimônias de casamento, e a mais retratada na iconografia grega, dada sua relevância de sanção pública à união celebrada naquelas festas, em geral muito elaboradas e
estendidas ao longo de vários dias, buscando propiciar, de todas as maneiras
possíveis, o enlace sexual que tornava legítima e consumada a união.
Vale observar o tom em crescendo da festa, com mistura de cantos, instrumentos de sopro, \textit{aulo}, e percussão, \textit{crótalo}, de líquidos -- nas ``crateras'',\footnote{\textit{Krátēres}, verso 29.} grandes jarros de larga boca, vinho misturado à água -- e arômatas  -- entre eles, o incenso,\footnote{\textit{Líbanos}, verso 30.} mencionado pela primeira vez, produto semítico --, que termina no apogeu da celebração dos noivos sob os auspícios de um canto a Apolo, o \textit{peã} em que se integra a cidade e com o qual louva o deus do arco e flecha -- da guerra que devastará a cidade e destruirá essa boda tão alegremente celebrada --, e o deus da lira que favorece a música, os cantos, a festa do tempo presente. 
Há, pois, no fragmento, um caráter epitalâmico, mas não há elementos consistentes para
qualificá"-lo como exemplar do subgênero mélico dos epitalâmios. Por fim, não se
pode deixar de lembrar que o casal Heitor e Andrômaca protagoniza no canto \textsc{vi}\footnote{Versos 369--502.} 
da \textit{Ilíada} uma das cenas mais comoventes de despedida
entre cônjuges afetuosamente próximos um do outro e do filho, o bebê Astiánax.
Essa cena empresta ao poema épico grande densidade trágica, pois, ao acenar
para a morte certa de Heitor, acena para a ruína de Troia e de seus elos mais
frágeis: mulheres e crianças, em geral mortas ou escravizadas pelos vitoriosos.
A carga dramática intensifica"-se pela ironia trágica de que Heitor perecerá
pelas mãos de Aquiles, que já haviam matado o pai e os irmãos de Andrômaca; Aquiles cuja sombra parece pairar sobre a canção, na expressão ``glória imperecível'', que profere na \textit{Ilíada} e que o fragmento sáfico enuncia pela boca de Ideu, e na palavra final da canção que termina, no texto grego e na tradução, com o adjetivo ``aos deuses símeis'', \textit{theoeikélois}, dado aos noivos em Safo, mas ao herói grego na \textit{Ilíada}.\footnote{\textsc{i}, verso 131; \textsc{xix}, verso 155.}
Essa lembrança confere ao fragmento, tão alegre e festivo, grande tristeza e
melancolia, e entrelaça casamento e morte, duas etapas de transição na trajetória
humana, cujas cerimônias guardam muitos paralelos no imaginário grego ao qual são característicos.\footnote{Redfield, 1982, p. 188.}}



\pagebreak
\section{Fragmento 141}

\begin{gkverse}
κῆ δ’ ἀμβροσίας μὲν\\
κράτηρ ἐκέκρατ’\\
Ἔρμαις δ’ ἔλων ὄλπιν θέοισ’ ἐοινοχόαισε.\\
κῆνοι δ’ ἄρα πάντες\\
καρχάσι’ ἦχον\\
κἄλειβον ἀράσαντο δὲ πάμπαν ἔσλα γάμβρωι
\end{gkverse}

\begin{verse}
\ldots{} e depois que uma cratera\\
de ambrosia foi misturada à água,\\
Hermes, tomando o jarro, vinhoverteu aos deuses.\\
E todos eles\\
seguravam cálices,\\
e libavam, e aguraram bons votos ao noivo \ldots{}
\end{verse}

\medskip

\paragraph{Comentário}
Ateneu (séculos \textsc{ii--iii} d.C.), \textit{Banquete dos eruditos} (10.425\,\textsc{cd}, 475\,\textsc{a}) é a principal fonte do fragmento. A cena  do  banquete nupcial pode ser da boda da Nereida Tétis e do mortal Peleu, pais de Aquiles, o grande herói grego da Guerra de Troia e da epopeia homérica \textit{Ilíada}. Ateneu afirma, antes da citação, que ``Alceu introduz Hermes como um servidor de vinho dos deuses, exatamente como Safo''. Note"-se a referência ao alimento dos deuses, ambrosia, \textit{ambrosías}, e a fusão do divino ao mortal na imagem do banquete em que o alimento dos deuses é objeto do verbo ``vinhoverteu'', \textit{eoinokhóēse},\footnote{Verso 3.} que encerra em si o líquido mais nobre dos mortais -- verbo visto já na festividade em espaço sacroerótico do Fr.\,2, em que Afrodite e néctar se combinam no \textit{vinhoverter} nas taças.


\pagebreak
\section{Fragmento 142}

\begin{gkverse}
Λάτω καὶ Νιόβα μάλα μὲν φίλαι ἦσαν ἔταιραι
\end{gkverse}

\begin{verse}
\ldots{} Leto e Níobe eram as mais caras \rlap{companheiras \ldots{}}
\end{verse}

\medskip

{\paragraph{Comentário} Preservado em Ateneu, o verso abaixo traz novamente,
agora no plural, o termo ``companheira'', \textit{étaira}, já visto no
126, desta vez ligando o \textit{eu} poético aos nomes de duas figuras míticas: a mãe de
Ártemis e Apolo, Leto, e Níobe, rainha que teve 14 filhos\footnote{Sete homens, sete
mulheres.} -- números que variam nas tradições --, e que, comparando"-se a Leto, insultou a deusa que tinha apenas dois.
Mas estes então puniram a mortal, matando"-lhe toda a prole, e ela, sucumbindo à dor que dela tudo drenou, metamorfoseou"-se em rocha. Não é possível afirmar
com certeza, mas o fragmento parece mítico, e nele as três figuras aparecem
estreitamente associadas, tanto pelo termo \textit{étairai}, quanto pelo
adjetivo que o precede.}



\pagebreak
\section{Leda e Zeus (Fr.\,166)}

\begin{gkverse}
φαῖσι δή ποτα Λήδαν ὐακίνθινον\\
<...> ὤιον εὔρην πεπυκάδμενον
\end{gkverse}

\begin{verse}
\ldots{} dizem que um dia Leda achou um ovo jacintino\\
na cor, coberto \ldots{}
\end{verse}

\medskip

{\paragraph{Comentário} Em certa tradição mítica, Zeus seduziu Leda, esposa de Tíndaro, disfarçando"-se
de cisne; o ovo por ela descoberto traria os gêmeos Castor e Polideuces, os
Dióscuros -- o primeiro mortal, e o segundo, imortal --, sendo a eles
permitido desfrutar de ambas as condições alternadamente. Na canção de Safo,
esse mito parece aludido, mas a cor do ovo contrasta de modo drástico com a usual,
e nos leva ao universo erótico das paisagens de enlace sexual, das quais o
jacinto é flor integrante. Cabe lembrar que os Dióscuros, noutras tradições,
são simplesmente filhos de Tíndaro e Leda, ou, noutras ainda, heróis de dupla
ascendência. A fonte do fragmento é Ateneu (2.57d).}




\chapter{Canções de recordação}

%\section{Memória e desejo}

\section{«\,Ode a Anactória\,» (Fr.\,16)}

\begin{gkverse}
Ο]ἰ μὲν ἰππήων στρότον οἰ δὲ πέσδων\\
οἰ δὲ νάων φαῖσ’ ἐπ[ὶ] γᾶν μέλαι[ν]αν\\
ἔ]μμεναι κάλλιστον, ἔγω δὲ κῆν’ ὄτ-\\
τω τις ἔραται

πά]γχυ δ’ εὔμαρες σύνετον πόησαι\\
π]άντι τ[ο]ῦ̣τ’, ἀ γὰρ πόλυ περσκέ̣θ̣ο̣ι̣σ̣α\\
κ̣άλ̣λο̣ς̣ [ἀνθ]ρ̣ώπων Ἐλένα [τὸ]ν ἄνδρα\\
τ̣ὸν̣ [    αρ̣]ι̣στον

κ̣αλλ[ίποι]σ̣’ ἔβα ’ς Τροΐαν πλέοι̣[σα\\
κωὐδ[ὲ πα]ῖδος οὐδὲ φίλων το[κ]ήων\\
π̣ά[μπαν] ἐμνάσθ<η>, ἀλλὰ παράγ̣α̣γ̣’ α̣ὔταν\\

\textnormal{[\textit{versos 12--4: ilegíveis e lacunares}]}

..]μ̣ε̣ νῦν Ἀνακτορί[ας ὀ]ν̣έ̣μναι-\\
σ’ οὐ] παρεοίσας,

τᾶ]ς <κ>ε βολλοίμαν ἔρατόν τε βᾶμα\\
κἀμάρυχμα λάμπρον ἴδην προσώπω\\
ἢ τὰ Λύδων ἄρματα κἀν ὄπλοισι\\
πεσδομ]άχεντας.

			   ].μεν οὑ δύνατον γένεσθαι\\
		   ].ν ἀνθρωπ[..(.) π]εδέχην δ’ ἅρασθαι\\

\textnormal{[\textit{versos 23--31: 23--7 perdidos, 28--31 ilegíveis e lacunares}]}


     τ’ ἐξ ἀδοκή[τω.
\end{gkverse}

\chapter*{}
\section*{}
\section*{}

\begin{verse}
Uns, renque de cavalos, outros, de soldados,\\
outros, de naus, dizem ser sobre a terra negra\\
a coisa mais bela, mas eu: o que quer\\*
que se ame.

De todo fácil fazer ver a\\
todos isso, pois a que muito superou\\			
em beleza os homens, Helena, o marido, \\
o mais nobre,

tendo deixado, foi para Troia navegando,\\
até mesmo da filha e dos queridos pais\\*
de todo esquecida, mas desencaminhou-a \ldots{}

agora traz-me Anactória à lembrança,\\*
a que está ausente, \ldots{}

Seu adorável caminhar quisera ver,\\
e o brilho luminoso de seu rosto,\\
a ver dos lídios as carruagens e a armada\\*
infantaria.

\ldots{} impossível vir a ser\\
\ldots{} humano \ldots{} partilhar e rezar\\
\ldots{}\\*
e inesperadamente.
\end{verse}

\pagebreak

\paragraph{Comentário} Ao lado dos Frs. 1 e 31, o 16 é seguramente um dos mais estudados da mélica \EP[2]
sáfica, sobretudo por conta da exemplificação mítica da afirmação feita no
\textit{priamel} -- como
se designa a estrofe inicial, em que a uma série de
negativas coloca"-se uma afirmativa, do ponto de vista da voz poética -- e da
alegada facilidade de sua compreensão,\footnote{Versos 5--6.} que soa aos nossos ouvidos
como ironia. Isso porque não está claro se o mito recordado, da fuga de Helena, esposa 
de Menelau (não nomeado), e Páris, o príncipe troiano não nomeado, é exemplo
negativo, o que se alinharia à imagem prevalecente na poesia grega antiga, após
os poemas homéricos -- nos quais, todavia, não há a condenação de Helena senão por ela mesma --, ou se positivo, o que tornaria singular a imagem do fragmento. A
ambiguidade que já caracteriza o olhar para a personagem naqueles poemas épicos também se verifica no Fr.\,16 de Safo: Helena é o \textit{exemplum} da afirmação da estrofe inicial sobre o \textit{tò kálliston}, ``a coisa mais bela'', mas é aquela que quebrou os contratos sociais mais relevantes de seus papéis de filha e esposa, pois abandonou pais e marido ``o mais nobre'', \textit{tòn\ldots{} áriston},\footnote{Verso 8.} talvez sob coação divina, se Afrodite é personagem da terceira estrofe. Tais crimes são lembrados por Alceu, o poeta lésbio contemporâneo a Safo, em similar dicção no Fr.\,283, que sobretudo condena Helena pela Guerra de Troia e seus incontáveis mortos, tal qual faz no Fr.\,42, mas enfocando outros elementos. Merece atenção, ainda, a
oposição \textit{éros}--guerra, sobre a qual se elaboram os versos, e a
referência ao rico reino oriental da Lídia, na Ásia Menor, com que estava
ligada a aristocracia da arcaica Mitilene. E a possibilidade de que haja um
viés metalinguístico relativo às escolhas temáticas da poeta da mélica arcaica.
A fonte do fragmento, \textit{Papiro de Oxirrinco} 1231, é a mesma dos Frs.~15, 17 e 22.
A jovem distante, não mais em Lesbos, parece ser uma das coreutas da associação coral de Safo, uma das \textit{parthénoi}, das moças, cuja partida pode coincidir com sua transição ao mundo do casamento. Resta a memória de sua beleza e graça, canta a canção que a celebra.




\pagebreak
\section{Fragmento 96}

\begin{gkverse}
] σαρδ.[\ldots{}]\\
     πόλ]λακι τυίδε̣ [.]ων ἔχοισα

ὠσπ.[\ldots{}].ώομεν, .[\ldots{}]..χ[\ldots{}]\\
    σε \dagger{}θεασικελαν ἀρι-\\
      γνωτα\dagger{}, σᾶι δὲ μάλιστ’ ἔχαιρε μόλπαι̣

νῦν δὲ Λύδαισιν ἐμπρέπεται γυναί-\\
    κεσσιν ὤς ποτ’ ἀελίω\\
      δύντος ἀ βροδοδάκτυλος <σελάννα>

πάντα περ<ρ>έχοισ’ ἄστρα φάος δ’ ἐπί-\\
    σχει θάλασσαν ἐπ’ ἀλμύραν\\
      ἴσως καὶ πολυανθέμοις ἀρούραις

ἀ δ’ <ἐ>έρσα κάλα κέχυται τεθά-\\
    λαισι δὲ βρόδα κἄπαλ’ ἄν-\\
      θρυσκα καὶ μελίλωτος ἀνθεμώδης 

πόλλα δὲ ζαφοίταισ’ ἀγάνας ἐπι-\\
    μνάσθεισ’ Ἄτθιδος ἰμέρωι\\
      λέπταν ποι φρένα κ[.]ρ\ldots{} βόρηται

κῆθι δ’ ἔλθην ἀμμ.[\ldots{}].ισα τό̣δ’ οὐ\\
    νωντα[\ldots{}]υστονυμ[̣..(.)] πόλυς\\
      γαρύει̣[\ldots{}(.)]αλον[̣\ldots{}(.)]τ̣ο ̣μέσσον

ε]ὔ̣μαρ[ες μ]ὲ̣ν οὐ.α.μι θέαισι μόρ-\\
    φαν ἐπή[ρατ]ον ἐξίσω-\\
      σθ̣αι συ[..]ρ̣ο̣ς ἔχη<ι>σθα[\ldots{}].νίδηον

\textnormal{[\textit{versos 24--5: ilegíveis e lacunares}]}

      καὶ δ[.]μ{[}̣	\qquad		{]}ος Ἀφροδίτα

καμ[̣		\qquad	] νέκταρ ἔχευ’ ἀπὺ\\
    χρυσίας [		       ]ν̣αν\\
      \ldots{}.(.)]απουρ̣[\qquad		] χέρσι Πείθω

\textnormal{[\textit{versos 30--2: ilegíveis e lacunares}]}

[\qquad			]ες τὸ Γεραίστιον\\
    {[}\qquad			   {]}ν ̣φίλαι

\textnormal{[\textit{versos 35--6: ilegíveis e lacunares}]}
\end{gkverse}


\begin{verse}
\ldots{} Sárdis \ldots{}\\
muita vez para cá \ldots{} ela tendo\\
\ldots{}\\
\ldots{} qual deusa manifesta,\\*
e [ela] muito se deleitava com tua dança-canção.

Mas agora ela se sobressai entre Lídias\\
mulheres como, depois do sol\\*
posto, a dedirrósea lua

supera todas as estrelas; e sua luz se esparrama\\
por sobre o salso mar \\*
e igualmente sobre multifloridos campos.

E o orvalho é vertido em beleza, e brotam\\
as rosas e o macio \\*
cerefólio e o trevo-mel em flor.

E [ela] muito agitada de lá para cá a \\
recordar a gentil Átis com desejo;\\*
decerto frágil peito \ldots{} se consome \ldots{}

\ldots{} canta \ldots{}

\ldots{} Fácil não \ldots{} com deusas quanto à forma\\
amável rivalizar \ldots{}\\
\ldots{}\\
\ldots{} ó Afrodite\\
\ldots{}

\ldots{} o néctar vertia da\\
áurea \ldots{}\\
\ldots{} mãos \ldots{} Peitó, deusa Persuasão\\
\ldots{} o Geraístio\\
\ldots{} queridas \ldots{}
\end{verse}

\medskip

{\paragraph{Comentário} O texto, preservado no \textit{Papiro de Berlim} 9722 (século \textsc{vi}
d.C.), está bastante danificado, sendo legíveis os versos 4--17.
A cena parece envolver uma 3\textsuperscript{a} pessoa do
singular, feminina, que no passado se relacionou à 2\textsuperscript{a} do
singular em chave erótica, como indicam a referência ao prazer e à canção,\footnote{Versos 4--5.} e à beleza e ao desejo.\footnote{Versos 6--17.} No tempo presente, 
em que estão separadas essas duas personagens, \textit{ela} encontra"-se na Lídia -- Sárdis, sua cidade mais importante, talvez no v.\,1 já seja nomeada --, onde sua beleza é proeminente, como provavelmente o fora em Lesbos, quando lá viveu junto ao grupo coral de Safo. No entanto, \textit{ela} sofre, saudosa de ``Átis''\footnote{Verso 17.} -- talvez o \textit{tu} da canção --, cuja figura a recordação lhe traz, fazendo
com que se consuma seu peito em desassossego erótico. A alternância temporal do
passado ao presente introduz o longo símile central,\footnote{Versos 6--14.} alavancado
na natureza, como é característico da linguagem erótica da poesia grega antiga.
Em Safo, essa linguagem é em especial elaborada; e no fragmento, o principal
elemento natural enfocado para o canto da beleza feminina é a feminina lua\footnote{Verso 8.} que, como Eos, a Aurora, nos poemas homéricos, é inesperadamente
chamada ``dedirrósea''. Ao erguer"-se, posto o sol, tinge"-se a lua
de seus tons laranja"-avermelhados ou rosados que esquentam seu branco prateado;
estamos, pois, no início do anoitecer. No verso 9, ganha relevo a luz da lua,
vibrante, já em plena noite; e no 12, o orvalho que vem ao findar"-se a noite,
avançada já a madrugada.

Não será esta a única vez em que Safo cantará a lua,
como mostram os Fr.\,34 e 154, inseridos nesta antologia. Sua escolha é
eloquente: a lua é elemento feminino, mesmo na língua grega, e tem a função de
regular os ciclos biológicos e o ritmo das marés, interferindo na fertilidade;
aproxima"-se, pois, da esfera da mulher, percebida na Antiguidade como ligada
aos líquidos, à umidade e à natureza, já que lhe cabe a crucial
reprodução no ciclo biológico humano. Chamo a atenção, por fim, para o cenário
florido e pulsante do símile, integrado, inclusive, pelas diletas flores de
Afrodite, as rosas, marcadas na adjetivação da lua. O símile conecta"-se
ao restante do fragmento, na medida em que ambos tratam da beleza feminina e
estão perpassados pelo erotismo, o que é frisado na estrofe dos versos 15--17.
Portanto, pode"-se dizer que a voz do fragmento insere sua canção no universo de
Afrodite, como no Fr.\,2.
E, referindo a própria coralidade, talvez cante a memória de uma coreuta que se foi, como seria o caso da Anactória do Fr.\,16, e da não nomeada jovem cuja despedida é recordada adiante, no Fr.\,94.
Os versos finais ficam cada vez mais precários, mas interessa notar que trazem o tema da beleza e do desejável, da festividade -- que remete ao Fr.\,2 -- e da presença divina de deusas da sedução, e a menção de um santuário de Posêidon na Eubeia (continente grego), obscura para nós. Salvo por este elemento, dos demais emana o mundo mais caracteristicamente sáfico, em que estão imersas a poeta, seu coro e as canções.}

\pagebreak
%\section{Separação (e solidão)}
\section{Fragmento 23}

\begin{gkverse}
]ἕρωτος  ἠλπ[\\
		       ]\\
		   αν]τιον εἰσίδωσ[\\
		       ] ’Ερμιόνα τεαυ[τα\\
	   ] ξάνθαι δ’ ’Ελέναι σ’ ἑίσ[κ]ην\\
	]κες \\
	].ις θνάταις, τόδε ἴσ[θι] τὰι σᾶι\\
	]παίσαν κέ με τὰν μερίμναν\\
	]λαισ’ ἀντιδ[..][.]α̣θοις δὲ̣\\
     ]\\
				]τας ὄχθοις\\
			          ]ταιν\\
			      παν]νυχίσ[δ]ην
\end{gkverse}

\begin{verse}
\ldots{} do desejo \ldots{}\\
\ldots{}\\
\ldots{} face a face contemplo \ldots{}\\
\ldots{} Hermíone tal \ldots{}\\
e comparar-te à loira Helena \ldots{}\\
\ldots{}\\
\ldots{} às mortais; e isto sabe, em tua\\
\ldots{} de todos os meus anseios\\
\ldots{}\\
\ldots{} nas margens \ldots{}\\
\ldots{}\\*
\ldots{} celebrar um festival noturno \ldots{} 
\end{verse}

\pagebreak

{\paragraph{Comentário} Preservado no rolo \textit{Papiro de Oxirrinco} 1231, que contém o Fr.\,15 e outros vistos, o Fr.\,23 traz elementos notáveis: a bela Helena -- que antes surge no Fr.\,16 -- e sua filha Hermíone, e o erotismo sobretudo nas menções ao desejo, ao olhar, à beleza de Helena -- projetada pelo loiro de seus cabelos, mesmo tom do ouro, o metal mais valioso, e do sol, astro essencial à vida -- e às ansiedades. Na última linha, \textit{pannykhísdēn} refere o celebrar de festa noturna, tal como outra forma do mesmo verbo no Fr.\,30, adiante. Não há propriamente separação, aqui, mas a fala à \textit{persona} feminina, comparada às personagens míticas decerto pela beleza compartilhada, desenha"-se como íntima, pessoal, e talvez em separado de um conjunto. Este seria o de celebrantes da boda, contexto provável também do Fr.\,30, em vista do rito noturno ritualístico, \textit{pannykhís}. Como veremos no Fr.\,111, a comparação dos noivos a figuras divinas ou míticas é frequente na dicção das canções relacionadas ao mundo do casamento, no qual, aliás, vimos Safo inserir a cena mítica troiana do Fr.\,44, do enlace de Heitor e Andrômaca, e do qual já algo disse no comentário ao Fr.\,112, e mais direi, chegando aos epitalâmios.}

\pagebreak
\section{Fragmento 49}

\begin{gkverse}
Ἠράμαν μὲν ἔγω σέθεν, Ἄτθι, πάλαι ποτά

\hspace*{27mm}\adforn{47}

σμίκρα μοι πάις ἔμμεν’ ἐφαίνεο κἄχαρις.
\end{gkverse}

\begin{verse}
Eu te desejei, Átis, há tempos, um dia \ldots{}

\hspace*{27mm}\adforn{47}

\ldots{} criança mirrada e sem graça me parecias ser \ldots{}
\end{verse}

\medskip

{\paragraph{Comentário} De novo, Átis é personagem de uma canção de erótica tonalidade, da qual temos o verso inicial preservado em Heféstion (7.7), fonte do Fr.\,102, entre outros. Recorda a voz poética sua paixão por Átis, exaurida no presente, razão pela qual estão ambas as personagens separadas. Depois, num outro verso, que não sabemos em que ponto da mesma canção seria entoado, Plutarco (séculos \textsc{i}--\textsc{ii} d.C., \textit{Diálogos sobre o amor} 751\,\textsc{d}),
sua fonte, conta que, ``falando a uma menina demasiado nova para o
casamento, Safo lhe diz'' palavras de clara reprovação, por conta de sua forma
física. }



\pagebreak
\section{Fragmento 88}

\begin{gkverse}
\textnormal{[\textit{versos 1--4 e 28: ilegíveis e lacunares}]}

].θέλοις οὐδυ̣[\\
].άσδοισ’ ὀλιγα[\\
].ένα φέρεσθα[ι

  ].φι̣α̣ τισ̣...[\\
  ].δ’ ἄδιον εἰσορ[\\
ο]ἶσθα καὔτα

λέ]λ̣αθ’ ἀλλονιά[\\
     ].αν τι̣ραδ[\\
     ]α̣ί̣ τις εἴποι

].σαν ἔγω τε γαρ[\\
]μ̣’ ἆς κεν ἔνηι μ’[\\
  ]α̣ι μελήσην

]φίλα φαῖμ’ ἐχύρα γέ[νεσθαι\\
   ]ενα[.]αις ἀτ̣[\\
          ]δ’ ὀνίαρ̣[ο]ς̣ [

        ]. πίκρος ὔμ[\\
        ]τα.θᾶδ̣[\\
        ].α τόδε δ’ ἴσ̣[θ

       ]ὤττι σ’ ἐ.[\\
]α φιλήσω[\\
]τ̣ω τι̣ λο[

]σσον γὰρ .[\\
          ]σ̣θαι βελέω[ν
\end{gkverse}

\pagebreak
\begin{verse}
\ldots{} eu desejaria \ldots{}\\
\ldots{}\\
\ldots{} carregar \ldots{}\\
\ldots{}\\
\ldots{} mais doce de ver \ldots{}\\
\ldots{} sabes tu mesma;

\ldots{} esqueceu \ldots{}\\ 
\ldots{}\\
\ldots{} alguém diria

\ldots{} pois eu \ldots{}\\
amarei \ldots{} até que haja em mim \ldots{}\\
\ldots{} será objeto de cuidado;\\
\ldots{} digo ser amiga confiável \\
\ldots{}\\
\ldots{} doloroso \ldots{}\\
\ldots{} amargo \ldots{}\\
\ldots{}\\
\ldots{} isto sabe\\
\ldots{} amarei \ldots{}\\
\ldots{}\\
\ldots{} de flechas\ldots{}
\end{verse}

\medskip

{\paragraph{Comentário} Tendo por fonte o \textit{Papiro de Oxirrinco} 2290 (fim do século \textsc{ii} ou início do \textsc{iii} d.C.), o fragmento traz a \textit{persona} a expressar volição e falar a um \textit{tu} feminino, afirmando que tipo de amizade é capaz de oferecer, no reiterado uso do verbo \textit{phileîn}, ``amar'', e do termo ``amiga'', \textit{phíla}. Algo de prazeroso se mescla a algo amargo e dolorido; e no último verso pode haver 
referência à caçadora deusa virgem, Ártemis.\footnote{Campbell, 1994, p. 113.} Há insistência, por fim, em que a destinatária saiba de algo que não se revela, e a menção do esquecimento. Difícil somar as partes, mas há na base dos versos  uma relação de amizade entre figuras femininas -- uma das quais, a \textit{persona} --, e talvez uma ruptura.}




\pagebreak
\section{Fragmento 94}

\begin{gkverse}
τεθνάκην δ’ ἀδόλως θέλω\\
ἄ με ψισδομένα κατελίμπανεν

πόλλα καὶ τόδ’ ἔειπέ̣  [μοι\\
ὤιμ’ ὠς δεῖνα πεπ[όνθ]αμεν,\\
Ψάπφ’, ἦ μάν σ’ ἀέκοισ’ ἀπυλιμπάνω.

τὰν δ’ ἔγω τάδ’ ἀμειβόμαν\\
χαίροισ’ ἔρχεο κἄμεθεν\\
μέμναισ’, οἶσθα γὰρ ὤς σε πεδήπομεν

αἰ δὲ μή, ἀλλά σ’ ἔγω θέλω\\
ὄμναισαι [...(.)].[...(.)].ε̣αι\\
ὀ̣σ̣[\qquad     - 10 -\qquad   ] καὶ κάλ’ ἐπάσχομεν

πό̣[λλοις γὰρ στεφάν]οις ἴων\\
καὶ βρ[όδων ...]κ̣ίων τ’ ὔμοι\\
κα..[ - 7 -\quad   ] πὰρ ἔμοι π<ε>ρεθήκα<ο>

καὶ πό̣λλαις ὐπαθύμιδας\\
πλέκ[ταις ἀμφ’ ἀπάλαι δέραι\\
ἀνθέων ἐ[  -  6  -  ] πεποημμέναις

καὶ π\ldots{}[\qquad      ]. μύρωι\\
βρενθείωι ̣[\qquad      ]ρ̣υ[...]ν\\
ἐξαλ<ε>ίψαο κα̣[ὶ βασ]ι̣ληίωι

καὶ στρώμν[αν ἐ] πὶ μολθάκαν\\
ἀπάλαν παρ[̣ \quad   ]ο̣ν̣ων\\ 
ἐξίης πόθο̣[ν \quad    ]νίδων

κωὔτε τισ[\quad       οὔ]τ̣ε̣ τι\\
ἶρον οὐδ’ ὐ[ \quad        ]\\
ἔπλετ’ ὄππ̣[οθεν ἄμ]μες ἀπέσκομεν,

οὐκ ἄλσος .[\qquad	       ].ρος

\textnormal{[\textit{versos 28--9: ilegíveis e lacunares}]}
\end{gkverse}

\pagebreak

\section*{}
\begin{verse}
\ldots{} morta, honestamente, quero estar;\\*
ela me deixava chorando

muito, e isto me disse:\\*
``Ah!, coisas terríveis sofremos,\\*
Ó Safo, e, em verdade, contrariada te deixo''.

E a ela isto respondi:\\*
``Alegra-te, vai, e de mim\\*
te recorda, pois sabes quanto cuidamos de ti;

mas se não, quero te\\*
lembrar \ldots{}\\*
\ldots{} e coisas belas experimentamos;

pois com muitas guirlandas de violetas\\*
e de rosas \ldots{} juntas\\*
\ldots{} ao meu lado puseste,			

e muitas olentes grinaldas\\*
trançadas em torno do tenro colo, \\*
de flores \ldots{} feitas;

e\ldots{} com perfume\\*
de flores \ldots{}\\*
digno de rainha, te ungiste,

e sobre o leito macio\\*
tenra \ldots{}\\*
saciavas [teu] desejo \ldots{}

Não havia \ldots{} nem algum\\*
santuário, nem \ldots{}\\*
do qual estivéssemos ausentes, 

nem bosque \ldots{}
\end{verse}

\pagebreak

{\paragraph{Comentário} Preservado no mesmo rolo papiráceo do Fr.\,96, o 94 é fortemente dramático ao
reencenar a separação entre o \textit{eu} -- ``Safo'' -- e \textit{ela}, revivida em
recordação detalhada, de tons eróticos cada vez mais intensos na
gradação corpo, perfumes, adornos, leito, e com participação no rito,
ao fim ainda legível do fragmento em que se destaca a coralidade própria à natureza do grupo de \textit{parthénoi}, de meninas virgens, liderado por Safo. Grupo em que dança, canto, atividades de culto e corais, a preparação para o \textit{gámos}, ``casamento'', e para a atuação no mundo do sexo faziam parte do que podemos chamar de \textit{paideía}, ``formação'' feminina. Grupo do qual já se acham distantes, na Lídia, as \textit{parthénoi} dos Frs.\,16 (Anactória) e 96, e a \textit{parthénos} cuja despedida a canção relata em chave consolatória -- o consolo a quem partiu e a quem ficou na memória feliz da coralidade compartilhada, da beleza, do prazer e do convívio com as amigas, de uma vida que, ao casar"-se e tornar"-se mulher, \textit{gyné}, deixa para trás.

Merece atenção a frequência do convite ao passado em Safo nesses Frs. 16, 94 e 96, o qual ``tem valor ideológico: nos momentos de fratura'', quando uma das jovens se separa do grupo ao mudar de \textit{status}, ``a poeta exorta a criar uma ponte entre passado e futuro, que mantenha viva a relação de companheirismo através da recordação. Esta operação tem um duplo significado: de um lado, consola quem parte e quem fica; de outro, reafirma os princípios constitutivos da comunidade, recordando"-lhe os valores e as atividades''.\footnote{Caciagli, 2009, p. 78.}
Ressalto, enfim, a força do tema da memória (e do esquecimento), marcado já no Fr.\,1, e que mesmo noutros fragmentos aqui não incluídos, como o 24 e 25, estaria presente, se aceitarmos a possibilidade de que, em seus respectivos e ilegíveis textos, haja formas verbais relativas ao lembrar ou esquecer.
}





% \pagebreak
% %\section{Imortalidade}
% \section{Fragmento 55}

% \begin{gkverse}
% κατθάνοισα δὲ κείσηι οὐδέ ποτα μναμοσύνα σέθεν\\
% ἔσσετ’ οὐδὲ \dagger{}ποκ’\dagger{} ὔστερον οὐ γὰρ πεδέχηις βρόδων\\
% τὼν ἐκ Πιερίας, ἀλλ’ ἀφάνης κἀν Ἀίδα δόμωι\\
% φοιτάσηις πεδ’ ἀμαύρων νεκύων ἐκπεποταμένα.
% \end{gkverse}

% \begin{verse}
% Morta jazerás, nem memória alguma futura\\
% de ti haverá, nem desejo, pois não partilhas das rosas\\
% de Piéria; mas invisível na casa de Hades\\*
% vaguearás esvoaçada entre vagos corpos \ldots{}
% \end{verse}

% \medskip

% {\paragraph{Comentário} Preservado na \textit{Antologia} (3.4.12) de Estobeu (século \textsc{v} d.C.), o fragmento traz um dos muitos momentos em que os poetas refletem em seus versos
% sobre o poetar e a própria poesia, que vai se configurar no imaginário grego
% como caminho para a imortalidade do nome.\footnote{Explorei o tema na poesia arcaica (Ragusa, 2018, pp. 143--152).} A linguagem é de ataque, de
% invectiva, e dirigida a um \textit{tu} feminino que pensa ser (hábil) poeta, mas
% não o é. Por isso, sua dupla morte ao descer ao mundo, reino de
% Hades, tornando"-se não mais alcançável sua figura aos olhos dos vivos, e
% apagada sua existência da memória destes.}



\pagebreak
\section{Fragmento 129}

\begin{gkverse}
ἔμεθεν δ’ ἔχηισθα λάθαν

\hspace*{17mm}\adforn{47}

ἤ τιν’ ἄλλον ἀνθρώπων ἔμεθεν φίληισθα

\end{gkverse}

\begin{verse}
\ldots{} e de mim memória não há \ldots{}

\hspace*{17mm}\adforn{47}

\ldots{} ou algum outro dos homens ame, além de mim \ldots{}
\end{verse}

\medskip

\paragraph{Comentário}
Citado no já referido tratado sobre pronomes de Apolônio, fonte do Fr.\,32, o 129 traz dois versos cuja conexão nos escapa, com a negação da memória e do desejo.


\section{Fragmento 147}

\begin{gkverse}
μνάσεσθαί τινα φα<ῖ>μι \dagger{}καὶ ἔτερον\dagger{} ἀμμέων
\end{gkverse}

\begin{verse}
\ldots{} digo, lembrar-se-á de nós alguém no porvir \ldots{}
\end{verse}

\medskip

{\paragraph{Comentário} Preservado em Dio Crisóstemo (séculos \textsc{i"-ii} d.C),\footnote{\textit{Oração} (37.47).} o fragmento talvez expresse a confiança na imortalidade por parte da poeta, em sua \textit{persona} dramática, em canção.}


\chapter{Desejos}

\section{Fragmento 95}

\begin{gkverse}
\textnormal{[\textit{versos 1--3 e 14--6: ilegíveis e lacunares}]}

Γογγυλα.[

ἦ τι σᾶμ’ ἐθε.[\\
παισι μάλιστα.[\\
μας γ’ εἴσηλθ’ ἐπ.[

εἶπον ὦ δέσποτ’, ἐπ.\\
ο]ὐ μὰ γὰρ μάκαιραν̣ [\\
ο]ὐδὲν ἄδομ’ ἔπαρθ’ ἀγα[

κατθάνην δ’ ἴμερός τις [ἔχει με καὶ\\
λωτίνοις δροσόεντας [ὄ-\\
χ[θ]οις ἴδην Ἀχερ[
\end{gkverse}

\pagebreak
\begin{verse}
\ldots{} Gongila \ldots{}

\ldots{} decerto um sinal \ldots{}

a todos e sobretudo \ldots{}\\
\ldots{} veio \ldots{}

\ldots{} disse: ``Ó senhor \ldots{}\\*
pois pela venturosa \ldots{}\\*
não me deleito em estar agitada \ldots{}

um desejo de morrer me toma, e\\*
com lótus orvalhadas as\\*
margens do Aqueronte ver \ldots{}
\end{verse}

\medskip

{\paragraph{Comentário} Um nome feminino e um forte desejo de rendição à morte marcam esse fragmento do
\textit{Papiro de Berlim }9722, também fonte dos Frs.~94 e 96. Ressalto a imagem do
rio do mundo dos mortos, o Aqueronte mencionado no Fr.\,65, cujas margens são usualmente pintadas
como densas de flores de lótus, flores desde a tradição egípcia associadas à morte. O \textit{eu} feminino parece dirigir"-se a um deus,
talvez Hermes, condutor que guia os mortos ao Hades, mensageiro que atravessa todas as fronteiras.}

\pagebreak
\section{Fragmento 121}

\begin{gkverse}
ἀλλ’ ἔων φίλος ἄμμι λέχος ἄρνυσο νεώτερον\\
οὐ γὰρ τλάσομ’ ἔγω σύν <τ’> οἴκην ἔσσα γεραιτέρα
\end{gkverse}

\begin{verse}
\ldots{} mas grato me sendo, toma o leito de [outra] mais nova,\\
pois não suportarei ser a mais velha numa \rlap{parceria \ldots{}}
\end{verse}

\medskip

{\paragraph{Comentário} Preservado em Estobeu (4.22.112), como o Fr.\,55, o 121 trabalha com a ideia da
reciprocidade, importante ao imaginário grego; o \textit{eu} feminino fala a um \textit{tu}
masculino, reclamando a necessidade do gesto de gratidão e de equiparação de
idade, em se tratando de alianças, talvez a do casamento.}

\section{Fragmento 126}

\begin{gkverse}
δαύοισ(’) ἀπάλας ἐτα<ί>ρας ἐν στήθεσιν
\end{gkverse}

\begin{verse}
\ldots{} que adormeças no peito macio de tua companheira \ldots{} 
\end{verse}

\medskip

{\paragraph{Comentário} A fonte é o \textit{Etimológico genuíno }(século \textsc{ix}). O termo empregado para
``companheira'', \textit{etaíras}, usado no Fr.\,142, marca uma amizade e aliança muito estreita entre o \textit{tu}
e \textit{ela}, e indica serem coetâneas essas personagens. O modo como o desejo é expresso parece acrescentar a essa relação o
ingrediente erótico, que de modo algum seria estranho ao convívio na associação coral feminina das \textit{parthénoi}, as virgens. Se o \textit{tu} referir uma delas, o íntimo convívio compartilhado pode estar em cena, como está no Fr.\,94 e no catálogo de belezas desfrutadas na coralidade, que lá se desenha.}




\pagebreak
\section{Fragmento 138}

\begin{gkverse}
στᾶθι \dagger{}κἄντα\dagger{} φίλος\\
καὶ τὰν ἐπ’ ὄσσοισ’ ὀμπέτασον χάριν
\end{gkverse}

\begin{verse}
\ldots{} posta-te [diante de mim], se és meu amigo,\\
e estende afora a graça de teus olhos \ldots{}
\end{verse}

\medskip

{\paragraph{Comentário} O foco nos olhos é comum quando se canta a beleza; nesse canto, o \textit{eu} chama o
\textit{tu} de \textit{phílos }, ``amigo'', termo que pode ou não ter conotação
erótica -- a primeira possibilidade sendo, creio, mais provável, inclusive
porque a fonte do fragmento, Ateneu (13.564\,\textsc{d}), que preservou também o Fr.\,141, afirma:
``Safo diz ao homem que é excessivamente admirado por sua forma e tido como belo''.}



\chapter{Dores de amor}

\section[«\,Phaínetaí moi\ldots{}\,» (Fr.\,31)]{«\,Phaínetaí moi\ldots{}\,» (Fr.\,31)}

\begin{gkverse}
Φαίνεταί μοι κῆνος ἴσος θέοισιν\\
ἔμμεν’ ὤνηρ, ὄττις ἐνάντιός τοι\\
ἰσδάνει καὶ πλάσιον ἆδυ φωνεί-\\
σας ὐπακούει

καὶ γελαίσας ἰμέροεν, τό μ’ ἦ μὰν\\
καρδίαν ἐν στήθεσιν ἐπτόαισεν\\
ὠς γὰρ <ἔς> σ’ ἴδω βρόχε’ ὤς με φώναι-\\
σ’ οὐδὲν ἔτ’ εἴκει,

ἀλλα \dagger{}καμ\dagger{} μὲν γλῶσσα \dagger{}ἔαγε\dagger{}, λέπτον\\
δ’ αὔτικα χρῶι πῦρ ὐπαδεδρόμηκεν,\\
ὀππάτεσσι δ’ οὐδὲν ὄρημμ’, ἐπιβρόμ-\\
μεισι δ’ ἄκουαι,

\dagger{}έκαδε\dagger{} μ’ ἴδρως ψῦχρος κακχέεται, τρόμος δὲ\\
παῖσαν ἄγρει, χλωροτέρα δὲ ποίας\\
ἔμμι, τεθνάκην δ’ ὀλίγω ’πιδεύης\\
φαίνομ’ ἔμ’ αὔτ[̣αι.

ἀλλὰ πὰν τόλματον, ἐπεὶ \dagger{}καὶ πένητα\dagger{}
\end{gkverse}

\pagebreak
\begin{verse}
Parece-me ser par dos deuses ele,\\
o homem, que oposto a ti\\
senta e de perto tua doce\\*
fala escuta,

e tua risada atraente. Isso, certo,\\
no peito atordoa meu coração;\\
pois quando te vejo por um instante, então\\*
falar não posso mais,

mas se quebra minha língua, e ligeiro\\
fogo de pronto corre sob minha pele,\\
e nada veem meus olhos, e\\*
zumbem meus ouvidos,

e água escorre de mim, e um tremor\\
de todo me toma, e mais verde que a relva\\
estou, e bem perto de estar morta\\
pareço eu mesma.

Mas tudo é suportável, se mesmo um pobre \rlap{homem \ldots{}}
\end{verse}

\medskip

{\paragraph{Comentário} O Fr.\,31 ou \textit{Phaínetaí moi}, ``Parece"-me\ldots{}'', tem por fonte
principal o famoso tratado \textit{Do sublime}.\footnote{10.1--3, século \textsc{i} d.C.?, \textit{Longino}.} Ao
explicar as maneiras de um texto alcançar a grandeza, seu autor de incerta identidade menciona os
``pensamentos elevados'';\footnote{ Cito as traduções do volume Hirata,
F. (trad., introdução, notas). \textit{Longino.} \textit{Do Sublime.} São
Paulo: Martins Fontes, 1996.} no quadro destes, cita o fragmento de Safo,
em que louva a ``escolha dos motivos'' e a ``concentração dos
motivos escolhidos''. ``Por exemplo Safo: as afecções consecutivas ao delírio
amoroso, a cada vez, ela as apreende como elas se apresentam sucessivamente e
na sua própria verdade. Mas onde mostra ela sua força? Quando ela é capaz, a
uma vez, de escolher e de ligar o que há de mais agudo e de mais
intenso nessas afecções''. A citação é sucedida por palavras que equiparam Safo
a Homero, ``o Poeta'': 
\pagebreak
\begin{quote}
Não admiras como, no mesmo momento, ela
procura a alma, o corpo, o ouvido, a língua, a visão, a pele, como se tudo isso
não lhe pertencesse e fugisse dela; e, sob efeitos opostos, ao mesmo tempo ela
tem frio e calor, ela delira e raciocina (e ela está, de fato, seja
aterrorizada, seja quase morta)? Se bem que não é uma paixão que se mostra
nela, mas um concurso de paixões! Todo esse gênero de acontecimentos cerca os
amantes, mas, como eu disse, a maneira de agrupá"-los, para relacioná"-los num
mesmo lugar, realiza a obra de arte. Da mesma maneira, a meu ver, para as
tempestades o Poeta escolhe as mais terríveis das consequências.
\end{quote}

A cena retratada eroticamente produz uma triangulação na qual o olhar do \textit{eu}
feminino contempla primeiramente um homem que se porta como audiência de um
\textit{tu} feminino, para depois concentrar"-se na contemplação dessa
personagem, que provoca sua excitação -- na imagem recorrente do peito que se agita ou,
literalmente, voa, como se vê no Fr.\,22 -- e crescente dominação erótica, ou
seja, fragmentação do corpo e da mente que leva à morte. Há um notável eco
metalinguístico no fragmento, em que a perda da voz, central na patologia
erótica, é crucial para poetas de tradição oral como Safo, que não existe sem
ela, e para a representação dramática da resposta à cena, que só se realiza a
partir justamente dessa perda. Temos o início do fragmento, mas não estamos
seguros de seu fim. Seu texto foi reelaborado por Catulo (século \textsc{i}
a.C.) no \textit{Poema 51},\footnote{ Cito"-o, em tradução de Oliva (1996):
\textit{Ele parece"-me ser par de um deus,/ Ele, se é fás dizer, supera os
deuses,/ Esse que todo atento o tempo todo/ Contempla e ouve"-te/ doce rir, o
que pobre de mim todo/ sentido rouba"-me, pois uma vez/ que te vi, Lésbia,
nada em mim sobrou/ De voz na boca/ Mas torpece"-me a língua e leve os membros
/ Uma chama percorre e de seu som/ Os ouvidos tintinam, gêmea noite/ cega"-me
os olhos./ O ócio, Catulo, te faz tanto mal./ No ócio tu exultas, tu vibras
demais./ Ócio já reis e já ricas cidades/ Antes perdeu.}} praticamente uma
tradução da canção sáfica.
Nos versos do Fr.\,31, impressiona com que habilidade Safo cria uma dimensão de intimidade; neles, como noutros, a poeta revela"-se ``a maior mestra em pseudointimidade''.\footnote{Scodel, 1996, p. 77.}}



\pagebreak
\section{Fragmento 36}

\begin{gkverse}
καὶ ποθήω καὶ μάομαι\ldots{}
\end{gkverse}

\begin{verse}
\ldots{} e desejo e enlouqueço \ldots{}
\end{verse}

\medskip

{\paragraph{Comentário} Nesse parco fragmento preservado, como o Fr.\,126, no \textit{Etimológico genuíno}, combina"-se um binômio recorrente na poesia erótica grega:
paixão e loucura. Vale anotar: a loucura erótica, provocada por Afrodite, será
tratada como um dos tipos de loucura definidos por Platão
no diálogo \textit{Fedro}, por exemplo.}


\section{Fragmento 48}

\begin{gkverse}
ἦλθες, \dagger{}καὶ\dagger{} ἐπόησας, ἔγω δέ σ’ ἐμαιόμαν,
ὂν δ’ ἔψυξας ἔμαν φρένα καιομέναν πόθωι.
\end{gkverse}

\begin{verse}
\ldots{} vieste, e eu ansiava por ti -- \\*
me esfriaste o peito que queimava com desejo \ldots{}
\end{verse}

\medskip

{\paragraph{Comentário} Preservado na \textit{Carta a Jâmblico} (183), nome do filósofo neoplatonista do
século \textsc{iv} d.C., de autoria do imperador romano Juliano, seu contemporâneo, o
breve fragmento elabora"-se a partir de outro motivo comum na concepção de
\textit{éros}: o de que paixão é fogo; logo, sua satisfação é como o apagar de
um incêndio, como diz o \textit{eu} feminino ao \textit{tu}, abaixo.}

\pagebreak
\section{Fragmento 51}

\begin{gkverse}
οὐκ οἶδ’ ὄττι θέω δύο μοι τὰ νοήματα
\end{gkverse}

\begin{verse}
\ldots{} não sei que faço: duas as minhas mentes \ldots{}
\end{verse}

\medskip

{\paragraph{Comentário} Como no Fr.\,36, temos aqui, possivelmente, o binômio paixão"-loucura, que fratura o amador. A
fonte do fragmento é o tratado \textit{Das negativas} (23), de Crísipo (filósofo
estoico, século \textsc{iii} a.C.).}   


\chapter{Sono}

\section{Fragmento 46}

\begin{gkverse}
ἔγω δ’ ἐπὶ μολθάκαν\\
τύλαν <κασ>πολέω \dagger{}μέλεα κἂν μὲν τετύλαγκας [ἀσπόλεα\dagger{}
\end{gkverse}

\begin{verse}
\ldots{} mas eu, sobre macias\\
almofadas quero deitar meus membros \ldots{}
\end{verse}

\medskip

{\paragraph{Comentário} Eis o que de legível há no texto corrompido do fragmento citado por Herodiano (fins do século \textsc{ii} d.C.), no tratado \textit{Sobre palavras anômalas} (\textit{beta} 39), que o atribui ao Livro \textsc{ii} da compilação alexandrina da obra de Safo, e com ele ilustra a palavra \textit{týle}, ``almofada'', em que a \textit{persona} talvez repouse o corpo. Mesmo em tão exíguo texto vemos a dimensão profundamente sensorial, tátil aqui, da dicção sáfica, cuja força está evidenciada nas canções.}



\pagebreak
\section{Fragmento 63}

\begin{gkverse}
Ὄνοιρε μελαινα[\\
φ[ο]ίταις ὄτα τ’ ὔπνος [

γλύκυς̣ θ̣[έ]ο̣ς, ἦ δεῖν’ ὀνίας μ[\\
ζὰ χῶρις ἔχην τὰν δυναμ[

ἔλπις δέ μ’ ἔχει μὴ πεδέχη[ν\\
μηδὲν μακάρων ἐλ̣[

ο̣ὐ̣ γάρ κ’ ἔον οὔτω[..́   \\
ἀθύρματα κα.[

γένοιτο δέ μοι[
\end{gkverse}

\begin{verse}
Ó Ôneiros, negra \ldots{}\\
vagueias, quando quer que o sono \ldots{}

doce deus, de fato terrivelmente angústias \ldots{}\\
manter \ldots{} apartada [tua] potência \ldots{}

mas a expectação me sustém, de não partilhar \ldots{}\\
nem \ldots{}  dos venturosos \ldots{}

pois eu não seria assim \ldots{}\\
berloques \ldots{}

viria a ser a mim \ldots{}
\end{verse}

\medskip

{\paragraph{Comentário} Tendo por fonte o \textit{Papiro de Oxirrinco} 1787, o mesmo do Fr.\,65, o 63 tem seu início preservado, com invocação em prece a Ôneiros, o deus Sonho, que move o sonho, \textit{ónar}, referido antes no Fr.\,134, e se associa ao sono ou ao deus que é seu soberano -- \textit{Hýpnos} é o nome de ambos. Este ou aquele deus é qualificado como \textit{glýkys}. ``doce'',\footnote{Verso 3.} e perturbações emocionais e expectativa compõem a cena que não podemos recompor.}



\pagebreak
\section{Fragmento 149}

\begin{gkverse}
ὄτα πάννυχος ἄσφι κατάγρει
\end{gkverse}

\begin{verse}
\ldots{} quando [o sono] de noite afora captura os seus [olhos] \ldots{}
\end{verse}

\medskip

{\paragraph{Comentário} Citado pelo gramático Apolônio Díscolo na discussão sobre pronomes, mesma fonte do Fr.\,32, o 149 traz apenas um verso que de novo canta o que se desenrola no correr da noite, \textit{pánnykhos}, como antes o Fr.\,23, e outros adiante.}


\section{Fragmento 151}

\begin{gkverse}
ὀφθάλμοις δὲ μέλαις νύκτος ἄωρος
\end{gkverse}

\begin{verse}
\ldots{} sobre os olhos sono da negra noite \ldots{}
\end{verse}

\medskip

 {\paragraph{Comentário} Citado no \textit{Etimológico Magno} (século \textsc{xii}), no verbete do termo sinônimo a \textit{hýpnos}, \textit{áōros}, o fragmento tem só um verso, sobre o sono, como o anterior.}

\pagebreak
\section{Fragmento 168\textsc{b}}

\begin{gkverse}
Δέδυκε μὲν ἀ σελάννα\\
καὶ Πληΐαδες μέσαι δὲ\\
νύκτες, παρὰ δ’ ἔρχετ’ ὤρα,\\
ἔγω δὲ μόνα κατεύδω.
\end{gkverse}

\begin{verse}
Imergiu a lua,\\
também as Plêiades; é\\
meia-noite, vai-se o tempo,\\
e eu sozinha durmo \ldots{}
\end{verse}

\medskip

{\paragraph{Comentário} Heféstion (11.5), referido já no comentário ao Fr.\,102, é a fonte principal do 168\,\textsc{b}, em que a natureza enquadra a imagem da solidão da voz poética, talvez devido a uma separação erótica.}


\chapter{Viagem}

\section{Fragmento 20}

\begin{gkverse}
\textnormal{[\textit{versos 1 e 22--5: ilegíveis e lacunares}]}

] ε, γάνος δὲ και̣..[\\
τ]ύχαι σὺν ἔσλαι\\
         λί]μενος κρέτησαι\\
  γ]ᾶς μελαίνας
    ]\\
    ]έλοισι ναῦται\\
  ]μ̣εγάλαις ἀήται[ς\\
  ]α κἀπὶ χέρσω\\
    ]\\
  .́ ]μοθεν πλέοι.\\
  ]δε τὰ φόρτι’ εἰκ[

\textnormal{[\textit{versos 14--9: ilegíveis e lacunares}]}

]ι̣ν ἔργα\\
] χέρσω [
\end{gkverse}

\chapter*{}
\section*{}


\begin{verse}
\ldots{} brilho\\
\ldots{}\\
\ldots{} com boa sorte \ldots{}\\
\ldots{} porto alcançar \ldots{} \\
\ldots{} da terra negra\\
\ldots{} \\
\ldots{} nautas\\
\ldots{} grandes ventos \ldots{}\\
e em terra firme\\
\ldots{}\\
\ldots{} navegar \ldots{}\\
\ldots{} carga \ldots{}\\
\ldots{}\\
\ldots{} trabalhos \ldots{}\\
\ldots{}\\
\ldots{} terra firme \ldots{}
\end{verse}

\medskip

{\paragraph{Comentário} Tendo por fonte o \textit{Papiro de Oxirrinco} 1231, o mesmo do Fr.\,15, o 20 traz uma viagem marinha difícil, ou usa essa imagem para falar das perturbações políticas da \textit{pólis}, no motivo conhecido como da ``nau do Estado''. Mas tão precário texto não permite navegar num ou noutro sentido com segurança.}

\chapter{Imagens da natureza}

\paragraph{\textsc{nota introdutória}}
Como observei ao comentar o Fr.\,96 de Safo, é notável em sua mélica o trabalho
da linguagem com imagens da natureza, de grande presença e força não raro
metafóricas.

\section{Fragmento 34}

\begin{gkverse}
ἄστερες μὲν ἀμφὶ κάλαν σελάνναν\\
ἂψ ἀπυκρύπτοισι φάεννον εἶδος,\\
ὄπποτα πλήθοισα μάλιστα λάμπη\\
γᾶν\ldots{}\\
\hspace*{23mm}\adforn{47}\\
ἀργυρία
\end{gkverse}

\begin{verse}
\ldots{} e as estrelas, em volta da bela lua,\\
de novo ocultam sua luzidia forma,\\
quando plena ao máximo ilumina\\
a terra \ldots{}\\
\hspace*{23mm}\adforn{47}\\
\ldots{} argêntea
\end{verse}

\medskip

{\paragraph{Comentário} A principal fonte desse fragmento concentrado no corpo celeste feminino da lua, antes cantada no Fr.\,96, é
um comentário de Eustácio (bispo de Tessalônica, século \textsc{xii} d.C.) à
\textit{Ilíada},\footnote{\textsc{viii}, 555.} em passo sobre o brilho dos astros no enredor da destacada lua.} 


\pagebreak
\section{Fragmento 42}

\begin{gkverse}
ταῖσι <--> ψῦχρος μὲν ἔγεντο θῦμος\\
πὰρ δ’ ἴεισι τὰ πτέρα
\end{gkverse}

\begin{verse}
\ldots{} e deles se tornou frio o peito,\\*
e suas asas se afrouxaram \ldots{}
\end{verse}

\medskip

{\paragraph{Comentário} Segundo a fonte do fragmento, um escólio (comentário antigo) a Píndaro
(\textit{Ode pítica} 1), poeta mélico da virada dos séculos \textsc{vi}--\textsc{v} a.C.,
``Safo diz dos pombos'' o que lemos nos versos que cita.}


\section{Fragmento 101\textsc{a}}

\begin{gkverse}
πτερύγων δ’ ὔπα\\
κακχέει λιγύραν ἀοίδαν,\\
ὄπποτα φλόγιον \dagger{}καθέ-\\
ταν\dagger{} ἐπιπτάμενον \dagger{}καταυδείη\dagger{}
\end{gkverse}

\begin{verse}
\ldots{} e de sob suas asas\\*
verte clara canção,\\*
quando o flamejante [verão] \ldots{}\\
\ldots{} voando \ldots{}
\end{verse}

\medskip

{\paragraph{Comentário} A fonte do fragmento, que também cita o 101, é Demétrio; ele afirma que seus versos tratam da cigarra. O texto apresenta"-se bastante corrompido da última palavra do verso 3 ao 4; traduzo, pois, os versos compreensíveis. Mas há interessante possibilidade de emenda a um dos termos problemáticos, o último, \textit{kataudeíē}, que lhe daria o sentido de ``tocar o aulo'', instrumento de sopro símil ao oboé, referido no Fr.\,44.}



\pagebreak
\section{Fragmento 135}

\begin{gkverse}
Τί με Πανδίονις, ὦ Εἴρανα, χελίδων\ldots{};
\end{gkverse}

\begin{verse}
Por que, ó Irene, a Pandionida, a andorinha a mim \ldots{}?
\end{verse}

\medskip

{\paragraph{Comentário} Citado em Heféstion, fonte do Fr.\,102, o 135 dirige"-se a Irene, também mencionada no Fr.\,91, para falar ``da filha de Pandião'', mítico rei de Atenas, chamada Procne, que se transformou em andorinha. Esposa de Tereu, rei da Trácia, e irmã de Filomela, que se transformou em rouxinol. Na tradição, conta"-se que Tereu, desejando"-a, violou a cunhada e por isso as irmãs dele se vingaram, matando Ítis, o próprio filho de Procne com o rei. Daí o canto triste da andorinha, ave que a \textit{persona} canta.}


\section{Fragmento 136}

\begin{gkverse}
ἦρος ἄγγελος ἰμερόφωνος ἀήδων
\end{gkverse}

\begin{verse}
\ldots{} mensageiro da primavera, o rouxinol de desejável voz \ldots{}
\end{verse}

\medskip

{\paragraph{Comentário} Segundo a fonte do fragmento, um escólio a Sófocles (século \textsc{v} a.C.) e à \textit{Electra},\footnote{Verso 149.} Safo já falava como na tragédia dos
rouxinóis.}

\section{Fragmento 143}

\begin{gkverse}
χρύσειοι <δ’> ἐρέβινθοι ἐπ’ ἀϊόνων ἐφύοντο
\end{gkverse}

\begin{verse}
\ldots{} e áureos grãos"-de"-bico cresciam nas margens \ldots{}
\end{verse}

\medskip

{\paragraph{Comentário} Ateneu (2.54\,\textsc{f}), fonte do Fr.\,141 já aqui visto, atribui a Safo a descrição.}


\pagebreak
\section{Fragmento 146}

\begin{gkverse}
μήτε μοι μέλι μήτε μέλισσα
\end{gkverse}

\begin{verse}
\ldots{} para mim, nem mel, nem abelha \ldots{}
\end{verse}

\medskip

{\paragraph{Comentário} O gramático Trifo, da era romana de Augusto (séculos \textsc{i} a.C.--\textsc{i} d.C.), 
cita como proverbiais estas palavras de Safo, no tratado \textit{Figuras de expressão} (25).
Notáveis são as reiterada assonâncias e aliterações na sequência de palavras, que costuram as sílabas pelo sentido e pelo som; translitero o verso: \textit{méte moi méli méte mélissa}.}


\section{Fragmento 167}

\begin{gkverse}
ὠίω πόλυ λευκότερον
\end{gkverse}

\begin{verse}
\ldots{} muito mais branco do que um ovo \ldots{}
\end{verse}

\medskip

{\paragraph{Comentário} Tendo por fonte Ateneu (2, 57\,\textsc{d}), que também preservou o Fr.\,141 e vários outros, as poucas palavras do 167 são dadas na sequência da citação do Fr.\,166, sobre Leda e o ovo de que nasce Helena, em certa tradição mítica. Aqui, o enfoque é na cor, e o ovo dá o padrão comparativo ao superlativo cujo uso é digno de atenção em Safo.}

\EP[2]

\section{Fragmento 168\textsc{c}}

\begin{gkverse}
ποικίλλεται\\
    γαῖα πολυστέφανος 
\end{gkverse}

\begin{verse}
\ldots{} ela adornou,\\
a terra de multiguirlandas \ldots{}
\end{verse}

\medskip

{\paragraph{Comentário} A fonte do fragmento é, como no caso do 101\textsc{a}, Demétrio, que trata do charme na expressão.}



\chapter[O cantar, as canções e as companheiras]{O cantar, as canções e as companheiras}

\vspace*{-1cm}


\paragraph{\textsc{nota introdutória}}
É importante na mélica sáfica a metalinguagem -- o poeta a falar de seu
poetar, a criticar o poetar de outros, como no Fr.\,55, a tratar de sua
audiência, a tratar de seus temas. Nesse item, insiro alguns fragmentos que
ilustram isso -- aqueles que estão mais legíveis no \textit{corpus} de Safo.


\section{Fragmento 70}

\begin{gkverse}
\textnormal{[\textit{versos 1--2 e 14: ilegíveis e lacunares}]}

]ν ̣δ’ εἶμ’ ε[\\

\textnormal{[\textit{versos 4--8: ilegíveis e lacunares}]}\\

]αρμονίας δ̣[\\
         ]αθην χόρον, ἄα[\\
   ]δ̣ε λίγηα.[\\
   ]ατόν σφι̣[\\
  ]παντεσσι[

\end{gkverse}

\begin{verse}
\ldots{} e vou \ldots{}\\
\ldots{}\\
\ldots{} de harmonia \ldots{}\\
\ldots{} dança \ldots{}\\
\ldots{} clara [canção] \ldots{}\\
\ldots{}\\
\ldots{} a todos \ldots{} 
\end{verse}

%\medskip

{\paragraph{Comentário} Preservado no \textit{Papiro de Oxirrinco} 1787, fonte do Fr.\,65, o 70 traz poucas palavras nos versos precários. Destaco \textit{harmonía} e \textit{khorón}, o coro que dança e, por extensão, canta, possivelmente referente ao coro de Safo que faz a \textit{performance}.}

\pagebreak
\section{Fragmento 118}

\begin{gkverse}
ἄγι δὴ χέλυ δῖα \dagger{}μοι λέγε\dagger{}\\
φωνάεσσα \dagger{}δὲ γίνεο\dagger{}
\end{gkverse}

\begin{verse}
\ldots{} vem, divina lira, fala-me\\*
e torna-te dotada de voz \ldots{}
\end{verse}

\medskip

{\paragraph{Comentário} Preservado no \textit{Sobre os tipos de estilo} (2.4), de Hermógenes (século \textsc{ii} d.C.),
o fragmento, segundo essa fonte, traz"-nos o momento em que ``Safo
questiona a sua lira, e a lira assim lhe responde''.}


\section{Fragmento 153}

\begin{gkverse}
πάρθενον ἀδύφωνον
\end{gkverse}

\begin{verse}
\ldots{} virgem dulcífona \ldots{}
\end{verse}

\medskip

{\paragraph{Comentário} Citado por Atílio Fortunato (metricista, século \textsc{iv} d.C.), na sua \textit{Arte} (28), o fragmento tem apenas duas palavras que trazem aos olhos e ouvidos a \textit{performance} do coro de meninas de Safo, ao destacar a habilidade de uma \textit{parthénos}, uma das moças, e o prazer erótico, sensorial de sua voz \textit{adýphōnon}.}


\pagebreak
\section{Fragmento 154}

\begin{gkverse}
Πλήρης μὲν ἐφαίνετ’ ἀ σελάν<ν>α\\
αἰ δ’ ὠς περὶ βῶμον ἐστάθησαν
\end{gkverse}

\begin{verse}
Em plenitude brilhava a lua, \\*
quando elas em volta do altar se postaram \ldots{}
\end{verse}

\medskip

{\paragraph{Comentário} Heféstion (11.3), fonte do Fr.\,102, citou os versos abaixo, provavelmente de abertura
de uma canção que giraria em torno de um rito sacrificial, à noite. Atividades rituais são características de associações corais femininas, como tenho frisado.}


\section{Fragmento 156}

\begin{gkverse}
πόλυ πάκτιδος ἀδυμελεστέρα\ldots{}\\
χρύσω χρυσοτέρα\ldots{}
\end{gkverse}

\begin{verse}
\ldots{} muito mais dulcissonante que a harpa \ldots{}\\*
\ldots{} mais áurea que o ouro \ldots{}
\end{verse}

\medskip

{\paragraph{Comentário} A fonte desse fragmento é, como do 101\textsc{a}, Demétrio (161\,\textsc{s}), que recorda o uso da
hipérbole.}

\pagebreak
\section{Fragmento 160}

\begin{gkverse}
\vinphantom{xxxxxxxxxxxxxx}τάδε νῦν ἐταίραις\\
ταὶς ἔμαις \dagger{}τέρπνα\dagger{} κάλως ἀείσω
\end{gkverse}

\begin{verse}
\ldots{} agora às minhas\\*
companheiras estas coisas aprazíveis belamente cantarei \ldots{}
\end{verse}

\medskip

{\paragraph{Comentário} A fonte do fragmento, como do 141, é Ateneu (13.571\,\textsc{d}). Nele ocorre o termo ``companheira'', \textit{étaira}, no plural.
Para alguns, esse termo em Safo levou à leitura de que a poeta estaria associada a uma \textit{hetaireía} feminina. Tratar"-se"-ia de um grupo aristocrático de iguais, ligados por laços de amizade e políticos. Porém, conforme enfatizei na introdução, não há nenhuma evidência de que existiria esse tipo de confraria típica do universo masculino no feminino; deslocá"-la a este é movimento de espelhamento carente de fundamentação. Logo, o que de mais seguro podemos dizer é que o termo, marcadamente afetivo, denota uma relação amistosa entre coevos, e é usado também nos Frs.\,126 e 142. Neste Fr.\,160, como no 126, indicaria uma fala dirigida às coreutas do grupo de Safo; e a \textit{persona} é decerto uma delas falando às amigas, suas iguais. Fica anunciada, em chave metalinguística, a bela \textit{performance}
de uma canção prazerosa às suas audiências interna e externa.}


\pagebreak
\section{Elogio}

\paragraph{\textsc{nota introdutória}}
O elogio e a censura consistem num dos pares opostos mais trabalhados na poesia
grega antiga, podendo mesmo ser tomados como categorias de organização dos seus gêneros.
Em Safo, há fragmentos que se concentram no
elogio, outros na
censura. Mesmo alguns textos demasiado mutilados e lacunares, com poucas palavras legíveis, parecem caminhar no sentido do binômio e de seus polos, como os Frs. 3 e 4 -- naquele, há referências à nobreza e beleza, à vileza, dores e ``censura'', \textit{óneidos};\footnote{Verso 5.} neste, talvez à beleza do ``rosto''\footnote{Verso 7.} e ao brilho.\footnote{Verso 6.} O poder de um e de outro só pode ser
minimamente compreendido se lembrarmos que a cultura da Grécia arcaica e
clássica pauta"-se pela ideia da vergonha; ou seja, o olhar público é
fundamental para a definição do lugar social ocupado pelo indivíduo; essa
definição repercute na sua linhagem e na sua comunidade. 

\section{Fragmento 41}

\begin{gkverse}
ταὶς κάλαισ’ ὔμμιν <τὸ> νόημμα τὦμον\\
       οὐ διάμειπτον
\end{gkverse}

\begin{verse}
\ldots{} para vós, as belas, meus pensamentos\\*
não mudaram \ldots{}
\end{verse}

\medskip

{\paragraph{Comentário} Apolônio Díscolo,\footnote{\textit{Sobre os pronomes} 124\,\textsc{c}.} fonte dos Frs.~32 e 33, cita os versos acima.
As destinatárias são elogiadas, e provavelmente os pensamentos são positivos e talvez expressem como a \textit{persona} as vê.}


\pagebreak
\section{Fragmento 82\textsc{a}}

\begin{gkverse}
Εὐμορφοτέρα Μνασιδίκα τὰς ἀπάλας Γυρίννως
\end{gkverse}

\begin{verse}
Mnasidica, mais formosa que a tenra Girino \ldots{}
\end{verse}

\medskip

{\paragraph{Comentário} Heféstion (11.5), citado aqui como fonte do Fr.\,102, preservou o 82\,\textsc{a}, que traz
um elogio à beleza física de uma moça, realçada pela comparação com outra
também louvada por um adjetivo, ``tenra'', \textit{apálas}, de marcada
carga erótica e empregado reiteradamente nos fragmentos da mélica sáfica que
adentram a esfera do erotismo, como o 94.}

\section{Fragmento 122}

\begin{gkverse}
ἄνθε’ ἀμέργοισαν παῖδ’ ἄγαν ἀπάλαν
\end{gkverse}

\begin{verse}
\ldots{} tenra virgem, colhendo flores \ldots{}
\end{verse}

\medskip

{\paragraph{Comentário} A fonte desse fragmento, como do 141 e de vários outros, Ateneu (12.554\,\textsc{b}), comenta: 

\begin{quote}
Pois é natural
que os que pensam ser belos e maduros colham flores. Por isso, dizem que
Perséfone e suas companheiras colhem flores, e Safo diz ter visto 
[citação do fragmento].
\end{quote}

 Ou seja, Ateneu sugere a aproximação entre as imagens
da menina virgem, filha da deusa Deméter, e da virgem ``tenra'', \textit{apálan}, de modo a ressaltar nesta a beleza e a
sensualidade. Vale lembrar que Perséfone, como conta o \textit{Hino homérico a Deméter} 
(início do século \textsc{vi} a.C), foi abduzida por Hades exatamente
quando brincava a colher flores num prado primaveril, ignorante dos perigos que
espreitam as belas e sensuais meninas virgens, \textit{parthénoi}. Cabe frisar, como fiz na introdução, que a virgindade delas,
na concepção grega, implica
apenas que estão na fase transitória entre infância e idade
adulta, a caminho do casamento, quando passam a ter vida sexual e, portanto, integram"-se
de vez à sociedade à qual pertencem, deixando para trás a dimensão algo
selvagem e indomada da fase virginal.}



\section{Censura}

\paragraph{\textsc{nota introdutória}}
Não temos no \textit{corpus }da mélica sáfica textos pertencentes ao gênero
poético do jambo, que no correr dos tempos acaba se configurando em essência como poesia de
ataque, de vituperação, de invectiva. Há, porém,
fragmentos de caráter jâmbico mais ou menos evidente; afinal, se um gênero
acabou por ter na vituperação seu elemento definidor, este não lhe é exclusivo,
mas integra vários da poesia grega antiga, desde a \textit{Ilíada}. Eis
alguns exemplos sáficos.

\section{Fragmento 3} 

\begin{gkverse}
]δώσην\\
κλ]ύτων μέντ’ ἐπ[\\
κ]άλων κἄσλων, σ[\\
.΄]οις, λύπης τέμ[\\
 ]μ’ ὄνειδος\\
]οιδήσαισ. ἐπιτα̣[\\
].΄αν, ἄσαιο. τὸ γὰρ [\\
]μον οὐκοὔτω μ[\\
] διάκηται,\\
  ] μη̣δ̣[\qquad    ]αζε,\\
  ]χις, συνίημ[\\ 
].ης κακότατο[ς\\
  ]μεν\\
  ]ν ἀτέραις με[\\
]η φρένας, εὔ[
\end{gkverse}
\pagebreak
\begin{verse}
\ldots{} dar \ldots{}\\
\ldots{} de ínclitos \ldots{}\\
\ldots{} de belos e também nobres \ldots{}\\
\ldots{} me vexas \ldots{}\\
\ldots{} censura \ldots{}\\
\ldots{} pois \ldots{}\\
\ldots{} mão desse modo \ldots{}\\
\ldots{} ele/a disposto/a \ldots{}\\
\ldots{}\\
\ldots{} compreendo \ldots{}\\
\ldots{} o mais vil \ldots{}\\
\ldots{}\\
\ldots{} as outras \ldots{}\\
\ldots{} sensos, \ldots{}
\end{verse}

\medskip

\paragraph{Comentário}
Preservado no precário \textit{Papiro de Berlim} (século \textsc{vii} d.C.), somado ao \textit{Papiro de Oxirrinco} 424 (século \textsc{iii} d.C.), o fragmento parece falar de amizade, de censura, \textit{óneidos}, em vexar -- e nessa atmosfera se inserem a \textit{persona} que é"-nos opaca tanto quanto o \textit{tu} indicado na  forma verbal do verso 4 (\textit{lýpēs}), e reflete sobre o bom/\,belo, \textit{kalós}, o nobre, \textit{esthlós}, e o feio/\,vil, \textit{kakós}, noções ético"-estéticas muito caras ao pensamento arcaico e seus valores tradicionais.


\pagebreak
%\section{Imortalidade}
\section{Fragmento 55}

\begin{gkverse}
κατθάνοισα δὲ κείσηι οὐδέ ποτα μναμοσύνα σέθεν\\
ἔσσετ’ οὐδὲ \dagger{}ποκ’\dagger{} ὔστερον οὐ γὰρ πεδέχηις βρόδων\\
τὼν ἐκ Πιερίας, ἀλλ’ ἀφάνης κἀν Ἀίδα δόμωι\\
φοιτάσηις πεδ’ ἀμαύρων νεκύων ἐκπεποταμένα.
\end{gkverse}

\begin{verse}
Morta jazerás, nem memória alguma futura\\
de ti haverá, nem desejo, pois não partilhas das rosas\\
de Piéria; mas invisível na casa de Hades\\*
vaguearás esvoaçada entre vagos corpos \ldots{}
\end{verse}

\medskip

{\paragraph{Comentário} Preservado na \textit{Antologia} (3.4.12) de Estobeu (século \textsc{v} d.C.), o fragmento traz um dos muitos momentos em que os poetas refletem em seus versos
sobre o poetar e a própria poesia, que vai se configurar no imaginário grego
como caminho para a imortalidade do nome.\footnote{Explorei o tema na poesia arcaica (Ragusa, 2018, pp. 143--152).} A linguagem é de ataque, de
invectiva, e dirigida a um \textit{tu} feminino que pensa ser (hábil) poeta, mas
não o é. Por isso, sua dupla morte ao descer ao mundo, reino de
Hades, tornando"-se não mais alcançável sua figura aos olhos dos vivos, e
apagada sua existência da memória destes.}


\pagebreak
\section{Fragmento 56}

\begin{gkverse}
οὐδ’ ἴαν δοκίμωμι προσίδοισαν φάος ἀλίω\\
ἔσσεσθαι σοφίαν πάρθενον εἰς οὐδένα πω χρόνον\\
τεαύταν
\end{gkverse}

\begin{verse}
\ldots{} não imagino que uma virgem, após ver a luz do sol,\\
terá em algum tempo futuro tal habilidade \ldots{}
\end{verse}

\medskip

{\paragraph{Comentário} Preservado em Crísipo (13), como o 51, o Fr.\,56 parece referir"-se à habilidade
poética de uma virgem, \textit{parthénos}, que em vida estaria condenada à
mediocridade.}


\section{Fragmento 57}

\begin{gkverse}
τίς δ’ ἀγροΐωτις θέλγει νόον\ldots{}\\
ἀγροΐωτιν ἐπεμμένα στόλαν\ldots{}\\
οὐκ ἐπισταμένα τὰ βράκε’ ἔλκην ἐπὶ τὼν σφύρων;
\end{gkverse}

\begin{verse}
\ldots{} e quem é a rusticona que encantou a mente \ldots{}\\*
vestida em rústica veste \ldots{}\\*
sem saber como erguer seus trapos aos tornozelos?
\end{verse}

\medskip

{\paragraph{Comentário} Ao citar os versos acima, em que roupa, atitude e modos de uma figura feminina são
duramente censurados, Ateneu (1.21\,\textsc{b}\,\textsc{c}), fonte também do Fr.\,141, declara: ``Safo
escarnece Andrômeda'', personagem abordada nos comentários aos Frs.\,130 e 133, talvez uma líder de coro de meninas, rival do de Safo.}


\pagebreak
\section{Fragmento 68} 

\begin{gkverse}
]ι ̣γάρ μ’ ἀπὺ τὰς ἐ[\\
ὔ]μως δ’ ἔγεν[το\\
   ] ἴσαν θέοισιν\\
 ]ασαν ἀλίτρα[\\
        Ἀν]δρομέδαν[.].αξ[\\
   ]αρ[...]α μά̣κα̣[ιρ] α\\
 ]ε̣ον δὲ τρόπον α[.].ύνη[\\
      ] κ̣όρο̣ν οὐ κατισχε.[\\
]κ̣α[...]. Τυνδαρίδαι[ς\\
]ασυ[.]...κα[.] χαρίεντ’ ἀ[\\
]κ’ ἄδολον [μ]ηκέτι συν[\\
] Μεγάρα.[...]ν̣α[...]α[

\textnormal{[\textit{versos 1--2 e 4--7: ilegíveis e lacunares}]}

\end{gkverse}

\begin{verse}
\ldots{} pois a mim desta \ldots{}\\
\ldots{}contudo tornou-se \ldots{}\\
\ldots{} par dos deuses \ldots{}\\
\ldots{} culpada \ldots{}\\
\ldots{} Andrômeda \ldots{}\\
\ldots{} venturosa \ldots{}\\
\ldots{} modo \ldots{}\\
\ldots{} ganância não foi refreada \ldots{}\\
\ldots{} Tindaridas \ldots{}\\
\ldots{} graciosa \ldots{}\\
\ldots{} sem dolo não mais \ldots{}
\end{verse}

\pagebreak
{\paragraph{Comentário} Preservado no duramente danificado rolo \textit{Papiro de Oxirrinco} 1787, em que se acham o Fr.\,65 e outros, o 68 traz, entre versos que mal se discernem, de novo a figura de Andrômeda, além de uma virgem associada ao grupo e Safo nos testemunhos antigos, Megara.  A linguagem parece lidar com a censura, e refere deuses ao nomeá"-los, \textit{théoisin},\footnote{Verso 3.} e pelo adjetivo \textit{mákaira}, ``venturosa'',\footnote{Verso 6.} que noutros fragmentos traduzi, e que só aos imortais pode ser atribuído -- no Fr.\,1, é dado a Afrodite. No verso 8, há algo talvez sobre o conter da ganância,\footnote{Ferrari, 2011, p. 158.} a qual é sempre condenada. Noto ainda a presença das Tindaridas, Helena e Clitemnestra, as filhas do rei de Esparta, Tíndaro, e Leda, casal de que falei no comentário ao Fr.\,166. Tal presença reforça a possibilidade de censura, pois as duas figuras míticas femininas sob tal signo se acham, ambas adúlteras, desertora -- aquela -- e assassina -- esta -- de seus respectivos maridos, os filhos de Atreu, Menelau e Agamêmnon. E por fim, veja"-se o adjetivo \textit{ádolon}, ``sem dolo'',\footnote{Verso 11.} que traz para a canção algo que se relacione justamente a um engano, um ardil.}



\pagebreak
\section{Fragmento 71}

\begin{gkverse}
]μισσε Μίκα\\
]ελα[..ἀλ]λά σ’ ἔγωὐκ ἐάσω\\
         ]ν̣ φιλότ[ατ’] ἤλεο Πενθιλήαν̣[\\
]δα κα̣[κό]τροπ’, ἄμμα[\\
    ] μέλ̣[ος] τι γλύκερον .[\\
    ]α μελλιχόφων[ος\\
    ]δει, λίγυραι δ’ ἄη[\\
        ]δροσ[ό]εσσα[
\end{gkverse}

\begin{verse}
\ldots{} Mica \ldots{}\\
\ldots{} mas eu não permitirei a ti \ldots{}\\
\ldots{} escolheste a amizade de uma Pentilida \ldots{}\\
\ldots{} ó maligna \ldots{}\\
\ldots{} uma doce canção \ldots{}\\
\ldots{} voz de mel \ldots{}\\
\ldots{} e claras [brisas] \ldots{}\\*
\ldots{} orvalhada \ldots{}
\end{verse}

\medskip

{\paragraph{Comentário} Em dado momento das convulsões políticas internas e da sucessão de regimes
tirânicos pós"-queda da tradicional aristocracia dos Pentilidas, surge Pítaco, que se tornou governante de
Mitilene, mas fez aliança com aquela linhagem por meio do casamento,
canta Alceu no Fr.\,70, envolvido nos conflitos,
rompendo com as facções aristocráticas rivais aos Pentilidas, que tinham apoiado, entre as quais, a
do grupo do poeta e de Safo. Acima, é notável como a traição de 
Mica, figura feminina, também se firma com a aliança da amizade com os Pentilidas. Termos
pesados são seguidos por referências luminosas, prazerosas, eróticas, mas o
fragmento, cuja precária fonte é o \textit{Papiro de Oxirrinco} 1787, o mesmo do Fr.\,65 e de outros, quase nada revela da canção perdida.}


\pagebreak
\section{Fragmento 91}

\begin{gkverse}
ἀσαροτέρας οὐδάμα πω Εἴρανα, σέθεν τύχοισαν
\end{gkverse}

\begin{verse}
\ldots{} nunca tendo encontrado outra mais abjeta que tu, Irena \ldots{}
\end{verse}

\medskip

{\paragraph{Comentário} A fonte do fragmento, como do 102, é Heféstion (11.5).
A mesma figura feminina é nomeada no Fr.\,135.}


\section{Fragmento 137}

\begin{gkverse}
θέλω τί τ’ εἴπην, ἀλλά με κωλύει\\
αἴδως\ldots{}
\end{gkverse}

\begin{verse}
\ldots{} quero algo te dizer, mas me impede\\
o pudor \ldots{}
\end{verse}

%\medskip

\paragraph{Comentário} \textls[-20]{Aristóteles (século \textsc{iv} a.C.), na \textit{Retórica} (1367\,\textsc{a}), afirma: ``Os homens se envergonham de dizer, de fazer ou de pretender fazer vilezas, exatamente \EP[2]
como Safo em sua resposta, quando Alceu falou: [citação do fragmento].''
O filósofo pressupõe um poema de Alceu respondido por um de Safo, esclarece um anônimo comentário antigo (escólio). Os
dois poetas, comentei na introdução, podem ter se conhecido na Mitilene arcaica, e
os antigos sempre os aproximavam, inclusive na iconografia.
Em outros entendimentos, os versos acima, que a edição adotada nesta antologia, a de Voigt, dá como de Safo, são atribuídos a Alceu, à poeta são atribuídos os versos sequenciais, que Voigt indica, mas sob suspeita: ``\ldots{} mas se ansiasses pelo honrado e pelo belo,/ e tua língua não se animasse a dizer algo vil,/ a vergonha não tomaria teus olhos,/ mas dirias tua fala\ldots{}''. Assim é em edições e traduções do tratado aristotélico, e em Ferrari.}\footnote{2010, p. 75.} Na edição Lobel"-Page\footnote{De 1955.} de Safo, que era a de autoridade antes de Voigt, tais versos suspeitos sequer são transcritos. A questão é, portanto, bastante espinhosa. Fiquemos, pois, com os versos que Voigt autoriza como sáficos.

\pagebreak
\section{Fragmento 144}

\begin{gkverse}
μάλα δὴ κεκορημένοις\\
Γόργως
\end{gkverse}

\begin{verse}
\ldots{} bem saturada de Gorgo \ldots{}
\end{verse}

\medskip

{\paragraph{Comentário} A fonte desse fragmento é o autor que também preservou o já visto Fr.\,46, Herodiano, no tratado
\textit{Sobre a declinação dos substantivos}. Nele, a personagem feminina é
criticada por uma voz sem qualquer identificação. Mas nos testemunhos, além de Andrômeda, circula também o nome de Gorgo, alvo de censura na canção, como poeta rival de Safo. Não temos evidência dessas poetas rivais que não sejam os testemunhos antigos. Não parece, contudo, implausível que Mitilene tivesse outras poetas líderes de grupos corais, como Safo, dadas as evidências sobre os numerosos  grupos corais femininos disseminados pelas \textit{póleis} gregas.}

\section{Fragmento 155}

\begin{gkverse}
πόλλα μοι τὰν Πωλυανάκτιδα παῖδα χαίρην
\end{gkverse}

\begin{verse}
\ldots{} minhas muitas saudações à filha de \rlap{Polianactides \ldots{}}
\end{verse}

\medskip

{\paragraph{Comentário} Para Máximo de Tiro (18), fonte desse fragmento, como do 47, Safo fala com ironia, tal qual
Sócrates fala a Íon na abertura do diálogo de Platão que leva o nome desse
rapsodo, \textit{Íon}, ou \textit{Sobre a inspiração poética}. Ora, a ironia
liga"-se à censura, tanto mais no paralelo socrático sugerido.
E o alvo da censura seria novamente Gorgo,\footnote{Ferrari, 2010, p. 81.} considerados os testemunhos antigos.}



\chapter[Epitalâmios: canções de casamento]{Epitalâmios: canções de casamento}

\paragraph{\textsc{nota introdutória}}
``Em Homero, a canção é um acompanhamento sempre presente no casamento, e a frequente menção de canções nupciais por autores de todos os períodos na literatura grega antiga atesta a continuada importância da canção para a cerimônia de casamento na Grécia antiga''.\footnote{Hague, 1983, p. 131.} Na era arcaica, são sáficas as canções que nos chegaram. Assim como o 112, os fragmentos arrolados neste tópico compõem o pequeno grupo
de representantes do subgênero mélico que teria sido compilado no nono livro de
Safo na Biblioteca de Alexandria, como ressaltei na introdução desta antologia.
Naquele fragmento, o elogio aos noivos se destaca; nestes, outros dos aspectos
da canção de casamento, inclusive o da linguagem e temática jocosas, que
revelam seus elos genéticos com a tradição popular.
Esses traços são muito próprios aos epitalâmios, pois não apenas é continuada a tradição grega de seu canto nos casamentos, mas há neles  ``um marcante nível de continuidade''.\footnote{Swift, 2010, p. 245.} Diga"-se ainda que os epitalâmios de Safo aqui apresentados ``exibem, com relação ao rito da boda, uma função pragmática precisa, por articular suas fases e realçar suas principais características''.\footnote{Ferrari, 2010, p. 119.} 

\pagebreak
\section{Fragmento 104 (\textsc{a--b})}

\begin{gkverse}
Ἔσπερε πάντα φέρηις ὄσα φαίνολις ἐσκέδασ’ Αὔως,\\
      φέρηις ὄιν, φέρηις αἶγα, φέρηις ἄπυ μάτερι παῖδα.

\hspace*{35mm}\adforn{47}

ἀστέρων πάντων ὀ κάλλιστος
\end{gkverse}

\begin{verse}
Ó Vésper, trazes tudo que a luzidia Eos espalhou:\\
trazes ovelha, trazes cabra, arrebatas da mãe a filha \ldots{}

\hspace*{35mm}\adforn{47}

\ldots{} dos astros todos o mais belo \ldots{} 
\end{verse}

\medskip

{\paragraph{Comentário} O fragmento \textsc{a}, composto pelo par de versos acima, tem por fonte Demétrio (141),
como o Fr.\,101; já o fragmento \textsc{b}, o verso traduzido em separado daquele par,
foi preservado na \textit{Oração 46}, de Himério, retórico antes citado no comentário ao Fr.\,112.
Ambos os fragmentos integram uma canção de Safo a Vésper, a estrela da tarde,
segundo Himério, e parecem aludir à procissão que sucede o banquete na casa da
noiva, iniciada no começo do anoitecer, para levá"-la à casa do noivo.
Noto que Demétrio destaca o uso que faz Safo da repetição -- característica da estilística epitalâmica, diga"-se -- da forma verbal \textit{phérēis}, da qual Vésper é sempre sujeito; uso que considera gracioso, \textit{kharientízetai}. O verbo ocorre uma vez no verso 1 e três vezes no 2, modificado, na última ocorrência, seu sentido modificado pelo termo sucessivo \textit{ápy} -- indicativo de afastamento, \textit{ápo}.\footnote{Em tradição anterior, segui o entendimento de que o verbo ao final diria ``trazes de volta à mãe a criança''. Todavia, adoto novo entendimento aqui, considerando, junto a Ferrari (2010, pp. 120--1) o que sabemos das etapas da boda e uma passagem no poeta latino Catulo (\textit{Ode} 62, versos 20--23), cujas odes epitalâmicas têm Safo como referente: ``\textit{Héspero, qual no céu é brilho mais cruel/ que tu, que a filha arrancas dos braços da mãe?/ Dos braços da mãe a arrancas, que se agarra,/ e a dás, menina casta, dom ao moço ardente?/ Rasa a urbe, o que mais cruel inimigos fazem?/ Hímen, ó Himeneu, vem Himeneu, Himeneu!}'', trad. Oliva 1996.} }




\pagebreak
\section{Fragmento 105 (\textsc{a--b})}

\begin{gkverse}
οἶον τὸ γλυκύμαλον ἐρεύθεται ἄκρωι ἐπ’ ὔσδωι,\\
ἄκρον ἐπ’ ἀκροτάτωι, λελάθοντο δὲ μαλοδρόπηες\\
οὐ μὰν ἐκλελάθοντ’, ἀλλ’ οὐκ ἐδύναντ’ ἐπίκεσθαι

\hspace*{35mm}\adforn{47}

οἴαν τὰν ὐάκινθον ἐν ὤρεσι ποίμενες ἄνδρες\\
πόσσι καταστείβοισι, χάμαι δέ τε πόρφυρον ἄνθος\ldots{}

\end{gkverse}

\begin{verse}
\ldots{} como o mais doce pomo enrubesce no ramo ao alto,\\
alto no mais alto ramo, e os colhedores o esquecem;\\*
não, não o esquecem -- mas não o podem alcançar \ldots{}

\hspace*{35mm}\adforn{47}

\ldots{} como o jacinto que nas montanhas homens, pastores,\\
esmagam com os pés, e na terra a flor purpúrea \ldots{}
\end{verse}

\medskip

{\paragraph{Comentário} Citados por Siriano, comentador bizantino do tratado de estilística de
Hermógenes (1.1), e por Demétrio (106), ambos já aqui mencionados respectivamente nos Frs.~118 e 101\textsc{a}, os fragmentos trazem duas imagens metafóricas distintas: no trio de versos (\textbf{a}), a da menina fruta madura, a maçã, sensual, desejada por todos,
mas não a todos alcançável; na dupla (\textbf{b}), a da perda da virgindade nas bodas.
Em ambos, é a natureza que alavanca a linguagem erótica.
E na segunda, o casamento faz"-se, como de fato é, uma espécie de morte,\footnote{Swift, 2010, p. 250.} porque provoca a transição irreversível de uma vida, a de \textit{parthénos}, ``moça'', para a outra, a de \textit{gyné}, ``mulher''. Como observei no comentário ao Fr.\,44, cujo tema é a boda mítica de Heitor e Andrômaca, são muitos e especificamente gregos os paralelos entre as cerimônias nupcial e fúnebre.\footnote{Redfield 1982, p. 188.}


\pagebreak
\section{Fragmento 106}

\begin{gkverse}
πέρροχος, ὠς ὄτ’ ἄοιδος ὀ Λέσβιος ἀλλοδάποισιν
\end{gkverse}

\begin{verse}
\ldots{} superior, como o cantor lésbio aos de outras terras \ldots{}
\end{verse}

\medskip

{\paragraph{Comentário} O fragmento tem por fonte Demétrio (146), como o 101\textsc{a}, cujo tratado diz que
``do homem excepcional assim fala Safo''. Se tal homem for o
noivo, há seu elogio em chave comparativa com o elevado \textit{status}
entre os antigos da tradição poético"-musical lésbio"-eólica.}


\section{Fragmento 107}

\begin{gkverse}
ἦρ’ ἔτι παρθενίας ἐπιβάλλομαι;
\end{gkverse}

\begin{verse}
\ldots{} será que ainda anseio pela virgindade?
\end{verse}

\medskip

{\paragraph{Comentário} O gramático Apolônio Díscolo, fonte do Fr.\,33, preservou o 107, no tratado
\textit{Sobre as conjunções} (1.223.24\,\textsc{ss}.). A virgindade, tema recorrente no epitalâmio, uma
vez que está em evidência sua ruptura no enlace sexual que deve consumar o
casamento, é pensada pelo \textit{eu} que nos fala.}


\section{Fragmento 108}

\begin{gkverse}
ὦ κάλα, ὦ χαρίεσσα
\end{gkverse}

\begin{verse}
Ó bela, ó graciosa \ldots{}
\end{verse}

\medskip

{\paragraph{Comentário} Citado em Himério,\footnote{\textit{Oração} 9.} fonte do Fr.\,104, este fragmento traz as palavras de condução do noivo à noiva, no aposento nupcial, informa a fonte, ao qual realçam a beleza dela.}


\pagebreak
\section{Fragmento 109 }

\begin{gkverse}
δώσομεν, ἦσι πάτηρ
\end{gkverse}

\begin{verse}
\ldots{} dá-la-emos, diz o pai \ldots{}
\end{verse}

\medskip

\paragraph{Comentário} Citado em anônimo comentário reunido junto a outros dispersos na coletânea \textit{Anedota Oxfordiana},\footnote{\textsc{i} 190, século \textsc{xix}.} o fragmento surge a propósito de um verso da \textit{Ilíada},\footnote{\textsc{i}, 528.} que com o termo equivalente (\textit{ê}) ao usado por Safo (\textit{êsi}), observa o comentador, expressa o dizer em 3ª pessoa do singular. Nele, há apenas três palavras, ligadas decerto à passagem da noiva de sua própria família à do noivo, pelas mãos do pai.


\section{Fragmento 110}

\begin{gkverse}
Θυρώρωι πόδες ἐπτορόγυιοι,\\
τὰ δὲ σάμβαλα πεμπεβόεια,\\
πίσσυγγοι δὲ δέκ’ ἐξεπόναισαν
\end{gkverse}

\begin{verse}
Os pés do porteiro têm sete braças,\\*
e as sandálias, couro de cinco bois -- \\*
e dez sapateiros nelas labutaram \ldots{}
\end{verse}

\medskip

{\paragraph{Comentário} Assim como o Fr.\,102, este é citado em Heféstion (7.6); e, segundo Pólux, fonte do Fr.\,54, a figura nele enfocada jocosamente, o guardião do leito nupcial ou tálamo, 
tinha por função impedir que os amigos da noiva acorressem a resgatá"-la do
quarto que a guardava junto ao noivo.
Tal figura tem contraparte real na cerimônia, o escolhido amigo do noivo, o real \textit{thyrōrós}, como é chamado, ressalta Caciagli,\footnote{2009, p. 68.} que se mantém em vigilância, enquanto os coros e danças prosseguem, do lado de fora do tálamo. Hesíquio (século \textsc{v} d.C.), no verbete ao termo em seu \textit{Léxico}, dá \textit{thyrōrós} o sinônimo \textit{paránymphos}, ``o amigo do noivo''. As amigas da noiva começam seu canto jocoso por ele, na figura do monstrinho, zombeteiramente, sublinha\footnote{Ferrari, 2010, pp. 124--5.} -- a zombaria, a jocosidade sendo da tradição popular das canções de casamento.}


\section{Fragmento 111}

\begin{gkverse}
Ἴψοι δὴ τὸ μέλαθρον\\
ὐμήναον,\\
ἀέρρετε, τέκτονες ἄνδρες\\
ὐμήναον,\\
γάμβρος \dagger{}(εἰσ)έρχεται ἴσος Ἄρευι\dagger{}\\
<ὐμήναον,>\\
ἄνδρος μεγάλω πόλυ μέζων\\
<ὐμήναον.>
\end{gkverse}

\begin{verse}
Ao alto o teto -- \\
Himeneu! -- \\
levantai, vós, varões carpinteiros! -- \\
Himeneu! -- \\
o noivo chega, qual Ares -- \\
Himeneu! -- \\
muito maior do que um varão grande --\\*
Himeneu!
\end{verse}

\medskip

{\paragraph{Comentário} Também citado em Heféstion (7.1), o fragmento louva a beleza do noivo símil a Ares,
deus da guerra, e altíssimo, canta em tom de brincadeira a descrição
hiperbólica. Na cena, alude"-se à entrada do noivo no aposento nupcial -- daí a
referência ao teto e a Himeneu, deus da boda, cuja presença na festa, por isso
solicitada aos gritos, garantiria o sucesso da união sexual para a qual o noivo está superlativamente equipado em sua virilidade. Da tradição popular das canções nupciais vem esse elemento da jocosidade e da malícia à beira da vulgaridade, que pode ter dois propósitos: aliviar a tensão de todo o processo do enlace sexual dos noivos em dado momento da cerimônia; afastar mau agouro, \textit{aprotopaica}, da festa, logo, da união nela celebrada e consumada. Mais: daquela tradição vem o refrão e a repetição de palavras, típicos da oralidade, que aqui e em outros epitalâmios de Safo encontramos.}


\section{Fragmento 113}

\begin{gkverse}
  οὐ γὰρ\\
ἀτέρα νῦν πάις, ὦ γάμβρε, τεαύτα
\end{gkverse}

\begin{verse}
Pois, ó noivo, jamais como agora outra menina como esta \ldots{}
\end{verse}

\medskip

{\paragraph{Comentário} Citado em Dionísio de Halicarnasso (25), mesma fonte do Fr.\,1, o fragmento traz canção em que, a certa altura, o coro se dirige ao noivo, celebrando a beleza ímpar da noiva.}


\pagebreak
\section{Fragmento 114}

\begin{gkverse}
[νύμφη) παρθενία, παρθενία, ποῖ με λίποισ’ α<π>οίχηι;\\
(παρθενία)  \dagger{}οὐκέτι ἤξω πρὸς σέ, οὐκέτι ἤξω\dagger{}.
\end{gkverse}

\begin{verse}
(\textsc{noiva}) ``Virgindade, virgindade, aonde vais, me abandonando?''

(\textsc{virgindade}) ``Nunca mais a ti voltarei, nunca mais voltarei''
\end{verse}

\medskip

{\paragraph{Comentário} Este, como o 101\textsc{a}, tem em Demétrio (140) sua fonte. O canto dialogado, forma já vista
nos Frs.~133 e 140, é de \textit{performance} coral. Novamente, a canção jocosa
centra"-se na incontornável perda da virgindade, usando o humor decerto como
meio de aliviar a tensão e o impacto de todo o processo que conduz a virgem à
idade adulta, em que passa a atuar na esfera do sexo, deixando para trás sua casa, a mãe e as amigas -- a existência de \textit{parthénos}.
``O casamento é um evento importante e que muda a vida; logo, é potencialmente assustador. O trauma ritualizado elabora os temores reais, apresentando"-os de um modo regulado. Articular essas ansiedades serve, pois, como forma de mitigá"-las, retratando"-as como parte normal e necessária da transição e, assim, uma resposta saudável, em vez de destrutiva, à mudança''.\footnote{Swift 2010, p. 248.}
Os termos do Fr.\,114 configuram uma ``linguagem reminiscente do lamento'',\footnote{Lardinois 2011, p. 164.} da \textit{moirología} -- canto de lamento pela perda (morte, casamento, partida da casa); e sua perspectiva é feminina, a refletir a ``ansiedade que a noiva sente em antecipação à noite nupcial e à nova vida com seu marido''.\footnote{Lardinois, pp. 164--5.}}



\pagebreak
\section{Fragmento 115}

\begin{gkverse}
Τίῳ σ᾿, ὦ φίλε γάμβρε, κάλως ἐικάσδω;\\
ὄρπακι βραδίνῳ σε μάλιστ᾿ ἐικάσδω.
\end{gkverse}

\begin{verse}
A que, ó caro noivo, belamente te comparo?\\*
A um ramo esguio sobretudo te comparo \ldots{}
\end{verse}

\medskip

{\paragraph{Comentário} Também preservado em Heféstion (7.6), esse fragmento faz o elogio do noivo jovem,\footnote{Ragusa e Rosenmeyer 2019, pp. 62--75.} porque comparado a jovem planta, de sua
graça física marcada no adjetivo à vergôntea, \textit{órpaki bradínōi}, chamando a atenção da noiva e tornando"-o atraente sexualmente aos seus olhos.
A comparação é feita pela técnica da \textit{eikasía},\footnote{Hague, 1983, pp. 132--9.} pela qual se exprime elogio, como é o caso, ou jocosidade; daí o uso do verbo \textit{eikázein} nos dois versos do fragmento. Tal recurso, bem como a própria repetição, são característicos da tradição popular das canções de casamento. e tradicional é igualmente o conjunto de imagens ``para expressar beleza e vigor juvenis: flores e primavera se enlaçam ao topos da assimilação da vida humana à fertilidade natural, enquanto o uso de um ramo para representar o vigor jovem se acha em Homero tanto para moços quanto para moças".\footnote{Swift, 2010, p. 246.}}

\pagebreak
\section{Fragmento 116}

\begin{gkverse}
χαῖρε, νύμφα, χαῖρε, τίμιε γάμβρε, πόλλα
\end{gkverse}

\begin{verse}
Salve, ó noiva, salve, ó digno noivo, muitas \ldots{}
\end{verse}

\medskip

{\paragraph{Comentário} Preservado por Sérvio (século \textsc{iv} d.C.), em seu comentário às \textit{Geórgicas} (1.31) de Virgílio (século \textsc{i} a.C.), o fragmento saúda os noivos, elogiando"-os.}


\section{Fragmento 117}

\begin{gkverse}
\dagger{}χαίροις ἀ νύμφα\dagger{}, χαιρέτω δ’ ὀ γάμβρος
\end{gkverse}

\begin{verse}
Saudações, ó noiva, saudações, ó noivo \ldots{}
\end{verse}

\medskip

{\paragraph{Comentário} Citado por Heféstion (4.2), o fragmento traz conteúdo similar ao do 116.}


\chapter{Festividades}

\paragraph{\textsc{nota introdutória}}
O Fr.\,2 fala em festividades que não podemos precisar; os epitalâmios,
caracterizados por estruturas métricas próprias, têm por texto/contexto a
cerimônia do casamento e seus festejos, os quais estão presentes em fragmentos não epitalâmicos: 23, 44, 141. No grupo a seguir, destaca"-se a festa,
ora ligada ao casamento, ora a ritos de celebração aos deuses, ora ao banquete,
evento da vida cotidiana masculina, largamente celebrado na poesia grega
antiga, em todos os seus gêneros.

\section{Fragmento 27}

\begin{gkverse}
\textnormal{[\textit{versos 1--3: ilegíveis e lacunares}]}

\ldots{}]. καὶ γὰρ δὴ σὺ πάις ποτ[\\
\ldots{}]ικ̣ης μέλπεσθ’ ἄγι ταῦτα[\\
\ldots{}] ζάλεξαι, κἄμμ’ ἀπὺ τωδεκ[\\
ἄ]δρα χάρισσαι

σ]τείχομεν γὰρ ἐς γάμον εὖ δε[\\
κα]ὶ σὺ τοῦτ’, ἀλλ’ ὄττι τάχιστα[\\
πα]ρ̣[θ]ένοις ἄπ[π]εμπε, θέοι[\\
]εν ἔχοιεν\\
      ] ὄδος̣ μ[έ]γαν εἰς Ὄλ[υμπον\\
       ἀ]νθρω[π\qquad]αίκ.[
\end{gkverse}
\pagebreak
\begin{verse}
\ldots{} e certa vez tu também menina \ldots{}\\
\ldots{} cantar-dançar vem! -- estas coisas \ldots{}\\
\ldots{} discutir, e para nós \ldots{}\\
\ldots{} abundantes deleites;\\
\ldots{} pois apressamo-nos à boda; bem \ldots{}\\
\ldots{} isso também tu, mas então rápido\\
as virgens envia, deuses \ldots{}\\
\ldots{} tivesse \ldots{}\\
\ldots{} caminho ao grande Olimpo\\*
\ldots{} mortais \ldots{}
\end{verse}

\medskip

{\paragraph{Comentário} Preservado no \textit{Papiro de Oxirrinco} 1231, como o Fr.\,15 e tantos outros, o 27 traz uma voz que busca convencer alguém  -- o \textit{tu}
pode ser a noiva ou a mãe da noiva -- a tomar parte na procissão de
ida ao casamento, liderando virgens certamente ligadas à noiva; os últimos versos
legíveis enveredam para uma sentença moralizante.
Caciagli\footnote{2009, p. 63.} nota que no fragmento ``afloram quase todos os aspectos que caracterizam a obra da poeta'', como os entrelaçados canto"-dança, \textit{mélpesthai}, e o desejo, \textit{érōs}, entrelaçados, como nos Frs. 22, 94, 96; ``termos conexos à juventude'' das personagens, recorrentes nas canções de Safo; o tema da memória, destacado nesta antologia, indicado no verso 1 (\textit{pote}); e ``a referência à cerimônia nupcial''. Nesse sentido, a linguagem faz"-se reflexo da canção da poeta, e a \textit{persona} faz"-se a líder das meninas do coro -- a \textit{khorodidáskalos}.\footnote{Frisam"-no Caciagli (2009, p.\,64) e Ferrari (2010, p.\,34).}}



\pagebreak
\section{Fragmento 30}

\begin{gkverse}
νύκτ[...].[

πάρθενοι δ[\\
παννυχίσδοι̣[σ]α̣ι[̣\\
σὰν ἀείδοισ̣[ι]ν φ[ιλότατα καὶ νύμ-\\
φας ἰοκόλπω.\\
ἀλλ’ ἐγέρθε̣ι̣ς, ἠϊθ[ε\\
στεῖχε σοὶς ὐμάλικ̣[ας\\
ἤπερ ὄσσον ἀ λιγ̣ύφω̣[νος\\
ὔπνον [ἴ]δωμεν.
\end{gkverse}

\begin{verse}
\ldots{} virgens \ldots{}\\
\ldots{} celebrando um festival noturno \ldots{}\\
\ldots{} teus amores cantariam e os da noiva\\*
de violáceo colo \ldots{}

mas, despertos,\footnote{A tradução busca enfatizar o sentido do acordar com o dia, mas é válida ainda a tradução por ``tendo se erguido'' (Ragusa, 2019a, p. 95). A opção depende do entendimento da celebração noturna (\textit{pannykhís}) do verso 3, que é a ocasião de \textit{performance}:  se sucede o banquete do casamento (Ferrari, 2010, p. 114), com o coro de moças a chamar o noivo e seus coevos a se erguerem da mesa; ou se é o momento em que o coro vem para o despertar celebrativo matinal dos noivos, ao fim da noite e raiar do dia (Stehle, 1997, pp.\,279--80), como prefiro pensar aqui.} os moços solteiros\ldots{}\\*
traz teus coevos \ldots{}\\
para que mais do que a clarissonante\ldots{}\\*
sono vejamos.
\end{verse}
\pagebreak
{\paragraph{Comentário} Tendo por fonte a mesma do Fr.\,27, o 30 igualmente revela uma linguagem muito característica do universo sáfico: a menção de moças virgens, o cantar e a festividade, destacados como temas dos fragmentos nesta antologia. Desse modo, pode"-se pensar como sendo da \textit{persona} da líder do coro, Safo, a
voz que fala ao noivo, como se nota no início e no fim, e da celebração
festiva e ritual noturna, \textit{pannykhís} -- como a do Fr.\,23 --, nomeada no verso 2, no verbo \textit{pannykhísdoisai}. Em seguida, a \textit{persona} parece concentrar"-se no grupo masculino que do
banquete, antes da procissão nupcial, tomava parte separadamente do feminino,
ligado à noiva e também presente. O fragmento, como o 27, tem caráter
epitalâmico, mas sua estrutura métrica não condiz com as desse subgênero
mélico cujos fragmentos acham"-se em tópico próprio.}

\pagebreak
\section{Fragmento 43} 

\begin{gkverse}
\textnormal{[\textit{versos 1--3: ilegíveis e lacunares}]}

] [κ]αλος\\
         ]. ἄκαλα κλόνει\\
         ] κάματος φρένα\\
       ]ε̣ κ̣ατισδάνε[ι]\\
       ] ἀλλ’ ἄγιτ’, ὦ φίλαι,\\
       ], ἄγχι γὰρ ἀμέρα.
\end{gkverse}

\begin{verse}
\ldots{} belo \ldots{}\\
\ldots{} serenidade \ldots{} agita\\
\ldots{} fadiga \ldots{} sensos\\
\ldots{} senta-se \ldots{}\\
\ldots{} mas vamos, ó minhas queridas,\\
\ldots{} pois perto o dia.
\end{verse}

\medskip

{\paragraph{Comentário} Tendo por fonte o \textit{Papiro de Oxirrinco} 1232, fonte do Fr, 44, o 43 talvez seja destinado a ritualística \textit{pannykhís}, como o anterior e o Fr.\,23. Isso segundo a indicação do último de seus parcos versos que é o final da canção perdida, que fala da aproximação do amanhecer. Ressalto a presença de \textit{phílai}, designando possivelmente as meninas do coro de Safo, responsáveis pela \textit{performance}. A atividade ritual, como enfatizei na introdução e no comentário do Fr.\,94, por exemplo, é de todo compatível e própria às atividades de grupos corais femininos, como atestam os partênios -- as canções para virgens de Álcman, o poeta mélico da Esparta mais ou menos contemporâneo a Safo.}


\pagebreak
\section{Fragmento 81}

\begin{gkverse}
\textnormal{[\textit{versos 1--3: ilegíveis e lacunares}]}

σὺ δὲ στεφάνοις, ὦ Δίκα, πέρθεσθ’ ἐράτοις φόβαισιν\\
ὄρπακας ἀνήτω συν<α>έρραισ̣’ ἀπάλαισι χέρσιν\\
εὐάνθεα \dagger{}γὰρ πέλεται\dagger{} καὶ Χάριτες μάκαιρα<ι>\\
μᾶλλον \dagger{}προτερην\dagger{}, ἀστεφανώτοισι δ’ ἀπυστρέφονται.
\end{gkverse}

\begin{verse}
\ldots{} e tu, ó Dica, cinge teus cachos com amáveis guirlandas,\\
tramando raminhos de aneto com mãos macias;\\
pois mesmo as Cárites venturosas voltam-se ao florido,\\*
sobretudo, mas ao não-coroado dão as costas.
\end{verse}

\medskip

{\paragraph{Comentário} Ateneu, fonte também do Fr.\,141, declara sobre a tradição de adornar"-se:
``Safo expressa com mais simplicidade a razão de nossa prática
de usar guirlandas, dizendo isto'', em cena talvez de adorno da noiva, a jovem em evidência nos versos, em busca do favor das deusas Graças do charme sedutor.
Dica, preparada algo ritualisticamente, é tornada ``ainda mais amável, mais bela, e, portanto, mais desejável''\footnote{Bartol, 1997, p. 79.} para seu \textit{gámos}, sua boda. A intimidade -- ilusão da dicção sáfica -- da fala à jovem insere a cena possivelmente na preparação da noiva para o banquete nupcial.}


\chapter{Vestes e adornos}

\paragraph{\textsc{nota introdutória}}
No universo feminino tão privilegiado na mélica sáfica, recebem atenção vestes e
adornos, que tornam mais belas e atraentes as figuras contempladas. Isso se
nota no Fr.\,22, em que o vestido integra a apreensão erótica do objeto do
desejo de quem com os olhos o detém; e no Fr.\,44 em que a beleza da noiva que chega a
Troia se reflete no dote cheio de tecidos e ornamentos. Eis mais dois fragmentos.


\section{Fragmento 39}

\begin{gkverse}
πόδα<ς> δὲ\\
ποίκιλος μάσλης ἐκάλυπτε, Λύδι-\\
ον κάλον ἔργον.
\end{gkverse}

\begin{verse}
\ldots{} e cobria\\*
seus pés sandália furta-cor, belo\\*
trabalho lídio \ldots{}
\end{verse}

\medskip

{\paragraph{Comentário} Citado num escólio à comédia \textit{A paz},\footnote{Verso 1174.} 
de Aristófanes (séculos \textsc{v}--\textsc{iv} a.C.),
o fragmento faz menção à Lídia que para a Grécia exportava variados e altamente
elaborados produtos de luxo, como a sandália de uma figura feminina, nos versos acima.}


\pagebreak
\section{Fragmento 62}

\begin{gkverse}
’Επτάξατε̣ [\\
δάφνας ὄτα̣[

πὰν δ’ ἄδιον[\\
ἢ κῆνον ἐλο[

καὶ ταῖσι μὲν ἀ̣[\\
ὀδοίπορος ἄν[\ldots{}]\ldots{}[

μύγις δέ ποτ’ εἰσάιον ἐκλ̣[\\
ψύχα δ’ ἀγαπάτασυ.[.́

τέαυτα δὲ νῦν ἔμμ̣[\\
ἴκεσθ’ ἀγανα[

ἔφθατε κάλαν[\\
τά τ’ ἔμματα κα̣[
\end{gkverse}

\begin{verse}
Vós vos agachastes \ldots{}\\
de louro \ldots{}

e tudo mais doce \ldots{}\\
do que aquele \ldots{}

e a essas \ldots{}\\
viajante \ldots{}

e a custo certa vez eu ouvi; \ldots{}\\
e o ânimo dileto \ldots{}

e agora tais vestes \ldots{}\\ 
chegar \ldots{} gentil \ldots{}

primeiro viestes; bela \ldots{}\\ \EP[1]
e as vestes \ldots{}
\end{verse}

\medskip

\textls[-10]{\paragraph{Comentário} Preservado no \textit{Papiro de Oxirrinco} 1787, fonte do Fr.\,65 e de outros, o 62, cujo início e final são marcados, conjuga elementos que colocam em cena figuras femininas, como parece. Seria o coro o \textit{vós}, e sua líder, a voz em 1ª pessoa do singular? Seriam vestes e a memória de uma jovem já ida o foco?}


\section{Fragmento 92} 

\begin{gkverse}
\textnormal{[\textit{versos 1--2 e 14--6: ilegíveis e lacunares}]}

πέπλον[...]π̣υ̣σ̣χ[\\
καὶ κλ̣ε̣[...]σαω[\\
κροκοεντα[\\
πέπλον πορφυ[ρ......]δ̣εξω̣[\\
χλαιναι περσ.[\\
στέφανοι περ[\\
καλ[.]ο̣σ̣σ̣α̣μ̣[\\
φρυ[\\
πορφ[υρ
\end{gkverse}

\begin{verse}
\ldots{} peplo \ldots{}\\ 
\ldots{}\\
açafroado \ldots{}\\
peplo purpúreo \ldots{}\\
mantos persas \ldots{}\\
guirlandas \ldots{}\\
\ldots{}\\
purpúreo \ldots{}
\end{verse}

\medskip

{\paragraph{Comentário} Tendo por fonte o \textit{Papiro de Berlim 9722}, que nos trouxe o Fr.\,96, entre outros, o 92 contém algumas palavras em inícios de versos, mas todas ligadas a vestes e adornos que são objeto de cuidado nos grupos corais femininos e tema de suas canções, como mostram Álcman e seus partênios, e as muitas canções de Safo que já vimos. Chamam a atenção as vestes, seja pelo luxo, expresso no colorido, seja pela forma ou pela maciez -- qualidade que, criando um contraste, marca pelo adjetivo \textit{ábrois'(i)} o pano rústico, \textit{lasíois'(i)}, que bem recobre alguém na linha única e lacunar do Fr.\,100.}



\section{Fragmento 125}

\begin{gkverse}
\dagger{}αυταόρα\dagger{} ἐστεφαναπλόκην
\end{gkverse}

\begin{verse}
\ldots{} eu mesma em meu tempo tecia \rlap{guirlandas \ldots{}}
\end{verse}

\medskip

{\paragraph{Comentário} Citado num escólio à comédia \textit{As tesmoforiantes},\footnote{Verso 401.} de Aristófanes, o fragmento faz menção a uma das atividades corais mais típicas das associações femininas, mencionada nos Frs.~81, 94, e a seguir, no 98. O escólio comenta que a prática liga"-se à juventude de mulheres de tempos antigos, como Safo.}



\chapter{Cleis}

\section{Fragmento 98 (\textsc{a--b})}

\begin{gkverse}
\ldots{}]θος ἀ γάρ με ἐγέννα̣[τ

σ]φᾶς ἐπ’ ἀλικίας μέγ[αν\\
κ]όσμον αἴ τις ἔχη φόβα<ι>σ̣[\\
π̣ορφύρωι κατελιξαμέ [να

ἔ̣μμεναι μά̣λα τοῦτο .[\\
ἀ̣λλα ξανθοτέρα<ι>ς ἔχη[\\
τ̣α<ὶ>ς κόμα<ι>ς δάϊδος προφ[

σ]τεφάνοισιν ἐπαρτία[ις\\
ἀ̣νθέων ἐριθαλέων [ \\
μ]ι̣τράναν δ’ ἀρτίως κλ[

π̣οικίλαν ἀπὺ Σαρδίω[ν\\
\ldots{}]. αονίας πόλ{ε}ις [

\hspace*{16mm}\adforn{47}
%\ast\quad\ast\quad\ast

-- σοὶ δ’ ἔγω Κλέι ποικίλαν [\\
-- οὐκ ἔχω -- πόθεν ἔσσεται; -- [\\
-- μιτράν<αν> ἀλλὰ τὼι Μυτιληνάωι [
\end{gkverse}

\chapter*{}
\section*{}

\begin{verse}
\ldots{} pois ela, a que me gerou \ldots{}


em sua época, era grande \\*
adorno, se alguém tinha os cachos\\*
atados em nó purpúreo;


era isso mesmo \ldots{} \\*
mas se alguém tinha a coma\\*
mais fulva que a tocha \ldots{},

com guirlandas [ornadas]\\*
de flores em flor \ldots{}\\*
Há pouco, [Cleis], uma fita

furta-cor de Sárdis \ldots{}\\*
\ldots{} cidades \ldots{}

Mas eu, a ti, Cleis, uma fita furta-cor \ldots{} --\\*
não tenho meios de tê-la; \\*
mas com o mitilênio \ldots{}

\hspace*{16mm}\adforn{47}

\ldots{} ter \ldots{}\\
\ldots{} furta-cor \ldots{}\\
estas coisas dos Cleanatidas \ldots{}\\
o exílio \ldots{}\\*
memoriais \ldots{} pois terrivelmente \rlap{devastado(a) \ldots{}}
\end{verse}

\pagebreak

{\paragraph{Comentário} Preservado em fontes do século \textsc{iii} a.C., o \textit{Papiro de Copenhagen} 301 (\textbf{a}) e o \textit{Papiro de Milão} 32 (\textbf{b}),
o fragmento é lido em chave biográfica, por conta do
nome de Cleis que, segundo fontes antigas, era filha da poeta; logo, o \textit{eu} seria
``Safo'', que se referiria, ainda, à sua mãe, a avó de Cleis, na
abertura. Há nele uma contraposição entre os tradicionais adornos, tidos como
elegantes no passado, e o adorno mais sofisticado e desejado no presente, a
\textit{mítra}, fita ou faixa adornada que cobria o cabelo, mas não as orelhas,
feita na Lídia, como a sandália do Fr.\,39 -- provas do influxo oriental na
cultura da ilha de Lesbos, ambas ditas ``furta"-cor'', na tradução do
adjetivo derivado do substantivo \textit{poikilía}, que carrega as ideias do
cintilar, da múltipla cor, do variegado, ou seja, da mistura de luz, formas ou
cor, que dificulta a direta e clara apreensão do objeto que se contempla, e que
justamente por isso associa"-se à sedução erótica e à astúcia. O \textit{eu} lamenta a
incapacidade de dar a Cleis a \textit{mítra}, devido à austeridade projetada no tempo presente, que contrasta com o passado. Que uma fita de cabelo Lídia ``possa chamar tanta atenção como um objeto capaz de perturbar os sonhos de uma jovem menina aristocrática é um dos sinais de um processo de aculturação que se espalhava pelas cidades eólias e jônicas da Ásia Menor e das ilhas próximas''.\footnote{Ferrari, 2010, p.\,5.} Mais: bem pode ser símbolo de \textit{status} e/ou de afinidades políticas. No fragmento, talvez por restrições políticas a \textit{mítra} esteja fora de alcance,  impostas pelo governo do ``mitilênio'', possivelmente
Pítaco, ou o adjetivo de um substantivo identificado a outra personagem. Há
ainda a menção ao exílio ``dos Cleanactidas'' ou ``de Cleanactides'', de memória
ainda viva em Lesbos, mas não sabemos como isso se ligaria ao
``mitilênio'', a Cleis, a ``Safo''. Cabe recordar que fontes antigas falam
de um período de exílio vivido por Safo em Siracusa, na Sicília (Magna Grécia),
por conflitos de seu grupo aristocrático com Pítaco, de quem falei no comentário do Fr.\,71, anteriormente.}



\pagebreak
\section{Fragmento 132}

\begin{gkverse}
Ἔστι μοι κάλα πάις χρυσίοισιν ἀνθέμοισιν\\
ἐμφέρη<ν> ἔχοισα μόρφαν Κλέις <  > ἀγαπάτα,\\
ἀντὶ τᾶς ἔγωὐδὲ Λυδίαν παῖσαν οὐδ’ ἐράνναν \ldots{}
\end{gkverse}

\begin{verse}
Tenho bela criança, portando forma símil\\*
à das áureas flores, Cleis, filha amada,\\*
por quem eu não [trocaria] toda a Lídia, nem a amável \ldots{}
\end{verse}

\medskip

{\paragraph{Comentário} Citado em Heféstion (15, 18s.), fonte do Fr.\,102, o 132 é o segundo e último em que vemos claramente o nome de Cleis, que o \textit{eu} introduz como sua filha, louvando-lhe a
beleza da juventude, na imagem das flores de ouro, e afirmando seu valor único no termo \textit{agapáta}, ``amada, querida'' -- que denota o que deve bastar ao contentamento. Por ela, nenhuma troca poderia ser feita, indica a elaboração do verso 3, nem pelo reino mais rico em ouro (a Lídia), nem, talvez, pela terra"-mãe, Lesbos.}



\chapter{Reflexões ético-morais}

\section{Fragmento 26} 

\begin{gkverse}
\textnormal{[\textit{versos 1 e 13--5: ilegíveis e lacunares}]}

ὄ]ττινα[ς γὰρ\\
εὖ θέω, κῆνοί με μά]λ̣ιστα πά[ντων\\
σίνονται̣

\textnormal{[\textit{versos 5--6: ilegíveis e lacunares}]}

           ].ιμ’ οὐ πρ[\\
  ]αι\\
  ] σέ, θέλω[\\
  ]το πάθη[\\
         ]αν, ἔγω δ’ ἔμ’ αὔται\\
τοῦτο συνοίδα
\end{gkverse}

\begin{verse}
\ldots{} pois aqueles a quem trato bem\\
são os que dentre todos sobretudo me\\
machucam \ldots{}\\
\ldots{} não \ldots{}\\
\ldots{} a ti, quero \ldots{}\\
\ldots{} sofrimentos \ldots{}\\
\ldots{} mas eu própria\\*
disso tenho consciência \ldots{}
\end{verse}

\medskip

{\paragraph{Comentário} A fonte do fragmento é o \textit{Papiro de Oxirrinco} 1231, como também do Fr.\,15. O que se lê parece centrar"-se numa discussão de caráter ético"-moral
a partir da experiência do \textit{eu} feminino que se mostra consciente da dolorosa
ingratidão, e não ingenuamente iludida.}



\pagebreak
\section{Fragmento 37}

\begin{gkverse}
κὰτ ἔμον στάλαγμον

\hspace*{16mm}\adforn{47}

τὸν δ’ ἐπιπλάζοντ’ ἄνεμοι φέροιεν\\
      καὶ μελέδωναι
\end{gkverse}

\begin{verse}
\ldots{} em minha dor \ldots{}

\hspace*{16mm}\adforn{47}

\ldots{} a ele que me censura, que o carreguem ventos
e anseios \ldots{}
\end{verse}

\medskip

{\paragraph{Comentário} A fonte é o tardio \textit{Etimológico genuíno}, como é o caso do 126. Dor e
rejeição à censura recebida e talvez indevida: eis o que canta o fragmento.}


\section{Fragmento 50}

\begin{gkverse}
ὀ μὲν γὰρ κάλος ὄσσον ἴδην πέλεται <κάλος>,\\
ὀ δὲ κἄγαθος αὔτικα καὶ κάλος ἔσ<σε>ται.
\end{gkverse}

\begin{verse}
\ldots{} pois o belo é belo enquanto se vê,\\*
mas o bom será de pronto também belo \ldots{}
\end{verse}

\medskip

{\paragraph{Comentário} Na \textit{Exortação à aprendizagem} (8.16), Galeno (século \textsc{ii} d.C.) cita esses dois versos
em que Safo contrapõe a beleza à bondade: ``Portanto, já que o auge da
juventude é como as flores primaveris, trazendo prazer de curta vida, é melhor
louvar a lésbia, quando ela diz''.}

\pagebreak
\section{Fragmento 52}

\begin{gkverse}
ψαύην δ’ οὐ δοκίμωμ’ ὀράνω \dagger{}δυσπαχέα\dagger{}
\end{gkverse}

\begin{verse}
\ldots{} não espero tocar o céu com meus dois braços \ldots{}
\end{verse}

\medskip

{\paragraph{Comentário} A fonte desse fragmento, como do 144, é Herodiano, no tratado \textit{Sobre
palavras anômalas} (2.912). Nele, o \textit{eu} afirma"-se consciente dos limites próprios da
mortalidade, que têm no alcance dos céus e no voo uma de suas imagens mais
fortes e recorrentes na tradição mítico"-poética grega.}


\section{Fragmento 58}

\begin{gkverse}
]ιμέναν νομίσδει\\
    ]αις ὀπάσδοι\\
ἔγω δὲ φίλημμ’ ἀβροσύναν,\qquad       ] τοῦτο καί μοι\\
τὸ λάμπρον ἔρος ἀελίω καὶ τὸ κά]λον λέλογχε.
\end{gkverse}

\begin{verse}
\ldots{} considera \ldots{}\\
\ldots{} concederia;\\
mas eu amo a delicadeza \ldots{} isso, e a mim\\
o desejo do sol deu por parte a luz e a beleza também.
\end{verse}

\medskip

{\paragraph{Comentário} Os dois últimos e mais legíveis versos\footnote{Versos 25--26.} do Fr.\,58, cuja fonte é o
\textit{Papiro de Oxirrinco} 1787, que preservou o 65, constam de outra, Ateneu (15.687\,\textsc{b}), que também cita o já visto Fr.\,141. Após a publicação em 2004 do
novo papiro de Safo -- que revelou uma nova canção fragmentária, em texto que se
sobrepõe a boa parte do que até então era editado como Fr.\,58 e que chamamos ``Canção sobre a velhice'', traduzida adiante --, concluiu"-se que aqueles
versos são independentes dos que os precedem e abririam uma nova canção. Eis acima, pois, a tradução do que seriam os versos 23--26 daquele papiro de Oxirrinco, que
trazem uma noção muito cara à mélica sáfica, da \textit{habrosýnē}, ``delicadeza'', de entranhada sensualidade em dimensão de sentido
estético, que alude a certo modo de vida aristocrático, marcado pelo luxo e pelo
refinamento que estimulavam a importação de objetos e costumes orientais
sob o signo de tais marcas. Note"-se a imagem do \textit{éros} do sol, que denota o desejo de viver, de vida, na solução para a tradução que me parece mais interessante ao contexto de citação.}



\section{Fragmento 120}

\begin{gkverse}
ἀλλά τις οὐκ ἔμμι παλιγκότων\\
ὄργαν, ἀλλ’ ἀβάκην τὰν φρέν’ ἔχω\ldots{}
\end{gkverse}

\begin{verse}
\ldots{} mas não sou das de têmpera rancorosa,\\*
mas tenho a mente serena \ldots{}
\end{verse}

\medskip

{\paragraph{Comentário} É fonte do fragmento o \textit{Etimológico magno} (2.43); nele, o \textit{eu} explica sua disposição.}


\section{Fragmento 148}

\begin{gkverse}
ὀ πλοῦτος ἄνευ ἀρέτας οὐκ ἀσίνης πάροικος\\
(ἀ δ’ ἀμφοτέρων κρᾶσις \dagger{}εὐδαιμονίας ἔχει τὸ ἄκρον\dagger{})
\end{gkverse}

\begin{verse}
\ldots{} a riqueza sem a excelência não é vizinha inofensiva,\\
mas a mistura de ambas traz a mais alta ventura \ldots{}
\end{verse}

\medskip

{\paragraph{Comentário} Um escólio a Píndaro\footnote{\textit{Ode olímpica} 2, verso 96\,\textsc{b}.} cita o fragmento,
que condena a separação entre a prosperidade material e a virtude ético"-moral.}

\pagebreak
\section{Fragmento 158}

\begin{gkverse}
σκιδναμένας ἐν στήθεσιν ὄργας\\
μαψυλάκαν γλῶσσαν πεφύλαχθαι
\end{gkverse}

\begin{verse}
\ldots{} a raiva espalhando-se \\*
no peito, proteger-se da língua tagarela \ldots{}
\end{verse}

\medskip

{\paragraph{Comentário} No tratado \textit{Sobre o refrear da cólera} (456\,\textsc{e}), Plutarco, fonte do Fr.\,49, cita o 158 como um conselho de Safo diante da seguinte situação:

\begin{quote}
Quando as pessoas estão bebendo, o que permanece silencioso é um peso
cansativo a seus companheiros; mas quando alguém está com raiva, nada é mais
digno do que a quietude, como exorta (\textit{paraineî}) Safo. 
\end{quote}}

Esse contexto realça o elemento parenético, muito disseminado na poesia grega arcaica e clássica, por sua natureza oral e pragmática, em essência, de estreitos laços com a vida cotidiana e função paidêutica -- no caso, junto ao coro de meninas da poeta.

\chapter{«\,Canção sobre a velhice\,»}

\paragraph{\textsc{nota introdutória}}
Como observei na anotação ao Fr.\,58, este foi quase totalmente reeditado com a 
publicação, em 2004, do \textit{Papiro de Colônia} 21351 (início do século \textsc{iii} a.C.), 
cujo texto se sobrepunha a boa parte daquele fragmento.\footnote{\textls[-25]{Tradução e comentários embasados na edição do texto grego dada em Buzzi \textit{et alii} (2008, p.~14) e em Greene e Skinner (2009 pp.~11 e 14--15), com a possibilidade de alguns suplementos sugeridos por Martin L.\,West (``The new Sappho''. \textit{\textsc{zpe}} 151, 2005, pp. 1--9), indicados no texto traduzido entre parênteses, como sempre nos fragmentos desta antologia. Tais suplementos são: o nome das Musas no primeiro verso preservado, cujo colo é qualificado; os adjetivos \textit{hápalos} e \textit{lêukai}, para adjetivar como ``tenra'' (\textit{ápalon}) a ``pele'' (\textit{khróa}) e ``brancos'' os cabelos de quem envelhece. As canções do novo papiro foram publicadas pela primeira vez por M.\,Gronewald e R.\,W.\,Daniel (``Ein neuer Sappho-Papyrus''. \textit{\textsc{zpe}} 147, 2004, pp. 1--8).}\looseness=-1} No início, a 
linguagem metapoética desenha uma cena de canto junto a ``meninas'', como há que entender o termo \textit{paîdes},\footnote{Verso 1.} considerada a associação liderada por Safo, 
integrada pelo favorecimento de divindades que devem ser as Musas,
dado o tema dos versos da abertura preservada. Em seguida, o tema da velhice
vem à tona e, com ele, a reiterada imagem dos cabelos que se tornam grisalhos,
e a provável referência à perda do frescor do corpo ou da pele, como parece cantar o precário Fr.\,21; depois, os
tormentos e preocupações, e a dificuldade de movimentação, que contrasta
drasticamente com a leveza de joelhos dançantes na juventude. Esse quadro é
motivo de dor, mas é inexorável e inelutável; daí a expressão do sentimento
agudo de impotência resignada e consolada contra o que é próprio da natureza mortal: envelhecer. Para
ilustrar tal verdade, em passagem gnômica, o \textit{eu} que dramatiza a líder do coro de meninas recorda o
mito da paixão de Eos, a Aurora, pelo jovem mortal troiano Titono, para o qual é nossa fonte
principal o \textit{Hino homérico \textsc{v}, a Afrodite },\footnote{Versos 218--238.} \textls[-20]{datado do
século \textsc{vii} a.C., e de autoria anônima. Em síntese, esse mito conta como a
deusa, tomada de paixão, pediu a Zeus que tornasse imortal seu amado por ela abduzido e levado ao Olimpo; a deusa,
porém, esqueceu"-se de que a mortalidade do homem não se concretiza apenas na
morte, mas na velhice. Assim, Titono tornou"-se imortal, mas imortalmente,
eternamente velho. Ora, sendo a paixão suscitada e sustentada pela
beleza do corpo e pela sua capacidade de atração, a esfera da velhice é
inadequada aos dons de Afrodite. Eos, então, acaba por se desinteressar por
completo do mortal, a quem encerra num quarto, do qual ressoa sem cessar sua
voz -- em certas tradições, esta é uma referência à metamorfose de Titono em
cigarra --, a ecoar de seu débil e velho corpo sempre a minguar. Merece nota o fato de
que Safo, com esse fragmento, ganha relevo na série de textos poéticos da
Grécia arcaica que trata da velhice, em gêneros variados, com ênfase na
decadência física que, no célebre Fr.\,1 -- uma elegia talvez completa -- de
Mimnermo, ganha conotação ético"-moral, e é descrita como obstáculo
intransponível à participação na esfera de Afrodite.
Dado o ingrediente erótico do mito que também esse poeta de meados do século
\textsc{vii} a.C. recorda noutra elegia -- o Fr.\,4, em que o presente de Titono é
julgado pior do que a própria morte, dada a primazia da paixão na sua perspectiva
--, é possível que a soma do canto e dança à
velhice e à referência mítica no novo fragmento de Safo apontem para um cenário
em que também Afrodite e/ou seu universo tomassem parte. Digno de nota, ainda,
é o modo como Eos se caracteriza, recordando a imagem homérica da Aurora
\textit{rhododáktylos}, ``dedirrósea'', no
epíteto composto \textit{brodópakhyn}, no Fr.\,53 atribuído às Cárites.
Ampliando o epíteto homérico, Safo amplia os rasgos
róseos"-avermelhados"-alaranjados do amanhecer, traçados não por dedos, mas pelos
braços que tanto desejaram enlaçar Titono, enquanto jovem e belo foi seu
corpo. Digna de nota, por fim, é a ressonância do Fr.\,26 de Álcman no novo de Safo, em que a voz autodramatizada do poeta -- como o é a voz de Safo na canção que traduzo -- canta às \textit{parthénoi}, as virgens de seu coro, a dor de seu próprio envelhecer que o impede de seguir acompanhando"-as nas \textit{performances} das canções.}

\begin{gkverse}
ἰ]ο̣κ[ό]λ̣πων κάλα δῶρα, παῖδες,\\
        τὰ]ν̣ φιλάοιδον λιγύραν χελύνναν

] π̣οτ̣’ [ἔ]ο̣ντα χρόα γῆρας ἤδη\\
         ἐγ]ένοντο τρίχες ἐκ μελαίναν

βάρυς δέ μ’ ὀ [θ]ῦμο̣ς ̣πεπόηται, γόνα δ’ [ο]ὐ φέροισι,\\
τὰ δή ποτα λαίψηρ’ ἔον ὄρχησθ’ ἴσα νεβρίοισι.

τὰ <μὲν> στεναχίσδω θαμέως ἀλλὰ τί κεν ποείην;\\
ἀγήραον ἄνθρωπον ἔοντ’ οὐ δύνατον γένεσθαι.

καὶ γάρ π̣[ο]τ̣α̣ Τίθωνον ἔφαντο βροδόπαχυν Αὔων\\
ἔρωι φ̣.. ̣ ̣α̣θ̣ε̣ισαν βάμεν’ εἰς ἔσχατα γᾶς φέροισα[ν,

ἔοντα̣ [κ]ά̣λ̣ο̣ν καὶ νέον, ἀλλ’ αὖτον ὔμως ἔμαρψε\\
χρόνωι π̣ό̣λ̣ι̣ο̣ν̣ γῆρας, ἔχ[ο]ν̣τ̣’ ἀθανάταν ἄκοιτιν.
\end{gkverse}

\begin{verse}
\ldots{} [das Musas] de violáceo colo os belos dons, meninas,\\
\ldots{} a melodiosa lira, amante do canto;

[tenra] outrora, agora é a pele da velhice,\\*
\ldots{} os cabelos, de negros [brancos] se tornaram.


Pesado se me fez o peito, e os joelhos não me suportam -- \\
os que um dia foram lépidos no dançar, quais os da corça.

Isso lamento sem cessar, mas que posso fazer?\\
O não-envelhecer não é possível ao ser humano.

Pois, certa vez, dizem que Eos de róseos braços,\\
com paixão \ldots{} carregando Titono aos confins da terra,

belo e jovem que era; mas mesmo a ele alcançou similmente\\
em tempo a grisalha velhice -- a ele que tinha imortal esposa.
\end{verse}

\chapter{Canto, velhice: um convite}


\section{Fragmento 21}

\begin{gkverse}
\textnormal{[\textit{versos 1--2: ilegíveis e lacunares}]}

]α̣νδ’ ὄλοφυν [....]ε̣.\\
] τρομέροις π.[..]α.λλα\\
]\\
  ] χρόα γῆρας ἤδη\\
   ]ν ἀμφιβάσκει\\
     ]ς πέταται διώκων\\
     ]\\
     ]τας ἀγαύας\\
   ]ε̣α, λάβοισα\\
   ] ἄεισον ἄμμι\\
τὰν ἰόκολπον\qquad         ]\\
        ]ρ̣ων μάλιστα\\
         ]ας π[λ]άναται
\end{gkverse}


\begin{verse}
\ldots{} lamento \ldots{}\\
\ldots{} trêmulos \ldots{}\\
\ldots{} a pele da velhice \ldots{}\\
\ldots{} em redor \ldots{}\\
\ldots{} voa, perseguindo\\
\ldots{} brilhante\\
\ldots{} ele pegando\\
\ldots{} canta tu a nós\\
a de violáceo colo \ldots{}\\
\ldots{} sobretudo\\
\ldots{} vagueia \ldots{}
\end{verse}

\medskip

{\paragraph{Comentário} Preservado no \textit{Papiro de Oxirrinco} 1231, em que estão vários
outros fragmentos vistos desde o Fr.\,15, o precário texto traz uma
cena em que, como podemos sugerir, a \textit{persona} da líder do grupo
faz um convite ao cantar, dirigindo"-se a uma das coreutas
imperativamente, após lamentar a própria velhice. Ressoam aqui os versos
da canção do novo fragmento que recorda o mito de Titono e Eos,
justamente a propósito do inevitável envelhecer aos seres humanos.
Naquele fragmento, o envelhecimento é temível à \textit{persona} da líder que canta lamentosamente às
coreutas a chegada do que será impeditivo ao dançar, ressaltando as
marcas no corpo -- pele, cabelos, joelhos -- e o peso das
preocupações e ansiedades no peito. Neste Fr.\,21, é possível que a
velhice seja obstáculo ao cantar que honra a figura de ``colo violáceo''
-- talvez uma das Musas pela mesma qualidade provavelmente referidas na
``Canção sobre a velhice'', ou Afrodite, ou uma coreuta
ou, como no Fr.\,30, uma noiva cujo colo é qualificado pelo adjetivo \textit{iókolpon}, ``de violáceo colo''. Se, porém, não pode a líder cantar, que
cante a ela, às demais \textit{parthénoi} do coro e à audiência uma das
coreutas, uma das jovens de seu grupo. Afinal, parece dizer a poeta, a
\textit{performance} não deve parar.}



\chapter{De cantos, cordas, prêmio: imortalidade?}

\paragraph{\textsc{nota introdutória}}
Do mesmo papiro que preservou a ``Canção sobre a velhice'' vem, de mais legível, um outro novo fragmento breve, do qual temos os versos finais. Neles, estão presentes festa, o canto da \textit{persona} e instrumentos de corda. Precisamente, a ``harpa'', \textit{pâktin}, que referida no Fr.\,22, e talvez a \textit{khelýnna}, um tipo de lira nomeado pela casca da tartaruga que lhe servia de caixa de ressonância e que designa o instrumento na ``Canção sobre a velhice''.
Tais cordófonos se combinam, ressoando as canções que já aqui ouvimos -- incluindo aquelas que cantam o próprio cantar. O ``prêmio'', \textit{géras}, talvez seja a ``grande fama das Musas'', \textit{kléos méga Moíseion}. No presente da canção, tempo enfatizado por duas vezes, emergem as ideias da morte, do estar sob a terra, e do estar vivo, sobre ela. Haveria no precário texto algo relativo à contemplação de Safo no Hades, morta, e a sua imortalização pela poesia, como vimos nos Frs.~55, 65 e 147? A resposta seria sim, e decerto, neste caso, ``prêmio'' maior e mais caro não há à poeta que continuamos a cantar:\footnote{Texto grego nas edições dadas em Buzzi \textit{et alii} (2008, pp.~21 e 57) e Greene e Skinner (2009, p.~10). As sugestões do suplemento que dá sentido ao ``prêmio'', aceita naquela, e do que traz a \textit{khelýnna} ao verso final, aceita nesta, baseiam-se nos estudos de Gronewald e Daniel, e de West, referidos à nota à ``Canção sobre a velhice''.}

\pagebreak

\begin{gkverse}
\textnormal{[\textit{versos 1--4: ilegíveis e lacunares}]}

      ] ε̣ὔχο̣μ[\\
]. νῦν θ̣αλ̣[ί]α γε̣[\\
] ν̣έρθε δὲ γᾶς γε̣[νοίμα]ν̣\\
]..ν̣ ἔχο̣ισαν γέρας ὠς [ἔ]οικεν,\\
]ζοεν ὠς νῦν ἐπὶ γᾶς ἔοισαν.\\
] λιγύραν, [α]ἴ ̣κεν ἔλοισα πᾶκτιν\\
        χε]λύν̣ν̣αν̣.αλαμοις ἀείδω. 
\end{gkverse}

\begin{verse}
\ldots{} rezo \ldots{}\\
\ldots{} agora festividade \ldots{}\\
\ldots{} sob a terra viria a ser;\\
\ldots{} tendo prêmio como é justo\\
\ldots{} como agora sobre a terra estando\\ 
\ldots{} se clara harpa agarrasse\\
\ldots{} lira \ldots{} eu canto.
\end{verse}


